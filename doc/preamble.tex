% Um alles auf einmal pflegen zu können, verwenden wir \ifhtml als
% Anzeige für tex4ht oder „normales“ (PDF)LaTeX.
%
\makeatletter

\let\ifhelp\iffalse
\let\ifhtml\iffalse
\let\ifpdf\iffalse
\let\ifdvi\iffalse

\def\mutabor@utput@html{
  \typeout{Producing HTML format.}
  \let\ifhtml\iftrue
}

\def\mutabor@utput@help{%
  \mutabor@utput@html
  \typeout{Subtype: .hhp .}
  \let\ifhelp\iftrue
}

\@ifundefined{mutabor@utput@\outputformat}{
  \typeout{Undefined output format%
    \@ifundefined{outputformat}{}{ \outputformat}.
  }
}{
  \csname mutabor@utput@\outputformat\endcsname
}

%\def\htmltrue{\let\ifhtml\iftrue}
%\def\htmlfalse{\let\ifhtml\iffalse}
%\ifx\ifhtml\undefined
%  \htmlfalse
%\else
%  \htmltrue
%\fi
% 
% Ein Kommentar aus dem alten Handbuch... ;-)
%
% =====================================================
%
% NOCH ZU KORRIGIEREN :
%
%
% - oberflächenspezifische Beschreibungen auslagern !
%
% =====================================================
%
% Koma-Skript liefert ein paar nützliche zusätzliche Definitionen
%
\documentclass[german,a4paper,BCOR1.0cm]{scrbook}
%
% Für tex4ht müssen wir auch tex4ht laden...
\ifhtml
\setcounter{tocdepth}{1}
% Grundeinstellungen liegen in der Datei „tshtml.cfg“
\usepackage[%tshtml,
mutabor,
% Es folgt das Hauptformat: hier xhtml
            html,
% CSS2 verwenden
%            css2,
% in der Log-Datei Informationen zur Konfiguration von tex4ht ausgeben
          info,
% Fußnoten (und Literaturzitate) erscheinen beim überfahren der Marke
%            mouseover,
% Auf der ersten Seite wird auch ein „Weiter“-Pfeil angezeigt
            next,
            sections+,
% HTML-Dateien möglichst kurz -- maximale Aufsplittungstiefe
            4,
 %            nominitoc,
% Index 2-Spaltig
            index=2,
            Gin-dim+,
% HTML-Zeichensatz: UTF-8
            charset=utf-8,
            hyperref,
            NoFonts,
%            fonts+
%            fonts,
]{tex4ht}
\makeatletter
\edef\ts@savecatcode{\noexpand\catcode`\noexpand\:\the\catcode`\:\relax} 
%\catcode`\:11\relax
\AtEndDocument{\makehhk}
\newcommand\tshhkentry{}
\newcommand\tsatoclink{}

%\:::HRefTag=macro:
%#1#2->\if \relax #2\relax \else \:TagHTag {#2}\fi \HCode {<\tag:A \:newlnch \if
% \relax #1\relax \NOHREF: {#2}\else \HREF: "\get:hfile {#1}\:sharp #1"\fi \if \
%relax #2\relax \else \space \NAME: "#2"\fi \:attr \empty:lnk >}.
%<insert>    \show\:::HRefTag

\expandafter\def\expandafter\mut@sharp\expandafter{\csname :sharp\endcsname}

\newif\ifwx@inrange
\wx@inrangefalse
\def\wx@item#1{%
  \def\wx@item@{#1}%
  \let\wx@indextext\wx@item@
}
\def\wx@subitem#1{%
  \def\wx@subitem@{\wx@item@{}: #1}
  \let\wx@indextext\wx@subitem@
}
\def\wx@subsubitem#1{%
  \def\wx@subsubitem@{\wx@subitem@{} -- #1}%
  \let\wx@indextext\wx@subsubitem@
}
\def\wx@LNK#1#2#3#4{%
  \ifwx@inrange
  \wx@inrangefalse
  \else
  \HCode{<li>^^J}%
  \HCode{<object type="text/sitemap" >^^J%
\space<param name="Name" value="}\wx@indextext\HCode{" >^^J%
\space<param name="Local" value="}#1\mut@sharp #2\HCode{" >^^J%
</object>^^J}% 
  \HCode{</li>^^J}%
  \fi
}
\def\loadwxindex{%
  {%
    \let\item\wx@item
    \let\subitem\wx@subitem
    \let\subsubitem\wx@subsubitem
    \let\LNK\wx@LNK
    \let\rangeto\wx@inrangetrue
    \InputIfFileExists{\jobname.wxi}{}{}%
  }%
}



\def\tsatoclink#1#2#3#4{%
  \typeout{Setting: \csname get:hfile\endcsname{#2} at \csname :sharp\endcsname #2}%
  \HCode{%
\tslinestart<object type="text/sitemap" >^^J
\tslinestart\space<param name="Name" value="}#4\HCode{" >^^J
\tslinestart\space<param name="Local" value="}\csname get:hfile\endcsname{#2}\csname :sharp\endcsname #2\HCode{" >^^J
\tslinestart</object>^^J}% "<
%   \expandafter\ifx \csname #3-def\endcsname\relax
%      \global \expandafter\let \csname #3-def\endcsname\def
%      \Link {#2}{#3}
%   \else
%      \Link {#2}{}
%   \fi 
%   {\Configure {ref}{}{}{}%
%     \let \EndLink =\empty
%     \let \H:Tag:attr \:gobbleII
%     \let \:::HRef \empty
%     \def \::hRef [##1]##2{}%
%     \def \::hRefTag [##1]##2##3{}%
%     \def \:::HRefTag ##1##2{}%
%     \Configure {cite}{}{}{}{}%
%     #4}%
%   \EndLink
}
\def\ts@hhkentry#1#2#3#4{%
  \def\tslinestart{#1\space}%
  \HCode{#1<li>^^J}%
  #3%
  \HCode{#1</li>^^J}%
}
\def\tshhkentry{\ts@hhkentry{}}
\def\tstocendsubparagraph{}
\def\tstocendparagraph{\endsubparagraph}

\def\deftocmacro#1#2#3{
  \expandafter\def\csname tstocstart#1\endcsname{%
    \HCode{#3\space<ul><!-- u#1 -->^^J}%
    \expandafter\let\csname tocstart#1\endcsname\relax
    \expandafter\def\csname tocstop#1\expandafter\endcsname{\csname tstocstop#1\endcsname}
  }

  \expandafter\def\csname tstocskip#1\endcsname{\csname tocstop#2\endcsname}

  \expandafter\def\csname tstocstop#1\endcsname{%
    \csname tocstop#2\endcsname
    \HCode{#3\space</ul><!-- i#1 -->^^J}%
    \expandafter\def\csname tocstop#1\endcsname{\csname tstocskip#1\endcsname}
  }

  \expandafter\def\csname tstoc#2\endcsname{%
    \csname tocstop#2\endcsname
    \csname tocstart#1\endcsname
    \expandafter\def\csname tocstart#2\endcsname{\csname tstocstart#2\endcsname}
    \HCode{#3\space\space<!-- a#2 -->^^J}%
    \ts@hhkentry{#3\space\space}%
  }%
}

\def\tstocpart{
  \tocstoppart
  \ts@hhkentry{}%
}

\deftocmacro{part}{chapter}{}
\deftocmacro{chapter}{section}{\space}
\deftocmacro{section}{subsection}{\space\space}
\deftocmacro{subsection}{subsubsection}{\space\space\space}
\deftocmacro{subsubsection}{paragraph}{\space\space\space\space}
\deftocmacro{paragraph}{subparagraph}{\space\space\space\space\space}
\def\tocstopsubparagraph{}

\def\gobblenl{\@ifnextchar[\@gobblenl{}}
\def\@gobblenl[#1]{}
{\catcode`\^^J=\active
  \gdef\mknlsp{%
    \def^^J{ }}%
}
\def\@nl@end{nl@end}
\def\@removenl#1#2\@nl@end{%
  \ifx#1^^J
  \ 
  \else
  #1
  \fi
  \ifx#2\relax
  \else
  \expandafter\@removenl
  \fi
}

\newcommand\removenewline[1]{%
  \ifx#1\relax
  \else
    \@removenl#1\@nl@end
  \fi
}

\newcommand\mutabortitle{}
\ifhelp
  \let\mutaborsavesubject\subject
  \def\subject#1{%
    \def\mutabortitle{#1}%
    \mutaborsavesubject{#1}%
  }
\else
  \let\mutaborsavetitle\title
  \def\title#1{%
    \def\mutabortitle{#1}%
    \mutaborsavetitle{#1}%
  }
\fi
\def\mutabordefaulttopic{\jobname.html}
\newcommand\tsarg{}
\def\tsarg#1{#1}
\let\tsdotocentry\tsarg
\expandafter\let\expandafter\mut@gobbleIV\csname :gobbleIV\endcsname
\expandafter\let\expandafter\mut@gobbleIII\csname :gobbleIII\endcsname
\expandafter\let\expandafter\mut@gobble\csname :gobble\endcsname
\newcommand\makehhk{%
  {\ignorespaces
    \let\par\relax
    \def\showname##1{\expandafter\show\csname ##1\endcsname}%
    \let\@gnewline\space
    \def\@newline{\space\mut@gobbleIII}%
    \let\texorpdfstring\@secondoftwo%
    \let\fontencoding\mut@gobble
    \let\fontfamily\mut@gobble
    \let\fontseries\mut@gobble
    \let\fontshape\mut@gobble
    \let\fontsize\mut@gobble
    \let\@setfontsize\mut@gobbleIII
    \let\usefont\mut@gobbleIV 
    \let\selectfont\relax
    \let\LARGE\relax
    \let\normalsize\relax
    % .hhp-Datei erstellen.
    \special{t4ht>\jobname.hhp}%
    \HCode{%
Contents file=\jobname.hhc^^J%
Index file=\jobname.hhk^^J%
Title=}{%
      \removenewline\expandafter{\mutabortitle}}%
    \HCode{^^J%
Default Topic=\mutabordefaulttopic^^J
Charset=UTF-8^^J}%
\special{t4ht<\jobname.hhp}%
\typeout{done}%
%\expandafter\show\csname a:TocLink\endcsname
    \let\doTocEntry\tsdotocentry
    \expandafter\let\csname a:TocLink\endcsname\tsatoclink
    \let\toclikesection\tshhkentry
    \let\toclikechapter\tshhkentry
    \let\tocaddchap\tshhkentry
    \let\tocpart\tshhkentry
    \let\tocsection\tshhkentry
    \let\tocchapter\tshhkentry
    \let\tocsubsection\tshhkentry
    \let\tocparagraph\tshhkentry
    \let\tocappendix\tshhkentry
    \let\tocminisec\tshhkentry
    \let\textsc\tsarg%
    \special{t4ht>\jobname.hhk}%
    \HCode{<ul>^^J}%
    \loadwxindex 
%    \catcode`\#11
    \InputIfFileExists{\jobname.4ct}{}{}%
\typeout{finished}%
    \HCode{</ul>^^J}%
    \special{t4ht<\jobname.hhk}%
\typeout{closed}%
%
    \let\toclikesection\tstocsection
    \let\toclikechapter\tstocchapter
    \let\tocaddchap\tstocchapter
    \let\tocpart\tstocpart
    \let\tocsection\tstocsection
    \let\tocchapter\tstocchapter
    \let\tocsubsection\tstocsubsection
    \let\tocparagraph\tstocparagraph
    \let\tocappendix\tstocchapter
    \def\tocminisec{\tocparagraph}%
%
    \let\tocstartpart\relax
    \let\tocstoppart\tstocskippart
    \let\tocstartchapter\relax
    \let\tocstopchapter\tstocskipchapter
    \let\tocstartsection\relax
    \let\tocstopsection\tstocskipsection
    \let\tocstartsubsection\relax
    \let\tocstopsubsection\tstocskipsubsection
    \let\tocstartsubsubsection\relax
    \let\tocstopsubsubsection\tstocskipsubsubsection
    \let\tocstartparagraph\relax
    \let\tocstopparagraph\tstocskipparagraph
    \let\tocstartsubparagraph\relax
    \let\tocstopsubparagraph\tstocskipsubparagraph
%
    \special{t4ht>\jobname.hhc}%
    \HCode{<ul>^^J}%
%    \catcode`\#11
    \InputIfFileExists{\jobname.4ct}{}{}%
    \HCode{</ul>^^J}%
    \special{t4ht<\jobname.hhc}%
  }%
  \relax
}
\fi
% Einstellungen für Latex laden
\usepackage[utf8]{inputenc}
\usepackage[T1]{fontenc}
\usepackage[german]{babel}
% Exteren Referenzen und Hyperref laden. 
% Das kann unterschiedlich ablaufen.
\ifhtml
 %  \Configure{html}{html.de}
  \ifpdfoutput{\pdfoutput0\relax}{}
   \usepackage{xr-hyper}
   \usepackage{hyperref}
\else
 %  \usepackage{mathpazo}
  \usepackage{xr-hyper}
  \usepackage[extension=pdf]{hyperref}
\fi
% Überschriften als Referenzen einfügen.
\usepackage{nameref}
\usepackage{color}
%\ifhtml
%\input nameref.4ht
%\fi
% Papierformat a4weit wird nicht empfohlen
%\usepackage{a4wide}
%
% Die üblichen Pakete für Grafik und Farben 
\usepackage{graphicx}
\usepackage{color}
%
% Das Handbuch wurde mit emTeX geschrieben. TeTeX versteht die emTeX
% specials. Also verwenden wir sie.
\usepackage{emlines}
% Makeindex laden
\usepackage{makeidx}
%
% file: macht uns die relativen Links kaputt :-(
%
\hyperlinkfileprefix{}
%
% Kolumnentitel erleichtern dem Leser die Orientierung
%
\AtBeginDocument{\pagestyle{headings}}
% 
% Titelei
\setkomafont{title}{\fontfamily{\rmdefault}\fontseries{bx}\huge}
\title{\rmfamily\texorpdfstring{\mutabor\\[\baselineskip]}{MUTABOR --}
 \LARGE\slshape Ein computergesteuertes Musikinstrument \\
	  zum Experimentieren mit\\
	  Stimmungslogiken und Mikrotönen}
\author{Volker Abel, Peter Reiss,\\ Rüdiger Krauße und Tobias Schlemmer}
\date{Programmversion $3.0x$ (\the\year)}
\ifhtml\else
  \publishers{\includegraphics[width=0.5\linewidth]{start}}
\fi
\lowertitleback{\footnotesize\copyright 1991, 1992 Volker Abel \& Peter Reiss\\
\copyright 2006 TU Dresden, Institut für Algebra}
%
% Index-Datei öffnen
%
\makeindex
\hyphenation{wei-te-re Ton-sys-tem Ton-sys-te-me}
%\parindent 0mm
%\parskip 5pt
%\textheight 18.5cm
%
% Jetzt wirds ein wenig wüst.
%
% Wir wollen mit wenig Aufwand die Labels auch für Hyperlinks
% verwenden. Damit sparen wir uns zusätzliche Anker.
% 
\makeatletter
%
% \XR@ext enthält die Erweiterung für Querverweise. Wenn wir auf
% Buchanfänge usw. verweisen wollen, kann uns xr-hyper nicht
% undbedingt helfen. (oder wir müsste das entsprechend definieren).
% Einfacher ist es wohl, wir nehmen den Dateinamen.
%
\newcommand\makefilename[1]{#1.\XR@ext}
%
%
% erstes von sechs Argumenten wiedergeben und Rest verwerfen
\def\ts@firstofsix#1#2#3#4#5#6{#1}
%
\ifhtml
% voreinstellungen, um „:“ als Buchstaben zu behandeln (für tex4ht)
% 
% Herausfiltern des Verlinkungsmakros aus Querverweis-Speichern
%
\def\ts@parse@ref@a#1#2\ts@end@parse{\ts@parse@ref@b#1\ts@end@parse}
\def\ts@parse@ref@b#1#2#3\ts@end@parse{#1{#2}}
%
% Setzen des Verweises.
% Das erste Argument mus zunächst expandiert werden, bevor es
% überhaupt mit den obigen makros geparst werden kann. Wir sichern uns
% das ganze in einem temporären Makro zusammen mit dem Ankertext. 
%
\def\ts@setref#1#2#3{%
  \expandafter\expandafter
  \expandafter\def
  \expandafter\expandafter
  \expandafter\ts@@tmp
  \expandafter\expandafter
  \expandafter{%
    \expandafter\ts@parse@ref@a#1\ts@end@parse{#2}}%
%
% Wir hacken uns in \@setref hineein.
% dort steht: \expandafter #2#1. Wir liefern aber alles, was wir
% brauchen schon in #2 mit, verwerfen also alles aus #1.
% wir machen daraus \expandafter\ts@firstofsix\expandafter\ts@@tmp#1
% damit wird zunächst #1 expandiert und dann verworfen.
% Genial nicht ;-)?
%
% Alternativ könnte man auch mit \@firstoftwo arbeiten. Hier muss man
% aufpassen, dass das erste Argument nicht falsch expandiert wird.
% \setref#1{\@firstoftwo{\ts@@tmp}}{#3}
%
  \@setref#1{% 
    \ts@firstofsix%
      \expandafter\ts@@tmp}{#3}%
}
%
% Jetzt können wir das eigentliche Linkmakro definieren. 
%
% Wenn die Referenz nicht definiert ist, setzen wir den Ankertext und
% rufen setref auf, damit die entsprechende Warnung ausgespuckt
% wird. Die ausgegebenen Fragezeichen setzen wir in weißer Farbe in
% eine 0pt breite Box (llap). Damit kommt es nur zu minimalen
% Verschiebungen (Kerning) zwischen undefiniert und definiert. 
%
% Ist die Referenz definiert, wird sie verwendet, um einen Link zu erzeugen.
\newcommand\ts@reflink[2]{%
  \@ifundefined{r@#1}{%
    \textcolor{red}{%
      #2%
      \color{white}{%
        \llap{%
          \@safe@activestrue
          \edef \RefArg {#1}
          \expandafter\ts@setref\csname r@#1\endcsname{{\@safe@activesfalse #2}}{#1}%
          \@safe@activesfalse
        }%
      }%
    }%
  }{%
    \@safe@activestrue
%    \let\::ref \T:ref
    \expandafter\ts@setref\csname r@#1\endcsname{{\@safe@activesfalse #2}}{#1}%
%    \def\::ref{\protect\T@ref}%
    \@safe@activesfalse
  }%
}
% : wiederherstellen
%\ts@savecatcode
\else
% Hier läuft es eigentlich genauso ab, wie bei der
% tex4ht-Variante. Nur werden jetzt die Verweise etwas anders kodiert,
% so dass man sie nicht wirklich neu parsen muss.
\newcommand\ts@reflink[2]{%
\begingroup%\tracingall
  \def\ts@tmp{{\@safe@activesfalse #2}}%
  \@ifundefined{r@#1}{%
    \textcolor{red}{%
      #2%
      \color{white}{%
        \llap{%
          \@safe@activestrue
          \expandafter\@setref\csname r@#1\endcsname{\ts@firstofsix\ts@tmp}{#1}%
          \@safe@activesfalse
        }%
      }%
    }%
  }{%
    \@safe@activestrue
    \expandafter\@setref\csname r@#1\endcsname{\ts@firstofsix\ts@tmp}{#1}%
    \@safe@activesfalse
  }%
%\show\ts@reflink
\endgroup
}
\fi
%\newcommand\ts@@reflink{\protect\ts@reflink}
%
% die anderen definierten Referenz-Makros sind auch \protect-et
% definiert. Also machen wir das auch mit einem Querverweis auf ein
% Label mit einem beliebigen Ankertext.
%
\newcommand\reflink{\protect\ts@reflink}
%
% Referenz durch Titel, ggf. Zusatz (wie z.\,B. Buchname bei externen
% Referenzen) und Seite bei nicht-HTML-Ausgabe. Der Zusatz wird in []
% angegeben. Das erledigen wir durch zwei Makros:
%
\newcommand\tsciteref[1]{%
  \frqq\nameref{#1}\flqq
  \@ifnextchar[{\ts@cite@ref{#1}}{\ts@cite@@ref{#1}}
}
% mit optionalem Argument: Leerzeichen+Zwischentext und dann
% Seitenangabe ohne zweites Arg.
\def\ts@cite@ref#1[#2]{ #2\ts@cite@@ref{#1}}
% Ohne optionales Argument: Seitenangabe
\def\ts@cite@@ref#1{%
  \ifhtml\else{}
  auf Seite \pageref{#1}%
  \fi
}
% So ziemlich alles, was man referenzieren kann:
% Verweis-Typ: Kapitel, Abschnitt usw. 
% Nummer, Titel, ggf. optionales Argument, ggf. Seite.
% Syntax \vollref{Label}[zwischentext] (letzteres wird durch
% \tsciteref eingeführt
\newcommand\vollref[1]{\autoref{#1} \tsciteref{#1}}
%
% Und hier sind wir aus den Interna heraus.
%
\makeatother
%
% Ein Makro zur vereinfachten Umwandlung der Mutabor-Hilfe mit Querverweisen.
%
\newcommand\tsreflink[2]{\reflink{sec:#2}{#1}}
%
% Der Name unseres Programmes
%
\newcommand\mutabor[1][]{\textsc{Mutabor#1}}
%
% Vorbereitung zur Verwendung des Pakets keystroke. Damit werden
% Tasten als Piktogramme angezeigt.
%
\usepackage{keystroke}%
%\providecommand\keystroke[1]{#1}
%
% Schlüsselworte und sonstiger Quelltext in Schreibmaschinenschrift.
%
\newcommand\keyword[1]{\texttt{#1}}
%
% Wir wollen die Handbücher ja nicht nur in die CD integrieren.
%
\ifhtml
  \newcommand\cdoronline[2]{#1}
\else
  \newcommand\cdoronline[2]{#2}
\fi

\newcommand\sourcecode{\texttt}%
\newcommand\filename{\texttt}

\newcommand\mutimage[2]{%
  \ifhtml
  \Picture{#1}%
  \else
  \includegraphics#2
  \fi
}

\newcommand\translate[1]{\textcolor{red}{#1}}

\endinput

