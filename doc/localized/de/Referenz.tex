% Um alles auf einmal pflegen zu können, verwenden wir \ifhtml als
% Anzeige für tex4ht oder „normales“ (PDF)LaTeX.
%
\makeatletter

\let\ifhelp\iffalse
\let\ifhtml\iffalse
\let\ifpdf\iffalse
\let\ifdvi\iffalse

\def\mutabor@utput@html{
  \typeout{Producing HTML format.}
  \let\ifhtml\iftrue
}

\def\mutabor@utput@help{%
  \mutabor@utput@html
  \typeout{Subtype: .hhp .}
  \let\ifhelp\iftrue
}

\@ifundefined{mutabor@utput@\outputformat}{
  \typeout{Undefined output format%
    \@ifundefined{outputformat}{}{ \outputformat}.
  }
}{
  \csname mutabor@utput@\outputformat\endcsname
}

%\def\htmltrue{\let\ifhtml\iftrue}
%\def\htmlfalse{\let\ifhtml\iffalse}
%\ifx\ifhtml\undefined
%  \htmlfalse
%\else
%  \htmltrue
%\fi
% 
% Ein Kommentar aus dem alten Handbuch... ;-)
%
% =====================================================
%
% NOCH ZU KORRIGIEREN :
%
%
% - oberflächenspezifische Beschreibungen auslagern !
%
% =====================================================
%
% Koma-Skript liefert ein paar nützliche zusätzliche Definitionen
%
\documentclass[german,a4paper,BCOR1.0cm]{scrbook}
%
% Für tex4ht müssen wir auch tex4ht laden...
\ifhtml
\setcounter{tocdepth}{1}
% Grundeinstellungen liegen in der Datei „tshtml.cfg“
\usepackage[%tshtml,
mutabor,
% Es folgt das Hauptformat: hier xhtml
            html,
% CSS2 verwenden
%            css2,
% in der Log-Datei Informationen zur Konfiguration von tex4ht ausgeben
          info,
% Fußnoten (und Literaturzitate) erscheinen beim überfahren der Marke
%            mouseover,
% Auf der ersten Seite wird auch ein „Weiter“-Pfeil angezeigt
            next,
            sections+,
% HTML-Dateien möglichst kurz -- maximale Aufsplittungstiefe
            4,
 %            nominitoc,
% Index 2-Spaltig
            index=2,
            Gin-dim+,
% HTML-Zeichensatz: UTF-8
            charset=utf-8,
            hyperref,
            NoFonts,
%            fonts+
%            fonts,
]{tex4ht}
\makeatletter
\edef\ts@savecatcode{\noexpand\catcode`\noexpand\:\the\catcode`\:\relax} 
%\catcode`\:11\relax
\AtEndDocument{\makehhk}
\newcommand\tshhkentry{}
\newcommand\tsatoclink{}

%\:::HRefTag=macro:
%#1#2->\if \relax #2\relax \else \:TagHTag {#2}\fi \HCode {<\tag:A \:newlnch \if
% \relax #1\relax \NOHREF: {#2}\else \HREF: "\get:hfile {#1}\:sharp #1"\fi \if \
%relax #2\relax \else \space \NAME: "#2"\fi \:attr \empty:lnk >}.
%<insert>    \show\:::HRefTag

\expandafter\def\expandafter\mut@sharp\expandafter{\csname :sharp\endcsname}

\newif\ifwx@inrange
\wx@inrangefalse
\def\wx@item#1{%
  \def\wx@item@{#1}%
  \let\wx@indextext\wx@item@
}
\def\wx@subitem#1{%
  \def\wx@subitem@{\wx@item@{}: #1}
  \let\wx@indextext\wx@subitem@
}
\def\wx@subsubitem#1{%
  \def\wx@subsubitem@{\wx@subitem@{} -- #1}%
  \let\wx@indextext\wx@subsubitem@
}
\def\wx@LNK#1#2#3#4{%
  \ifwx@inrange
  \wx@inrangefalse
  \else
  \HCode{<li>^^J}%
  \HCode{<object type="text/sitemap" >^^J%
\space<param name="Name" value="}\wx@indextext\HCode{" >^^J%
\space<param name="Local" value="}#1\mut@sharp #2\HCode{" >^^J%
</object>^^J}% 
  \HCode{</li>^^J}%
  \fi
}
\def\loadwxindex{%
  {%
    \let\item\wx@item
    \let\subitem\wx@subitem
    \let\subsubitem\wx@subsubitem
    \let\LNK\wx@LNK
    \let\rangeto\wx@inrangetrue
    \InputIfFileExists{\jobname.wxi}{}{}%
  }%
}



\def\tsatoclink#1#2#3#4{%
  \typeout{Setting: \csname get:hfile\endcsname{#2} at \csname :sharp\endcsname #2}%
  \HCode{%
\tslinestart<object type="text/sitemap" >^^J
\tslinestart\space<param name="Name" value="}#4\HCode{" >^^J
\tslinestart\space<param name="Local" value="}\csname get:hfile\endcsname{#2}\csname :sharp\endcsname #2\HCode{" >^^J
\tslinestart</object>^^J}% "<
%   \expandafter\ifx \csname #3-def\endcsname\relax
%      \global \expandafter\let \csname #3-def\endcsname\def
%      \Link {#2}{#3}
%   \else
%      \Link {#2}{}
%   \fi 
%   {\Configure {ref}{}{}{}%
%     \let \EndLink =\empty
%     \let \H:Tag:attr \:gobbleII
%     \let \:::HRef \empty
%     \def \::hRef [##1]##2{}%
%     \def \::hRefTag [##1]##2##3{}%
%     \def \:::HRefTag ##1##2{}%
%     \Configure {cite}{}{}{}{}%
%     #4}%
%   \EndLink
}
\def\ts@hhkentry#1#2#3#4{%
  \def\tslinestart{#1\space}%
  \HCode{#1<li>^^J}%
  #3%
  \HCode{#1</li>^^J}%
}
\def\tshhkentry{\ts@hhkentry{}}
\def\tstocendsubparagraph{}
\def\tstocendparagraph{\endsubparagraph}

\def\deftocmacro#1#2#3{
  \expandafter\def\csname tstocstart#1\endcsname{%
    \HCode{#3\space<ul><!-- u#1 -->^^J}%
    \expandafter\let\csname tocstart#1\endcsname\relax
    \expandafter\def\csname tocstop#1\expandafter\endcsname{\csname tstocstop#1\endcsname}
  }

  \expandafter\def\csname tstocskip#1\endcsname{\csname tocstop#2\endcsname}

  \expandafter\def\csname tstocstop#1\endcsname{%
    \csname tocstop#2\endcsname
    \HCode{#3\space</ul><!-- i#1 -->^^J}%
    \expandafter\def\csname tocstop#1\endcsname{\csname tstocskip#1\endcsname}
  }

  \expandafter\def\csname tstoc#2\endcsname{%
    \csname tocstop#2\endcsname
    \csname tocstart#1\endcsname
    \expandafter\def\csname tocstart#2\endcsname{\csname tstocstart#2\endcsname}
    \HCode{#3\space\space<!-- a#2 -->^^J}%
    \ts@hhkentry{#3\space\space}%
  }%
}

\def\tstocpart{
  \tocstoppart
  \ts@hhkentry{}%
}

\deftocmacro{part}{chapter}{}
\deftocmacro{chapter}{section}{\space}
\deftocmacro{section}{subsection}{\space\space}
\deftocmacro{subsection}{subsubsection}{\space\space\space}
\deftocmacro{subsubsection}{paragraph}{\space\space\space\space}
\deftocmacro{paragraph}{subparagraph}{\space\space\space\space\space}
\def\tocstopsubparagraph{}

\def\gobblenl{\@ifnextchar[\@gobblenl{}}
\def\@gobblenl[#1]{}
{\catcode`\^^J=\active
  \gdef\mknlsp{%
    \def^^J{ }}%
}
\def\@nl@end{nl@end}
\def\@removenl#1#2\@nl@end{%
  \ifx#1^^J
  \ 
  \else
  #1
  \fi
  \ifx#2\relax
  \else
  \expandafter\@removenl
  \fi
}

\newcommand\removenewline[1]{%
  \ifx#1\relax
  \else
    \@removenl#1\@nl@end
  \fi
}

\newcommand\mutabortitle{}
\ifhelp
  \let\mutaborsavesubject\subject
  \def\subject#1{%
    \def\mutabortitle{#1}%
    \mutaborsavesubject{#1}%
  }
\else
  \let\mutaborsavetitle\title
  \def\title#1{%
    \def\mutabortitle{#1}%
    \mutaborsavetitle{#1}%
  }
\fi
\def\mutabordefaulttopic{\jobname.html}
\newcommand\tsarg{}
\def\tsarg#1{#1}
\let\tsdotocentry\tsarg
\expandafter\let\expandafter\mut@gobbleIV\csname :gobbleIV\endcsname
\expandafter\let\expandafter\mut@gobbleIII\csname :gobbleIII\endcsname
\expandafter\let\expandafter\mut@gobble\csname :gobble\endcsname
\newcommand\makehhk{%
  {\ignorespaces
    \let\par\relax
    \def\showname##1{\expandafter\show\csname ##1\endcsname}%
    \let\@gnewline\space
    \def\@newline{\space\mut@gobbleIII}%
    \let\texorpdfstring\@secondoftwo%
    \let\fontencoding\mut@gobble
    \let\fontfamily\mut@gobble
    \let\fontseries\mut@gobble
    \let\fontshape\mut@gobble
    \let\fontsize\mut@gobble
    \let\@setfontsize\mut@gobbleIII
    \let\usefont\mut@gobbleIV 
    \let\selectfont\relax
    \let\LARGE\relax
    \let\normalsize\relax
    % .hhp-Datei erstellen.
    \special{t4ht>\jobname.hhp}%
    \HCode{%
Contents file=\jobname.hhc^^J%
Index file=\jobname.hhk^^J%
Title=}{%
      \removenewline\expandafter{\mutabortitle}}%
    \HCode{^^J%
Default Topic=\mutabordefaulttopic^^J
Charset=UTF-8^^J}%
\special{t4ht<\jobname.hhp}%
\typeout{done}%
%\expandafter\show\csname a:TocLink\endcsname
    \let\doTocEntry\tsdotocentry
    \expandafter\let\csname a:TocLink\endcsname\tsatoclink
    \let\toclikesection\tshhkentry
    \let\toclikechapter\tshhkentry
    \let\tocaddchap\tshhkentry
    \let\tocpart\tshhkentry
    \let\tocsection\tshhkentry
    \let\tocchapter\tshhkentry
    \let\tocsubsection\tshhkentry
    \let\tocparagraph\tshhkentry
    \let\tocappendix\tshhkentry
    \let\tocminisec\tshhkentry
    \let\textsc\tsarg%
    \special{t4ht>\jobname.hhk}%
    \HCode{<ul>^^J}%
    \loadwxindex 
%    \catcode`\#11
    \InputIfFileExists{\jobname.4ct}{}{}%
\typeout{finished}%
    \HCode{</ul>^^J}%
    \special{t4ht<\jobname.hhk}%
\typeout{closed}%
%
    \let\toclikesection\tstocsection
    \let\toclikechapter\tstocchapter
    \let\tocaddchap\tstocchapter
    \let\tocpart\tstocpart
    \let\tocsection\tstocsection
    \let\tocchapter\tstocchapter
    \let\tocsubsection\tstocsubsection
    \let\tocparagraph\tstocparagraph
    \let\tocappendix\tstocchapter
    \def\tocminisec{\tocparagraph}%
%
    \let\tocstartpart\relax
    \let\tocstoppart\tstocskippart
    \let\tocstartchapter\relax
    \let\tocstopchapter\tstocskipchapter
    \let\tocstartsection\relax
    \let\tocstopsection\tstocskipsection
    \let\tocstartsubsection\relax
    \let\tocstopsubsection\tstocskipsubsection
    \let\tocstartsubsubsection\relax
    \let\tocstopsubsubsection\tstocskipsubsubsection
    \let\tocstartparagraph\relax
    \let\tocstopparagraph\tstocskipparagraph
    \let\tocstartsubparagraph\relax
    \let\tocstopsubparagraph\tstocskipsubparagraph
%
    \special{t4ht>\jobname.hhc}%
    \HCode{<ul>^^J}%
%    \catcode`\#11
    \InputIfFileExists{\jobname.4ct}{}{}%
    \HCode{</ul>^^J}%
    \special{t4ht<\jobname.hhc}%
  }%
  \relax
}
\fi
% Einstellungen für Latex laden
\usepackage[utf8]{inputenc}
\usepackage[T1]{fontenc}
\usepackage[german]{babel}
% Exteren Referenzen und Hyperref laden. 
% Das kann unterschiedlich ablaufen.
\ifhtml
 %  \Configure{html}{html.de}
  \ifpdfoutput{\pdfoutput0\relax}{}
   \usepackage{xr-hyper}
   \usepackage{hyperref}
\else
 %  \usepackage{mathpazo}
  \usepackage{xr-hyper}
  \usepackage[extension=pdf]{hyperref}
\fi
% Überschriften als Referenzen einfügen.
\usepackage{nameref}
%\ifhtml
%\input nameref.4ht
%\fi
% Papierformat a4weit wird nicht empfohlen
%\usepackage{a4wide}
%
% Die üblichen Pakete für Grafik und Farben 
\usepackage{graphicx}
\usepackage{color}
%
% Das Handbuch wurde mit emTeX geschrieben. TeTeX versteht die emTeX
% specials. Also verwenden wir sie.
\usepackage{emlines}
% Makeindex laden
\usepackage{makeidx}
%
% file: macht uns die relativen Links kaputt :-(
%
\hyperlinkfileprefix{}
%
% Kolumnentitel erleichtern dem Leser die Orientierung
%
\AtBeginDocument{\pagestyle{headings}}
% 
% Titelei
\setkomafont{title}{\fontfamily{\rmdefault}\fontseries{bx}\huge}
\title{\rmfamily\texorpdfstring{\mutabor\\[\baselineskip]}{MUTABOR --}
 \LARGE\slshape Ein computergesteuertes Musikinstrument \\
	  zum Experimentieren mit\\
	  Stimmungslogiken und Mikrotönen}
\author{Volker Abel, Peter Reiss,\\ Rüdiger Krauße und Tobias Schlemmer}
\date{Programmversion $3.0x$ (\the\year)}
\ifhtml\else
  \publishers{\includegraphics[width=0.5\linewidth]{start}}
\fi
\lowertitleback{\footnotesize\copyright 1991, 1992 Volker Abel \& Peter Reiss\\
\copyright 2006 TU Dresden, Institut für Algebra}
%
% Index-Datei öffnen
%
\makeindex
\hyphenation{wei-te-re Ton-sys-tem Ton-sys-te-me}
%\parindent 0mm
%\parskip 5pt
%\textheight 18.5cm
%
% Jetzt wirds ein wenig wüst.
%
% Wir wollen mit wenig Aufwand die Labels auch für Hyperlinks
% verwenden. Damit sparen wir uns zusätzliche Anker.
% 
\makeatletter
%
% \XR@ext enthält die Erweiterung für Querverweise. Wenn wir auf
% Buchanfänge usw. verweisen wollen, kann uns xr-hyper nicht
% undbedingt helfen. (oder wir müsste das entsprechend definieren).
% Einfacher ist es wohl, wir nehmen den Dateinamen.
%
\newcommand\makefilename[1]{#1.\XR@ext}
%
%
% erstes von sechs Argumenten wiedergeben und Rest verwerfen
\def\ts@firstofsix#1#2#3#4#5#6{#1}
%
\ifhtml
% voreinstellungen, um „:“ als Buchstaben zu behandeln (für tex4ht)
% 
% Herausfiltern des Verlinkungsmakros aus Querverweis-Speichern
%
\def\ts@parse@ref@a#1#2\ts@end@parse{\ts@parse@ref@b#1\ts@end@parse}
\def\ts@parse@ref@b#1#2#3\ts@end@parse{#1{#2}}
%
% Setzen des Verweises.
% Das erste Argument mus zunächst expandiert werden, bevor es
% überhaupt mit den obigen makros geparst werden kann. Wir sichern uns
% das ganze in einem temporären Makro zusammen mit dem Ankertext. 
%
\def\ts@setref#1#2#3{%
  \expandafter\expandafter
  \expandafter\def
  \expandafter\expandafter
  \expandafter\ts@@tmp
  \expandafter\expandafter
  \expandafter{%
    \expandafter\ts@parse@ref@a#1\ts@end@parse{#2}}%
%
% Wir hacken uns in \@setref hineein.
% dort steht: \expandafter #2#1. Wir liefern aber alles, was wir
% brauchen schon in #2 mit, verwerfen also alles aus #1.
% wir machen daraus \expandafter\ts@firstofsix\expandafter\ts@@tmp#1
% damit wird zunächst #1 expandiert und dann verworfen.
% Genial nicht ;-)?
%
% Alternativ könnte man auch mit \@firstoftwo arbeiten. Hier muss man
% aufpassen, dass das erste Argument nicht falsch expandiert wird.
% \setref#1{\@firstoftwo{\ts@@tmp}}{#3}
%
  \@setref#1{% 
    \ts@firstofsix%
      \expandafter\ts@@tmp}{#3}%
}
%
% Jetzt können wir das eigentliche Linkmakro definieren. 
%
% Wenn die Referenz nicht definiert ist, setzen wir den Ankertext und
% rufen setref auf, damit die entsprechende Warnung ausgespuckt
% wird. Die ausgegebenen Fragezeichen setzen wir in weißer Farbe in
% eine 0pt breite Box (llap). Damit kommt es nur zu minimalen
% Verschiebungen (Kerning) zwischen undefiniert und definiert. 
%
% Ist die Referenz definiert, wird sie verwendet, um einen Link zu erzeugen.
\newcommand\ts@reflink[2]{%
  \@ifundefined{r@#1}{%
    \textcolor{red}{%
      #2%
      \color{white}{%
        \llap{%
          \@safe@activestrue
          \edef \RefArg {#1}
          \expandafter\ts@setref\csname r@#1\endcsname{{\@safe@activesfalse #2}}{#1}%
          \@safe@activesfalse
        }%
      }%
    }%
  }{%
    \@safe@activestrue
%    \let\::ref \T:ref
    \expandafter\ts@setref\csname r@#1\endcsname{{\@safe@activesfalse #2}}{#1}%
%    \def\::ref{\protect\T@ref}%
    \@safe@activesfalse
  }%
}
% : wiederherstellen
%\ts@savecatcode
\else
% Hier läuft es eigentlich genauso ab, wie bei der
% tex4ht-Variante. Nur werden jetzt die Verweise etwas anders kodiert,
% so dass man sie nicht wirklich neu parsen muss.
\newcommand\ts@reflink[2]{%
\begingroup%\tracingall
  \def\ts@tmp{{\@safe@activesfalse #2}}%
  \@ifundefined{r@#1}{%
    \textcolor{red}{%
      #2%
      \color{white}{%
        \llap{%
          \@safe@activestrue
          \expandafter\@setref\csname r@#1\endcsname{\ts@firstofsix\ts@tmp}{#1}%
          \@safe@activesfalse
        }%
      }%
    }%
  }{%
    \@safe@activestrue
    \expandafter\@setref\csname r@#1\endcsname{\ts@firstofsix\ts@tmp}{#1}%
    \@safe@activesfalse
  }%
%\show\ts@reflink
\endgroup
}
\fi
%\newcommand\ts@@reflink{\protect\ts@reflink}
%
% die anderen definierten Referenz-Makros sind auch \protect-et
% definiert. Also machen wir das auch mit einem Querverweis auf ein
% Label mit einem beliebigen Ankertext.
%
\newcommand\reflink{\protect\ts@reflink}
%
% Referenz durch Titel, ggf. Zusatz (wie z.\,B. Buchname bei externen
% Referenzen) und Seite bei nicht-HTML-Ausgabe. Der Zusatz wird in []
% angegeben. Das erledigen wir durch zwei Makros:
%
\newcommand\tsciteref[1]{%
  \frqq\nameref{#1}\flqq
  \@ifnextchar[{\ts@cite@ref{#1}}{\ts@cite@@ref{#1}}
}
% mit optionalem Argument: Leerzeichen+Zwischentext und dann
% Seitenangabe ohne zweites Arg.
\def\ts@cite@ref#1[#2]{ #2\ts@cite@@ref{#1}}
% Ohne optionales Argument: Seitenangabe
\def\ts@cite@@ref#1{%
  \ifhtml\else{}
  auf Seite \pageref{#1}%
  \fi
}
% So ziemlich alles, was man referenzieren kann:
% Verweis-Typ: Kapitel, Abschnitt usw. 
% Nummer, Titel, ggf. optionales Argument, ggf. Seite.
% Syntax \vollref{Label}[zwischentext] (letzteres wird durch
% \tsciteref eingeführt
\newcommand\vollref[1]{\autoref{#1} \tsciteref{#1}}
%
% Und hier sind wir aus den Interna heraus.
%
\makeatother
%
% Ein Makro zur vereinfachten Umwandlung der Mutabor-Hilfe mit Querverweisen.
%
\newcommand\tsreflink[2]{\reflink{sec:#2}{#1}}
%
% Der Name unseres Programmes
%
\newcommand\mutabor[1][]{\textsc{Mutabor#1}}
%
% Vorbereitung zur Verwendung des Pakets keystroke. Damit werden
% Tasten als Piktogramme angezeigt.
%
\usepackage{keystroke}%
%\providecommand\keystroke[1]{#1}
%
% Schlüsselworte und sonstiger Quelltext in Schreibmaschinenschrift.
%
\newcommand\keyword[1]{\texttt{#1}}
%
% Wir wollen die Handbücher ja nicht nur in die CD integrieren.
%
\ifhtml
  \newcommand\cdoronline[2]{#1}
\else
  \newcommand\cdoronline[2]{#2}
\fi

\newcommand\sourcecode{\texttt}%
\newcommand\filename{\texttt}

\newcommand\mutimage[2]{%
  \ifhtml
  \Picture{#1}%
  \else
  \includegraphics#2
  \fi
}

\endinput


\subject{Referenzhandbuch}
\externaldocument{handbuch}
\externaldocument{bedienung}
\begin{document}
\ifhtml
\setcounter{tocdepth}{1}
\fi

\maketitle
%\ifhtml\else\clearpage\fi
\ifhtml\ifhelp\else
\section*{Verfügbare Formate dieses Handbuchs}
Diese Dokumentation können sie in folgenden Formaten lesen und ausdrucken:

\HCode{<?php require_once(\dq \cdoronline{}\urlbase includes/hilfsfkt.inc\dq);?>\Hnewline
  <div class=\dq dokumentation download tex4ht-vorspann\dq>\Hnewline
<table class=\dq beitraege\dq>\Hnewline
  <tbody class=\dq beitraege\dq >\Hnewline
  <tr class=\dq beitraege\dq >	\Hnewline
    <th
      class=\dq beitraege\dq ><a href=\dq
      \jobname.html\dq class=\dq link\dq>\jobname.html</a></th>\Hnewline
   <td class=\dq beitraege\dq >Das Dokument, das Sie gerade lesen. </td>\Hnewline
  </tr>\Hnewline
  <tr class=\dq beitraege\dq >\Hnewline
    <th
      class=\dq beitraege\dq ><?=dateilink(\dq \jobname.ps\cdoronline{}{.bz2}\dq )?></th>\Hnewline
    <td class=\dq beitraege\dq >PostScript zum Ausdrucken</td>\Hnewline
  </tr>\Hnewline
  <tr class=\dq beitraege\dq >\Hnewline
    <th class=\dq beitraege\dq ><?=dateilink(\dq \jobname.pdf\dq )?></th>\Hnewline
    <td class=\dq beitraege\dq >PDF zum Ausdrucken</td>\Hnewline
  </tr>\Hnewline
  </tbody>\Hnewline
</table>\Hnewline
</div>}
\Css{div.tex4ht-vorspann { 
    margin-bottom: 2em; 
    width:30em;
    width:auto;
    max-width:30em;
}}
\fi
\else
\tableofcontents
\fi

\iffalse

% Nummerierung der Kapitel
% damit es linear weitergeht (Nach dem Anhang...)
%
%\setcounter{chapter}{\value{savedchapter}}%
%                           % Muss im Bedienungshandbuch 
%                           % gesetzt werden (vor \appendix)
%\setcounter{savedchapter}{\value{page}} % muss nach \mainmatter
%                                % wiederhergestellt werden.
%\setcounter{section}{0}%
%\def\thechapter{\arabic{chapter}}% 
%\global\let\chapter\savedchaptercommand
%\mainmatter
%\setcounter{page}{\value{savedchapter}}
%\makeatletter
%\xdef\@chapapp{\chaptername}%
%\ifx\theHchapter\undefined
%\else
%\renewcommand\theHchapter    {\arabic{chapter}}
%\renewcommand\theHfigure     {\theHchapter.\arabic{figure}}
%\renewcommand\theHtable      {\theHchapter.\arabic{table}}
%\renewcommand\theHsection    {\theHchapter.\arabic{section}}
%\def\Hy@chapapp{\Hy@chapterstring}
%\fi
%\makeatother
\fi


\hyphenation{Har-mo-nie-na-me
	     Har-mo-nie-na-men
	     kon-sis-tent kon-sis-ten-te kon-sis-ten-ter}

\newcommand{\bi}{\begin{itemize}}
\newcommand{\ei}{\end{itemize}}


\pagestyle{headings}

\chapter{Sprachreferenz\label{cha:sprachreferenz}}
\label{eins}

Das Referenzhandbuch soll einerseits ein kurzes
Nachschlagewerk sein, andererseits stellt es die Basisbeschreibung
der Logiksprache \mutabor{} dar. Es ist nicht dafür geeignet, die
Logiksprache zu erlernen, dafür sei an dieser Stelle auf das
Bedienungshandbuch verwiesen, sondern gibt bei zweifelhaften
Fragen exakte Auskünfte. Das Referenzhandbuch ist dafür gedacht,
die Funktionalität der Logiksprache festzulegen. Wenn ein Benutzer
von \mutabor{} feststellt, daß eine von ihm programmierte Logik
nicht so funktioniert, wie er es wünscht, so ist dieses
Referenzhandbuch die Entscheidungsinstanz, anhand derer man
feststellen kann, ob das Logikprogramm falsch ist, oder ob
\mutabor{} nicht korrekt funktioniert.
Dieser Abschnitt beschreibt die Logiksprache \mutabor{} in der Version 3.0.


\chapter{Lexikalische Konventionen}\label{cha:lexik-konv}
\label{zwei}

Ein Programm besteht aus einem Text, der in einer Datei gespeichert ist.
Es wird in einer einzigen Phase übersetzt.

\section{Eingabesymbole}\label{sec:eingabesymbole}
\index{Eingabesymbole}
\label{zweieins}

Es  gibt fünf Klassen\index{Wortklassen} von Eingabesymbolen:
Namen\index{Namen},  reservierte  Worte\index{Worte!reservierte},
Zahlen,
 Operatoren  und  andere
Trenner.  Leerzeichen,\index{Leerzeichen} Tabulatorzeichen,
Vertikal-Tabulatoren,
Zeilentrenner,\index{Zeilentrenner} Seitenvorschub\index{Seitenvorschub}
sowie Kommentare\index{Kommentare},
allgemein mit Zwischenraum  bezeichnet,  werden
ignoriert,  abgesehen davon,  daß sie zur Trennung direkt
benachbarter Namen, Zahlen oder reservierter Worte nötig sind.

Wenn  die Eingabe bis zu einem bestimmten Zeichen schon in Eingabesymbole
zerlegt  wurde,  wird als nächstes
Eingabesymbol die längstmögliche  Kette
von Zeichen aufgefasst, die ein Eingabesymbol darstellen kann.


\section{Kommentare}\label{sec:kommentare}
\index{Kommentare}
\label{zweizwei}

Kommentare\index{Kommentar} werden in doppelte Anführungszeichen (\texttt{\dq})
\index{\dq (Anführungszeichen)}eingeschlossen.
Kommentare  können  nicht
verschachtelt\index{Verschachtelung} werden.
Kommentare dürfen sich über mehrere Zeilen erstrecken.

\section{Namen}\label{sec:namen}
\label{zweidrei}

Ein Name\index{Namen} besteht aus einer Folge von
Buchstaben und Ziffern;  das
erste Zeichen muss ein Buchstabe sein.  Dabei zählen
der Unterstrich  (\_)\index{\_ (Unterstrich)} und der Apostroph
(\textbf{'})\index{' (Apostroph)}
zu den Buchstaben.\footnote{Der Unterstrich zählt zu den Buchstaben, damit
man aussagekräftige Namen gut lesbar schreiben kann, z.\,B.
\texttt{LOGIK Drittel\_Ton\_Mutierend} ist besser als
\texttt{LOGIK dritteltonmutierend}
und der Apostroph zählt als Buchstabe,
damit man Töne mit einfach oder mehrfach gestrichenen
Namen versehen kann, z.\,B. \texttt{TON e${}'''$ }}
Große und kleine Buchstaben werden gleich behandelt.
Namen können beliebig lang sein.
Zur  Unterscheidung von Namen wird der gesamte Name herangezogen,
und nicht nur die ersten paar Buchstaben, wie dies bei anderen
Programmiersprachen üblich ist\index{Signifikanz}.

In den Büchern von \mutabor{} werden die Begriffe \emph{Name}\index{Namen}
und \emph{Bezeichner}\index{Bezeichner}
parallel und mit gleicher Bedeutung verwendet. Ferner werden bei einigen
Syntaxgraphen auch Begriffe der Art \emph{Intervallname}, \emph{Tonname},\ldots
verwendet. Diese Begriffe sind syntaktisch gesehen einfache Namen.
Es soll damit zum Ausdruck kommen, daß Namen einer bestimmten Art von
Objekten gemeint sind. Dies gehört prinzipiell nicht zur Syntax der
Logiksprache, sondern wird erst später semantisch überprüft. Es erhöht
jedoch das Verständnis, wenn direkt erkennbar ist, welche Art von Namen
in Frage kommt.

\section{Reservierte Worte}\label{sec:reservierte-worte}
\label{zweivier}

Die  folgenden  Worte sind reserviert\index{Worte!reservierte}
 und können  nur  mit  ihrer
vordefinierten Bedeutung verwendet werden:

%\vspace{1cm}
\begin{center}
\begin{tabular}{|llll|}
	      \hline
   ANSONSTEN & FORM   & HARMONIE & INTERVALL \\
   LOGIK  & MIDIIN   & MIDIKANAL & MIDIOUT    \\
   TASTE     & TON    & TONSYSTEM &   UMSTIMMUNG \\
   WURZEL    &        &           &              \\
 \hline
\end{tabular}
\end{center}
%\vspace{1cm}

Ebenso dürfen die folgenden englischen Schlüsselworte
benutzt werden. Sie sind ebenfalls reserviert und
dürfen nicht als Namen benutzt werden.

\begin{center}
\begin{tabular}{|llll|}
	      \hline
   ELSE & SHIFTED   & PATTERN & INTERVAL \\
   LOGIC  & MIDIIN   & MIDICHANNEL & MIDIOUT    \\
   KEY     & TONE    & TONESYSTEM &  RETUNING \\
   ROOT    &        &           &              \\
 \hline
\end{tabular}
\end{center}

Alle anderen Wörter dürfen als Namen oder Bezeichner verwendet werden.

\section{Zahlen}\label{sec:zahlen}
\index{Zahlen}
\label{zweifuenf}

Zahlen können ganzzahlig oder mit Dezimalpunkt geschrieben werden.
Eine Schreibweise mit Zehnerexponent ist nicht
zugelassen.\index{Zahlen} Ferner kann eine Zahl auch mit dem Zeichen
\textbf{\#}\index{\#} beginnen und weitere Hex-Ziffern\index{Hex-Ziffern}
enthalten. Diese Schreibweise bedeutet, daß die Zahl zur Basis 16
ausgewertet wird. In dieser Hex-Schreibweise ist kein Dezimalpunkt
erlaubt, folglich sind nur ganzzahlige Werte darstellbar.

Ähnlich wie bei Namen werden Zahlen im Syntaxdiagramm auch als \emph{Faktor}
oder \emph{Verankerungs"=Taste} bezeichnet,
um ihre jeweilige semantische Bedeutung
hervorzuheben. Es handelt sich aber syntaktisch gesehen um ganz gewöhnliche
Zahlen.

An einigen Stellen sind nur ganze Zahlen zugelassen (z.\,B. als
Verankerungs-Taste, Tonsystem-Breite,\ldots) und an anderen Stellen
sind auch Komma"=Zahlen (mit Punkt statt Komma) zugelassen (Tonfrequenz,
In\-ter\-vall\-gr"o"se,\ldots).



\chapter{Syntax"=Schreibweise}\label{cha:syntax-schreibweise}
\label{drei}

In der Syntax-Darstellung in diesem Buch werden die Grammatikbegriffe
\emph{in eckigen Kästchen} und reservierte Worte und Zeichen \emph{in
ovalen Kästchen} geschrieben. Die Syntax erhält man, indem man auf den
Pfeilen entlang durch das Syntaxdiagramm läuft.
Alternative Formulierungen erscheinen dabei als Verzweigungen
und Wiederholungen ergeben sich, indem das Syntaxdiagramm Zyklen enthält.
Ein Beispiel:

\input{syn1.pic}

Das Wort \emph{TON} muss genau so geschrieben werden, während das Konstrukt
\emph{EinzelTon} noch genauer aufgelöst werden muss. Ferner ergibt sich durch
die Schleife, daß mehrere Konstrukte von der Art \emph{EinzelTon}
aufeinander folgen können.


\chapter{Das Logikprogramm}\label{cha:das-logikprogramm}
Das Logikprogramm ist der oberste Grammatikbegriff und entspricht
dem gesamten Programm. Das Logikprogramm selbst ist eine beliebige
Folge von Deklarationen\index{Deklarationen}.
Insbesondere ist hervorzuheben, daß auch
gar nichts, also ein leeres File, ein zulässiges Programm ist.
In diesem Fall ist keine Logik vorhanden und die normale
gleichstufig temperierte Stimmung eingestellt.

Es können Intervalle, Töne, Tonsysteme, Umstimmungen, Harmonien,
Logiken und MIDI-Kanäle deklariert werden. Die einzelnen Elemente können
in beliebiger Reihenfolge deklariert werden. Insbesondere kann auf
Elemente Bezug genommen werden, die erst später deklariert werden.

\input{syn2.pic}

\chapter{Intervall}\index{Intervall}\label{cha:intervall}

Eine Intervalldeklaration definiert einzelne Intervalle und die
dazugehörenden Werte. Intervallnamen dürfen nicht doppelt definiert
werden, sie dürfen jedoch gleiche Namen tragen,
wie Objekte einer anderen Art.
Der Wert eines Intervalls ist ein Frequenzverhältnis.
Der Wert eines Intervalls kann auf eine von drei Arten angegeben werden:
\bi
  \item als Zahlenverhältnis,
  \item als Wurzelausdruck\index{Wurzel} oder
  \item als Kombination aus anderen Intervallen.
\ei
Es bedeutet dabei
\texttt{Zahl1 : Zahl2} das Verhältnis von Zahl1 zu Zahl2 und
\texttt{A WURZEL B} ist die \texttt{A}te Wurzel aus \texttt{B}.
Wenn Intervalle aus anderen Intervallen zusammengesetzt werden,
so dürfen keine gegenseitigen Abhängigkeiten entstehen, sondern
die Werte der Intervalle müssen in konsistenter Form berechenbar sein.
Alle Intervalle die zum Zusammensetzen anderer Intervalle benutzt werden,
müssen innerhalb des Logik-Programms deklariert werden.
Optional kann einem Intervall auch ein Faktor\index{Faktor}
 vorangestellt werden, dann gilt der Wert dieses Intervalls
entsprechend mehrfach. Da es sich bei Intervallen eigentlich um
Frequenzverhältnisse handelt, ergibt sich der Wert eines zusammengesetzten
Intervalles multiplikativ. Zum Beispiel wird das Intervall
\begin{lstlisting} 
INTERVALL I = 3 oktave - 5 terz + 17 cent 
\end{lstlisting}
 nach folgendem Schema
berechnet: \[I = \mathtt{oktave}^{3} * \mathtt{terz}^{-5} * \mathtt{cent}^{17}\]

\input{syn3.pic}

\input{syn5.pic}

\chapter{Ton}\label{cha:ton}
\index{Ton}

Eine Tondeklaration definiert einzelne Töne und die
dazugehörenden Werte. Tonnamen dürfen nicht doppelt definiert
werden, sie dürfen jedoch gleiche Namen tragen,
wie Objekte einer anderen Art.
Der Wert eines Tons ist eine Zahl, die der Frequenz des Tons
entspricht. Der Wert eines Tons kann eine einfache Zahl sein oder
sich aus einem Bezugston\index{Bezugston}
 ergeben, der möglicherweise
um bestimmte Intervalle verschoben wird.
Im ersteren Fall wird die Frequenz\index{Frequenz}
 des Tons direkt angegeben.
Im anderen Fall folgt nach dem Gleichheitszeichen der Name eines
Bezugstons, sowie eine Folge von Intervallen, mit denen die Frequenz des
Bezugstons multipliziert wird, bis sie den gewünschten Ton ergibt.
Grundsätzlich müssen alle Töne, die als Bezugston herangezogen werden,
im Logikprogramm definiert sein. Ferner darf die Definition aller Töne
keine gegenseitigen Abhängigkeiten enthalten.

Sämtliche Töne ergeben nach deren Berechnung immer eine feste Frequenz.
Ein Tonname steht also als Synonym für eine
absolute Frequenz\index{Frequenz!absolute}, deren
Wert sich multiplikativ ergibt. Beispiel:
\begin{lstlisting} 
TON  b = c - 2 quint + oktave 
\end{lstlisting}
wird berechnet als \[Frequenz(\mathtt{b}) = Frequenz(\mathtt{c}) * \mathtt{quint}^{-2} * \mathtt{oktave}\].

\input{syn4.pic}



\chapter{Tonsystem}\label{cha:tonsystem}
\index{Tonsystem}
Eine Tonsystemdeklaration definiert einzelne Tonsysteme und die
dazugehörenden Werte. Tonsystemnamen dürfen nicht doppelt definiert
werden, sie dürfen jedoch gleiche Namen tragen,
wie Objekte einer anderen Art.
Ein Tonsystem besteht aus einer Verankerungstaste\index{Verankerungstaste},
einer Menge von Tönen, der Fundamentaltonleiter, 
und einem Periodenintervall\index{Periodenintervall}.
Das Periodenintervall kann auch aus mehreren Intervallen
kombiniert werden.
Die Fundamentaltonleiter wiederum
besteht aus einer Folge von Tönen, die in eckigen Klammern
``\textbf{[ ]}''\index{[ ] (eckige Klammern)}
 stehen und durch Kommata getrennt sind.
Sie entspricht einem Bereich auf der Klaviatur. Die Zuordnung der Töne
zu den Tasten ergibt sich durch die Verankerungstaste.
Sie entspricht der MIDI-Nummer\index{MIDI!Nummer} des ersten Tones der
Fundamentaltonleiter.\footnote{Das eingestrichene $c'$ entspricht
der MIDI-Tastennummer 60.}
Von dieser Position aus werden nach rechts folgend die weiteren Töne
den Klaviaturtasten zugeordnet. Die nächste Taste rechts daneben entspricht
folglich dem zweiten Ton der Fundamentaltonleiter, etc\ldots\ .
Wenn nach diesem Schema
alle Töne der Fundamentaltonleiter zugeordnet sind, wird die gesamte
Fundamentaltonleiter sowohl nach unten, wie nach oben wiederholt und
dabei um das Periodenintervall\label{Periodenintervall}
verschoben, bis die Grenzen der Klaviatur erreicht sind.

Bei der Deklaration der Fundamentaltonleiter\index{Fundamentaltonleiter}
dürfen einzelne Töne fehlen, nicht jedoch die trennenden Kommata.
Diese fehlenden Töne bewirken,
daß die entsprechende Taste keinen Ton erzeugt.
Fehlende Töne werden bei der Zuordnung
$Klaviaturtaste \rightarrow Ton$ normal
mitgezählt, als wären sie vorhanden.



\input{syn6.pic}


\chapter{Umstimmung}\label{cha:umstimmung}
\index{Umstimmung}
Eine Umstimmungsdeklaration definiert einzelne Umstimmungen und die
dazugehörenden Werte. Umstimmungsnamen dürfen nicht doppelt definiert
werden, sie dürfen jedoch gleiche Namen tragen,
wie Objekte einer anderen Art.
Eine Umstimmung enthält Regeln, nach denen die aktuelle Stimmung
\index{Stimmung!aktuelle} zu
verändern ist. Der Wert der Änderung kann sowohl absolut als auch
relativ zum alten Zustand angegeben werden. Bei einer absoluten Änderung
wird eine Zahl oder ein symbolischer Parameter angegeben, der den
neuen Wert darstellt, ohne Rücksicht auf den vorherigen Zustand.
Bei einer relativen Änderung wird der bisherige Wert als Bezugswert
genommen und um einen gewissen Betrag verändert. Semantisch bedeutet das
Symbol \textbf{@}\index{*A@\textat (Klammeraffe)} „\emph{der alte Wert}“,
auf den Bezug genommen werden soll.

Falls die Umstimmung von einem oder mehreren Parametern\index{Parameter}
 abhängig sein soll,
so ist für jeden Parameter\index{Parameter!symbolischer Name}
 ein symbolischer Parameter-Name anzugeben.
Ein Parameter-Name kann später im Umstimmungs-Ausdruck benutzt werden.

\input{syn8.pic}


Es gibt sieben Arten von Umstimmungs-Ausdrücken, die jeweils einzelne
Merkmale der aktuellen Stimmung ver\-än\-dern:

\input{syn7.pic}

\begin{description}
  \item[Umstimmung – Verankerung]\index{Umstimmung!Verankerung}
      entspricht einer Verschiebung
      der Fundamentaltonleiter. Dabei wird angegeben, wo sich die
      neue Verankerungs-Taste befinden soll. Diese neue Verankerungstaste
      behält ihre Frequenz bei und alle anderen Töne ergeben sich,
      indem relativ zur neuen Verankerungstaste die gesamte
      bisherige Intervallstruktur\index{Intervallstruktur}
      übernommen wird.

      Im allgemeinen folgt aus einer Veränderung der Verankerungstaste auch
      eine Veränderung der Töne.

      \input{syn9.pic}

      \input{syn26.pic}

  \item[Umstimmung – Breite]\index{Umstimmung!Breite}
      entspricht einer Veränderung der Breite
      der Fundamentaltonleiter.\index{Fundamentaltonleiter!Breite der}
      Der erste Ton der Fundamentaltonleiter bleibt erhalten und alle
      Töne rechts davon werden im aktuellen Zustand festgehalten.
      Dann wird abgezählt, wieviele der festgehaltenen Töne benötigt
      werden, um die neue Breite der Fundamentaltonleiter zu erhalten.
      Diese Töne bilden dann die neue Fundamentaltonleiter.
      Das Intervall von der Verankerungstaste zur ersten Taste nach der
      Fundamentaltonleiter ergibt das neue Periodenintervall.

      Im allgemeinen folgt aus der Veränderung der Breite auch eine Veränderung
      des Periodenintervalls. Eine Verkürzung der Breite, die zu einer
      Breite kleiner 1 führt, wird ignoriert. Eine Verbreiterung, die
      zu einer Breite von mehr als 60 Tasten führt, wird ignoriert.

      \input{syn10.pic}


  \item[Umstimmung – Töne]\index{Umstimmung!Töne}
      entspricht einer Änderung einzelner Töne
      innerhalb der Fundamentaltonleiter.
      Hier wird für jeden Ton der Fundamentaltonleiter ein Rechenausdruck
      angegeben, um den sich der Ton zu verändern hat. Diese Rechenausdrücke
      sind durch Kommata getrennt und beziehen sich in ihrer Reihenfolge
      auf die entsprechenden Töne, wie sie bei einem Tonsystem
      angegeben werden.
      Soll ein Ton unverändert
      erhalten bleiben, so ist dafür das Symbol \texttt{@}\index{*A@\textat (Klammeraffe)}
      zu schreiben, weil
      `\emph{nichts}' bedeutet, daß der Ton stumm geschaltet wird.
      Falls die Anzahl der Rechenausdrücke kleiner ist, als die
      Breite der Fundamentaltonleiter, so bleiben die überzähligen Töne
      unverändert. Falls die Breite der Fundamentaltonleiter kleiner ist,
      als die Anzahl der Rechenausdrücke, so bleiben die
      überzähligen Rechenausdrücke ohne Effekt.

      \input{syn11.pic}

      \input{syn12.pic}

  \item[Umstimmung – Periode]\index{Umstimmung!Periode}
      entspricht einer Änderung des
      Periodenintervalls\index{Periodenintervall}.
      Hier wird der Wert des
      Periodenintervalls verändert, so daß die Töne der Fundamentaltonleiter
      erhalten bleiben, aber alle anderen Töne neu berechnet werden,
      da sie sich aus der Fundamentaltonleiter und der Periodenverschiebung
      ergeben.

      \input{syn13.pic}



  \item[Umstimmungs"=Bund]\index{Umstimmung!Bund}
      Ein Umstimmungsbund umfasst eine Folge von
      Aktionen.\index{Aktionen} Die einzelnen
      Aktionen können andere Umstimmungen,
      Tonsysteme oder Logiken aufrufen, oder auch MIDI-Nachrichten
      senden. Dabei dürfen keine gegenseitigen
      Abhängigkeiten entstehen. Die einzelnen Aktionen werden in der
      aufgeführten Reihenfolge ausgeführt.
      Die Parameter\index{Parameter} eines
      Umstimmungsbundes können an die einzelnen Aufrufe weitergegeben
      werden, sofern der aufgerufene Name eine Umstimmung ist und
      diese Umstimmung auch entsprechend viele Parameter akzeptiert.
      Die mitgegebenen Parameter werden in der gleichen Reihenfolge
      den Parameter-Namen der Umstimmung zugeordnet.


      \input{syn14.pic}

      \input{syn28.pic}

      \input{syn23.pic}

      \input{syn17.pic}


    \item[Umstimmungs-Alternative]\index{Umstimmung!Alternative} Eine
      Umstimmungs-Alternative ermöglicht die Auswahl einer von
      mehreren Alternativen in Abhängigkeit eines
      Parameters\index{Parameter} der Umstimmung (der
      Selektor\index{Selektor}).  Die Alternativen bestehen jeweils
      aus der Konstanten, die einen der möglichen Fälle bestimmt, und
      einer Folge von Aktionen, welche durchgeführt werden, falls die
      Auswahl für diese Alternative zutrifft. Die einzelnen Aktionen
      einer Alternative sind mit Komma (\texttt{,}) getrennt, während
      komplette Alternativen nicht durch ein besonderes Trennzeichen
      getrennt sind. Die Konstanten der einzelnen Alternativen müssen
      ganzzahlig und eindeutig sein. Am Ende der Alternativen darf
      einmalig die Spezialkonstante \texttt{ansonsten}\index{Ansonsten}
      stehen. Falls der Selektor mit einer der Konstanten
      übereinstimmt, wird diese Alternative durchgeführt. Falls der
      Selektor mit keiner Konstanten übereinstimmt und die
      Spezialkonstante \texttt{ansonsten} vorhanden ist, wird die
      Alternative des \texttt{ansonsten}-Zweigs durchgeführt. Falls der
      Selektor mit keiner Konstanten übereinstimmt und die
      Spezialkonstante \texttt{ansonsten} nicht vorhanden ist, wird keine
      der Alternativen durchgeführt, der Aufruf dieser Umstimmung
      bleibt in diesem Fall folgenlos.

      \input{syn16.pic}

      \item[Aktion MIDI"=Ausgabe]\index{Aktion!MIDI"=Ausgabe}
	MIDI-Ausgaben werden mit dem Wort \emph{MIDIOUT}\index{MIDIOUT}
	 eingeleitet und
sind eine in runde Klammern eingeklammerte
und durch Kommata getrennte Folge von Zahlen. Bei Zahlen ist die sedezimale
Schreibweise zulässig (\#\emph{ziffern}).\index{\#} Die Zahlen
müssen im Bereich von 0 bis 255 liegen, also Bytes sein. Es erfolgt keine
Kontrolle, ob es sich um eine sinnvolle MIDI-Nachricht handelt.

      \input{syn22.pic}

\end{description}

\chapter{Harmonie}\label{cha:harmonie}
\index{Harmonie}

Eine Harmoniedeklaration definiert einzelne Harmonien und die
dazugehörenden Werte. Harmonienamen dürfen nicht doppelt definiert
werden, sie dürfen jedoch gleiche Namen tragen,
wie Objekte einer anderen Art.
Der Wert einer Harmonie ist eine Folge von Tasten-Nummern,\index{Tasten!Nummer}
welche einzelne Tasten innerhalb der Fundamentaltonleiter identifizieren.
Die Tasten der Klaviatur, welche außerhalb der Fundamentaltonleiter
stehen, werden in die Fundamentaltonleiter hineinprojiziert und
bilden die Projektionstonleiter,\index{Projektionstonleiter}
welche die gleiche Breite
hat, wie die Fundamentaltonleiter, aber stattdessen die Information
enthält, welche (projizierten) Tasten (Tonigkeiten\index{Tonigkeit})
gerade gedrückt sind.
Eine Harmonie wiederum wird erkannt, wenn in der Projektionstonleiter genau
die angegebenen Tastennummern gedrückt sind. Dabei können einzelne
Tasten unberücksichtigt bleiben. Die entsprechenden Tastennummern sind
durch einen Stern (*)\index{*} gekennzeichnet.
Optional kann eine Bezugstaste\index{Bezugstaste}\index{Tasten!Bezugs-}
mit angegeben werden.
Ihr Wert ändert nicht das Harmonie-Erkennungsmuster, sondern gibt
bei der Harmonieform-Analyse eine relative Verschiebung
der Bezugstonigkeit\index{Bezugstonigkeit}
 in dem Wert \texttt{ABSTAND}\index{ABSTAND} an.
Falls keine Bezugstaste angegeben ist, so wird ihr Wert als $0$ angenommen,
also ohne Verschiebung der Bezugstonigkeit.

\input{syn15.pic}



\chapter{Logik}\label{cha:logik}
\index{Logik}
Eine Logikdeklaration definiert einzelne Logiken und die
dazugehörenden Werte. Logiknamen dürfen nicht doppelt definiert
werden, sie dürfen jedoch gleiche Namen tragen,
wie Objekte einer anderen Art.
Logiken werden durch einen Auslöser\index{Auslöser} aktiviert und
bestehen aus einer Einstimmung\index{Einstimmung}
 und einer Menge von Anweisungen\index{Anweisung!Menge von}.
Die Einstimmung darf ein Tonsystem oder eine Umstimmung sein.
Die Einstimmung darf fehlen, in diesem Fall wird die vorherige Stimmung
beibehalten. Die Menge der Anweisungen darf ebenfalls leer sein, in diesem
Fall bleibt die Logik in der aktuellen Stimmung stehen.
Eine Anweisung wiederum besteht ebenfalls aus einem Auslöser, sowie einer
Liste von Aktionen.
Die Aktionen dürfen der Aufruf einer Umstimmung, eines Tonsystems
oder einer Logik sein, oder das Senden einer MIDI-Nachricht.

Grundsätzlich existieren zwei Mengen von Auslösern: Die Auslöser, welche
eine der Logiken aktivieren, und die Auslöser, die innerhalb der
gerade aktiven Logik die Anweisungen anwählen. Erstere nennen wir
\emph{globale Auslöser}\index{Auslöser!globale}
 und letztere \emph{lokale Auslöser}\index{Auslöser!lokale}. Beim Eintreffen
eines Ereignisses wird zuerst die Menge der globalen Auslöser
und dann die Menge der lokalen Auslöser überprüft, ob das Ereignis
den Auslöser auslöst. Innerhalb der globalen und lokalen Auslöser wird
in der Reihenfolge ihrer Deklaration gesucht und beim ersten zutreffenden
Auslöser die Suche beendet.
Wenn ein globaler Auslöser zutrifft, so wird in die entsprechende Logik
gewechselt. Wenn ein lokaler Auslöser zutrifft, so werden die entsprechenden
Aktionen innerhalb der aktuellen Logik aufgerufen.
Innerhalb der Menge der globalen Auslöser ist der Spezialauslöser
\texttt{ANSONSTEN}\index{ANSONSTEN} nicht erlaubt.

\input{syn24.pic}

\input{syn27.pic}

\section{Auslöser}\label{sec:ausloser}
\index{Auslöser}

Auslöser können wahlweise eine der 26 Buchstabentasten, eine Harmonie,
eine Harmonieform,
ein MIDI-Ereignis oder der Spezialauslöser \texttt{ansonsten} sein.

\input{syn18.pic}

\subsection{Tasten}\index{Tasten}\label{sec:tasten}

Dieser Auslöser\index{Auslöser!Taste}
 besteht aus dem Wort \emph{Taste}\index{Tasten}, gefolgt von einem der
26 Buchstaben. Wenn innerhalb von \mutabor{} die angegebene Taste auf der
Tastatur gedrückt wird, so gilt dieser Auslöser als erfüllt.
Der Auslöser bezieht sich jeweils auf die \mutabor"=Box, das Instrument bzw.\ MIDI-Kanal,
das zuletzt aktiv war. Die ersten zehn Boxen können
mit den Zifferntasten \Ctrl+\keystroke{0} bis \Ctrl+\keystroke{9}
angewählt wurden. Die Instrumente/MIDI-Kanäle 1–9 können mit den
Zifferntasten \keystroke{1} bis \keystroke 9 ausgewählt werden.
Man kann also nur die
Instrumente/MIDI-Kanäle 1--9 mit einem Tasten-Auslöser steuern.

      \input{syn19.pic}

\subsection{Harmonien}\label{sec:harmonien}
\index{Harmonien}
Harmonie-Auslöser\index{Auslöser!Harmonie}
 können wahlweise eine Harmonie oder eine Harmonieform
\index{Auslöser!Harmonie-Form}
erkennen. Die Harmonie wird beim Namen genannt und die enthaltenen
Tasten-Nummern beziehen sich auf die entsprechenden Ton-Nummern der
Fundamentaltonleiter\index{Fundamentaltonleiter}.
Alle gedrückten Tasten werden in die
Fundamentaltonleiter hineinprojiziert und das entstehende Tastenmuster
wird mit der Harmonie verglichen. Bei Übereinstimmung gilt die
Harmonie als erkannt. Tasten, die in der Harmonie mit einem
Stern (*)\index{*} gekennzeichnet sind, werden bei diesem
Vergleich nicht berücksichtigt.
Einschränkenderweise kann man mit der Vortaste\index{Vortaste} angeben,
welche Tonigkeit\index{Tonigkeit}
der tiefste Ton haben muss, damit diese Harmonie gilt. Entsprechend muss
bei Angabe der Nachtaste\index{Nachtaste} der höchste Ton
die Tonigkeit der Nachtaste
haben. Bei Angabe von Vortaste und Nachtaste müssen beide Bedingungen
erfüllt sein. Die Tonigkeit eines Tons bezieht sich hier auf die Breite
der aktuellen Fundamentaltonleiter.\footnote{Diese Breite ist nicht
auf 12 festgelegt, sondern in weiten Grenzen frei wählbar.}
Wenn Vortaste oder Nachtaste eine Tonigkeit haben, die nicht in der
Harmonie vorkommt, so kann diese Harmonie nicht gespielt werden, da sie
sich selbst widerspricht. Solche Harmonien sind „\emph{unmögliche}“
Auslöser\index{Auslöser!unmögliche}
 und werden mit einer Warnung gekennzeichnet. Praktisch
bleiben sie ohne Effekt, da sie nie zutreffen können.

Bei einer Harmonieform\index{Harmonieform} wird überprüft,
ob die Harmonie relativ zu jedem
Ton der Fundamentaltonleiter zutrifft. Wird eine solchermaßen
verschobene Harmonie erkannt, gilt der Auslöser als erfüllt
und der Wert \texttt{ABSTAND}\index{ABSTAND} wird entsprechend der Harmonieanalyse gesetzt.

Wenn bei der Harmonie eine Bezugstaste\index{Bezugstaste}
 angegeben wurde, dann wird
die Harmonie-Analyse relativ um die Bezugstaste verschoben durchgeführt.

    \input{syn20.pic}

\subsection{MIDI-Ereignisse}\index{Auslöser!MIDI-Ereignis}\label{sec:midi-ereignisse}
MIDI-Ereignisse werden mit dem Wort \emph{MIDIIN}\index{MIDIIN}
eingeleitet und
sind eine eingeklammerte Folge von Zahlen. Bei Zahlen ist die sedezimale
Schreibweise zulässig (\#\emph{ziffern}).\index{\#}
Die erste Zahl muss im Bereich von
128 bis 255 liegen, also ein Status-Byte sein. Alle weiteren Zahlen
müssen im Bereich von 0 bis 127 liegen, also Daten-Bytes sein.
Beim Status-Byte müssen die unteren vier Bits (Kanal-Nummer) auf Null stehen.
Die tatsächliche Kanalnummer wird dem jeweiligen
Instrument/MIDI-Kanal zugeordnet,
so daß ein MIDI-Ereignis sich jeweils auf das richtige
Instrument/MIDI-Kanal bezieht.
Die Status-Bytes\index{Status-Byte}
 \emph{Note-On\index{Note-On} (\#90)}, \emph{Note-Off\index{Note-Off}
  (\#80)} und
\emph{System-Byte\index{System-Byte}
 (\#F0 - \#FF)} werden nicht in die MIDI-Analyse
einbezogen.

    \input{syn21.pic}

\section{Aktionen}\index{Aktionen}\label{sec:aktionen}

Aktionen werden innerhalb einer Logik ausgelöst und können wahlweise
ein Tonsystem aufrufen, eine Umstimmung aufrufen,
zu einer Logik wechseln oder eine MIDI-Nachricht
senden. Beim Aufruf einer Umstimmung müssen genau so viele Parameter\index{Parameter}
mitgegeben werden, wie die aufgerufene Umstimmung benötigt. Die mitgegebenen
Parameter werden in der gleichen Reihenfolge den Parameter-Namen
der Umstimmung zugeordnet. Als Parameter können Zahlen angegeben werden,
sowie der Wert \emph{ABSTAND}\index{ABSTAND}.
Der Wert ABSTAND entspricht den Gegebenheiten der
zuletzt durchgeführten Harmonie-FORM-Analyse, wobei der Wert
ABSTAND bis zur nächsten Harmonie-FORM-Analyse beibehalten wird.

Eine Logik kann eine einzelne Aktion ausführen oder eine in geschweiften
Klammern stehende Aktions-Liste aus\-füh\-ren. Im Falle der Aktions-Liste
werden die einzelnen Aktionen der Liste in genau der angegebenen
Reihenfolge ausgeführt.


\subsection{MIDI-Ausgaben}\index{MIDI!Ausgabe}\label{sec:midi-ausgaben}
\index{Aktion!MIDI-Ausgabe}

MIDI-Ausgaben werden mit dem Wort \emph{MIDIOUT}\index{MIDIOUT}
 eingeleitet und
sind eine eingeklammerte Folge von Zahlen. Bei Zahlen ist die sedezimale
Schreibweise zulässig (\#\emph{ziffern}). Die Zahlen
müssen im Bereich von 0 bis 255 liegen, also Bytes sein. Es erfolgt keine
Kontrolle, ob es sich um eine sinnvolle MIDI-Nachricht handelt.

    \input{syn22.pic}

\chapter{MIDI-Kanäle}\label{cha:midi-kanale}
\index{MIDI!Kanal}
\label{MIDIKANAL}

Eine MIDI-Kanal-Deklaration definiert die Zuordnung, auf welchen
MIDI-In-Kanal \mutabor{} reagieren soll, und wieviele MIDI-Out-Kanäle
den einzelnen Eingabekanälen zugeordnet werden. Dies ist für die
Tonerzeugung nach dem Multi-16-Verfahren\index{Multi-16-Verfahren}
 nötig. 
\iffalse
Wird als
Synthesizer-Treiber das Modell FB-01\index{FB-01} eingestellt, so gilt nur
die Angabe des MIDI-In-Kanals, der Ausgabekanal wird ignoriert.
\fi
Die MIDI-In-Kanäle werden auch als \emph{Instrumente} bezeichnet, da
für jeden MIDI-In-Kanal separat das gesamte Logikprogramm gilt.
Es können bis zu 16 Instrumente gleichzeitig und in
verschiedenen Logiken spielen.
Werden alle 16 Instrumente/MIDI-Kanäle
und das Multi-16-Verfahren benutzt, so kann jedes Instrument nur
noch einen einzigen
Ton gleichzeitig spielen, da nicht mehr als eben diese 16 Midikanäle
auch als Ausgabe-Kanäle verfügbar sind. 
\iffalse
Beim Modell FB-01 von Yamaha wird
eine spezielle System-exklusive Ansteuerungsmethode verwendet,
so daß hier bis zu 16 Instrumente zu je 16 gleichzeitigen Tönen
spielbar sind.
\fi

Beim Multi-16-Verfahren wird
jedem Eingabekanal ein Bereich von MIDI-Out-Kanälen zugeordnet. Die Größe
dieses Bereichs entspricht der Anzahl Töne,
die mit diesem Eingabekanal gleichzeitig gespielt werden können.
Die Kanal-Information des empfangenen Statusbytes bestimmt, für
welchen Eingabekanal die Meldung gilt.
Jeder Eingabekanal hat für sich das gesamte Logik-Programm
zur Verfügung und arbeitet unabhängig von den anderen Eingabekanälen.
Bei der Analyse einer Harmonie und bei der Analyse einer MIDI-in-Nachricht
wird berücksichtigt, auf welchen Eingabekanal es sich bezieht.
Auf dem Bildschirm kann mit den Zifferntasten 1 bis 9 zwischen den
Eingabekanälen 1 bis 9 umgeschaltet werden. Ein Tasten-Auslöser
bezieht sich immer auf den am Bildschirm angezeigten Eingabekanal.
Falls keine MIDI-Kanäle deklariert werden, und auch keine Kanalzuordnung
in der Konfigurationsdatei angegeben ist,
gilt die Deklaration
\begin{lstlisting}
	MIDIKANAL  1  ->  1 - 16
\end{lstlisting}
als automatisch gegeben.

\input{syn25.pic}

\section{Worterklärungen}\label{sec:worterklarungen}
\begin{description}
\item[Taste] Die Klaviatur-Tasten werden als Abstand in
  Tastatur-Halbtönen zur Verankerungstaste angegeben. Die
  Verankerungstaste hat somit die Nummer 0, die nächste Taste weiter
  rechts die Nummer 1, die nächste die 2 usw.

\item[Selektor] Der Selektor einer Umstimmungs-Alternative ist der
  Parameter, nach dem sich die Alternativen ausrichteten. In der Regel
  ist dies ein Parameter (d.\,h. der Parametername) der Umstimmung und
  repräsentiert so eine ganze Zahl.

\item[Buchstabe] Hier können sie einen der 26 Standard-Buchstaben des
  lateinischen Alphabets angeben (also keine Umlaute). Wenn Sie dann
  auf der Computer"=Tastatur diese Taste drücken wird das zugeordnete
  Ereignis ausgelöst. Es wird dabei nicht zwischen Klein- und
  Großbuchstaben unterschieden.

\item[Verankerungs-Taste] Die Verankerungs-Taste ist die
  Klaviatur-Taste, bei der die Fundamentaltonleiter beginnt und wird
  als MIDI-Nummer der Taste angegeben (also als Zahl). Das
  eingestrichene c (c') hat die Nummer 60, die anderen Tasten sind
  aufsteigend bzw. absteigend durchnummeriert, d.\,h.  c\#' hat die
  Nummer 61, d' die 62 usw.

\item[Konstanten] Die Konstanten ganze Zahlen, die festlegen, bei
  welchem Wert des Selektors in die
  darauf folgende Umstimmung verzweigt wird.

\item[Namen] Viele Aufrufe und Zuordnungen laufen in \mutabor{}
  über die Namen ab. Beim deklarieren eines jeden Objektes müssen Sie
  diesem einen Namen geben, den Sie später als Referenz in anderen
  Deklarationen angeben können.

\item[Zahlen] Zahlen können in \mutabor{} als Ganze Zahlen oder
  Dezimalzahlen (mit Punkt als Komma) angegeben werden.

\item[MIDI-Werte] MIDI-Werte sind ganze Zahlen zwischen 0 und 255, die
  von MIDI-Geräten als Befehle und Daten ausgewertet werden.
  Zusätzlich ist hier die hexadezimale Schreibweise erlaubt, der ein
  Gatterzeichen (\#) voran zu stellen ist, also z.\,B. \#9C, \#A0, \#78,
  \#E0 \dots
\end{description}

%\chapter{Konfiguration}
%\index{Konfiguration}
%
%Die Datei \texttt{MUTABOR.CFG} enthält die globalen Voreinstellungen bezüglich
%der an den Instrumenten angeschlossenen Synthesizer-Treiber und
%der Voreinstellungen der Kanalzuordnung bei den MIDI-Kanälen.
%Als Treibernummern sind bisher vorgesehen:
%\begin{description}
% \item[1] System-Exclusiv, Typ Yamaha FB01
% \item[2] Standard Midi nach Multi-16-Methode
%\end{description}
%Falls hier MIDI-Kanäle deklariert sind, so gilt deren Kanal-Zuordnung,
%es sei denn, das Logikprogramm gibt eine eigene Kanal-Zuordnung vor.
%Falls keinerlei Kanalzuordnung getroffen wird, so gilt die Zuordnung
%\texttt{1 -> 1 - 16} als gegeben.




\printindex
%\input{handbuch.ind}

\end{document}
