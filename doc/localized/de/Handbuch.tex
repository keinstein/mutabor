% Um alles auf einmal pflegen zu können, verwenden wir \ifhtml als
% Anzeige für tex4ht oder „normales“ (PDF)LaTeX.
%
\makeatletter

\let\ifhelp\iffalse
\let\ifhtml\iffalse
\let\ifpdf\iffalse
\let\ifdvi\iffalse

\def\mutabor@utput@html{
  \typeout{Producing HTML format.}
  \let\ifhtml\iftrue
}

\def\mutabor@utput@help{%
  \mutabor@utput@html
  \typeout{Subtype: .hhp .}
  \let\ifhelp\iftrue
}

\@ifundefined{mutabor@utput@\outputformat}{
  \typeout{Undefined output format%
    \@ifundefined{outputformat}{}{ \outputformat}.
  }
}{
  \csname mutabor@utput@\outputformat\endcsname
}

%\def\htmltrue{\let\ifhtml\iftrue}
%\def\htmlfalse{\let\ifhtml\iffalse}
%\ifx\ifhtml\undefined
%  \htmlfalse
%\else
%  \htmltrue
%\fi
% 
% Ein Kommentar aus dem alten Handbuch... ;-)
%
% =====================================================
%
% NOCH ZU KORRIGIEREN :
%
%
% - oberflächenspezifische Beschreibungen auslagern !
%
% =====================================================
%
% Koma-Skript liefert ein paar nützliche zusätzliche Definitionen
%
\documentclass[german,a4paper,BCOR1.0cm]{scrbook}
%
% Für tex4ht müssen wir auch tex4ht laden...
\ifhtml
\setcounter{tocdepth}{1}
% Grundeinstellungen liegen in der Datei „tshtml.cfg“
\usepackage[%tshtml,
mutabor,
% Es folgt das Hauptformat: hier xhtml
            html,
% CSS2 verwenden
%            css2,
% in der Log-Datei Informationen zur Konfiguration von tex4ht ausgeben
          info,
% Fußnoten (und Literaturzitate) erscheinen beim überfahren der Marke
%            mouseover,
% Auf der ersten Seite wird auch ein „Weiter“-Pfeil angezeigt
            next,
            sections+,
% HTML-Dateien möglichst kurz -- maximale Aufsplittungstiefe
            4,
 %            nominitoc,
% Index 2-Spaltig
            index=2,
            Gin-dim+,
% HTML-Zeichensatz: UTF-8
            charset=utf-8,
            hyperref,
            NoFonts,
%            fonts+
%            fonts,
]{tex4ht}
\makeatletter
\edef\ts@savecatcode{\noexpand\catcode`\noexpand\:\the\catcode`\:\relax} 
%\catcode`\:11\relax
\AtEndDocument{\makehhk}
\newcommand\tshhkentry{}
\newcommand\tsatoclink{}

%\:::HRefTag=macro:
%#1#2->\if \relax #2\relax \else \:TagHTag {#2}\fi \HCode {<\tag:A \:newlnch \if
% \relax #1\relax \NOHREF: {#2}\else \HREF: "\get:hfile {#1}\:sharp #1"\fi \if \
%relax #2\relax \else \space \NAME: "#2"\fi \:attr \empty:lnk >}.
%<insert>    \show\:::HRefTag

\expandafter\def\expandafter\mut@sharp\expandafter{\csname :sharp\endcsname}

\newif\ifwx@inrange
\wx@inrangefalse
\def\wx@item#1{%
  \def\wx@item@{#1}%
  \let\wx@indextext\wx@item@
}
\def\wx@subitem#1{%
  \def\wx@subitem@{\wx@item@{}: #1}
  \let\wx@indextext\wx@subitem@
}
\def\wx@subsubitem#1{%
  \def\wx@subsubitem@{\wx@subitem@{} -- #1}%
  \let\wx@indextext\wx@subsubitem@
}
\def\wx@LNK#1#2#3#4{%
  \ifwx@inrange
  \wx@inrangefalse
  \else
  \HCode{<li>^^J}%
  \HCode{<object type="text/sitemap" >^^J%
\space<param name="Name" value="}\wx@indextext\HCode{" >^^J%
\space<param name="Local" value="}#1\mut@sharp #2\HCode{" >^^J%
</object>^^J}% 
  \HCode{</li>^^J}%
  \fi
}
\def\loadwxindex{%
  {%
    \let\item\wx@item
    \let\subitem\wx@subitem
    \let\subsubitem\wx@subsubitem
    \let\LNK\wx@LNK
    \let\rangeto\wx@inrangetrue
    \InputIfFileExists{\jobname.wxi}{}{}%
  }%
}



\def\tsatoclink#1#2#3#4{%
  \typeout{Setting: \csname get:hfile\endcsname{#2} at \csname :sharp\endcsname #2}%
  \HCode{%
\tslinestart<object type="text/sitemap" >^^J
\tslinestart\space<param name="Name" value="}#4\HCode{" >^^J
\tslinestart\space<param name="Local" value="}\csname get:hfile\endcsname{#2}\csname :sharp\endcsname #2\HCode{" >^^J
\tslinestart</object>^^J}% "<
%   \expandafter\ifx \csname #3-def\endcsname\relax
%      \global \expandafter\let \csname #3-def\endcsname\def
%      \Link {#2}{#3}
%   \else
%      \Link {#2}{}
%   \fi 
%   {\Configure {ref}{}{}{}%
%     \let \EndLink =\empty
%     \let \H:Tag:attr \:gobbleII
%     \let \:::HRef \empty
%     \def \::hRef [##1]##2{}%
%     \def \::hRefTag [##1]##2##3{}%
%     \def \:::HRefTag ##1##2{}%
%     \Configure {cite}{}{}{}{}%
%     #4}%
%   \EndLink
}
\def\ts@hhkentry#1#2#3#4{%
  \def\tslinestart{#1\space}%
  \HCode{#1<li>^^J}%
  #3%
  \HCode{#1</li>^^J}%
}
\def\tshhkentry{\ts@hhkentry{}}
\def\tstocendsubparagraph{}
\def\tstocendparagraph{\endsubparagraph}

\def\deftocmacro#1#2#3{
  \expandafter\def\csname tstocstart#1\endcsname{%
    \HCode{#3\space<ul><!-- u#1 -->^^J}%
    \expandafter\let\csname tocstart#1\endcsname\relax
    \expandafter\def\csname tocstop#1\expandafter\endcsname{\csname tstocstop#1\endcsname}
  }

  \expandafter\def\csname tstocskip#1\endcsname{\csname tocstop#2\endcsname}

  \expandafter\def\csname tstocstop#1\endcsname{%
    \csname tocstop#2\endcsname
    \HCode{#3\space</ul><!-- i#1 -->^^J}%
    \expandafter\def\csname tocstop#1\endcsname{\csname tstocskip#1\endcsname}
  }

  \expandafter\def\csname tstoc#2\endcsname{%
    \csname tocstop#2\endcsname
    \csname tocstart#1\endcsname
    \expandafter\def\csname tocstart#2\endcsname{\csname tstocstart#2\endcsname}
    \HCode{#3\space\space<!-- a#2 -->^^J}%
    \ts@hhkentry{#3\space\space}%
  }%
}

\def\tstocpart{
  \tocstoppart
  \ts@hhkentry{}%
}

\deftocmacro{part}{chapter}{}
\deftocmacro{chapter}{section}{\space}
\deftocmacro{section}{subsection}{\space\space}
\deftocmacro{subsection}{subsubsection}{\space\space\space}
\deftocmacro{subsubsection}{paragraph}{\space\space\space\space}
\deftocmacro{paragraph}{subparagraph}{\space\space\space\space\space}
\def\tocstopsubparagraph{}

\def\gobblenl{\@ifnextchar[\@gobblenl{}}
\def\@gobblenl[#1]{}
{\catcode`\^^J=\active
  \gdef\mknlsp{%
    \def^^J{ }}%
}
\def\@nl@end{nl@end}
\def\@removenl#1#2\@nl@end{%
  \ifx#1^^J
  \ 
  \else
  #1
  \fi
  \ifx#2\relax
  \else
  \expandafter\@removenl
  \fi
}

\newcommand\removenewline[1]{%
  \ifx#1\relax
  \else
    \@removenl#1\@nl@end
  \fi
}

\newcommand\mutabortitle{}
\ifhelp
  \let\mutaborsavesubject\subject
  \def\subject#1{%
    \def\mutabortitle{#1}%
    \mutaborsavesubject{#1}%
  }
\else
  \let\mutaborsavetitle\title
  \def\title#1{%
    \def\mutabortitle{#1}%
    \mutaborsavetitle{#1}%
  }
\fi
\def\mutabordefaulttopic{\jobname.html}
\newcommand\tsarg{}
\def\tsarg#1{#1}
\let\tsdotocentry\tsarg
\expandafter\let\expandafter\mut@gobbleIV\csname :gobbleIV\endcsname
\expandafter\let\expandafter\mut@gobbleIII\csname :gobbleIII\endcsname
\expandafter\let\expandafter\mut@gobble\csname :gobble\endcsname
\newcommand\makehhk{%
  {\ignorespaces
    \let\par\relax
    \def\showname##1{\expandafter\show\csname ##1\endcsname}%
    \let\@gnewline\space
    \def\@newline{\space\mut@gobbleIII}%
    \let\texorpdfstring\@secondoftwo%
    \let\fontencoding\mut@gobble
    \let\fontfamily\mut@gobble
    \let\fontseries\mut@gobble
    \let\fontshape\mut@gobble
    \let\fontsize\mut@gobble
    \let\@setfontsize\mut@gobbleIII
    \let\usefont\mut@gobbleIV 
    \let\selectfont\relax
    \let\LARGE\relax
    \let\normalsize\relax
    % .hhp-Datei erstellen.
    \special{t4ht>\jobname.hhp}%
    \HCode{%
Contents file=\jobname.hhc^^J%
Index file=\jobname.hhk^^J%
Title=}{%
      \removenewline\expandafter{\mutabortitle}}%
    \HCode{^^J%
Default Topic=\mutabordefaulttopic^^J
Charset=UTF-8^^J}%
\special{t4ht<\jobname.hhp}%
\typeout{done}%
%\expandafter\show\csname a:TocLink\endcsname
    \let\doTocEntry\tsdotocentry
    \expandafter\let\csname a:TocLink\endcsname\tsatoclink
    \let\toclikesection\tshhkentry
    \let\toclikechapter\tshhkentry
    \let\tocaddchap\tshhkentry
    \let\tocpart\tshhkentry
    \let\tocsection\tshhkentry
    \let\tocchapter\tshhkentry
    \let\tocsubsection\tshhkentry
    \let\tocparagraph\tshhkentry
    \let\tocappendix\tshhkentry
    \let\tocminisec\tshhkentry
    \let\textsc\tsarg%
    \special{t4ht>\jobname.hhk}%
    \HCode{<ul>^^J}%
    \loadwxindex 
%    \catcode`\#11
    \InputIfFileExists{\jobname.4ct}{}{}%
\typeout{finished}%
    \HCode{</ul>^^J}%
    \special{t4ht<\jobname.hhk}%
\typeout{closed}%
%
    \let\toclikesection\tstocsection
    \let\toclikechapter\tstocchapter
    \let\tocaddchap\tstocchapter
    \let\tocpart\tstocpart
    \let\tocsection\tstocsection
    \let\tocchapter\tstocchapter
    \let\tocsubsection\tstocsubsection
    \let\tocparagraph\tstocparagraph
    \let\tocappendix\tstocchapter
    \def\tocminisec{\tocparagraph}%
%
    \let\tocstartpart\relax
    \let\tocstoppart\tstocskippart
    \let\tocstartchapter\relax
    \let\tocstopchapter\tstocskipchapter
    \let\tocstartsection\relax
    \let\tocstopsection\tstocskipsection
    \let\tocstartsubsection\relax
    \let\tocstopsubsection\tstocskipsubsection
    \let\tocstartsubsubsection\relax
    \let\tocstopsubsubsection\tstocskipsubsubsection
    \let\tocstartparagraph\relax
    \let\tocstopparagraph\tstocskipparagraph
    \let\tocstartsubparagraph\relax
    \let\tocstopsubparagraph\tstocskipsubparagraph
%
    \special{t4ht>\jobname.hhc}%
    \HCode{<ul>^^J}%
%    \catcode`\#11
    \InputIfFileExists{\jobname.4ct}{}{}%
    \HCode{</ul>^^J}%
    \special{t4ht<\jobname.hhc}%
  }%
  \relax
}
\fi
% Einstellungen für Latex laden
\usepackage[utf8]{inputenc}
\usepackage[T1]{fontenc}
\usepackage[german]{babel}
% Exteren Referenzen und Hyperref laden. 
% Das kann unterschiedlich ablaufen.
\ifhtml
 %  \Configure{html}{html.de}
  \ifpdfoutput{\pdfoutput0\relax}{}
   \usepackage{xr-hyper}
   \usepackage{hyperref}
\else
 %  \usepackage{mathpazo}
  \usepackage{xr-hyper}
  \usepackage[extension=pdf]{hyperref}
\fi
% Überschriften als Referenzen einfügen.
\usepackage{nameref}
%\ifhtml
%\input nameref.4ht
%\fi
% Papierformat a4weit wird nicht empfohlen
%\usepackage{a4wide}
%
% Die üblichen Pakete für Grafik und Farben 
\usepackage{graphicx}
\usepackage{color}
%
% Das Handbuch wurde mit emTeX geschrieben. TeTeX versteht die emTeX
% specials. Also verwenden wir sie.
\usepackage{emlines}
% Makeindex laden
\usepackage{makeidx}
%
% file: macht uns die relativen Links kaputt :-(
%
\hyperlinkfileprefix{}
%
% Kolumnentitel erleichtern dem Leser die Orientierung
%
\AtBeginDocument{\pagestyle{headings}}
% 
% Titelei
\setkomafont{title}{\fontfamily{\rmdefault}\fontseries{bx}\huge}
\title{\rmfamily\texorpdfstring{\mutabor\\[\baselineskip]}{MUTABOR --}
 \LARGE\slshape Ein computergesteuertes Musikinstrument \\
	  zum Experimentieren mit\\
	  Stimmungslogiken und Mikrotönen}
\author{Volker Abel, Peter Reiss,\\ Rüdiger Krauße und Tobias Schlemmer}
\date{Programmversion $3.0x$ (\the\year)}
\ifhtml\else
  \publishers{\includegraphics[width=0.5\linewidth]{start}}
\fi
\lowertitleback{\footnotesize\copyright 1991, 1992 Volker Abel \& Peter Reiss\\
\copyright 2006 TU Dresden, Institut für Algebra}
%
% Index-Datei öffnen
%
\makeindex
\hyphenation{wei-te-re Ton-sys-tem Ton-sys-te-me}
%\parindent 0mm
%\parskip 5pt
%\textheight 18.5cm
%
% Jetzt wirds ein wenig wüst.
%
% Wir wollen mit wenig Aufwand die Labels auch für Hyperlinks
% verwenden. Damit sparen wir uns zusätzliche Anker.
% 
\makeatletter
%
% \XR@ext enthält die Erweiterung für Querverweise. Wenn wir auf
% Buchanfänge usw. verweisen wollen, kann uns xr-hyper nicht
% undbedingt helfen. (oder wir müsste das entsprechend definieren).
% Einfacher ist es wohl, wir nehmen den Dateinamen.
%
\newcommand\makefilename[1]{#1.\XR@ext}
%
%
% erstes von sechs Argumenten wiedergeben und Rest verwerfen
\def\ts@firstofsix#1#2#3#4#5#6{#1}
%
\ifhtml
% voreinstellungen, um „:“ als Buchstaben zu behandeln (für tex4ht)
% 
% Herausfiltern des Verlinkungsmakros aus Querverweis-Speichern
%
\def\ts@parse@ref@a#1#2\ts@end@parse{\ts@parse@ref@b#1\ts@end@parse}
\def\ts@parse@ref@b#1#2#3\ts@end@parse{#1{#2}}
%
% Setzen des Verweises.
% Das erste Argument mus zunächst expandiert werden, bevor es
% überhaupt mit den obigen makros geparst werden kann. Wir sichern uns
% das ganze in einem temporären Makro zusammen mit dem Ankertext. 
%
\def\ts@setref#1#2#3{%
  \expandafter\expandafter
  \expandafter\def
  \expandafter\expandafter
  \expandafter\ts@@tmp
  \expandafter\expandafter
  \expandafter{%
    \expandafter\ts@parse@ref@a#1\ts@end@parse{#2}}%
%
% Wir hacken uns in \@setref hineein.
% dort steht: \expandafter #2#1. Wir liefern aber alles, was wir
% brauchen schon in #2 mit, verwerfen also alles aus #1.
% wir machen daraus \expandafter\ts@firstofsix\expandafter\ts@@tmp#1
% damit wird zunächst #1 expandiert und dann verworfen.
% Genial nicht ;-)?
%
% Alternativ könnte man auch mit \@firstoftwo arbeiten. Hier muss man
% aufpassen, dass das erste Argument nicht falsch expandiert wird.
% \setref#1{\@firstoftwo{\ts@@tmp}}{#3}
%
  \@setref#1{% 
    \ts@firstofsix%
      \expandafter\ts@@tmp}{#3}%
}
%
% Jetzt können wir das eigentliche Linkmakro definieren. 
%
% Wenn die Referenz nicht definiert ist, setzen wir den Ankertext und
% rufen setref auf, damit die entsprechende Warnung ausgespuckt
% wird. Die ausgegebenen Fragezeichen setzen wir in weißer Farbe in
% eine 0pt breite Box (llap). Damit kommt es nur zu minimalen
% Verschiebungen (Kerning) zwischen undefiniert und definiert. 
%
% Ist die Referenz definiert, wird sie verwendet, um einen Link zu erzeugen.
\newcommand\ts@reflink[2]{%
  \@ifundefined{r@#1}{%
    \textcolor{red}{%
      #2%
      \color{white}{%
        \llap{%
          \@safe@activestrue
          \edef \RefArg {#1}
          \expandafter\ts@setref\csname r@#1\endcsname{{\@safe@activesfalse #2}}{#1}%
          \@safe@activesfalse
        }%
      }%
    }%
  }{%
    \@safe@activestrue
%    \let\::ref \T:ref
    \expandafter\ts@setref\csname r@#1\endcsname{{\@safe@activesfalse #2}}{#1}%
%    \def\::ref{\protect\T@ref}%
    \@safe@activesfalse
  }%
}
% : wiederherstellen
%\ts@savecatcode
\else
% Hier läuft es eigentlich genauso ab, wie bei der
% tex4ht-Variante. Nur werden jetzt die Verweise etwas anders kodiert,
% so dass man sie nicht wirklich neu parsen muss.
\newcommand\ts@reflink[2]{%
\begingroup%\tracingall
  \def\ts@tmp{{\@safe@activesfalse #2}}%
  \@ifundefined{r@#1}{%
    \textcolor{red}{%
      #2%
      \color{white}{%
        \llap{%
          \@safe@activestrue
          \expandafter\@setref\csname r@#1\endcsname{\ts@firstofsix\ts@tmp}{#1}%
          \@safe@activesfalse
        }%
      }%
    }%
  }{%
    \@safe@activestrue
    \expandafter\@setref\csname r@#1\endcsname{\ts@firstofsix\ts@tmp}{#1}%
    \@safe@activesfalse
  }%
%\show\ts@reflink
\endgroup
}
\fi
%\newcommand\ts@@reflink{\protect\ts@reflink}
%
% die anderen definierten Referenz-Makros sind auch \protect-et
% definiert. Also machen wir das auch mit einem Querverweis auf ein
% Label mit einem beliebigen Ankertext.
%
\newcommand\reflink{\protect\ts@reflink}
%
% Referenz durch Titel, ggf. Zusatz (wie z.\,B. Buchname bei externen
% Referenzen) und Seite bei nicht-HTML-Ausgabe. Der Zusatz wird in []
% angegeben. Das erledigen wir durch zwei Makros:
%
\newcommand\tsciteref[1]{%
  \frqq\nameref{#1}\flqq
  \@ifnextchar[{\ts@cite@ref{#1}}{\ts@cite@@ref{#1}}
}
% mit optionalem Argument: Leerzeichen+Zwischentext und dann
% Seitenangabe ohne zweites Arg.
\def\ts@cite@ref#1[#2]{ #2\ts@cite@@ref{#1}}
% Ohne optionales Argument: Seitenangabe
\def\ts@cite@@ref#1{%
  \ifhtml\else{}
  auf Seite \pageref{#1}%
  \fi
}
% So ziemlich alles, was man referenzieren kann:
% Verweis-Typ: Kapitel, Abschnitt usw. 
% Nummer, Titel, ggf. optionales Argument, ggf. Seite.
% Syntax \vollref{Label}[zwischentext] (letzteres wird durch
% \tsciteref eingeführt
\newcommand\vollref[1]{\autoref{#1} \tsciteref{#1}}
%
% Und hier sind wir aus den Interna heraus.
%
\makeatother
%
% Ein Makro zur vereinfachten Umwandlung der Mutabor-Hilfe mit Querverweisen.
%
\newcommand\tsreflink[2]{\reflink{sec:#2}{#1}}
%
% Der Name unseres Programmes
%
\newcommand\mutabor[1][]{\textsc{Mutabor#1}}
%
% Vorbereitung zur Verwendung des Pakets keystroke. Damit werden
% Tasten als Piktogramme angezeigt.
%
\usepackage{keystroke}%
%\providecommand\keystroke[1]{#1}
%
% Schlüsselworte und sonstiger Quelltext in Schreibmaschinenschrift.
%
\newcommand\keyword[1]{\texttt{#1}}
%
% Wir wollen die Handbücher ja nicht nur in die CD integrieren.
%
\ifhtml
  \newcommand\cdoronline[2]{#1}
\else
  \newcommand\cdoronline[2]{#2}
\fi

\newcommand\sourcecode{\texttt}%
\newcommand\filename{\texttt}

\newcommand\mutimage[2]{%
  \ifhtml
  \Picture{#1}%
  \else
  \includegraphics#2
  \fi
}

\endinput


\subject{Programmier- und Bedienungshandbuch}
\externaldocument{Referenz}
\externaldocument{Bedienung}
\begin{document}
\ifhtml
\setcounter{tocdepth}{1}
\fi

\maketitle
%\ifhtml\else\clearpage\fi
\ifhtml\ifhelp\else
\section*{Verfügbare Formate dieses Handbuchs}
Diese Dokumentation können sie in folgenden Formaten lesen und ausdrucken:

\HCode{<?php require_once(\dq\cdoronline{}\urlbase includes/hilfsfkt.inc\dq);?>\Hnewline
  <div class=\dq dokumentation download tex4ht-vorspann\dq>\Hnewline
<table class=\dq beitraege\dq>\Hnewline
  <tbody class=\dq beitraege\dq >\Hnewline
  <tr class=\dq beitraege\dq >	\Hnewline
    <th
      class=\dq beitraege\dq ><a href=\dq
      \jobname.html\dq class=\dq link\dq>\jobname.html</a></th>\Hnewline
   <td class=\dq beitraege\dq >Das Dokument, das Sie gerade lesen. </td>\Hnewline
  </tr>\Hnewline
  <tr class=\dq beitraege\dq >\Hnewline
    <th
      class=\dq beitraege\dq ><?=dateilink(\dq \jobname.ps\cdoronline{}{.bz2}\dq )?></th>\Hnewline
    <td class=\dq beitraege\dq >PostScript zum Ausdrucken</td>\Hnewline
  </tr>\Hnewline
  <tr class=\dq beitraege\dq >\Hnewline
    <th class=\dq beitraege\dq ><?=dateilink(\dq \jobname.pdf\dq )?></th>\Hnewline
    <td class=\dq beitraege\dq >PDF zum Ausdrucken</td>\Hnewline
  </tr>\Hnewline
  </tbody>\Hnewline
</table>\Hnewline
</div>}
\Css{div.tex4ht-vorspann { 
    margin-bottom: 2em; 
    width:30em;
    width:auto;
    max-width:30em;
}}
\fi
\else
\tableofcontents
\fi

\addchap{Vorwort}
\index{Projektgründung}
Das Projekt MUTABOR~II wurde im August 1987
gegründet. Das ursprüngliche Ziel war es, das Instrument \mutabor{},
welches im Jahre 1984 im Rahmen des Forschungsvorhabens
"`Mathematische Musiktheorie"' an der TH Darmstadt\index{TH Darmstadt} gebaut wurde, auf
einen handels\-üblichen Rechner zu übertragen, da der Prototyp von
\mutabor{} ein Unikat ist.

Aus diesem zunächst nur als kosmetische Korrektur geplanten Projekt
entstand im Laufe der Zeit ein völlig neues und umfassenderes Konzept
eines Instrumentes zum Experimentieren mit Stimmungslogiken und
Mikrotönen. MUTABOR~II wurde erstmals im Mai 1991 auf dem vierten
internationalen Symposium für Mikrotonforschung im Mozarteum Salzburg
vorgestellt.

MUTABOR 3 ist ein darauf aufbauendes Projekt, welches eine 
breitere Palette an technischen Möglichkeiten bereitstellt. MUTABOR 3
ist ein Projekt der TU Dresden.

\begin{tabular}{l@{\ \ --\ \ \ }l}
\mutabor{}  & Einerseits lateinisch: Ich werde verändert werden \\
\mutabor{}  & andererseits \textbf{Mut}ierende \textbf{a}utomatisch 
               \textbf{b}etriebene \textbf{Or}gel  \\
\mutabor{}  & oder das Zauberwort aus Kalif Storch \\
\mutabor{}  & aber auch Mut, ab dem Ohr seltsame Dinge zu hören \\
\mutabor{}  & und Mut, dass das Ohr abfallen könnte \\
\mutabor{}  & \ldots \\
\end{tabular}

\index{Danksagung} Auf diesem Wege danken die Autoren allen, die ihnen
bei der Entwicklung dieses Programmes so hilfreich zur Seite gestanden
haben. Explizit genannt seien Herr Levigion, Herr Dr. Pense und Herr
Dr. Schmitt, Uni Mainz, die zum konzeptuellen Entwurf viele Ideen
eingebracht haben, Herrn Prof. Ganter und Herrn Prof. Wille, TH
Darmstadt, deren Projekt "`\mutabor{}"' hier seine Weiterentwicklung
gefunden hat.

\vspace{2cm}

\begin{quotation}
\parbox[t]{7cm}{\it
"`Feinste Tonunterschiede werden mit dem computergesteuerten
Musikinstrument MUTABOR II hörbar."'\\ \rm (Darmstädter Echo)
} \hfill \parbox[t]{7cm}{\it
"`Für Experimentalmusiker bietet das computergestützte Gerät [\dots{}] einen
schier unerschöpflichen Fundus für eigene Kreationen. [\dots{}] Das Instrument
macht feinste Tonabfolgen in die Höhe oder Tiefe, über die man bisher nur
theoretisch fachsimpeln konnte, endlich hörbar."' \rm (Frankfurter Rundschau)}
\end{quotation}

\ifhtml\else\clearpage\fi

%\input{storch.pic}

%\vspace{-1cm}

\iffalse
\index{Seriennummer}
\fbox{
  \small \sc
  \parbox{14.6cm} { Bedienungshandbuch zur Version 2.1 / Juli 1992 \\
    Die Lizenznummer\index{Lizenznummer} Ihrer MUTABOR~II-Version lautet
    \underline{ \# \hspace{3cm}}
\index{Seriennummer}
  }
}

\paragraph{Lieferumfang}
\index{Lieferumfang}
Ihnen sollten folgende Dinge bei der Lieferung von MUTABOR~II
vorliegen:

\begin{itemize}
\item Die PROGRAMMDISKETTE mit folgenden Dateien:\\ 
\filename{MUTABOR.PRG, MUTABOR.CFG, README.TXT}, 
einem Ordner \filename{EXAMPLES} mit 
Beispielen und einem Ordner mit einem Public-Domain-Editor
als Geschenk. Die Datei \filename{README.TXT} enthält letzte
 Informationen, die nicht mehr ins Handbuch aufgenommen
werden konnten.
\item Dieses BEDIENUNGSHANDBUCH
\item Ein BEIHEFT mit der Beschreibung der Bedienungs--Oberfläche
\end{itemize}

\paragraph{\copyright 1991, 1992 V.Abel/P.Reiss}
\index{Copyright}
Alle Rechte vorbehalten. Jede auch auszugsweise Vervielfältigung der Dokumentation oder des MUTABOR-II-Softwareprogramms wird strafrechtlich verfolgt. Die Übertragung der Softwareprogramme auf Datenträger jeglicher Art zu einem anderen Zwecke als dem der Datensicherung ist nicht gestattet. Die Rechte an der Dokumentation und die Rechte an dem Softwarepaket
MUTABOR~II liegen bei Volker Abel und Peter Reiss. 
\fi

\addsec{Einschränkung der Gewährleistung}
\index{Garantie} Inhaltliche Änderungen des Handbuchs und der
Softwareprogramme behalten wir uns ohne Ankündigung vor. Es wird keine
Haftung für die Richtigkeit des Inhaltes dieses Handbuchs oder
Schäden, die sich aus dem Gebrauch der Softwareprogramme ergeben,
übernommen. Ebenso können wir die Funktionsfähigkeit der Software
nicht garantieren -- \emph{da sich Fehler, trotz aller Bemühungen, nie
  vollständig vermeiden lassen, sind wir für Hinweise jederzeit
  dankbar.}

\addsec{Warenzeichen}
\index{Warenzeichen}
Innerhalb dieses Handbuchs wird auf Warenzeichen Bezug genommen, die nicht explizit als solche ausgewiesen sind. Aus dem Fehlen einer Kennzeichnung kann also nicht geschlossen werden, dass ein Name frei von Rechten Dritter ist.

\iffalse
\vfill

\begin{center}
\index{Autoren}
\vspace{1cm}
\textbf{Mutabor Soft, Gesellschaft bürgerlichen Rechts}
\vspace{-1cm}
\end{center}

Volker Abel       \hfill   Peter Reiss \\
Taubenweg 14      \hfill   Liebigstr. 9 \\
W-6101 Roßdorf    \hfill   W-6100 Darmstadt \\
Tel: 06154/81888  \hfill   Tel+Fax: 06151/292257 \\


\tableofcontents
\fi
%
%===============================================================
%


\part{Einleitung}
\label{teilEinleit}


\chapter{Das Konzept von \mutabor{}}\label{cha:das-konzept-von-mutabor}

Es gibt eine Fülle verschiedener Konzepte mikrotonaler
Kompositionstechnik. Nahezu jeder Komponist hat ein eigenes
Tonsystem und eigene Intonationsvorschriften. Die Tatsache, dass
bis heute noch keine standardisierte Notation für Mikrotöne von
allen Komponisten und Musikwissenschaftlern akzeptiert wird, lässt
auf die ausgesprochene Vielfalt auf diesem Gebiet schließen.

\index{MUTABOR~II!Planung}
Bei der Planung von \mutabor{} wurde versucht,
möglichst vielen Komponisten, Instrumentalisten und
Musikwissenschaftlern ein Instrument in die Hand zu geben, das
sich durch größtmögliche Flexibilität auszeichnet. Das Konzept,
mit dem man dies realisieren kann, ist die Entwicklung einer
möglichst universellen Programmiersprache zum Programmieren von
Tonsystemen und Stimmungslogiken. Der Anwender kann sich die
Stimmung seines Instrumentes ohne großen Aufwand individuell
programmieren. Dies erspart den Bau vieler verschiedener
Spezialinstrumente, ist kostengünstig und zeitsparend.

\index{Sampler}
Durch die Möglichkeit, mit \mutabor{} hochwertige Sampler
ansteuern zu können, ist eine Klangqualität zu erreichen, die weit
über die Qualität synthetischer Instrumente, die bisher zur
Erzeugung von Mikrotönen gebaut wurden, hinausgeht. Die
handelsübliche Hardware gestattet momentan eine
Intonationsgenauigkeit, \index{Intonationsgenauigkeit}
die (je nach Instrument) immerhin zwischen
0,8 und 2,4 Cent liegt (Ein Cent entspricht einem Hundertstel eines
gleichstufig gestimmten Halbtones).\index{Cent}

\mutabor{} ist einfach zu bedienen und ohne große Spezialkenntnisse
nutzbar. Sie benötigen weder Programmierkenntnisse, noch benötigen Sie
Hardwarekenntnisse. Es wird lediglich vorausgesetzt, dass Ihnen die
grundlegende Bedienung des Computers bekannt ist. Sie sollten also
wissen, wie man den Rechner ein- bzw. ausschaltet. Die "`Maus"' sollte
bekannt sein, ebenso wie das Einlegen einer Diskette in das
Diskettenlaufwerk und das Starten eines Programms. Diese Kenntnisse
können Sie dem entsprechenden Bedienungshandbuch des Computers
entnehmen.

\section{Ausführbare Mikrotöne}\label{sec:ausf-mikr}
Eines der großen Probleme bei der praktischen Arbeit mit
mikrotonaler Musik ist ihre Ausführbarkeit. Da es sich hierbei um
für das "`klassische"' Ohr ungewohnte Intervalle handelt, könnte es\index{Intervalle!ungewohnte}
insbesondere für den praktizierenden Musiker von Nutzen sein, eine
genaue Vorgabe des zu spielenden Stückes zu bekommen. Eine
mikrotonale Komposition ohne Hörbeispiel und ohne Anleitung vom
Komponisten oder einer ähnlich autorisierten Person 
einfach  "`vom Blatt"' zu spielen ist extrem schwierig, 
vielleicht gar unmöglich.

Mit \mutabor{} haben Sie ein Instrument erworben,
welches es Ihnen er\-mög\-licht, auf beliebigen Tonsystemen bei
exakter Intonation zu spielen\footnote{Natürlich im Rahmen der
technischen Möglichkeiten, siehe weiter oben. Das aber ist ein
Problem des angeschlossenen Synthesizers/Samplers und nicht der
Software.}. Dies könnte als Vorgabe für ein natürliches Instrument
dienen, oder als Werkzeug bei der Komposition mikrotonaler Stücke.
Insbesondere für vokale Intonationsübungen zur Aufführung von
Musikstücken z.\,B. in barocken Tonsystemen kann \mutabor{} dem
Interpreten ein wichtiges Hilfsmittel sein.

\section{Mutierende Stimmungen}\label{sec:muti-stimm}

Statische Tonsysteme, also Tonsysteme, deren Stimmung sich
während eines Musikstückes nicht ändert, lassen sich mit gewissen
Einschränkungen auf speziell gestimmten Klavieren realisieren. Ist
das Klavier erst einmal gestimmt, so kann man sehr leicht mit dem
Tonsystem experimentieren.

\index{MUTABOR!Konzept}
Das Konzept vom \mutabor{} geht über solche
statischen Tonsysteme hinaus. Sie können {\em mutierende}, also bewegte,\index{Stimmung!mutierende}
"`sich verändernde"' Stimmungen programmieren und im freien Spiel
anwenden --- in Echtzeit. Dies ist bisher auf keinem (mechanischen)
Instrument möglich. Die hohe Rechengeschwindigkeit der erweiterten
Version 2.1 er\-mög\-licht ein verzögerungsfreies Spielen selbst bei
komplexen Stimmungslogiken.

Mutierende Stimmungen ergaben sich bisher fast ausschließlich als
Nebeneffekte bei bestimmten Intonationen, z.\,B. dem "`Wandern im
tonalen Netz"' bei gewissen Modulationen. Aufgrund ihrer
Komplexität und ihrer schwierigen Ausführbarkeit konnten
mutierende Stimmungen bisher nicht in der kompositorischen Praxis
verwendet werden. MUTABOR~II erschließt --- bei
einfacher Bedienung --- nun erstmals einen Teil des Gebietes der
mutierenden Stimmungen für den experimentierenden Komponisten.

\section{Einsatz dieser Dokumentation}\label{sec:eins-dies-dokum}

\index{Dokumentation}
Im Lieferumfang von \mutabor{} ist dieses Handbuch
enthalten. Es gliedert sich in zwei wesentliche Teile, das "`Be\-die\-nungs\-hand\-buch"' und das
"`Referenzhandbuch"'.

Teil \ref{teilEinleit} des
Be\-die\-nungs\-hand\-buchs\index{Handbücher} beschreibt die Arbeit
mit \mutabor{} und gibt eine Anleitung zu Installation und Bedienung
der Software. In Teil \ref{teilSprache} erfolgt eine Einführung in die
Programmiersprache von \mutabor{}, so dass Sie nach dem Durcharbeiten
dieses Abschnittes imstande sein sollten, Ihre eigenen
Stimmungslogiken zu entwickeln. Wenn Sie mit diesen grundlegenden
Programmiertechniken ver\-traut sind, erfahren Sie
wei\-ter\-füh\-ren\-de Konstruktionen im Teil \ref{teilFortgesch}
dieses Handbuches.  Das Bedienungs\-hand\-buch dient als Lehrgang für
den Umgang mit \mutabor{} und die Programmierung von Stimmungslogiken.
Sie lernen hier die Programmierung von statischen Tonsystemen und von
einfachen mutierenden Stimmungen.

Sie werden beim Durcharbeiten des Bedienungshandbuches oft auf
Wiederholungen des Stoffes stoßen. \index{MUTABOR!Didaktik}Wir
halten es aus didaktischen Gründen für sinnvoll, einen Lehrgang für
das Programmieren mit \mutabor{} nicht so weit in der Information zu
reduzieren, dass einmal genannte Fakten als für immer präsent
vorausgesetzt werden können.  Somit finden Sie an mehreren Stellen im
Handbuch z.\,B. eine Beschreibung, wie man Intervalle deklariert. Falls
Sie auf die wiederholte Beschreibungen eines Sachverhaltes treffen,
der Ihnen bereits geläufig ist, so lesen Sie einfach darüber hinweg.
Der eindeutige Vorteil dieser Methode liegt darin, dass Anwender, die
bestimmte Strukturen noch nicht hundertprozentig verstanden haben,
diese an verschiedenen Stellen des Handbuches von verschiedenen
Standpunkten aus beschrieben bekommen, so dass wir annehmen, dass {\em
  wirklich jeder nach dem Lesen des Handbuches in der Lage ist,
  eigenständig Stimmungslogiken zu programmieren.}

Für die Programmierpraxis und als schnelles Nachschlagewerk dient das
"`Referenzhandbuch"', in dem alle Befehle der
Programmiersprache\index{MUTABOR!Referenzhandbuch} von \mutabor{}
zusammengefasst und \emph{präzise definiert} sind.  Eine Beschreibung
der Benutzeroberfläche finden Sie in dem Beiheft
"`\href{\makefilename{bedienung}}{Bedienungsanleitung der
  Oberfläche}"', da das \mutabor{} --
System mittlerweile auf verschiedenen Rechnern implementiert ist, die
sich nicht im Kern, sondern nur in der Benutzeroberfläche
unterscheiden. Das Benutzerhandbuch ist für alle \mutabor{} --
Implementationen gültig, Abweichungen der einzelnen Versionen, die
sich wie gesagt nicht in der Leistungsfähigkeit, sondern nur in der
Bedienung unterscheiden, sind in dem Beiheft, welches speziell die
Bedienung auf Ihrem Rechner beschreibt, beschrieben. Die
Benutzeroberfläche von \mutabor{} ist sehr einfach und übersichtlich,
so dass Sie, falls Sie bereits ein wenig Erfahrung mit anderen
Programmen haben, auch ohne das Beiheft zu lesen, sofort mit
\mutabor{} arbeiten können.

\section{Womit anfangen?}\label{sec:womit-anfangen-}

\iffalse
\index{Originaldisketten} Als allererstes sollten Sie eine Kopie der
Originaldiskette an\-fer\-ti\-gen\footnote{Das Kopieren von Disketten
  ist im Handbuch des Computers beschrieben.} und die Originaldiskette
danach an einem sicheren Platz verwahren. Wir machen Sie nochmals
darauf aufmerksam, dass die Weitergabe der Software und/oder der
Dokumentation an Dritte von uns strafrechtlich verfolgt wird.

\emph{Arbeiten Sie nur mit Kopien.
Sie riskieren sonst den Verlust Ihrer ganzen Daten.}
\fi

Bevor Sie mit der Installation beginnen, sollten Sie sich das
Kapitel "`Installation"' in aller Ruhe durchlesen. Für die
Installation werden Sie kaum mehr als fünf Minuten benötigen. Wenn
die Installation abgeschlossen ist, können Sie im Grunde sofort
mit \mutabor{} arbeiten.

Trotzdem ist es wichtig, dass Sie sich mit der Bedienung von
\mutabor{} vertraut machen. Wenn Sie Ihre eigenen
Stimmungslogiken entwickeln wollen, ist es unabdingbar, dass Sie
das Kapitel über das "`Programmieren von Stimmungslogiken"'
sorgfältig studieren.


\chapter{Die Installation der Software}\label{cha:die-installation-der}

\index{Installation}

\iffalse
\section{Installation auf Diskette}\label{sec:inst-auf-disk}

Zur Installation auf Diskette fertigen Sie einfach eine Kopie der
Originaldiskette an. Zusätzlich müssen Sie einen Text-Editor auf die
Arbeitsdiskette kopieren, falls Sie eigene Logikprogramme schreiben
wollen. Sie können gegebenenfalls auch den von uns als Geschenk
beigefügten Public-Domain-Editor verwenden. Falls Sie nur fertige
Logikprogramme laufen lassen wollen, ist der Text-Editor nicht nötig.
Sie können aber auch jeden anderen Text-Editor verwenden. Achten Sie
dabei darauf, dass der Editor \textbf{keine}
Steuerzeichen\index{Steuerzeichen!keine} oder Formatierungen in die
Programmdatei schreibt, der von Ihnen erstellte Quelltext also aus
reinen ASCII-Zeichen besteht.

\section{Installation auf Festplatte}\label{sec:inst-auf-festpl}

Zur Installation auf Festplatte legen Sie einen eigenen Ordner
an, der z.\,B. \filename{MUTABOR.II} heißen kann. Kopieren Sie den
Inhalt der Originaldiskette in diesen Ordner. Ferner sollte
auf Ihrem Computer ein Text-Editor vorhanden sein. Für den
Text-Editor gilt sinngemäß dasselbe wie bei der Installation
auf Diskette.
\fi

Mutabor ist kostenlos im Internet verfügbar. Sie können das
Installationsprogramm auf unserer Homepage unter
\begin{center}
\cdoronline{\texttt{\href{../../resources/lch/mutabor_web.lch}{http://www.math.tu-dresden.de/\textasciitilde mutabor/}}}
{\url{http://www.math.tu-dresden.de/~mutabor/}}.
\end{center}
herunterladen.

\section{Installation auf Festplatte}\label{sec:inst-auf-festpl-1}

Sie installieren \mutabor{} auf Ihrer Festplatte, indem sie einfach das
Installationsprogramm ausführen. Mit wenigen Klicks sind sie dann
haben sie eine betriebsbereite Version auf Ihrer Festplatte. 

\section{Installation auf anderen Medien}\label{sec:inst-auf-ander}

\mutabor{} ist anspruchslos. Es kann von so ziemlich allen Medien
ausgeführt werden, auf denen es Platz findet. Sie können \mutabor{}
mit dem Installationsprogramm auf dem Medium installieren. Möchten Sie
ihre Lieblingseinstellungen gleich mit auf dem Medium platzieren,
können sie auch gleich den gesamten Ordner kopieren, in dem \mutabor{}
installiert ist. Sie können dann mit dem Medium zu einem anderen
Computer wechseln und mit Mutabor arbeiten, ohne es erst installieren
zu müssen.

\section{Anschluss des Synthesizers}\label{sec:anschl-des-synth}
\label{ANSCHLUSS}
\index{Anschluss des Synth.}
Für den Betrieb von \mutabor{} benötigen Sie noch
eine Klangerzeugungseinheit und eine Klaviatur. Sie können zu
diesem Zweck jedes MIDI-Keyboard und jeden handelsüblichen
Synthesizer oder Sampler benutzen, sofern dafür ein Treiber
existiert\footnote{Siehe Abschnitt "`Arbeitsweise der Synthesizer-
Treiber"'}.

Es gibt nun zwei mögliche Konfigurationen: entweder sind Klaviatur
und Klangerzeugung in einem Gerät untergebracht, oder Klaviatur
(Masterkeyboard) und Klangerzeugung sind getrennt.
\index{MIDI!Local Off}
Wenn Klaviatur und Klang\-er\-zeu\-gung in einem Gerät untergebracht
sind, so ist es wichtig, dass Sie am Synthesizer/Sampler die
Betriebsart "`LOCAL OFF"' einstellen, da sonst bei jedem
Tastenanschlag zwei Töne erklingen (einmal vom Synthesizer und einmal über MIDI
von \mutabor{}). Schlagen Sie hierzu im Handbuch des Gerätes nach
(Stichwort MIDI-Funktionen).

\subsection{Synthesizer mit Klaviatur}\label{sec:synth-mit-klav}

Verbinden Sie nun mit einem MIDI-Kabel die MIDI-OUT-Buchse des
Synthesizers mit der MIDI-IN-Buchse am Computer bzw. MIDI-Interface.
Dann verbinden Sie die MIDI-IN-Buchse des Synthesizers mit der MIDI-
OUT-Buchse am Rechner (bzw. Interface).  Damit \mutabor{} die
MIDI-Informationen des Keyboards verstehen kann, ist es wichtig, dass
Sie als MIDI- \index{MIDI!Sendekanal} Sendekanal den Kanal 1 an Ihrem
Masterkeyboard einstellen.%
\footnote{Genauere Informationen über die MIDI-Spezifikation entnehmen
  Sie dem Abschnitt "`Arbeitsweise der Synthesizer-Treiber"'.  Die
  Voreinstellung des MIDI-Kanals 1 kann im Logikprogramm geändert
  werden. Näheres beschreibt
  \vollref{MIDIKANAL}[im \href{\makefilename{referenz}}{Referenzhandbuch}]}

\subsection{Separate Klaviatur}\label{sec:separate-klaviatur}

Verbinden Sie die MIDI-OUT-Buchse der Klaviatur (Masterkeyboard)
mit der MIDI-IN-Buchse des Computers und die MIDI-IN-Buchse
des Synthesizers mit der MIDI-OUT-Buchse des Computers. Achten Sie
auch hier darauf, dass der Master auf dem MIDI-Kanal 1 sendet!


\chapter{\mutabor{} starten}\label{cha:mutabor-starten}


\index{MUTABOR!Programm starten} Wenn Sie nun mit \mutabor{}
arbeiten wollen, so starten Sie einfach das Programm \filename{MUTABOR},
welches sich je nach Konfiguration auf der Festplatte im
\mutabor{}-Ordner oder auf Ihrer Programmdiskette
befindet.\footnote{Siehe Abschnitt "`Installation"'} Da \mutabor{} auf
verschiedenen Computern mit unterschiedlicher Bedienungsoberfläche
läuft, befindet sich eine genaue Spezifikation der Benutzeroberfläche
in dem kleinen Beiheft "`Die Benutzeroberfläche"'

Um einen einwandfreien Betrieb von \mutabor{} zu ermöglichen, ist es
sehr wichtig, dass alle Geräte richtig miteinander verbunden sind, und
dass der angeschlossene Synthesizer bzw.\ Sampler für den Betrieb mit
\mutabor{} eingestellt ist.

Da \mutabor{} nur über einen "`Trick"' \index{Trick}
Mikrotöne erzeugen kann, ist es unabdingbar, dass der Synthesizer dafür
korrekt eingestellt ist. Leider können wir an dieser Stelle keine
allgemeingültige Abfolge von Tasten angeben, die Sie auf Ihrem Gerät
drücken müssen, um es für Mikrotöne vorzubereiten. Wir können also nur
eine Beschreibung geben, \emph{was} zu tun ist, nicht aber, \emph{wie}
es konkret auf Ihrem Gerät einzustellen ist. 

\index{Synthesizer!Arbeitsweise}

Leider haben es die Hersteller handelsüblicher Synthesizer versäumt,
in das MIDI-Protokoll Befehle zur direkten Ansteuerung von Mikrotönen
einzubauen.\footnote{Einzige Ausnahme bildet hier der
Synthesizer FB-01 von Yamaha. Hier gibt es eine System-Exklusiv-
Meldung der Gestalt "`Schalte die Note c mit der Feinstimmung +42
Feinstimmeinheiten an"', wobei eine Feinstimmeinheit einem
Hundertachtundzwanzigstel Halbton entspricht. Leider lässt die
Klangqualität dieses Gerätes nach heutigen Maßstäben zu wünschen übrig
und außerdem wird es nicht mehr produziert. Wahrlich schade, dass
Yamaha diese für unsere Zwecke so sinnvolle Option aus ihrem Programm
gestrichen hat. \iffalse
Trotzdem haben wir einen speziellen Treiber für dieses
Gerät geschrieben, denn es erreicht von allen uns bekannten
Synthesizern noch immer die beste Intonationsgenauigkeit.
\fi
Seit Version 3 können wir diesen Treiber leider auch nicht mehr unterstützen.} Wir müssen
daher einen kleinen Trick anwenden, um dem Synthesizer/""Sampler
trotzdem Mikrotöne entlocken zu können.

\index{Mikrotöne!Erzeugung}\index{Pitch-Bender}
Die meisten Synthesizer haben einen "`Pitch-Bender"'. Das ist ein
Drehrad, mit dessen Hilfe man die Frequenz der Töne während des Spiels
anheben oder absenken kann. Und zwar je nach Einstellung am
Synthesizer zwischen einem und zwölf Halbtönen. Diese Anhebung erfolgt
in Schritten von einem Vierundsechzigstel des Maximalintervalls. Wenn
man also die Weite des Pitch-Bendings auf einen Halbton einstellt, so
kann man damit eine Auflösung von $\frac{1}{64}$ Halbton erreichen, was
theoretisch 1,6 Cent entspricht.

Für ein solches Pitch-Bending existiert glücklicherweise auch ein
MIDI-Code, mit dem man diese Funktion über MIDI ansteuern kann. Leider
verändert eine Pitch-Bend-Nachricht nicht nur einen gezielten Ton,
sondern alle momentan liegenden Töne. Auf diese Weise ist es also noch
nicht möglich, das c' um 4 Cent zu erhöhen und gleichzeitig das f'' um
35 Cent zu erniedrigen.

Doch zu guter Letzt ist auch dieses Problem in den Griff zu bekommen.
Wenn man sechzehn Synthesizer
hintereinander schaltet, so dass jeder Synthesizer nur \emph{einen} Ton
spielt, so könnte man jedem Gerät eine \emph{eigene} 
Pitch-Bend-Information geben --- 
und das Problem wäre gelöst. Nun hat zwar nicht
jeder sechzehn gleiche Synthesizer, aber in diesem Punkt kommt uns die
moderne Technik zur Hilfe: die meisten Synthesizer lassen sich in
einer "`Multi-Mode"'-Betriebsart ansteuern. D.\,h. der Synthesizer tut
so, als ob er aus sechzehn eigenständigen Synthesizern gleichen Typs
bestehen würde. Man kann nun jeden der sechzehn 'Synthesizer im
Synthesizer' gesondert ansprechen, und somit erreichen, dass bis zu
sechzehn gleichzeitig liegende Töne jeweils eine \emph{eigene}
Feinstimminformation bekommen können.

Dabei sind jedoch folgende Dinge von großer Wichtigkeit:
\begin{itemize}\index{Kanal!MIDI}
\item Auf jedem MIDI-Kanal muss dasselbe Instrument (Klangfarbe)
  eingestellt werden, damit mehrere gleichzeitig liegende Töne auch
  mit derselben Klangfarbe erklingen.
\item Auf jedem Kanal muss die Reichweite des Pitch-Benders auf 1
  Halbton eingestellt werden, damit die Feinstimmungen korrekt
  durchgeführt werden, denn eine Feinstimmungseinheit entspricht
  $\frac{1}{64}$ Weite. Diese "`Pitch Bend Range"' ist bei den meisten
  Synthesizern ein Parameter der Klangfarbe und nicht des Multis.
\end{itemize}

Diese Einstellungen erfordern gewisse Kenntnisse über die Bedienung
des benutzten Synthesizers, über die dieses Handbuch aufgrund der
Mannigfaltigkeit des Angebotes von Synthesizern und Samplern leider
keine genaue Beschreibung liefern kann. Glücklicherweise sind nur die
beiden oben genannten Einstellungen nötig, und wenn Sie erst einmal
wissen, wie Sie Ihren Synthesizer richtig einzustellen haben, dürfte
dies kein Problem mehr darstellen. Falls Sie dennoch mit der
Einstellung Ihres Synthesizers Schwierigkeiten haben, so stehen wir
Ihnen gerne mit Rat und Tat beiseite, denn an solchen technischen
Kleinigkeiten sollte der Einsatz von \mutabor{} nicht scheitern.


Nachdem Sie die Software installiert haben und alle Geräte richtig
miteinander verbunden und eingestellt sind, 
können Sie mit \mutabor{}
arbeiten. 

Außerdem befindet sich auf der Originaldiskette ein Ordner,
in dem sich die Quelltexte diverser
Stimmungslogiken befinden, unter anderem auch viele in diesem Handbuch
aufgeführte Beispiele.
Diese Beispiele können direkt benutzt werden, oder als Vorlage
dienen, um zu experimentieren. Man kann diese Beispiele mit dem
Text-Editor\index{Beispiele!eigene} verändern und sich danach die geänderten Beispiele
anhören.

\chapter{Die Entwicklung von Logikprogrammen}\label{cha:die-entwicklung-von}

\section{Stimmungslogiken programmieren}\label{sec:stimm-progr}

Irgendwann wollen Sie sicherlich eigene Stimmungslogiken entwickeln
oder vielleicht die Demo-Stimmungen modifizieren. Jede Stimmungslogik
wurde in einer speziell entworfenen Programmiersprache für
Stimmungslogiken programmiert. Man formuliert in
der\index{Programmiersprache} Programmiersprache eine Folge von
Anweisungen, die den Computer im Laufzeitmodul veranlassen, den
Synthesizer richtig zu stimmen. Man schreibt also ein \emph{Programm}.
Wenn der Programmtext einmal auf dem Papier entworfen worden ist, muss
er in den Computer eingegeben und auf einem Datenträger wie Diskette
oder Festplatte gespeichert werden. Dies geschieht mit Hilfe eines
\emph{Text"=Editors}. Genauso wie Sie mit einer Textverarbeitung Briefe
schreiben können, werden Sie mit dem Texteditor den Programmtext für
Ihre Logiken eingeben und verändern. \mutabor{} verfügt über einen
programminternen Texteditor, der den Bedürfnissen des Programmierens
genügt.

Wenn dies geschehen ist und Sie Ihre Logik gerne im freien Spiel
anwenden möchten, ist es notwendig, dass der Programmtext vom
sogenannten "`Compiler"' in interne Tabellen übersetzt wird. Das
Logikprogramm, welches Sie mit dem Texteditor eingegeben und
abgespeichert haben, besteht nur aus einer Ansammlung von für den
Computer bedeutungslosen Zeichen. Um diesen Text "`verstehen"' zu
können, und insbesondere um Ihre Stimmungslogik ausführen zu können,
muss dieser "`Quelltext"' übersetzt werden.  Diese Aufgabe übernimmt
der \emph{Compiler}. Während des Übersetzens überprüft der Compiler
automatisch Ihr Programm auf seine syntaktische Richtigkeit. Falls der
Quelltext fehlerhaft ist, wird der Übersetzungsvorgang abgebrochen und
auf dem Bildschirm erscheint eine Fehlermeldung, die Ihnen anzeigt, wo
der Fehler gefunden wurde, und welcher Natur er ist. Sie sollten
daraufhin wieder den Editor aufrufen, den Fehler verbessern und einen
erneuten Übersetzungsvorgang\index{Fehler!was nun ?}  starten.

Wenn der Compiler keinen Fehler meldet, können Sie nun direkt mit der Stimmungslogik
experimentieren. Und
hier noch einmal die drei Schritte der Programmentwicklung auf
einen Blick:\index{Programmentwicklung}
\begin{itemize}
\item Entwerfen Sie auf dem Papier den Programmtext Ihrer
  Stimmungslogik.
\item Geben Sie ihn im Texteditor ein und speichern Sie die Datei auf
  Diskette oder Festplatte.
\item Compilieren Sie das Logikprogramm in \mutabor{} und aktivieren Sie
  es.\footnote{Der Ladevorgang selbst wird mit Hilfe des Compilers
    durchgeführt.}
\end{itemize}

\section{Die Schritte einer Pro\-gramm\-ent\-wick\-lung}\label{sec:die-schritte-einer}

An einem Beispiel sollen Sie nun lernen, wie Sie ein Logikprogramm
entwickeln und zum Laufen bringen können.  Nehmen wir an, Sie hätten
folgendes Logikprogramm geschrieben und wollten mit dieser Logik
experimentieren\footnote{Viele im Handbuch abgedruckte Logikprogramme
  finden Sie auch auf der Installationsdiskette oder im Internet. Sie
  brauchen also die wichtigsten Beispiellogiken aus dem Handbuch nicht
  mühsam `abzutippen'. In einer Fußnote wird jeweils der Dateiname
  angegeben; dieses Beispiel finden Sie unter dem Dateinamen
  \texttt{drittel.mut}}:
\label{DRITTEL}\index{Dritteltöne}
\begin{verbatim}
INTERVALL
    Drittelton = 18 Wurzel 2
TON
    a = 440
TONSYSTEM
    Drittel = 69 [a] Drittelton
LOGIK
    Drittel Taste D = Drittel [ ]
\end{verbatim}

(Dies ist ein einfaches Beispielprogramm; den Umgang mit der
Programmiersprache von \mutabor{} lernen Sie in Teil \ref{teilSprache} und \ref{teilFortgesch} dieses Handbuches.)

\subsection{Datei auswählen}\label{sec:datei-auswahlen}

Jedes Logikprogramm, das Sie auf \mutabor{} programmieren, wird unter
einem beliebigen Dateinamen (z.\,B. mit der Erweiterung \filename{.mut} )
auf Diskette oder Festplatte gespeichert\footnote{Ihr Computer
  verwaltet Dateien unter einem Dateinamen, welcher aus zwei Teilen
  besteht.  Beide Teile sind durch einen Punkt voneinander getrennt,
  z.\,B. \filename{lernen.doc}.  Den ersten Teil nennen wir "`Namen"', den
  zweiten Teil "`Erweiterung"'. Der Name darf bei einigen Rechnern
  maximal acht Zeichen enthalten, muss mit einem Buchstaben beginnen
  und darf danach auch Ziffern oder das "`Tiefstrich"'-Zeichen (\_)
  enthalten. Die Erweiterung darf bei einigen Systemen maximal drei
  Zeichen umfassen.}. Im Laufe Ihrer Arbeit mit \mutabor{} werden Sie
also verschiedene Logikprogramme geschrieben haben, die sich dann auf
einem Speichermedium (Diskette oder Festplatte) befinden.

\index{MUTABOR!Compiler}
Bevor Sie nun mit Ihrem Logikprogramm im freien Spiel experimentieren
können, muss der Quelltext auf seine grammatikalische Richtigkeit
überprüft und in die computerinternen Stimmungstabellen umgewandelt werden. Diese
Aufgabe übernimmt der Compiler. Sie müssen lediglich den Menüpunkt
"`Logik laden"' anklicken und einen kleinen Moment 
warten\footnote{Bei größeren Programmen oder einer langsamen 
Festplatte oder Diskettenbetrieb kann dieser "`kleine Moment"' 
manchmal zu einem "`großen Moment"' werden \dots{}}.

Wenn Ihr Programm fehlerfrei übersetzt werden konnte, so meldet
dies \mutabor{} und man kann die Logik starten. 
Wenn Ihr Quelltext an einer Stelle syntaktisch falsch ist, macht
Sie der Compiler auf den Fehler aufmerksam und unterbricht seine
Arbeit. Sie können nun erneut den Texteditor aufrufen, den Fehler
verbessern und den Compiler zu einem neuen Übersetzungsversuch
anklicken. Die häufigsten Programmierfehler und deren Beseitigung
finden Sie im Anhang "`Fehlerursachen"'. Nebenbei sei bemerkt,
dass der Compiler nur den ersten gefundenen Fehler meldet und dann
die Übersetzung abbricht. Man muss diesen Fehler korrigieren und
nochmal übersetzen, bis keine Fehler mehr gemeldet werden.
Erst wenn ein Logikprogramm ohne Fehlermeldung übersetzt werden
konnte, kann man es ausprobieren.

\index{Logik aufrufen} Nachdem Ihr Logikprogramm erfolgreich übersetzt
und somit geladen wurde, steht der Ausführung nichts mehr im Wege.
Auf dem Bildschirm erscheint das Laufzeitmodul und Sie können Ihre
Stimmungslogiken im freien Spiel anwenden.


%
%===============================================================
%


\part[Beschreibung der Programmiersprache]{Beschreibung der
Programmiersprache von \mutabor{}}\label{part:beschr-der-progr}
\label{teilSprache}

\index{Einführung} Dieser Teil des Handbuches gibt Ihnen eine
Einführung in das Programmieren von Stimmungslogiken. An vielen
Beispielen werden Sie die Programmierfunktionen von \mutabor{}
kennen lernen.  Solche Beispiele sind immer in einer
Schreibmaschinenschrift gedruckt, z.\,B. :
\begin{verbatim}
INTERVALL
    Quinte = 3:2
\end{verbatim}
Diese Schrift bedeutet, dass es sich um Programme oder
Programmfragmente handelt, die Sie genau so, wie sie im Handbuch
stehen, im Editor eingeben können.

Die Programmiersprache von \mutabor{} hat drei verschiedene
Grundelemente: \emph{Tonsysteme, Umstimmungen\/} und \emph{Logiken\/}.
Diese drei Elemente seien im folgenden beschrieben.  Beachten Sie, dass
wir für eine Programmiersprache präzise definieren müssen, was wir
unter einem Tonsystem (Umstimmung, Logik, \dots) verstehen wollen. In
der Musiktheorie wird der Begriff des Tonsystems\index{Tonsystem}
keineswegs einheitlich verstanden. Hier jedoch ist es nötig, den
Begriff des Tonsystems mathematisch exakt zu definieren. Diese
Definition ist so geartet, dass sie dem intuitiven Verständnis
möglichst nahe kommt. Es soll also mit dieser Definition für die
MU\-TA\-BOR-Pro\-gram\-mier\-zwecke eine Art "`\mutabor{}"=Tonsystem"'
geschaffen\index{Tonsystem!MUTABOR-eigenes} werden, welches den hier
angegebenen Regeln genügt und als eindeutig definierter Begriff
benutzbar ist. Des intuitiven Verständnisses wegen behalten wir im
folgenden den Begriff "`Tonsystem"' bei und meinen damit immer das
hier definierte "`\mutabor{}"-Tonsystem"'. Diese Präzisierung gilt
sinngemäß auch für alle anderen Elemente der Programmiersprache
(Umstimmung, Logik, \dots).  \textbf{Wir wollen diese Begriffe keineswegs
  generell für die Musiktheorie so definiert wissen, sondern lediglich
  innerhalb dieser formalen Logiksprache eine Basis schaffen, mit der
  eine eindeutige Stimmungslogik programmierbar wird.}

\paragraph{Kommentare} \index{Kommentare} können an beliebigen Stellen
in einem Logikprogramm
geschrieben werden und dienen dem menschlichen Leser zur 
Dokumentation des Programms. Kommentare stehen in 
doppelten Anführungszeichen 
\begin{verbatim} 
" Dies ist ein Kommentar " 
\end{verbatim}
können sich über mehrere Zeilen erstrecken und haben keinerlei Einfluss
auf die sonstige Funktionsweise des Logikprogramms.

\chapter{Tonsysteme}\label{cha:tonsysteme}
Ein Tonsystem ist eine statische, d.\,h. unveränderliche Stimmung.
Mit einem bestimmten System werden alle Tasten der Klaviatur wie
bei einer Klavierstimmung gestimmt, d.\,h. jeder Taste der Klaviatur
wird eine bestimmte Frequenz zugewiesen, einschließlich der
Möglichkeit, dass eine Taste keinen Ton hervorbringt.
\index{Tonsystem}

\section{Die vier Parameter eines Tonsystems}\label{sec:die-vier-parameter}
\index{Grundparameter}

Jedes Tonsystem, das mit \mutabor{} programmiert
werden kann, lässt sich durch genau vier Parameter exakt
beschreiben.
\subsection{Die Töne der Fundamentaltonleiter}\label{sec:die-tone-der}
\index{Fundamentaltonleiter!Töne}

Mit Fundamentaltonleiter (FT) bezeichnen wir einen bestimmten
zusammenhängenden Ausschnittsbereich der Klaviatur. Dieser Bereich
kann beliebig viele Tasten umfassen und an jeder beliebigen Stelle
der Klaviatur beginnen; zum Beispiel die Tasten vom
eingestrichenen c' bis zum eingestrichenen h', oder nur eine
Taste, usw.\dots{} Jeder Taste dieses Bereiches wird eine Frequenz
zugewiesen, die beim Anschlagen derselben erklingen soll
(einschließlich der Möglichkeit, dass eine Taste keinen Ton
erzeugt, also stumm geschaltet ist).

\subsection{Die Breite der FT}\label{sec:die-breite-der}
\index{Fundamentaltonleiter!Breite}

Die Anzahl an Tasten, welche die Fundamentaltonleiter  darstellen,
nennen wir Breite der FT.

\subsection{Die Verankerungstaste}\label{sec:die-verank}
\index{Fundamentaltonleiter!Anker}
\index{Verankerungstaste}
Die Taste der Klaviatur, auf der die FT beginnt, nennen wir
Verankerungstaste (Es ist natürlich auch möglich, dass die
Verankerungstaste keinen Ton erzeugt). Der MIDI-Standard hat alle
Tasten der Klaviatur nummeriert. Das eingestrichene c' hat die
Nummer 60, das eingestrichene a' die Nummer 69 (a' liegt 9
Tasten bzw. Halbtöne über c').

\subsection{Das Periodenintervall}\label{sec:das-peri}
\index{Fundamentaltonleiter!Periode}

Bisher haben wir nur einen kleinen Ausschnitt der Klaviatur mit Tönen,
d.\,h. Frequenzen belegt. Wir benötigen also noch eine Art
Reproduktionsvorschrift, wie man aus der FT die komplette
Frequenzbelegung der Klaviatur\index{Klaviatur!Frequenzbelegung}
berechnen kann. Dies geschieht mit einem einzigen Parameter, dem
"`Periodenintervall"'. Gewöhnlicherweise soll die Struktur der
Fundamentaltonleiter \emph{blockweise wiederholt} werden, und zwar um
ein bestimmtes Intervall verschoben. Dieses Intervall nennen wir das
"`Periodenintervall"'. \index{Periodenintervall} Somit haben wir nun
alle vier Parameter besprochen, die ein Tonsystem definieren. Diese
sehr einfache Definition des Begriffes "`Tonsystem"', die zunächst aus
einer Verallgemeinerung des Begriffes der Tonleiter entstanden ist,
bietet dem Anwender bereits eine unglaubliche Vielfalt an
Möglichkeiten, das Instrument zu stimmen. Selbstverständlich gibt es
darüber hinaus noch statische mikrotonale Strukturen, die sich nicht in
dem Schema Verankerungstaste -- Töne -- Periodenintervall einordnen
lassen, solche Systeme lassen sich aber behelfsweise durch ein
Tonsystem mit der Breite der gesamten Klaviatur programmieren, so dass
wirklich jede Taste eine individuelle Frequenz zugewiesen bekommt.
Glücklicherweise basieren die meisten herkömmlichen
Tonsystemstrukturen auf unserem Prinzip der "`Fundamentaltonleiter"'
und sind somit sehr leicht zu realisieren. Die folgende Grafik soll
noch einmal den Aufbau eines Tonsystems aus den vier Parametern {\it
  Verankerungstaste, Töne, Breite \/} und \emph{Periodenintervall}
veranschaulichen:

\begin{center}
\input{ft.pic}
\index{Fundamentaltonleiter!Abbildung}
\end{center}

\section{Eine mathematische Betrachtung}\label{sec:eine-math-betr}
Die Definition dieses Tonsystems lässt sich mathematisch exakt
formulieren:
\index{Berechnungsformel}
\label{MATHE}

\index{Fundamentaltonleiter!Berechnungsformel}
Es seien $\alpha$ die Verankerungstaste des Tonsystems (als MIDI-
Nummer), $\delta$ die Breite und $\psi$ das Periodenintervall.
Außerdem sei eine Funktion $\xi ( \tau )$ mit $ 0 \le \tau \le (
\delta - 1 )$ definiert, die als Ergebnis die Frequenz des $\tau$-ten Tones der Fundamentaltonleiter liefert.
Durch diese vier Parameter wird das Tonsystem vollständig
beschrieben. Wenn nun die (MIDI-)Taste $t$ gedrückt wird, so muss der
Synthesizer eine Frequenz von $y$ Hertz spielen, wobei sich $y$ als
Funktion von $\alpha$,$\delta$,$\psi$,$\xi$ und $t$ wie folgt
schreiben lässt:
\[ y= \xi ((t-\alpha) \bmod \delta ) \cdot \psi ^ {\;(t-\alpha )
\mbox{\rm div}\;\; \delta } \]
$ (t-\alpha ) \/{\rm div}\/ \delta $ entspricht dem Abstand der Taste
von der FT\footnote{$mod$ bedeutet hier den Rest bei der Teilung, z.\,B.
24 $mod$ 7 = 3; $div$ bedeutet ganzzahliger Quotient der 
Teilung, z.\,B. 14 $div$ 5 =
2}, gemessen in Einheiten der FT-Breite und
$(t-\alpha) \bmod \delta $ steht für die Taste der FT, die der
angeschlagenen entspricht, also eine Art Tonigkeit.
Anschaulicher lässt sich die Formel schreiben als:
\[ y= \xi ( {\rm Bezugston} ) \cdot \psi ^ {\/{\rm Abstand\;zur\;FT}} \]

Die intervallischen Zusammenhänge zwischen den Tönen der
\index{Intervallstruktur}Fundamentaltonleiter nennen wir \emph{Intervallstruktur}.


\section{Hilfsmittel: Intervalle und Töne}\label{sec:hilfsm-interv-und}
Bisher musste die Erstellung der Grundparameter eines Tonsystems
müh\-sam mit dem Taschenrechner geschehen. Die Frequenzen der
Fundamentaltöne mussten berechnet werden; alle Werte als Kommazahlen
eingegeben werden, \ldots um Ihnen diese mühsame Arbeit zu ersparen,
und um die Struktur eines programmierten Tonsystems deutlich erkennbar
zu machen, stellt Ihnen die Programmiersprache von \mutabor{} zwei
wichtige Hilfsmittel zur Verfügung:

\subsection{Intervalldeklarationen}\label{sec:interv}
\index{Intervalle}

In einem Tonsystem stehen die Töne der Fundamentaltonleiter (FT)
in einer bestimmten Intervallbeziehung zueinander. Die Aussage "`G
liegt um eine Quinte höher als C"' bedeutet, dass man die Frequenz
von "`C"' mit dem Faktor "`Quinte"' multiplizieren muss, um die
Frequenz von "`G"' zu erhalten. Da solche Intervalle für die
Definition von Tonsystemen von so grundlegender Bedeutung sind,
können Sie sich in einem \mutabor{}-Programm Ihre
Intervalle selbst definieren. Ein paar Beispiele:
\begin{verbatim}
INTERVALL
    Quinte     = 3 : 2
    Terz       = 5 : 4
    Halbton    = 12 Wurzel 2
    Oktave     = 2 : 1
    Drittelton = 18 Wurzel 2
    Cent       = 1200 Wurzel 2
    Syn_Komma  = 4 Quint - 2 Oktave - Terz
  
\end{verbatim}
\index{Intervalle!zulässige} Zulässige Zuweisungen sind: Proportionen
(z.\,B. \sourcecode{32:19}, aber auch \sourcecode{3.14159:4}) und Wurzelausdrücke (
z.\,B. \sourcecode{9 Wurzel 1.34}\footnote{9 Wurzel 1.34 ist zu lesen als 9te
  Wurzel aus 1.34 ($\sqrt[9]{1.34}$)und nicht 1.34te Wurzel aus 9}).
Komplexe Ausdrücke, also Intervalldeklarationen, die sich auf andere
Intervalle beziehen (siehe letzte Deklaration im Beispiel), sind
ebenfalls erlaubt, genaueres darüber lesen Sie im Anhang
\ref{anhang:programmieren}.

\subsection{Tondeklarationen}\label{sec:tondeklarationen}
\index{Töne}

Nachdem Sie nun alle Intervalle, auf die Ihr Tonsystem aufbaut,
mit einem Namen und einem Wert versehen haben, können Sie die
Zusammenhänge zwischen den Tönen bestimmen. Nehmen wir das
Beispiel von oben: "`G liegt um eine Quinte höher als C"'. Das würde
man ganz einfach so formulieren:
\begin{verbatim}
TON
    G = C + Quinte
\end{verbatim}
Bei den Tondeklarationen werden also die \emph{intervallischen
  Zusammenhänge} zwischen den Tönen definiert. Die Vorgehensweise ist
analog der des Klavierstimmens.\index{Klavierstimmen} Man beginnt das
Klavierstimmen normalerweise mit der Festlegung des Kammertones. Somit
haben Sie auch bei \mutabor{} die Möglichkeit, einem\footnote{oder
  auch mehreren verschiedenen Tönen} Ton einen festen Frequenzwert
zuzuweisen, z.\,B.
\begin{verbatim}
TON
    a = 440
\end{verbatim}
Hier wird dem Ton \sourcecode{a} eine Frequenz von \sourcecode{440} Hertz zugewiesen. Solche
"`absoluten"' Zuweisungen sind insofern von elementarer Bedeutung,
als innerhalb der Tondeklarationen immer \emph{mindestens ein Ton} eine\index{Absolutfrequenz}
Absolutfrequenz zugewiesen bekommen muss, da das Tonsystem sonst
nicht eindeutig bestimmt ist. Wenn Sie z.\,B. nur die Töne C und G
wie folgt bestimmen:
\begin{verbatim}
TON
    C = G + Quinte
    G = C - Quinte
\end{verbatim}
und keine weiteren Töne deklarieren, so ist ein Tonsystem, welches
diese Töne benutzt, nicht eindeutig bestimmt (C und G hängen
ohne Fußpunkt endlos-rekursiv voneinander ab).
Die Gesamtheit aller Tondeklarationen darf keine solche "`Zyklen"'
enthalten.\index{Zyklen} Somit wäre auch folgende Sammlung von Tönen nicht
konsistent und würde eine Fehlermeldung des Compilers hervorrufen:
\begin{verbatim}
TON
    a = f + Terz
    c = a - Terz + Quinte - Oktave
    g = c + Quinte
    f = g - Halbton
    d = g + Quinte - Oktave
    h = 480
    e = h - Quinte
\end{verbatim}
Zwar ist hier ein Ton absolut deklariert (\sourcecode{h=480}), aber die Töne
a,c,g,f und d sind geschlossen miteinander verkettet und somit
nicht eindeutig bestimmt, was Sie dem folgenden Diagramm entnehmen können:

\input{abhaeng.pic}

\index{Töne!falsche Abhängigkeiten}In diesen Deklarationen steckt noch
ein zweiter gravierender Fehler. Die Töne c,g,a und f hängen zyklisch
voneinander ab. Wenn Sie Ihre Tondeklarationen in Form solcher Graphen
darstellen, erkennen Sie sofort eventuelle Fehler. Nicht eindeutig
bestimmte Töne zeichnen sich dadurch aus, dass sie keine Verbindung zu
einem Ton mit Absolutfrequenz haben. Dies ist, als ob Sie Ihrem
Klavierstimmer zwar sagen, wie er von einem Ton zum anderen stimmen
soll, ihm aber nicht sagen, wo er anfangen soll \ldots Es ist wichtig
zu bemerken, dass die Reihenfolge, in der die Töne deklariert werden
völlig bedeutungslos ist. Im Gegensatz zu vielen Programmiersprachen,
wie Pascal, \ldots \ gilt bei \mutabor{} nicht das Prinzip "`erst
deklarieren, dann benutzen"'. Wichtig ist nur, dass die Verknüpfung der
Töne in sich konsistent ist. Der große Vorteil dieser Tondeklarationen
besteht einfach darin, dass der \emph{strukturelle Zusammenhang} der
Töne unmittelbar -- und auch für jemanden, der noch nie mit \mutabor{} II
gearbeitet hat -- ersichtlich wird.

Desweiteren können solche Tonbezüge auch faktorielle
Zusammensetzungen von Intervallen enthalten, z.\,B.
\begin{verbatim}
TON
    c' = 264
    e' = c' + 4 Quinte - 2 Oktave
\end{verbatim}
Die Frequenz des Tones e' berechnet sich also\footnote{Wir nehmen an,
  dass die Intervalle \sourcecode{Quinte} und \sourcecode{Oktave} wie in den vorigen
  Beispielen als \sourcecode{3:2} bzw. \sourcecode{2:1} deklariert wurden.} aus der
Frequenz der Tones c' (${}=264\,\mathrm{Hz}$) , von dort aus vier Quinten nach oben
(${}=1336{,}5\,\mathrm{Hz}$), und dann wieder zwei Oktaven nach unten. e' erklingt
also hier mit $334\frac{1}{8}\,\mathrm{Hz}$.

\subsection{Zusammenfassung}\label{sec:zusammenfassung}
Wie in der Einleitung bereits gesagt wurde, ist es beim Entwurf einer
Programmiersprache -- wie überhaupt im Umgang mit Computern -- absolut
notwendig, eine \emph{präzise} Sprache zu sprechen, also nur mit {\em
  wohldefinierten Begriffen} zu hantieren. Der Computer kann nur
Stimmungslogiken "`verstehen"', deren Programmtext sich 100\%ig an
die vorgegebene Grammatik (Syntax) der Sprache hält. Damit dem Leser
die Syntax der Programmiersprache von Anfang an verinnerlichen kann,
sind die Beispiele einerseits zwar an der Basis musiktheoretischer
Begriffe orientiert, sie sollen andererseits aber auch einen
Abstraktheitsgrad erreichen, dass mittels einer abstrakten Syntax auch
eine völlig neue musikalische Semantik geschaffen werden kann.

Doch zurück zu konkreten Tönen. Die verschiedenen Möglichkeiten einer
Tondeklaration lauten:
\begin{itemize}
\item[a)] Absolutfrequenz in Hertz, z.\,B. \sourcecode{XX = 546.3324}
\item[b)] Verknüpfung, genauer: Bezugston $\pm$  Distanz, z.\,B.\\
\sourcecode{XX = YY + 3 Quinte - 4 Terz + 17 Oktave} oder\\
\sourcecode{ZZ = QQ - Halbton} 
\end{itemize}

\section{Tonsystem-Deklaration}\label{sec:tonsyst-dekl}
\index{Tonsystem!Deklaration}
Nachdem mit der Deklaration von Tönen das Fundament des
Tonsystems gelegt wurde, kommen wir nun zu dessen Deklaration:
Jedes Tonsystem bekommt -- genauso wie Intervalle und 
Töne -- einen Namen. Auf diesen folgen
die Parameter Verankerungstaste, Töne der FT und zuletzt das
Periodenintervall (Die Breite der FT ist implizit durch die Anzahl
der Tasten in der FT gegeben.). Ein Beispiel:
\begin{verbatim}
TONSYSTEM
    C_Dur = 60 [ c,des,d,es,e,f,fis,g,as,a,b,h ] Oktave
\end{verbatim}
Dieses Tonsystem ist auf der Taste 60 (=eingestrichenes c')
verankert, die FT enthält die Töne c,des,d, \dots, h und das
Periodenintervall ist die Oktave. Alle Töne und das Periodenintervall
müssen irgendwo im Programm definiert werden; die Benutzung eines
nicht definierten Bezeichners führt zu einer Fehlermeldung des
Compilers.
Wenn Sie in Ihrem Programm Intervalle, Töne und ein Tonsystem
deklariert haben, es mit dem Compiler übersetzt haben und
aufrufen, so ist Ihr Tonsystem trotzdem noch nicht aufrufbar. Wenn Sie
es hören bzw. spielen wollen, so müssen Sie dem Programm
noch die Zeilen
\begin{verbatim}
LOGIK
    Name Taste x = Name [ ]
\end{verbatim}
hinzufügen, wobei Sie statt "`Name"' den tatsächlichen Namen Ihres
Tonsystems einsetzen und statt 'x' irgendeinen Buchstaben. Die
Oberfläche auf dem Bildschirm zeigt dann in einer "`Tastenliste"' an, dass die Taste x
die Logik \emph{Name} aktiviert. So können Sie im Laufzeitmodul Ihre
Stimmungslogiken (bzw. indirekt Ihre Tonsysteme) per Tastendruck
aktivieren. (Den Sinn dessen lesen Sie im Abschnitt
"`Logiken"' bzw. "`Anweisungen, Auslöser und Aktionen"') .

\section{Beispiele für Tonsysteme}\label{sec:beisp-fur-tonsyst}
\subsection{'Reines' C-Dur}\label{sec:reines-c-dur}

Dieses Programm beschreibt einen Ausschnitt aus dem tonalen Netz
(siehe Abschnitt "`Beschreibung der Demonstrationslogiken"') mit der
Tonigkeit C als Zentrum\footnote{Beispielprogramm c\_dur.mut}.

\index{C-Dur, reines}
\label{CDUR}
\begin{verbatim}
INTERVALL
    Quinte = 3 : 2
    Terz   = 5 : 4
    Oktave = 2 : 1
TON
    c   = a - Terz + Quinte - Oktave
    des = f - Terz
    d   = g + Quinte - Oktave
    es  = g - Terz
    e   = c + Terz
    f   = c - Quinte + Oktave
    fis = d + Terz
    g   = c + Quinte
    as  = c - Terz + Oktave
    a   = 440
    b   = c - 2 Quint + 2 Oktave
    h   = g + Terz
TONSYSTEM
    C_Dur = 60 [ c,des,d,es,e,f,fis,g,as,a,b,h ] Oktave
LOGIK
    C_Dur Taste C = C_Dur [ ]
\end{verbatim}

\subsection{Einige gleichschwebende Temperaturen}\label{sec:einige-gleichschw-te}
Trotz der weit verbreiteten Sprechweise von "`gleichschwebenden"'
Temperaturen ist es korrekter, von \emph{gleichstufigen} Stimmungen zu
sprechen.  Gleichstufige Teilungen der Oktave in beliebig viele Stufen
sind besonders einfach zu pro\-gram\-mieren\footnote{Beispielprogramm
  gleich.mut}. In manchen Fällen macht einem aber die 12er --
Periodizität der Klaviatur zu schaffen. Dieses Problem wird im
nächsten Beispiel angegangen.
\label{GLEICHSCH}
\index{Gleichschwebend}
\index{Gleichstufig}
\begin{verbatim}
LOGIK
    Halb Taste H = Halb [ ]
    Drittel Taste D = Drittel [ ]
    Viertel Taste V = Viertel [ ]
TONSYSTEM
    Halb = 69 [a] Halbton
    Drittel = 69 [a] Drittelton
    Viertel = 69 [a] Viertelton
TON
    a=440
INTERVALL
    Halbton = 12 Wurzel 2
    Viertelton = 24 Wurzel 2
    Drittelton = 18 Wurzel 2
\end{verbatim}
An diesem Beispiel sehen Sie, dass die Reihenfolge, in der Sie die
Deklarationsteile schreiben, keine Rolle spielt.

\subsection{Eine pentatonische Skala}\label{sec:eine-pent-skala}
\index{Pentatonik} Hier sehen Sie zwei verschiedene Ausführungen des
gleichen Tonsystems, welche sich nur durch unterschiedlichen
Spielkomfort unterscheiden. Die Fundamentaltonleiter dieses
(gleichschwebend) pentatonischen Tonsystems umfasst fünf Töne. Die
erste Logik "`\sourcecode{PENTA1}"' realisiert dies zwar sehr einfach, aber
die Spielbarkeit leidet spürbar unter dieser Einfachheit. Das Problem
ist hier die auf zwölfstufige Tonsysteme ausgelegte Klaviatur. Die
Oktave liegt zwischen a', d', g' - und bevor man mit dieser
Tastenbelegung umgehen kann, sind doch einige Umgewöhnungen nötig.

Dieses Problem ist in der zweiten Logik "`\sourcecode{PENTA2}"' beseitigt.
Hier sind nur die schwarzen Tasten der Klaviatur belegt - die
pentatonische Skala ist einwandfrei spielbar (Die 440 Hertz liegen
jetzt auf der Taste cis'=61). Wenn Sie in einem Tonsystem bestimmte
Tasten nicht belegen wollen, also stumm schalten möchten, so lassen
Sie einfach den Platz zwischen den Kommata
frei\footnote{Beispielprogramm penta.mut}.
\label{PENTA}
\begin{verbatim}
LOGIK
    PENTA1 Taste P = Penta1 [ ]
    PENTA2 Taste Q = Penta2 [ ]
TONSYSTEM
    Penta1 = 69 [a] Pentaton
    Penta2 = 60 [,a,,next1,,,next2,,next3,,next4,] Oktave
INTERVALL
    Pentaton = 5 Wurzel 2
    Oktave   = 2 : 1
TON
    a = 440
    next1 = a + Pentaton
    next2 = a + 2 Pentaton
    next3 = a + 3 Pentaton
    next4 = a + 4 Pentaton
\end{verbatim}
Betrachten wir nochmals das
Tonsystem "`\sourcecode{Penta2}"'. Auf der
Verankerungstaste (60=c') liegt kein Ton. Die darauf folgende
Taste (61=cis') bekommt den Ton a, etc. Wichtig ist der leere
Platz vor der geschlossenen eckigen Klammer. Hätten wir einfach
\begin{verbatim}
    Penta2 = 60 [,a,,next1,,,next2,,next3,,next4] Oktave
\end{verbatim}
geschrieben, so würde sich die FT bereits auf der Taste h' um eine
Oktave versetzt wiederholen und wir hätten noch mehr Chaos als bei
der Logik "`\sourcecode{PENTA1}"'. Wenn Sie also eine spezielle Tastenbelegung
für Ihr Tonsystem wünschen, so ist es wichtig, genau zu
berücksichtigen, welche Tasten zur Fundamentaltonleiter gehören, und welche nicht.

\chapter{Umstimmungen}\label{cha:umstimmungen}
\index{Umstimmung}\index{Stimmung!mutierende}
\section{Grundlage für mutierende Stimmungen}\label{sec:grundl-fur-muti}
Kommen wir nun zum eigentlich interessanten Teil von \mutabor{}, der
Programmierung von \emph{mutierenden Stimmungen}. Hierbei handelt es
sich, wie in der Einleitung bereits beschrieben, um Stimmungen, bei
denen die Taste-Frequenz-Zuordnung nicht fest ist, sondern im Laufe
des Spiels verändert wird.

\mutabor{} kann solche Veränderungen der Stimmung, kurz: \emph{Umstimmungen},
nicht nur von außen steuerbar machen (z.\,B. Knopfdruck, MIDI-Ereignis,
\dots), sondern Umstimmungen als Reaktion auf die gespielten Harmonien
definieren. Doch bevor wir auf solche Umstimmautomatiken eingehen,
soll zunächst der Umstimmvorgang selbst genau betrachtet werden.

Sie können in einer Umstimmung jeden der vier Tonsystem-Parameter
verändern. Veränderungen von Verankerungstaste und/oder Breite der
Fundamentaltonleiter verändern zusätzlich andere Parameter, um gewisse
Strukturmerkmale des Tonsystems sinnvoll zu transformieren. Diese
beiden Umstimmungsarten sind deshalb trotz ihres einfachen Aufrufs
eigentlich komplexere Umstimmungen.

\subsection{Das aktuelle Tonsystem}\label{sec:das-aktu-tonsyst}
\index{Tonsystem!aktuelles}
Die momentane Stimmung von \mutabor{} lässt sich auch
bei mutierenden Stimmungen als Tonsystem charakterisieren. Diesen
Zustand nennen wir "`aktuelles Tonsystem"'. \textbf{Das "`aktuelle Tonsystem"'
ist also die Stimmung von \mutabor{} im aktuellen
Moment.} Diese Stimmung kann explizit im Programm angegeben sein
(als Tonsystemdeklaration). In den meisten Fällen aber wird sich
das "`aktuelle Tonsystem"' als Produkt einer Folge von relativen
Umstimmungen ergeben und nirgendwo explizit angegeben worden sein.
\index{Tonsystem!aktuelles}

Wenn im Zusammenhang von Umstimmungen und später Stimmungslogiken
vom "`alten Tonsystem"' die Rede ist, so ist das Tonsystem gemeint,
das unmittelbar vor der Umstimmung das "`aktuelle Tonsystem"' war.

\subsection{Die zwei Umstimmtypen}\label{sec:die-zwei-umst}
Da der Begriff des Tonsystems bereits eingeführt ist, können wir
eine Umstimmung wie folgt definieren:
\index{Umstimmung!Definition}
\emph{Eine Umstimmung ist der Wechsel von einem Tonsystem in ein
anderes.} Das aktuelle Tonsystem wird verändert, es "`mutiert"'.
Dieser Wechsel kann auf zweierlei Arten vonstatten gehen, und dies
unterscheidet die zwei grundlegenden Umstimmtypen bei \mutabor{}:

\begin{itemize}
\item \index{Umstimmung!absolut}Absolute Umstimmungen.  Hier wird der
  alte Zustand des Tonsystems nicht berücksichtigt, und ohne jeden
  Zusammenhang einfach alle (bzw. ein) Parameter neu gesetzt. Eine
  absolute Umstimmung wäre es z.\,B., wenn der dritte Ton der FT auf
  $375{,}8\,\mathrm{Hz}$ gestimmt, oder das Periodenintervall auf
  "`Septime"' gesetzt wird. Dabei spielt es keine Rolle, welchen Wert
  der Parameter vor der Umstimmung hatte. (Daher der Begriff
  "`absolute"' Umstimmung)
\item \index{Umstimmung!relativ}Relative Umstimmungen.  Hier wird der
  neue Wert des Parameters aus dem aktuellen Zustand berechnet. Die
  Zuweisung ist also nicht absolut oder kontextfrei. Der neue Zustand
  lässt sich als "`relativ"' zum alten beschreiben. Eine relative
  Umstimmung wäre es z.\,B., wenn die Verankerungstaste um eine Taste
  angehoben wird, oder der fünfte Ton der FT um ein syntonisches Komma
  erniedrigt wird.  Relative Umstimmungen er\-mög\-lichen es Ihnen,
  "`auf endlich vielen Tasten unendlich viele Töne spielen zu können"'
\end{itemize}

\section{Absolute Umstimmungen}\label{sec:absol-umst}
Die Syntax der Umstimmungen ist an die Deklaration des Tonsystems
angelehnt. Dort steht \emph{vor} den eckigen Klammern die
Verankerungstaste, \emph{in } den Klammern die Töne der FT und {\em
hinter} den Klammern das Periodenintervall. Wenn Sie in einer
Umstimmung die Verankerungstaste verändern wollen, so schreiben Sie
den Umstimmausdruck \emph{vor} eckige Klammern; Veränderungen der Töne
gehören \emph{zwischen} eckige Klammern, ebenso eine Veränderung der
Breite der FT; soll das Periodenintervall geändert werden, so steht der
Ausdruck \emph{hinter} eckigen Klammern. (Dies gilt für absolute und
relative Umstimmungen gleichermaßen.)

Sie können in einer Umstimmung jeden der vier Grundparameter
ver\-än\-dern. Welche Konsequenzen solche Umstimmungen jeweils haben
und wie sie genau formuliert werden können, lesen Sie in den folgenden
Abschnitten.

\subsection{Verändern des Periodenintervalls}\label{sec:verand-des-peri}
\index{Umstimmung!Periodenintervall} Eine Änderung des
Periodenintervalls hat keine Auswirkungen auf die Töne der FT, es
werden also nur diejenigen gedrückten Tasten umgestimmt, die nicht im
Bereich der FT liegen.  Probieren Sie folgende Logiken einmal
aus\footnote{D.\,h. Eingabe mit dem Editor, Speichern (Text-Datei), vom
  Compiler übersetzen lassen und aufrufen.} und schalten Sie während
des Spiels zwischen ihnen hin und her\footnote{Indem Sie die Tasten
  E,Z und D drücken.}. Sie werden deutlich hören, wie sich das
Periodenintervall ändert.  (Da Logiken noch nicht besprochen wurden,
hier zur Erklärung: Eine Logik der Gestalt
\begin{verbatim}
LOGIK
    Name Taste x = Umstimmungsname [ ]
\end{verbatim}
führt bei ihrem Aufruf durch Drücken der Taste x (einmalig) die
angegebene Umstimmung durch.)
Doch nun das Programm\footnote{Beispielprogramm periode.mut}:
\label{PERIODE}
\begin{verbatim}
INTERVALL
    Erstens = 12 Wurzel 2
    Zweitens = 13 Wurzel 2
    Drittens = 14 Wurzel 2
UMSTIMMUNG
    E = [ ] Erstens
    Z = [ ] Zweitens
    D = [ ] Drittens
LOGIK
    Eins Taste E = E [ ]
    Zwei Taste Z = Z [ ]
    Drei Taste D = D [ ]
\end{verbatim}
Sie werden sich bei diesem Beispiel wahrscheinlich fragen, mit
welchem Tonsystem das Laufzeitmodul initialisiert wird, da hier
kein Tonsystem angegeben ist. Hierbei gilt grundsätzlich: \emph{Das
Laufzeitmodul wird immer mit einer gleichstufigen
Halbtontemperierung initialisiert}, also mit
\begin{verbatim}
INTERVALL Halb = 12 Wurzel 2
TON a = 440
TONSYSTEM Start = 69 [ a ] Halb
LOGIK keine_Logik = Start [ ]
\end{verbatim}
In unserem Beispiel mit den Logiken Eins, Zwei und Drei, welche
bei Drücken der entsprechenden Taste die Umstimmungen E, Z bzw. D
durchführen wird das Periodenintervall verändert. Bei einer solchen
Veränderung müssen
alle liegenden Tasten, außer Tasten im FT-Bereich, neu gestimmt
werden. Da der FT-Bereich hier nur eine Taste (69) umfasst, bleibt
nur diese Taste (=a') von den Umstimmungen unberührt.


\subsection{Verändern der Verankerungstaste}\label{sec:verand-der-verank}
\index{Umstimmung!Verankerungstaste}

Die allgemeine Form einer Veränderung der Verankerungstaste lautet
\begin{verbatim}
UMSTIMMUNG
    Name = NeuerWert [ ]
\end{verbatim}
Wenn Sie die Verankerungstaste neu setzen wollen, z.\,B. auf die
Taste 57, so lautet die Umstimmung hierzu:
\begin{verbatim}
UMSTIMMUNG
    Anker_Neu = 57 [ ]
\end{verbatim}
Wenn nun in einer Logik die Umstimmung \sourcecode{Anker\_Neu} aufgerufen
wird, so wird die Verankerungstaste auf 57 gesetzt, alle anderen
Tonsystemparameter behalten ihren alten Wert. Obwohl scheinbar nur ein
Parameter verändert wird, handelt es sich hierbei um den komplexesten
Umstimmungstyp. Eine Veränderung des Ankers impliziert nämlich eine
Veränderung fast aller Töne der FT, da nicht nur die Verankerungstaste
neu gesetzt wird, sondern die komplette Intervallstruktur des alten
Tonsystems auf die neue Verankerungstaste verschoben wird. Hier
wird\index{Verankerungstaste} also die Struktur des alten Tonsystems
auf einen neuen Bezugston kopiert, welcher dann zum neuen Anker (oder
Zentrum) wird.

Eine Veränderung der Verankerungstaste geht folgendermaßen vonstatten:
\begin{itemize}
\item Der Computer merkt sich die Intervallstruktur des aktuellen
  Tonsystems, also das Intervall zwischen 1. und 2. Ton der FT,
  zwischen 1. und 3. Ton der FT. usw\dots
\item Der Parameter 'Verankerungstaste' bekommt den in der Umstimmung
  angegebenen neuen Wert zugewiesen.
\item Die Frequenz, die im alten Tonsystem ertönen würde, wenn die
  neue Verankerungstaste ge\-drückt würde, wird als Frequenz des 1.
  Tons der neuen FT eingetragen.
\item Die Intervallstruktur des alten Tonsystems, die der Rechner sich
  gemerkt hat, bestimmt nun die Töne der neuen FT, unter Bezugnahme
  auf den neuen Grundton (=1. Ton) der FT. Somit erhält z.\,B. der 6.
  Ton der neuen FT die Frequenz, welche sich berechnet, wenn man die
  Frequenz der neuen Verankerungstaste mit dem Intervallfaktor
  multipliziert, den sich der Rechner als Intervall zwischen dem 1.
  und dem 6. Ton der alten FT gemerkt hat.
\end{itemize}

Am besten lässt sich dieser Vorgang an einem Beispiel erklären. Das
Laufzeitmodul kann in einem Protokollfenster sowohl Tonsystemdaten als
auch die Intervallrelationen zwischen den Tönen der
Fundamentaltonleiter anzeigen.  Man kann diesen Vorgang also am
Bildschirm sehr gut verfolgen und nachvollziehen.  Nehmen wir an, das
aktuelle Tonsystem sei
\label{ANKER}
\begin{verbatim}
INTERVALL
    TerzGross = 5 : 4
    TerzKlein = 6 : 5
    Oktave    = 2 : 1
TON
    i = 440
    j = i + TerzKlein
    k = j + TerzGross
    l = k + TerzKlein
TONSYSTEM
    Meier = 60 [i,j,k,l] Oktave
\end{verbatim}

Die Abbildung zeigt die Frequenzen der Töne dieses Tonsystems:

\begin{center}
\input{anker1.pic}
\end{center}

Sie können anhand dieser Grafik auch die dem Tonsystem
zugrunde liegende Intervallstruktur erkennen. Ausgehend von der
Verankerungstaste, welche mit einer Frequenz von $440\,\mathrm{Hz}$
belegt ist, gehen die drei Intervalle $\frac{6}{5},\frac{3}{2}$ und
$\frac{9}{5}$ aus. Deshalb lauten die Frequenzen der vier Töne der
Fundamentaltonleiter $440\,\mathrm{Hz}$, $528\,\mathrm{Hz}$,
$660\,\mathrm{Hz}$ und $792\,\mathrm{Hz}$. Mit Hilfe des
Periodenintervalls Oktave (\sourcecode{2:1}) ergeben sich die Frequenzen
aller übrigen Tasten.

Nun soll die Verankerungstaste von ihrem alten Wert ($=60$) auf den
neuen Wert $61$ angehoben werden, also\footnote{Das komplette
Beispielprogramm zur Veränderung der Verankerungstaste heißt
anker.mut}
\begin{verbatim}
UMSTIMMUNG
    Schmitt = 61 [ ]
\end{verbatim}
Wie sieht nun das neue, durch die Umstimmung „Schmitt“ entstandene
Tonsystem aus?

\begin{center}
\input{anker2.pic}
\end{center}

Die Intervallstruktur wurde aus dem alten Tonsystem auf den neuen
Ankerton übertragen. Dieses Prinzip liegt allen Veränderungen des
Verankerungstones zugrunde.

Um die Intervallstruktur des Tonsystems bei einer Änderung der
Verankerungstaste beizubehalten, ist es also meist notwendig, auch die
Frequenzzuordnungen der Töne der Fundamentaltonleiter zu verändern.
Wenn Sie eine solche Umstimmung durchführen, während eine oder mehrere
Töne liegen, so muss deren Frequenz korrigiert werden\footnote{Siehe
dazu auch Abschnitt "`Korrektur der Frequenzen"'}

Doch hier noch eine Randbemerkung: Bei Tonsystemen der Breite 1
hat eine Veränderung der Verankerungstaste keine hörbare
Auswirkung. (Man überlege sich, warum!)

\subsection{Verändern der Breite der FT}\label{sec:verandern-der-breite}
\index{Umstimmung!Breite}

Eine recht seltene Umstimmungsart ist die Veränderung der Breite 
der Fundamentaltonleiter. Sie können mit diesem Umstimmungstyp 
die Breite der FT auf einen beliebigen neuen Wert setzen. 
Es gibt deshalb drei verschiedene Fälle:

\begin{itemize}

\item Der erste Fall ist trivial: die neue Breite entspricht der alten
  Breite -- damit wird gar nichts verändert.

\item Wenn die neue Breite kleiner ist als die aktuelle Breite, so
  wird das Periodenintervall neu gesetzt, und zwar auf den Wert des
  Intervalls zwischen dem ersten Ton der FT und der ersten Taste nach
  der neuen Breite. Dann werden alle Töne der FT vergessen, die
  jenseits der neuen Breite lagen.

\item Wenn die neue Breite größer ist als die aktuelle Breite, so wird
  ebenfalls das Periodenintervall auf das Intervall zwischen dem
  ersten Ton der FT und dem ersten Ton nach der neuen Breite gesetzt.
  Dann werden die Töne, um die die FT erweitert wurde, gemäß dem alten
  Tonsystem berechnet und eingetragen.

\end{itemize} 

Man kann also sagen, dass eine Veränderung der Breite der FT 
im allgemeinen auch eine Veränderung der Töne der FT und 
des Periodenintervalls nach sich zieht.


Syntaktisch sieht diese Konstruktion so aus:
\begin{verbatim}
UMSTIMMUNG
  Neue_Breite = [ << 7 >> ]
\end{verbatim}
(Die doppelten Kleiner-als- bzw.\ Größer-als-Zeichen (\sourcecode{<<} bzw.
\sourcecode{>>} sollen an eine Breitenangabe in Zeichnungen, wie z.\,B.
\mbox{$\longleftarrow 4 cm \longrightarrow$} erinnern.) In diesem
Beispiel wurde die Breite der FT auf den neuen Wert 7 gesetzt.

Eine praktische Bedeutung der Breitenänderung ist das {\em
  Expandieren} \index{Expandieren} von Tonsystemen zur Benutzung von
Harmonie-Auslösern, die eine größere Breite der FT voraussetzen. Da
der Begriff des Harmonieauslösers noch nicht erklärt wurde, sei hier
nur soviel bemerkt, dass gewisse Konstruktionen eine FT der Breite 12
verlangen. Wenn Sie ein gleichstufig temperiertes Halbtonsystem
programmieren, das die Breite 12 hat, so würden Sie bisher
\begin{verbatim}
INTERVALL
  Halbton = 12 Wurzel 2
  Oktave  = 2 : 1

TON
  c   = a - 9 Halbton       cis = a - 8 Halbton
  d   = a - 7 Halbton       dis = a - 6 Halbton
  e   = a - 5 Halbton       f   = a - 4 Halbton
  fis = a - 3 Halbton       g   = a - 2 Halbton
  gis = a - 1 Halbton       a   = 440
  ais = a + 1 Halbton       h   = a + 2 Halbton

TONSYSTEM
  Halb12 = 60 [ c,cis,d,dis,e,f,fis,g,gis,a,ais,h ] Oktave
\end{verbatim}
schreiben müssen.

Eine starke Vereinfachung dieser Stimmung stellt nun eine auf 
12 expandierte FT der (alten) Breite 1 dar. Programmieren 
Sie einfach ein Halbtonsystem mit der Breite 1, wie Sie es 
im Beispiel auf Seite \pageref{GLEICHSCH} kennen gelernt haben. Wenn Sie 
nun die Breite der FT auf 12 setzen, das Tonsystem also expandieren, 
so stellt sich keinerlei hörbarer Effekt ein, da das Periodenintervall 
dieser Änderung angeglichen wird\footnote{Eine solche Expansion funktioniert 
auf nichtveränderunde Weise nur bei einer Vergrößerung um 
ein ganzzahliges Vielfaches der alten Breite.}. Auf diese Weise 
haben Sie ohne Mühe ein gleichstufiges Halbtonsystem der Breite 
12 programmiert:

\begin{verbatim}
INTERVALL
  Halbton = 12 Wurzel 2

TON
  a = 440

TONSYSTEM
  Halb1 = 69 [ a ] Halbton

UMSTIMMUNG
  Expandieren = [ << 12 >> ]
\end{verbatim}

Wenn eine Logik die Umstimmung \sourcecode{Expandieren} durchführt, während 
das Tonsystem \sourcecode{Halb1} aktiv ist, so wird der gewünschte Zustand 
eines gleichstufigen Halbtonsystems mit der Breite 
12 eingestellt\footnote{Später lernen Sie ein Verfahren 
kennen, mit dem man diese beiden Schritte in einer 
einzigen Umstimmung zusammenfassen kann (Umstimmungsbund).}. 

\subsection{Verändern der Töne der FT}\label{sec:verandern-der-tone}
\index{Umstimmung!Töne}

Dies ist die wohl vielfältigste Umstimmungsart. Während die bisher
er\-klär\-ten Umstimmungstypen die Töne der FT zumindest in ihrem
intervallischen Zusammenhang nicht manipuliert haben, so können
Sie mit diesem Umstimmungstyp die Struktur des aktuellen Tonsystems
gezielt verändern.
Ein weiteres Beispiel: Nehmen wir an, Sie spielten gerade das
\begin{verbatim}
TONSYSTEM    C_Dur = 60 [ c,des,d,es,e,f,fis,g,as,a,b,h ] Oktave
\end{verbatim}
und wollen die Töne des,es und as in cis,dis und gis umwandeln
(Wir nehmen der Einfachheit halber an, dass alle diese Töne im
Tondeklarationsteil bereits deklariert worden sind). Die
Umstimmung hierzu lautet
\begin{verbatim}
UMSTIMMUNG
    Be = [@,cis,@,dis,@,@,@,@,gis,@,@,@]
\end{verbatim}
Sie werden sich nun fragen, was das "`Klammeraffensymbol"' hier zu
suchen hat. Hätte man eine solche Umstimmung nicht einfacher
\begin{verbatim}
    Be = [,cis,,dis,,,,,gis,,,]
\end{verbatim}
schreiben können? Aber erinnern wir uns, was ein Leerschritt
zwischen den Kommata bei Tonsystemen bedeutet: die Taste wird
"`gesperrt"'. Und wenn wir hier den Klammeraffen weggelassen hätten,
wäre auch genau das passiert. Alle Tasten, außer cis,dis und gis
wären gesperrt worden (Durch die Anwendung von Leerstellen können
Sie gezielt in einer Umstimmung bestimmte Tasten sperren).
Die Benutzung des Klammeraffen hingegen bedeutet: "`\emph{alter Wert}"',
d.\,h. es wird der Wert genommen, den der Parameter vor der
Umstimmung hatte. Wenn Sie in einer Umstimmung nur den
Klammeraffen benutzen, so wird nichts verändert. Der Klammeraffe
kann auch bei allen anderen Umstimmungsarten benutzt werden:
\begin{verbatim}
UMSTIMMUNG
    Otto = @ [ ]
    Meier = [<<@>>]
    Fritz = [ ] @
    Schorsch = [@,@,@,@,@,@,@,@,@,@,@]
\end{verbatim}
All diese Umstimmungen bewirken nichts und haben für die Praxis
sicherlich keine Bedeutung, aber sie führen uns zur zweiten Kategorie
von Umstimmungen, den \emph{relativen
  Umstimmungen}\index{Umstimmung!relative}. Hier wird das
Klammeraffensymbol (\sourcecode{@}) gebraucht, um in einer Formel zur
Bestimmung des neuen Wertes auf den "`alten Wert"' zugreifen zu
können.

\section{Relative Umstimmungen}\index{Umstimmung!relativ}\label{sec:relat-umst}
Die Funktionsweise der relativen Umstimmungen ist die gleiche wie
bei den absoluten. Der einzige Unterschied besteht darin, dass dem
umzustimmenden Parameter kein fester Wert zugewiesen wird, sondern
dass sich der neue Wert aus dem alten Zustand mit einer einfachen Formel
berechnen lässt.

Dies ist nützlich, wenn der neue gewünschte Zustand eines Parameters
nicht absolut bekannt ist, sondern sich aus dem alten Zustand ergibt.
Dies ist der Fall, wenn z.\,B. die Verankerungstaste um eins erhöht
werden soll. Dies schreibt man so:
\begin{verbatim}
UMSTIMMUNG
    Anker = @ + 1 [ ]
\end{verbatim}
Das Klammeraffensymbol (\sourcecode{@}) bedeutet hier 'alter Wert'. Bei dieser
Umstimmung wird die Verankerungstaste auf 'alter Wert plus
eins' gesetzt, sprich um eins erhöht.
Wenn die Verankerungstaste zum Zeitpunkt des Umstimmungsaufrufs
den Wert 56 hat, so ist ein Aufruf der Umstimmung \sourcecode{Anker}
gleichbedeutend mit der (absoluten) Umstimmung
\begin{verbatim}
UMSTIMMUNG
    Von_56 = 57 [ ]
\end{verbatim}
Relative Umstimmungen können genauso wie absolute Umstimmungen jeden
der Grundparameter Anker, Breite und Töne des aktuellen Tonsystems
sowie das Periodenintervall verändern. Die nachfolgend aufgeführten
Umstimmungen sind alle gültig:
\begin{verbatim}
UMSTIMMUNG
    Anker  = @ + 1 [ ]
    Breite = [ << @ - 2 >> ]
    Toene  = [@ , @ , @ + 2 Quinte - Oktave ,
              @ - 3 Terz + Oktave , @ , @ + Syn_Komma , , ]
    Periode= [ ] @ + Syn_Komma
\end{verbatim}
An dieser Stelle sei bemerkt, dass man als Namen keines der {\it
reservierten Worte} verwenden darf. Dazu zählt das Wort \sourcecode{Ton}.
Wenn man beispielsweise diese Umstimmungen deklariert,
\begin{verbatim}
UMSTIMMUNG
    Anker  = @ + 1 [ ]
    Breite = [ << @ - 2 >> ]
    Ton  = [ @ , @ , @ + 2 Quinte - Oktave ,
             @ - 3 Terz + Oktave , @ , @ + Syn_Komma , , ]
    Periode= [ ] @ + Syn_Komma
\end{verbatim}
so erhält man einen Syntaxfehler, weil das Wort \sourcecode{Ton} eine
vorgegebene Bedeutung innerhalb der Programmiersprache hat. Erlaubt
hingegen ist, wenn ein reserviertes Wort Bestandteil eines Bezeichners
ist, so ist es zwar verboten, eine Umstimmung namens \sourcecode{Ton} zu
deklarieren, aber eine Umstimmung \sourcecode{Tontaube} ist natürlich
erlaubt.

Da die vier verschiedenen Umstimmtypen in ihrer Bedeutung und
Handhabung sehr unterschiedlich sind, folgt hier noch einmal eine
Auflistung der gültigen Rechenzeichen bei relativen Umstimmungen:

\begin{center}
\begin{tabular}{|c|c|}
\hline
Parameter    &  gültige Rechenzeichen \\ \hline
Anker        &  +   -                 \\
Breite       &  +   -   *   /         \\
Periode      &  +   -                 \\
Töne der FT  &  wie bei TON-Deklaration   \\ \hline
\end{tabular}
\end{center}

\vspace{3 mm}
Bereichsüber- bzw. Unterschreitungen werden im Laufzeitmodul
abgefangen und ignoriert. Wenn z.\,B. die Breite der FT~$1$ beträgt
und die Umstimmung
\begin{verbatim}
UMSTIMMUNG
    Halbieren = [ << @/2 >> ]
\end{verbatim}
aufgerufen wird, so wird die angegebene Umstimmung ignoriert, da die
Breite der FT mindestens 1 betragen muss. Der Anschauung halber folgt
nun noch einmal eine Liste gültiger relativer Umstimmungen:

\begin{verbatim}
INTERVALL
  cent = 1200 wurzel 2

UMSTIMMUNG
  Verdopplung    = [ << @ * 2 >> ]
  Trichter       = [ ] @ + 3 cent
  Temp_nach_rein = [ @ , @ , @ , @ , @ - 14 cent , @ - 2 cent ,
                     @ , @ + 2 cent ]
  Transponieren  = @ - 2 [ ]
  Was_ist_das    = @ + 4
\end{verbatim}
Die letzte Umstimmung ist syntaktisch falsch, da nicht eindeutig ist,
ob die Verankerungstaste oder das Periodenintervall verändert werden
soll. Aus diesem Grunde ist es immer notwendig, die zunächst als
überflüssig erscheinenden eckigen Klammern zu schreiben.

Detailliertere Informationen über Umstimmungen mit einer exakten
Spezifikation der Vorgehensweise in Ausnahmesituationen entnehmen Sie
bitte dem Abschnitt "`Handhabung von Grenzfällen"'.

\section{Umstimmungsbünde}\index{Umstimmungsbund}\label{sec:umstimmungsbunde}
Mit den bisher gezeigten Umstimmungsarten können Sie nur jeweils {\em
  einen} Parameter verändern. Manchmal ist es jedoch erforderlich,
zwei oder drei Parameter in einem Umstimmvorgang zu ändern. Es wäre
nun prinzipiell denkbar, einfach die Syntax der Tonsystemdeklaration
zu übernehmen, um eine gleichzeitige Veränderung von Breite und
Periodenintervall zu erreichen, also etwa
\begin{verbatim}
UMSTIMMUNG
	    Breite_und_Periode = [ << @+9 >> ] Quint
\end{verbatim}
Leider ist eine solche Konstruktion in ihrem Ergebnis nicht eindeutig.
Es macht einen deutlichen Unterschied, ob zuerst das Periodenintervall
verändert wird und dann die Breite, oder umgekehrt\footnote{überlegen
Sie sich, wie diese Umstimmung unterschiedlich auf das Periodenintervall
eines Tonsystems der Breite 1 wirkt (z.\,B. das Tonsystem Drittel). \\
a) zuerst wird die Breite auf den Wert 10 erhöht, dabei ändert sich das
Periodenintervall, was jedoch egal ist, da es anschließend auf den
absoluten Wert Quinte gesetzt wird. \\ b) Wird hingegen zuerst das
Periodenintervall auf Quinte gesetzt und anschließend die Breite um 9 Tasten (also auf den Wert 10)
expandiert, so wird das Periodenintervall dieser Breitenänderung
angepasst und erhält einen ungemein größeren Wert.}. Sie können deshalb
in einem Umstimmungsbund mehrere Umstimmungen nacheinander auflisten,
um so eine eindeutige Reihenfolge festzulegen. Obiges Beispiel würde
demnach wie folgt umgeschrieben werden müssen:
\begin{verbatim}
UMSTIMMUNG
	    Breite   = [ << @+9 >> ]
	    Periode  = [ ] Quint
	    Entweder = { Breite, Periode }
	    Oder     = { Periode, Breite }
\end{verbatim}
Der Umstimmungsbund \sourcecode{Entweder} vergrößert zuerst die Breite des
momentanen Tonsystems um neun, um dann das Periodenintervall auf Quinte
zu setzen; der Umstimmungsbund \sourcecode{Oder} führt die gleichen
Umstimmungen in umgekehrter Reihenfolge durch.

In einem Umstimmungsbund können Umstimmungen, Tonsysteme und sogar Logiken\footnote{Bei
Namensmehrdeutigkeiten wird hier erstrangig zugunsten der Logik,
dann zugunsten der Umstimmung entschieden
!} und andere Um\-stim\-mungs\-bün\-de\footnote{Hierbei ist zu beachten, dass
keine Rekursionen entstehen dürfen.} aufgelistet werden. 
Alle aufgelisteten Elemente eines Umstimmungsbundes müssen
im Programm deklariert sein.

\chapter{Logiken}\label{cha:logiken}
Kommen wir nun zur "`Schaltzentrale"' unserer Stimmung - der Logik.
Hier bestimmen Sie, welche Ereignisse Umstimmungen auslösen sollen.
Die Logik ist außerdem das einzige Objekt aus Ihrem Programm, das die
Verbindung zur Benutzeroberfläche des Laufzeitmoduls herstellt. Alle
anderen Objekte wie Intervalle, Tonsysteme, \dots{} sind rein
"`deklarative Elemente"', die erst dann im Laufzeitmodul eine Wirkung
zeigen, wenn sie von einer Logik benutzt werden. Aus diesem Grund
mussten Sie bisher, selbst wenn Sie nur ein einfaches Tonsystem spielen
wollten, dem Programm eine Logik der Gestalt
\begin{verbatim}
LOGIK
    Name Taste x = Name [ ]
\end{verbatim}
hinzufügen. Die Deklaration von Tonsystemen hat im Laufzeitmodul
noch keine Auswirkung. Nur Elemente, die direkt oder indirekt von einer
Logik benutzt werden, werden in das Laufzeitmodul eingetragen.

Eine Logik besteht aus drei Teilen: dem \emph{Auslöser}, der {\em
  Einstimmung} und dem \emph{Aktionsteil}. Die allgemeine Struktur
einer Logik lautet also:

\sourcecode{LOGIK Name Auslöser = Einstimmung [ Aktionen ]}

\section{Der Auslöser}\index{Auslöser}\label{sec:der-ausloser}
Um eine Stimmungslogik zu aktivieren, ist irgend eine Art von externem
Signal nötig. Dies kann z.\,B. ein Tastendruck auf der Computertastatur
sein. Wenn z.\,B. die Stimmungslogik \sourcecode{Meier} durch das Drücken der
Taste 'M' aktiviert werden soll, so schreiben Sie
\begin{verbatim}
LOGIK
      Meier Taste M = ...
\end{verbatim}
\mutabor{} kennt noch weitere Auslöser für Stimmungslogiken, die aber
erst im Teil \ref{teilFortgesch} dieses Handbuches beschrieben werden.

\section{Die Einstimmung}\index{Einstimmung}\label{sec:die-einstimmung}
Hier kann ein Tonsystemname oder ein Umstimmungsname stehen. Das
angeführte Tonsystem (bzw. die Umstimmung) wird einmalig aufgerufen,
und zwar in dem Moment, in dem Sie im Laufzeitmodul die Logik durch
Tastendruck\footnote{bzw. durch einen anderen Auslöser} aktivieren.

Eine Einstimmung dient als Initialisierung des aktuellen Tonsystems.
Wenn Sie hier ein Tonsystem eintragen, spricht man von einer
"`harten"' Initialisierung, weil der alte Zustand komplett vergessen
wird und alle Parameter auf einen neuen Wert eingestellt werden, gemäß
dem Einstimmungs-Tonsystem. Tonsysteme werden immer dann als
Einstimmung benutzt, wenn die Logik einen fest definierten und somit
immer gleichen Anfangszustand haben soll.  Und natürlich, wenn Sie,
wie im Abschnitt "`Tonsysteme"', nur ein nicht-mutierendes Tonsystem
spielen wollen. In diesem Fall besteht die Logik aus einer Einstimmung
(nämlich dem gewünschten Tonsystem) und einer leeren Aktionsliste ("`
\sourcecode{[ ] } "').

Wenn Sie eine Umstimmung als Einstimmung angeben, so wird nur ein Teil
der Parameter verändert. Solche "`weichen"' Einstimmungen sind mit
Vorsicht zu genießen, da verschiedene Ausgangszustände die Logik
unterschiedlich initialisieren. So passieren möglicherweise ungewollte
Effekte. Nichtsdestotrotz sind Umstimmungen als Einstimmung einer
Logik manchmal nützlich, insbesondere wenn nur die Breite des
Tonsystems expandiert werden soll.

Außerdem ist es möglich, keine Einstimmung anzugeben. Dann wird
das aktuelle Tonsystem beim Aufruf der Logik nicht verändert, es
findet beim Aufruf der Logik überhaupt keine Initialisierung
statt.

\subsection{Eine Randbemerkung}\label{sec:eine-randbemerkung}

Sie können jetzt eine "`leere"' Logik programmieren:
\index{Logik!leere}
\begin{verbatim}
LOGIK
    Leer Taste L = [ ]
\end{verbatim}
Der Sinn einer solchen Logik ist folgender: nehmen wir an, Sie
hätten eine mutierende Stimmung programmiert, auf ihr gespielt und
einen Punkt erreicht (sprich: eine Stimmung), den Sie gerne
unverändert weiterspielen wollen. Leider würde die Automatik der
mutierenden Stimmung diesen gewünschten Zustand bald wieder verlassen.

Wenn Sie nun eine "`leere"' Logik aktivieren, so wird, da sie keine
Einstimmung hat, eben der aktuelle gewünschte Zustand beibehalten;
und, da in der leeren Logik keine Aktionen definiert sind, dieser
Zustand auch nicht verändert.

Doch nun wird es Zeit, von solchen statischen Logiken zu den
mutierenden, automatischen Logiken zu kommen.

\section{Harmonien als Auslöser für Um\-stim\-mun\-gen}\label{sec:harm-als-ausl}
\index{Harmonien!Auslöser}
Innerhalb der eckigen Klammern einer Stimmungslogik können Sie nun
beliebig viele Anweisungen der Gestalt

\sourcecode{Auslöser -> Aktion}

\noindent angeben.

In den meisten Fällen wird die Aktion eine Umstimmung sein, die das
aktuelle Tonsystem verändert. Die Auslöser regeln nun eindeutig, unter
welchen Bedingungen die Umstimmungen durchgeführt werden sollen.

Am einfachsten können Sie innerhalb einer Stimmungslogik Umstimmungen
per Tastendruck erzeugen. Wenn Sie z.\,B. den Unterschied zwischen einer
Naturseptime (7:4) und der Septime als Schichtung zweier Quarten
(16:9) unmittelbar vergleichen möchten, so ist es sinnvoll, diese
Umstimmungen mit einer Taste als Auslöser zu versehen, also
z.\,B.\footnote{Beispielprogramm naturspt.mut}:
\label{NATURSPT}
\begin {verbatim}
LOGIK
      Reines_C_Dur Taste R = C_Dur [
                                    Taste N -> Natursept
                                    Taste S -> Normalsept
      ]
UMSTIMMUNG
      Natursept  = [ @,@,@,@,@,@,@,@,@,@,@-Septkomma,@ ]
      Normalsept = [ @,@,@,@,@,@,@,@,@,@,@+Septkomma,@ ]
INTERVALL
      Septkomma = 64:63
\end{verbatim}
Das Tonsystem \sourcecode{C\_Dur} ist bereits des öfteren weiter oben benutzt
worden (Siehe Seite \pageref{CDUR}). Versuchen Sie diese
Stimmungslogik einmal, schlagen Sie einen vollen $C^7$-Akkord
(c',e',g',b') an und wechseln Sie per Tastendruck
 (N bzw. S) zwischen
den beiden Septimen -- ein sehr interessanter Vergleich!

Dieses einfache Prinzip zum Auslösen von Umstimmungen bietet zwar
bereits eine Reihe an Möglichkeiten, ist aber dennoch recht
unflexibel. Das Hinzukommen eines jeden neuen Bedienungselementes ist
für den Spieler ungewohnt und fremd. Wer möchte schon gerne
während des Spiels ständig einen Knopf am Computer drücken, um die
rechte Intonation zu bekommen.

Deshalb besitzt \mutabor{} die Fähigkeit zu einer einfachen
programmierbaren harmonischen Analyse des gespielten Musikstücks.
\index{Harmonische Analyse}\index{Analyse!harmonische}
Sie können Umstimmungen jetzt von gespielten Harmonien abhängig
machen. Der Spieler braucht sich an
keine neuen Knöpfe zu gewöhnen -- die Umstimmung passiert
\emph{automatisch}.

Wo liegen nun die praktischen Anwendungen einer solchen
harmoniegekoppelten Umstimmungsautomatik?  Nehmen wir an, Sie wollen
Stücke in tonaler Dur-Moll-Harmonik spielen, aber mit rein gestimmten
Terzen und Quinten. Zu diesem Zweck können Sie eine sehr einfache
Stimmungslogik programmieren, die Terzen, Quinten und Dreiklänge als
solche erkennt und die passende Umstimmung durchführt, so dass diese
Intervalle rein intoniert werden können\footnote{Eine solche
  Stimmungslogik finden Sie im Abschnitt \ref{MININETZ}.}.

\subsection{Die Projektionstonleiter}\label{sec:die-proj}
\index{Projektionstonleiter}
Um die Funktionsweise von mutierenden Stimmungslogiken verstehen
zu können, müssen wir den Begriff der Projektionstonleiter
einführen.

Die gesamte Klaviatur lässt sich als blockweise Aneinanderreihung von
Tonleitern verstehen. Eine besondere Tonleiter ist die
Fundamentaltonleiter. In ihr stecken die Informationen der vier
Grundparameter: Töne, Breite, Anker und Periode. Bedenken Sie, dass
trotz der 12-periodischen Struktur der Klaviatur die
Fundamentaltonleiter -- und somit alle momentanen "`Tonleitern"' nicht
auf eine Breite von 12 Tasten festgelegt sind! Somit kann es möglich
sein, dass die Klaviatur bei einem Tonsystem der Breite 7 vom c' zum g'
zum d'' \dots{} in Blöcke unterschiedlicher Tastenstruktur unterteilt ist!

Die Projektionstonleiter enthält nun keine Frequenz- oder Tonwerte,
sondern die Information, welche Tasten der Klaviatur momentan
gedrückt sind, und zwar \emph{in die FT projiziert}. Ein Beispiel: Ihre
FT ist auf Taste c' (=60) verankert und habe die Breite 12, also
eine ganz normale C-Dur-Tonleiter. Bezeichnen wir den Zustand
"`Taste gedrückt"' mit '*' und den Zustand "`Taste nicht gedrückt"'
mit '+'. Solange wir keine Taste auf der Klaviatur anschlagen, hat die\index{Tasten!Muster}
Projektionstonleiter das Aussehen "`++++++++++++"'. Wenn wir jetzt die
Taste c' (=60) anschlagen, so ändert sich der Zustand der
Projektionstonleiter zu "`*+++++++++++"', da die erste Taste einer
Tonleiter gedrückt ist.
Wenn wir einen C-Dur-Akkord anschlagen, (Tasten c', g' und e'), so
enthält die PT das Muster "`*+++*++*++++"'.

Warum heißt dieses Gebilde Projektionstonleiter? Weil es ohne
Bedeutung ist, in welcher Tonleiter (oder Lage) eine Taste
angeschlagen wird,\index{Tonleiter}\index{Lage}\index{Tonigkeit} alles
wird in die FT hineinprojiziert. Musikalisch gesprochen ist die
Projektionstonleiter also nichts anderes als die Menge der momentan
gedrückten \emph{Tonigkeiten}. So ist der Zustand der PT gleich
"`++*+++*++*++"', wenn man die Tasten d'(=62),fis'(=66) und a'(=69)
anschlägt, und ebenso, wenn man die Tasten d (=50),a' (=69) und fis"'
(=78) drückt. Bei einem zwölfstufigen Tonsystem gibt es also -- laut
Kombinatorik -- (theoretisch) 4096 verschiedene mögliche Zustände der
PT (inklusive keiner gedrückten Taste), allgemein gibt es $2^{Breite}$
mögliche Zustände einer PT.

\begin{center}
\input{pt.pic}
\end{center}

Ein zweites, etwas seltsames Beispiel: Betrachten wir ein Tonsystem,
wie wir es für die Einstimmung von gleichstufig temperierten
Tonsystemen benutzt haben. Setzen wir den Anker auf die Taste a' (=69)
und die Breite der FT auf 1 (Die Töne und des Periodenintervalls sind
für diese Betrachtung nicht von Bedeutung).  Überlegen wir uns einmal,
welche verschiedenen Zustände die PT haben kann. Ein seltsames
Ergebnis: nur zwei. Wenn keine Taste auf der gesamten Klaviatur
gedrückt ist, so bekommt die PT den Inhalt "`+"'. Und sobald irgend
eine oder mehrere Tasten gedrückt werden bekommt die PT den Inhalt
"`*"'. Warum? Nun, weil alles in die FT projiziert wird, und die FT
hat nur die Breite eins.

Fassen wir zusammen: \emph{die PT enthält die gleiche Anzahl
"`Tasten"', wie die FT breit ist. In ihr wird der in den Bereich
der FT projizierte Zustand der Klaviatur festgehalten.}

\section{Harmoniedeklarationen}\index{Harmonie!Deklaration}\label{sec:harm}
Nach dieser etwas komplizierten Einführung des Begriffes der
Projektionstonleiter kommen wir nun zu dessen praktischer Anwendung,
der \emph{Harmoniedeklaration}. Am einfachsten lässt sich dies an
einigen Beispielen verdeutlichen.

Nehmen wir an, wir befänden uns in einem zwölfstufigen Tonsystem
mit Verankerungstaste c' (=60). Wir wollen einen C-Dur-Akkord
(Tasten c,e und g) als Auslöser für eine Umstimmung benutzen.
Diese Harmonie (\sourcecode{C\_DUR}) deklarieren wir als:
\begin{verbatim}
HARMONIE
    C_DUR = { 0,4,7 }
\end{verbatim}
Die Harmonie \sourcecode{C\_DUR} wird erkannt, sobald die 0., die 4. und die 7.
Taste der Projektionstonleiter gedrückt sind. Bitte führen Sie sich
vor Augen, dass der Begriff "`Harmonie"'\index{Harmonie!Tastenmuster}
in diesem Zusammenhang lediglich ein bestimmtes Tastenmuster
auf der Klaviatur meint.

Neben den aufgeführten Tasten einer Harmonie, die gedrückt sein {\em
  müssen}, können Sie auch Tasten angeben, deren Zustand bei der
Harmonieanalyse \emph{nicht} berücksichtigt werden soll:

\begin{verbatim}
HARMONIE
    Dur2 = { 0,4,7,*10 }
\end{verbatim}

Der Zustand einer mit einem Stern versehenen Taste wird bei der
Harmonieanalyse ignoriert. Die Harmonie \sourcecode{Dur2} ist also erfüllt,
sowohl beim Spielen der Harmonie \sourcecode{\{0,4,7,10\} }, als auch \sourcecode{\{
  0,4,7 \} }.

\subsection{Eine Anmerkung}\label{sec:eine-anmerkung}
Die hier vorgestellten Prinzipien einer Harmonieanalyse entstammen --
wie unschwer zu erkennen ist -- den Begriffen einer einfachen
dur-moll-tonalen Harmonik. Wir denken jedoch, dass diese
Harmonieanalyse von \mutabor{} vielfältigere Verwendung finden kann,
da hier praktisch alle möglichen Kombinationen liegender Tasten eine
Aktion (z.\,B. Umstimmung) veranlassen können.

Oft taucht der Wunsch auf, Umstimmungen nicht nur von harmonischen --
und somit momentanen -- Ereignissen, sondern auch von {\em
  melodischen} Prinzipien\index{Melodisches Prinzip} auslösen zu
können. Leider stellt dies zwei gravierende Probleme prinzipieller
Art: einerseits ist die Live-Verwendbarkeit eine wichtige Fähigkeit
von \mutabor{}. Ob eine Umstimmung oder allgemeiner eine Aktion
stattfinden soll, oder nicht, muss sofort, also nur unter
Berücksichtigung vergangener und aktueller, nicht aber zukünftiger
Ereignisse entschieden werden können. Melodische Strukturen benötigen
leider meistens Entscheidungsstrukturen der Art "`vor zwei Minuten
hätte eine Umstimmung stattfinden sollen"'. Das zweite Problem bei
melodischen Auslösern ist, dass sie nur schwer in allgemeiner Weise als
Programmiersprache formuliert werden können. Sie können jedoch sicher
sein, dass wir an diesem Problem arbeiten. Für Ideen, Vorschläge oder
Anregungen sind wir jederzeit dankbar.


Obwohl den meisten Beispielen ein zwölfstufiges Tonsystem zugrunde
liegt, können Sie praktisch bei jeder beliebigen Breite für ein
Tonsystem Harmonieerkennungen verwenden. Leider sind wegen der
Struktur der Klaviatur z.\,B. 7-stufige Tonsysteme denkbar ungünstig zu
spielen, so dass man meistens bestimmte Tasten stumm schaltet, wie in
der Beispiellogik \sourcecode{PENTA2} in Abschnitt \ref{PENTA} gezeigt
wurde. Schön wäre es natürlich, wenn es eine MIDI-Klaviatur gäbe, die
nicht auf die Zwölferstruktur festgelegt ist.

\section{Der Aktionsteil}\index{Anweisung}\index{Aktion}\label{sec:der-aktionsteil}\label{sec:der-aktionsteil-1}
Hier wird die Umstimmautomatik einer mutierenden Stimmung
definiert. Eine Umstimmautomatik besteht aus einer
Aneinanderreihung von Anweisungen. Wir wollen in diesem Abschnitt
besprechen, welche Anweisungen Sie zur Programmierung einer
mutierenden Stimmung verwenden können.

Wie der Begriff schon andeutet, soll es sich bei der
Umstimmautomatik um einen im Hintergrund arbeitenden Mechanismus
handeln, der die Stimmung des Instrumentes immer im richtigen
Moment verändert. Bereits bei der Besprechung von Umstimmungen
haben Sie einen solchen Mechanismus kennen gelernt: die Umstimmung
'auf Knopfdruck'. Leider ist es während des Spiels nicht immer
möglich, die eine Hand zum Computer zu bewegen, um dort eine Taste
zur Umstimmung zu betätigen. Deshalb können Sie
\mutabor{} so programmieren, dass bei der Erkennung
von bestimmten Tastenkombinationen (Harmonien) eine Umstimmung
durchgeführt wird. Welche Tastenkombination nun welche Umstimmung
hervorrufen soll, können Sie frei bestimmen.

Es gibt insbesondere in der Anwendung tonaler Harmonik bestimmte
Strukturen, die über pure "`Tastenkombinationen"' herausgehen.
Deshalb wäre es meistens sehr umständlich, wenn Sie alle möglichen
Tastenkombinationen angeben müssten, die die Umstimmung erzeugen
sollen. Die Analyse der Klaviatur zu jedem Zeitpunkt einer Änderung
steht in einem Zusammenhang mit dem aktuellen Tonsystem, insbesondere
mit seiner Breite. Ein sinnvoller Auslöser für Umstimmungen ist also
die Erkennung einer Harmonie, bzw. eine Beziehung zwischen den
gedrückten Tasten und der Fundamentaltonleiter, was dann z.\,B. so etwas
wie einen "`C-Dur-Akkord ohne Berücksichtigung von Lage und
Umkehrung"' ergibt.  Allgemein können wir also wieder

\sourcecode{Auslöser -> Aufruf}

als Regel für das Laufzeitmodul definieren, beim Erkennen eines
bestimmten Auslösers (z.\,B. einer Harmonie) den angegebene Aufruf
durchzuführen (z.\,B. eine Umstimmung aufrufen). Damit können wir
bereits ein erstes Programm mit einer automatisch mutierenden
Stimmung schreiben.\index{Stimmung!mutierende}

Nehmen wir folgendes Beispiel:
Das im Abschnitt "`Tonsysteme"' beschriebene Programm zum Spielen in
einem 'reinen' C-Dur soll die Grundlage für unsere mutierende
Stimmung sein. Dieses Tonsystem definiert den Ton b als Quinte
unter dem Ton f. Nehmen wir nun an, wir hätten den Wunsch, beim
Anschlagen der Sept c-b das b nicht als Quinte unter f zu
intonieren, sondern als Naturseptime über c. Ferner soll bei
einstimmigem Melodiespiel das b wieder als Quinte unter f benutzt
werden.\index{Naturseptime}

Die Automatik, die dies realisieren kann, benötigt also zwei
verschiedene Zustände, nämlich den Zustand "`Natur"', in dem das
Intervall c-b eine Naturseptime ist und den Zustand "`Normal"', in
dem der Ton b als Quinte unter f genommen wird.
Die einfachste Realisierung dieser zwei Zustände wären zwei
Logiken, bei denen diese Zustände als Einstimmung angegeben sind.
Dieses Beispiel funktioniert ohne Klaviaturanalyse.
Im Programmtext also:
\begin{verbatim}
INTERVALL
    Quinte       = 3:2
    Terz         = 5:4
    Naturseptime = 7:4
    Oktave       = 2:1
TON
    c   = a - Terz + Quinte - Oktave
    des = f - Terz
    d   = g + Quinte - Oktave
    es  = g - Terz
    e   = c + Terz
    f   = c - Quinte + Oktave
    fis = d + Terz
    g   = c + Quinte
    as  = c - Terz + Oktave
    a   = 440
    b   = f - Quinte + Oktave
    h   = g + Terz
    b2  = c + Naturseptime
TONSYSTEM
    C_Dur_Rein = 60 [ c,des,d,es,e,f,fis,g,as,a,b,h ] Oktave
UMSTIMMUNG
    Natur =  [ @,@,@,@,@,@,@,@,@,@, b2 ,@ ]
LOGIK
    C_Normal Taste C = C_Dur_Rein [ ]
    C_Natur  Taste N = Natur [ ]
\end{verbatim}
In diesem Programm wird zunächst ein Tonsystem mit einem 'reinen'
C-Dur programmiert, wie Sie es aus dem Kapitel "`Tonsysteme"'
bereits kennen. Zusätzlich wird das Intervall Naturseptime als \sourcecode{7:4}
und der Ton \sourcecode{b2} als \sourcecode{c+Naturseptime} deklariert. Nun folgt eine
Umstimmung: die Umstimmung \sourcecode{Natur} setzt die Taste b auf den Ton
\sourcecode{b2}, also eine Naturseptime über c. Alle anderen Töne der FT
bleiben bei der Umstimmung unverändert (Klammeraffensymbol).
Zum Umschalten wurden hier einfach zwei aktionslose Logiken
deklariert, die die beiden Zustände \sourcecode{Natur} und \sourcecode{C\_Dur\_Rein} als
Einstimmung benutzen. Sie können jetzt per Knopfdruck
zwischen den beiden Septimen wechseln.
Wenn Sie aber gerne beide Hände zum Spielen frei haben möchten,
und dennoch nicht auf den "`Septimenwechsel"' verzichten wollen, so

müssen Sie sich eine Automatik programmieren.
Überlegen wir uns dazu, welche Harmonien die Auslöser für
Umstimmungen sind.
Die Naturseptime soll in einem Zweiklang c-b erklingen.
Deklarieren wir also die Harmonie\index{Septakkord}
\begin{verbatim}
HARMONIE    Septakkord = { 0, *4, *7, 10 }
\end{verbatim}
und die Aktion
\begin{verbatim}
LOGIK
    Mutierend Taste M = C_Dur_Rein [ Septakkord -> Natur ]
\end{verbatim}
Was bedeutet dies? Die Harmoniedeklaration sollte klar sein. Ein
Septakkord ist das Tastenmuster, bei dem auf jeden Fall die 0. und
die 10. Taste der Projektionstonleiter gedrückt sind, eventuell
die 4. und/oder die 7. Taste, aber keine anderen. Wenn der
Computer während des Spiels diese Harmonie erkennt, so soll die
Umstimmung \sourcecode{Natur} durchgeführt werden - das ist alles.
Diese Logik benutzt als Einstimmung das 'normale' C-Dur und
schaltet auf Naturseptime, sobald ein C-Septakkord erkannt wurde.

Die Schwäche
dieses Programms liegt in der Tatsache, dass der Zustand
"`Naturseptime"', wenn er einmal erreicht ist, nicht mehr verlassen
werden kann; aber gerade das wollen wir ja erreichen: z.\,B. beim
einstimmigen Melodiespiel soll das b wieder als Quinte unter f
interpretiert werden. Wir brauchen also noch eine zweite Aktion,
die in den alten Zustand zurückschaltet, und zwar genau dann, wenn
kein Septakkord mit c-b angeschlagen wird.
Hierzu deklarieren wir die "`Harmonie"' \sourcecode{Be} mit
\begin{verbatim}
HARMONIE
    Be = { 10 }
\end{verbatim}
(Hier erkennen Sie deutlich, dass der Begriff "`Harmonie"' eher im
Sinne von "`Tastenmuster"' zu verstehen ist.) Schließlich fügen wir
der Logik die Aktion
\begin{verbatim}
LOGIK
    Mutierend Taste M = C_Dur_Rein [ Septakkord -> Natur
                                             Be -> Normal ]
\end{verbatim}
hinzu. Damit haben wir unser Ziel erreicht. Zwecks Übersichtlichkeit
hier nochmal der Steuerungsteil des
Programms\footnote{Beispielprogramm c\_b.mut}:
\label{CB}
\begin{verbatim}
UMSTIMMUNG
    Natur =  [ @,@,@,@,@,@,@,@,@,@, b2 ,@ ]
    Normal = [ @,@,@,@,@,@,@,@,@,@, b,@ ]
HARMONIE
    Septakkord = { 0, *4, *7, 10 }
    Be = { 10 }
LOGIK
    Statisch  Taste s = C_Dur_Rein [ ]
    Mutierend Taste m = C_Dur_Rein [ Septakkord -> Natur
                                     Be         -> Normal ]
\end{verbatim}
Programmieren Sie diese Logik einmal und spielen Sie sie dann
folgendermaßen:
\begin{enumerate}
\item Durch Drücken der Taste S aktivieren Sie die Logik \sourcecode{
    Statisch}. Schlagen Sie das Intervall f-b an. Sie hören eine reine
  Quart. Die Septime c-b wird aber nicht als Naturseptime intoniert.
\item Aktivieren Sie die Logik \sourcecode{Mutierend}. Schlagen Sie wieder
  das Intervall f-b an. Die Quart ist rein.
\item Drücken Sie den Ton b. Wenn Sie daraufhin den Ton c mit einem
  sehr sanften Anschlag betätigen, können Sie hören, wie \mutabor{} den
  Ton b zur Naturseptime umstimmt, da die Harmonie \{0,10\} erkannt
  wurde.
\item Lassen Sie das c wieder los; sie hören nun, wie das b wieder
  umgestimmt wird - die Harmonie \{10\} wurde vom Computer erkannt.
\end{enumerate}
\subsection{Eine Randbemerkung}\label{sec:eine-randbemerkung-1}
Bitte beachten Sie beim Programmieren von eigenen mutierenden
Stimmungen, dass das Laufzeitmodul von \mutabor{} die
Aktionsvorschriften \emph{nach jeder Veränderung des Klaviaturzustands}
durchgeht. Welche Konsequenzen dies hat, können Sie an dem
Septimenbeispiel erfahren:

Aktivieren Sie die Logik "`Mutierend"' und schlagen Sie c-b an. Es
erklingt die Naturseptime. Wenn Sie nun das Intervall f-b
anschlagen, so erwarten Sie eine nicht reine Quart, was auch im
Sinne der Umstimmautomatik liegt, da ja nur bei einem einzelnen b
wieder auf die normale Stimmung zurückgeschaltet werden soll. Und
hier taucht nun ein prinzipielles Problem auf. Wenn Sie das
Intervall f-b so anschlagen, dass zuerst das f und dann das b
erklingt, so werden Sie das b als Naturseptime über c hören. Wenn Sie
dasselbe Intervall aber in der Reihenfolge b, f anschlagen, so
erklingt die Quart wieder rein.

Dieser Effekt resultiert aus der
Tatsache, dass \mutabor{} bei jeder Änderung des \index{Klaviaturzustand!Änderung des}
Klaviaturzustandes die Aktionseinträge durchgeht und eventuell
umstimmt. Wenn Sie das b vor dem f anschlagen, so haben Sie für
einen kurzen Moment den Zustand {10} vorliegen - und der führt
laut Umstimmautomatik zur Umstimmung "`Normal"'. Solange Sie sehr
langsam spielen, können Sie solche Seiteneffekte leicht
kontrollieren. Wenn Sie aber mehrere Töne gleichzeitig anschlagen,
so ist die Reihenfolge nicht mehr kontrollierbar. Programmieren
Sie Ihre Logiken also möglichst so, dass solche als "`Hauptmuster"'
fehl gedeuteten "`Übergangsmuster"' nicht auftreten können. (Falls
Sie eine Lösung für dieses prinzipielle Problem haben, so
schreiben Sie uns bitte!)

\section{Differenziertere Harmonieanalyse}
\label{UNMOEGLICHER}
\index{Harmonieanalyse!differenziert} Die gewöhnliche Harmonie- oder
besser: Tastaturanalyse berücksichtigt nur einen Vergleich mit der
Projektionstonleiter.  Haben wir z.\,B. ein gewöhnliches 12-stufiges
Tonsystem programmiert, so erlaubt eine "`Harmonieanalyse"' im Sinne
tonaler Dur-Moll-Harmonik nur die Erkennung des Akkordgrundtones. Auf
Lage oder Umkehrung wird keinerlei Rücksicht genommen. Ob man den
C"=Dur"=Akkord in enger oder in weiter Lage, mit der Quinte als
Grundton, mit Terzverdoppelung, oder wie auch immer anschlägt, alles
fällt unter die Kategorie "`Dur = \{ 0,4,7 \}"', sofern nur die Tasten
c, e und g irgendwie gedrückt sind.

Um eine differenziertere Klaviaturanalyse zu er\-mög\-li\-chen, stellt
das Lauf\-zeit\-mo\-dul zusätzlich Informationen über die höchste und
die tiefste momentan gedrückte Taste zur Verfügung. Diese
Informationen können als zusätzliche Bedingungen bei der
Harmonieerkennung benutzt werden.

Allerdings haben Sie keinen Zugriff auf die absolute Tastennummer,
sondern auf die Nummer der entsprechenden FT-Taste, also die Tonigkeit. Wie alle
anderen Tasten bei der Harmonieerkennung werden auch die beiden
"`Randtasten"' als in die FT projiziert angegeben. Sie können also
nicht abfragen, ob die tiefste gedrückte Taste ein dis'' ($=75$)
ist, sondern nur, ob der soundsovielte Ton des Projektionstonleiter
gedrückt ist\footnote{Was auch bedeutend mehr Sinn ergibt}
Ein Beispiel:
\begin{verbatim}
HARMONIE
    D_Dur = { 2,6,9 }
UMSTIMMUNG
    Gelb = [ ] MiniOktave
LOGIK
    Braun Taste b = Zwoelf_Stufig [ D_Dur -> Gelb ]
\end{verbatim}
Dies ist eine simple Stimmungslogik, die beim Erkennen eines 
D-Dur-Akkordes die Umstimmung \sourcecode{Gelb} durchführt und damit das
Periodenintervall auf \sourcecode{MiniOktave} verändert. Hierbei spielen Lage\index{Lage}\index{Umkehrung}
und Umkehrung des Akkordes keine Rolle\footnote{Die Logik benötigt,
falls Sie sie ausprobieren wollen, irgendein 12-stufiges Tonsystem
mit Verankerungstaste 60 als Einstimmung, da sonst die Harmonieerkennung nicht
funktioniert, siehe Abschnitt "`Verändern der Breite der FT"', und
außerdem eine Definition des Intervalls "`MiniOktave"'.}. Hauptsache
ist, dass die Tasten d, fis und a irgendwo gedrückt sind.

Wir können diese Aufrufbedingung nun einschränken, indem wir die
Automatik so programmieren, dass nur bei einem D-Dur-Akkord mit dem
fis als tiefstem Ton die Umstimmung stattfindet. Dies formuliert
man so:
\begin{verbatim}
HARMONIE
    D_Dur = { 2,6,9 }
UMSTIMMUNG
    Gelb = [ ] MiniOktave
LOGIK
    Braun Taste b = Zwoelf_Stufig [ 6 ~ D_Dur -> Gelb ]
\end{verbatim}
Eine 6 vor der Harmonie D\_Dur bedeutet, dass die tiefste Taste 'vom
Typ 6' sein muss. Die Tonigkeits-Nummer der tiefsten Taste und der
Harmoniename werden durch das Tilde-Zeichen ($\sim$) getrennt.
Beachten Sie, dass im folgenden Beispiel die Umstimmung Meier nie
erreicht werden kann:
\begin{verbatim}
HARMONIE Moll = {0,3,7}
UMSTIMMUNG Meier = 34 [ ]
LOGIK Nie Taste n = [ 2 ~ Moll -> Meier ]
\end{verbatim}
(Die Harmonie Moll wird bei gedrückter 2 nicht erkannt, und eben
eine gedrückte 2 ist Voraussetzung für das Aktivieren der
Umstimmung Meier)
Doch bleiben wir bei normalen Fällen:
Wenn der D-Dur-Akkord als höchste Taste ein d haben soll, so
schreibt man:
\begin{verbatim}
LOGIK
    Braun Taste b = Zwoelf_Stufig [ D_Dur ~ 2 -> Gelb ]
\end{verbatim}
Mit Hilfe der erweiterten Harmonieanalyse ist es also möglich,
nicht nur den Typ eines Akkordes, sondern auch dessen Lage und
Umkehrung zu berücksichtigen.

\chapter{Komplexe Stimmungslogiken}\label{cha:kompl-stimm}

In diesem Teil des Handbuches zu \mutabor{} haben Sie die
grundlegenden Programmierkenntnisse erworben, die es Ihnen
ermöglichen, beliebige statische Tonsysteme zu programmieren. Darüber
hinaus können Sie einfache Harmonieerkennungen programmieren, die
beliebige Parameter des momentanen Tonsystems verändern können.

Die Programmiersprache von \mutabor{} enthält noch
diverse wei\-ter\-füh\-ren\-de Konstruktionen, die bisher noch nicht
erwähnt wurden. Alle weiteren Möglichkeiten
erfahren Sie im nun folgenden Teil des Handbuches.

Es ist jedoch nicht sinnvoll, diesen Teil zu studieren, ohne bereits
einige praktische Erfahrungen im Programmieren von Tonsystemen und
einfachen mutierenden Stimmungen gemacht zu haben. Falls Sie also
gerade dabei sind, dieses Handbuch im "`Trockendurchgang"' zu
studieren, so möchten wir Sie bitten, mit dem Lesen einen Moment
aufzuhören und alles, was Sie bisher über das Programmieren von
Stimmungslogiken gelernt haben, einmal in der Praxis auszuprobieren.
Dies kostet zwar Zeit, aber Klavierspielen erlernt man ja auch nicht
durch das Lesen von Noten, sondern nur durch praktisches Üben. Sie
werden sehr schnell ein Gefühl für mikrotonale Strukturen entwickeln
und Stimmungen nach Ihren eigenen Wünschen entwerfen -- oder
historische Stimmanweisungen nachvollziehen und auf Ihrer MUTABOR II
einstellen und ausprobieren. Erst wenn Sie mit diesen einfachen
Programmiertechniken wirklich vertraut sind, sollten Sie sich auf den
nun folgenden Teil des Handbuches stürzen.


%
%===============================================================
%


\part{Fortgeschrittenes Programmieren}\label{part:fortg-progr}
\label{teilFortgesch}

\chapter{Resum\'{e}e}\label{cha:resumee}
Dieser Teil der Dokumentation zu \mutabor{} befasst sich mit
weiterführenden Programmiertechniken zum Erstellen komplexer
mutierender Stimmungslogiken.

Dieses Kapitel zeigt noch einmal in Kurzform die grundlegenden
Programmiermöglichkeiten von statischen und einfachen mutierenden
Stimmungslogiken auf.

\section{Statische Tonsysteme}\label{sec:statische-tonsysteme}

Ein Tonsystem besteht aus einer Verankerungstaste, einer Menge von
Tönen und einem Periodenintervall. Töne stehen in fest definierten
intervallischen Zusammenhängen zueinander, wobei mindestens ein Ton
eine absolute Frequenz derart zugewiesen bekommt, dass sich mittels
der intervallischen Beziehungen für jeden Ton eine eindeutige
Frequenz berechnen lässt. Um diese Berechnungen braucht sich der
Anwender nicht zu kümmern, er gibt lediglich die Intervallbeziehungen
zwischen den Tönen an und gruppiert dann die gewünschten Töne zu
Tonsystemen.

\subsection{Beispiel: Pythagore\"{\i}sche Tonleiter auf den weißen Tasten}\label{sec:beisp-pyth-tonl}\index{Tonleiter!Pythagoreische}
\label{PYTHAGO}

\begin{verbatim}
INTERVALL
	   Oktave = 2:1
	   Quint = 3:2
TON
	   c = 260.1
	   d = g + Quint - Oktave
	   e = a + Quint - Oktave
	   f = c - Quint + Oktave
	   g = c + Quint
	   a = d + Quint
	   h = e + Quint
TONSYSTEM
	   Pythago = 60 [ c, ,d, ,e,f, ,g, ,a, ,h ] Oktave
\end{verbatim}

Die schwarzen Tasten der Klaviatur bekommen keinen Ton zugewiesen,
sind also stumm geschaltet\footnote{Beispielprogramm pythago.mut}.

Um dieses Tonsystem im Laufzeitmodul spielen zu können, müssen wir
noch eine Stimmungslogik hinzufügen, die unser Tonsystem als
Einstimmung benutzt und ansonsten "`leer"' ist. Außerdem definieren
wir bei dieser Gelegenheit einen Auslöser, welcher die Logik
aktivieren soll, und zwar sinngemäß die Taste P.

\begin{verbatim}
LOGIK  Pythagoras TASTE P = Pythago [ ]
\end{verbatim}

Genaue Definitionen über die Syntax finden Sie im Referenzhandbuch.
Hier stehen alle möglichen Sprachkonstruktionen in grafischer
Darstellung aufgelistet. Dieses Handbuch legt die Funktionalität der
Logiksprache fest. Wenn Sie feststellen, dass eine programmierte Logik
nicht so funktioniert, wie Sie es wünschen, so ist das
Referenzhandbuch die Entscheidungsinstanz, anhand derer Sie
feststellen können, ob das Logikprogramm falsch ist, oder \mutabor{}
nicht korrekt funktioniert.

Es sei hierbei darauf hingewiesen, dass der \mutabor{}"=Compiler nicht
zwischen Groß- und Kleinschreibung unterscheidet. Die Stimmungslogik

\begin{verbatim}
LoGiK  PyThAgorAs TAstE p = PytHAgo [ ]
\end{verbatim}

ist also mit der obigen völlig identisch.

\section{Einfache Umstimmungen}\label{sec:einf-umst}

Eine Umstimmung definiert eine Veränderung eines oder mehrerer
Parameter des aktuellen Tonsystems. Wenn nur ein Parameter verändert
werden soll, so benutzen Sie eine einfache Umstimmung, falls Sie
mehrere Parameter verändern möchten, so müssen Sie die einzelnen
Parameteränderungen als einzelne Umstimmungen programmieren und in
einem Umstimmungsbund in der gewünschten Reihenfolge auflisten. Ein
Umstimmungsbund darf außerdem Tonsystemnamen beinhalten, wobei bei
Ambivalenzen zugunsten der Umstimmung entschieden
wird.\footnote{Namensmehrdeutigkeiten werden vom Compiler in Form einer
Warnung angezeigt.}

Jeder Parameter kann entweder einen neuen, absoluten Wert zugewiesen
bekommen, also z.\,B.

\begin{verbatim}
UMSTIMMUNG
	    Anker49  = 49 [ ]
	    Breite17 = [ << 17 >> ]
	    TonC     = [ c ]
	    PeriodeQ = [ ] Quinte
\end{verbatim}

oder der neue Wert soll sich aus dem aktuellen Zustand berechnen:

\begin{verbatim}
UMSTIMMUNG
	    Ankerplus  = @ + 1 [ ]
	    Breitehalb = [ << @/2 >> ]
	    Septneu    = [ @,@,@,@,@,@,@,@,@,@,@-Septkomma,@ ]
	    Persil     = [ ] @ + 4cent
\end{verbatim}

Wenn eine Umstimmung die Töne der Fundamentaltonleiter verändern
soll, ist natürlich auch eine Mischung von absoluten und relativen
Umstimmungen möglich:
\index{Umstimmung!relativ}\index{Umstimmung!absolut}

\begin{verbatim}
UMSTIMMUNG
	    Toene_neu = [ @, ,@+Komma, ,e,f,fis, @+12Quint-7Oktave]
\end{verbatim}

Wenn eine Taste keinen Wert zugewiesen bekommt, so wird sie gesperrt.
Es gibt daher zwei Umstimmungsarten, die leicht verwechselt werden:

\begin{verbatim}
UMSTIMMUNG
	    Typ1 = [  ,  ,  ,  ,  ,  ,   ]
	    Typ2 = [ @, @, @, @, @, @, @ ]
\end{verbatim}

Bei \sourcecode{Typ1} werden die ersten sieben Tasten stumm geschaltet, bei
\sourcecode{Typ2} behalten die Tasten ihre Frequenzwerte bei, es ändert sich
also nichts.

\section{Logiken}\label{sec:logiken}

Im Teil \ref{teilSprache} dieses Handbuches haben Sie bereits einfache
Methoden zur Harmonieanalyse\footnote{Der verwendete Begriff
  "`Harmonieanalyse"' ist hier nicht im musikwissenschaftlichen Sinne
  gemeint. Vielleicht wäre die Bezeichnung "`Klaviaturanalyse"'
  sinnvoller. Da wir aber in einer Stimmungslogik die
  "`Klaviaturanalyse"' zum Zweck einer -- wenn auch sehr
  eingeschränkten -- tatsächlichen Harmonieanalyse eingeführt haben,
  wollen wir in diesem Handbuch weiterhin den Begriff
  "`Harmonieanalyse"' verwenden, mit dem ausdrücklichen Hinweis, dass
  damit eigentlich eine "`Klaviaturanalyse"' gemeint ist. (Es ist
  durchaus denkbar, dass in zukünftigen Versionen von \mutabor{} eine
  Harmonieanalyse möglich wird, die diesen Begriff auch verdient.)}
kennen gelernt. Zunächst definieren Sie die Harmonien, die bestimmte
Aktionen auslösen sollen, z.\,B.
\begin{verbatim}
HARMONIE
	  C_Dur = { 0, 4, 7 }
	  F_Dur = { 5, 9, 0 }
\end{verbatim}
In einer Stimmungslogik können Sie nun festlegen, welche Aktion beim
Erkennen der entsprechenden Harmonie durchgeführt werden soll, z.\,B.
%\pagebreak[3]
\begin{verbatim}
LOGIK
    Pythagoras_2 Taste P = Pythago [
                            C_Dur -> PythagoC
                            F_Dur -> PythagoF
                           ]
\end{verbatim}



\chapter{Umstimmungen mit Parametern}\label{cha:umst-mit-param}
\index{Umstimmung!Parameter}\index{Parameter}
Es entstehen häufig Situationen, in denen ein bestimmter Parameter
des aktuellen Tonsystems in verschiedenen Umstimmungen verändert 
werden soll.

Betrachten wir folgenden Ausschnitt aus einem Logikprogramm:
\begin{verbatim}
LOGIK
    Modulation Taste M = C_Dur [  Taste B -> Anker70
                                  Taste F -> Anker65
                                  Taste C -> Anker60
                                  Taste G -> Anker67
                                  Taste D -> Anker62
                               ]
UMSTIMMUNG
    Anker60 = 60 [ ]
    Anker62 = 62 [ ]
    Anker65 = 65 [ ]
    Anker67 = 67 [ ]
    Anker70 = 70 [ ]
\end{verbatim}
Das Tonsystem \sourcecode{C\_Dur} ist das bereits aus vielen Beispielen
bekannte "`reine C-Dur"'\footnote{Wir halten es für wichtig, auch an
  dieser Stelle zu betonen, dass Stimmungslogiken auf \mutabor{} in
  keinster Weise an zwölfstufige Tonsysteme oder gar an Prinzipien der
  reinen Stimmung gebunden sind. Die Beispiele im Handbuch sind nur
  der Anschaulichkeit halber meist in Begriffen der zwölfstufigen
  tonalen Harmonik dargestellt, da wohl die Mehrzahl der Leser mit
  diesen Strukturen vertraut (wenn auch nicht unbedingt einverstanden)
  sein dürfte.}, dieses finden Sie auf Seite \pageref{CDUR}.
Bekanntermaßen ist in diesem Tonsystem z.\,B. die Quinte d---a keine
reine Quinte (3:2). Die Stimmungslogik \sourcecode{Modulation} kann nun per
Tastendruck auf einen neuen Bezugston modulieren.\index{Modulieren}
Dies geschieht, indem in der entsprechenden Umstimmung die
Verankerungstaste neu gesetzt wird, und das heißt, dass die
Intervallstruktur des Tonsystems \sourcecode{C\_Dur} auf einen neuen
Bezugston übertragen wird\footnote{Dies lässt sich im Laufzeitmodul
  sehr schön zeigen, indem Sie sich vor und nach einer Modulation im
  Protokollfenster das Tonsystem und die Intervallstruktur anzeigen
  lassen, siehe Begleitbuch "'Die Benutzeroberfläche"` .}. Eine solche
"`Modulation"' verändert also die Töne der Fundamentaltonleiter (bis
auf den neuen Grundton), behält aber die Intervallstruktur bei. Im
Zustand "`C\_Dur"' ist die Quinte d---a unrein, wenn Sie aber nach
"`D\_Dur"' oder "`G-Dur"' modulieren, so ertönt diese Quinte im
Frequenzverhältnis 3:2.

Doch konzentrieren wir uns nun auf die Umstimmungen. Alle fünf
Umstimmungen sind \emph{vom selben Typ}: sie weisen der Verankerungstaste
einen neuen (absoluten) Wert zu. Da es sehr häufig vorkommt, dass
verschiedene Umstimmungen ein und denselben Parameter verändern, haben
wir eine Konstruktion eingeführt, welche alle Umstimmungen gleichen
Typs in einer einzigen zusammenfassen kann: \emph{Umstimmungen mit
Parametern\footnote{Im folgenden bezeichnen die Begriffe 'Parameter' 
und 'Platzhalter' denselben Sachverhalt. An einigen Stellen 
benutzen wir deshalb den Ausdruck 'Platzhalter', um ihn begrifflich 
von den 'Parametern' eines Tonsystems zu unterscheiden.}}. 
Wir schreiben einfach
\begin{verbatim}
UMSTIMMUNG
        Anker_neu ( x )  =  x [ ]
\end{verbatim}
und aktivieren diese Umstimmung mit verschiedenen Werten für den
Platzhalter x:
\begin{verbatim}
LOGIK
    Modulation Taste M = C_Dur [ Taste B -> Anker_neu(70)
                                 Taste F -> Anker_neu(65)
                                 Taste C -> Anker_neu(60)
                                 Taste G -> Anker_neu(67)
                                 Taste D -> Anker_neu(62)
                               ]
\end{verbatim}

Dies gleicht einer einfachen Konstruktion aus der Mathematik: Man
definiert eine Funktion, z.\,B. $ f(x) = \sqrt{x+4} $ und erhält daraus
später konkrete Ergebnisse, indem man feste Werte einsetzt: $ f(12) =
4 $ . Genauso arbeiten Umstimmungen mit Parametern: man definiert den
Typ der Umstimmung und konkretisiert dies \emph{beim
  Aufruf}\index{Aufruf!mit Parametern} in der Logik.

Einen Platzhalter können Sie bei Veränderungen der Verankerungstaste
und bei Veränderungen der Breite der Fundamentaltonleiter benutzen.
Die vier möglichen Umstimmungen sind also:

\begin{verbatim}
UMSTIMMUNG
         Anker_absolut(Neu)     = Neu [ ]
         Anker_relativ(Distanz) = @ + Distanz [ ]
         Breite_absolut(Neu)    = [ << Neu >> ]
         Breite_relativ(Faktor) = [ << @ * Faktor >> ]
\end{verbatim}
%         Periode_absolut        = [ ] Terz + Quinte
%         Periode_relativ        = [ ] @ - 2.4 cent

Natürlich sind bei relativen Umstimmungen mit Parametern alle sonst 
gültigen Rechenzeichen erlaubt (Tatsächlich gibt es also acht 
verschiedene Typen).

Auf der rechten Seite des Gleichheitszeichens dürfen nur solche Namen
benutzt werden, die auch in der Klammer links stehen, da sonst keine
eindeutige Zuordnung möglich ist. Die Umstimmung 
\begin{verbatim} 
    UMSTIMMUNG Versuch_es_doch (Meier) = @ + Schmitt [ ]
\end{verbatim}
führt zu einer Fehlermeldung des Compilers, da der Platzhalter \sourcecode{Schmitt}
nirgendwo definiert worden ist. 

\chapter{Differenziertere Harmonieanalyse}\label{cha:diff-harm}

\section{Harmonieformen}\index{Harmonieform}\label{sec:harmonieformen}
Die Harmonieform ist eine 'translative' Verallgemeinerung der Harmonie auf 
den gesamten Bereich der Projektionstonleiter. Wenn die Harmonie-\emph{Form} 
\{ 0,4,7 \} in einem Tonsystem der Breite 12 und der Verankerungstaste 60 
(=c') als Auslöser z.\,B. für eine Umstimmung eingetragen wurde, so wird 
diese Umstimmung nicht nur beim Erkennen der Harmonie \{0,4,7\} 
durchgeführt, sondern ebenfalls beim Erkennen der Harmonie \{1,5,8\} , 
sowie \{2,6,9\}, \dots{} und auch über die Breite der Fundamentaltonleiter 
hinaus, dann aber modulo derselben: \{9,1,4\}

Tonal interpretiert stellt die Harmonie \{0,4,7\} innerhalb eines 
Tonsystems der Breite 12 und der Verankerungstaste 60 (=c') einen 
C-Dur-Dreiklang dar, während die Harmonie\emph{form}
 \{0,4,7\} als Auslöser 
auf jeden Dur-Akkord (F-Dur, Cis-Dur, \dots{}) reagiert.

Denken wir uns nun eine Stimmungslogik aus, die (ausgehend von einer 
gleichstufigen Temperierung) bei einem beliebigen Durakkord alle \index{Komma!syntonisches}
Töne um ein syntonisches Komma erhöht, und bei einem beliebigen 
Mollakkord alle Töne um dasselbe erniedrigt. Die Umstimmungen hierzu 
sind sehr einfach\footnote{Beispielprogramm aufab.mut}:
\label{AUFAB}
\begin{verbatim}
INTERVALL
       SynKomma = 81:80
UMSTIMMUNG
       Hinauf = [@+SynKomma,@+SynKomma,@+SynKomma,@+SynKomma,
                 @+SynKomma,@+SynKomma,@+SynKomma,@+SynKomma,
                 @+SynKomma,@+SynKomma,@+SynKomma,@+SynKomma]
       Hinab  = [@-SynKomma,@-SynKomma,@-SynKomma,@-SynKomma,
                 @-SynKomma,@-SynKomma,@-SynKomma,@-SynKomma,
                 @-SynKomma,@-SynKomma,@-SynKomma,@-SynKomma]
       Anker60 = 60 [ ]
       Breite12 = [<<12>>]
       Init = { Anker60, Breite12 }
\end{verbatim}
Zusätzlich enthält diese Umstimmungsdeklaration die Umstimmungen \sourcecode{Anker60}, 
\sourcecode{Breite12} und den Umstimmungsbund \sourcecode{Init}, welcher diese beiden der Reihenfolge 
nach durchführt. Zu Beginn der Stimmungslogiken wird der Umstimmungsbund 
\sourcecode{Init} als Einstimmung aufgerufen, um alle Werte zu initialisieren. Um 
die Taste c' als Anker der Fundamentaltonleiter einzustellen, wird die 
Umstimmung \sourcecode{Anker60} aktiviert. Die Umstimmung \sourcecode{Breite12} wird benötigt, um 
die Breite der Fundamentaltonleiter auf zwölf Tasten zu expandieren, da 
die Harmonieanalyse eines Akkordes \{0,4,7\} nicht in einem Tonsystem der 
Breite 1, wie es beim Aufruf des Laufzeitmoduls eingestellt ist, möglich 
ist (siehe Abschnitt "`Handhabung von Grenzfällen"').
In der Stimmungslogik benutzen wir die Harmonien \sourcecode{Dur} 
und \sourcecode{Moll}:
\begin{verbatim}
HARMONIE
	   Dur  = { 0, 4, *7, *10 }
	   Moll = { 0, 3, *7 }
\end{verbatim}

Ein Durakkord besteht hier aus Grundton und großer Terz, mit eventueller 
Quinte und großer Septime; ein Mollakkord aus Grundton, kleiner Terz und 
eventueller Quinte.

Würden wir die Stimmungslogik
\begin{verbatim}
LOGIK AufAb1 Taste A = Init [ Dur  -> Hinauf
                              Moll -> Hinab ]
\end{verbatim}
aktivieren, so würde nur bei einem \textbf{C}-Dur-Akkord bzw. 
\textbf{c}-moll-Akkord die Hinauf- bzw. Hinab-Umstimmung durchgeführt, 
nicht aber bei einem Fis-Dur-Akkord oder einem h-moll-Akkord.
Nur durch die Angabe, dass die Harmonie als Harmonieform zu analysieren ist, 
erreichen wir das Ziel, bei \emph{jedem} Dur- bzw. Mollakkord Umzustimmen:
\begin{verbatim}
LOGIK
       AufAb2 Taste B = Init [ FORM Dur  -> Hinauf
                               FORM Moll -> Hinab ]
\end{verbatim}

In diesem Fall gibt es zwölf äquivalente Logikprogramme, die alle exakt 
den gleichen Prozess durch\-führ\-en. Es spielt jetzt nämlich keine 
Rolle mehr, 
welchen Durakkord wir in der Harmoniedeklaration explizit angeben. Wir 
hätten, da es sich nur um eine Harmonieformanalyse handelt, ebenso die 
Harmonien
\begin{verbatim}
HARMONIE
       Dur  = { 1,5,*8,*11 }
       Moll = { 4,7,*11 }
\end{verbatim}
deklarieren können.

Die Stimmungslogik \sourcecode{Aufab1} hingegen würde dabei nicht mehr auf C-Dur bzw.\ 
c-moll reagieren, sondern auf Cis-Dur bzw.\ e-moll! Nur bei 
Harmonieformen ist also die Wahl des 'Grundtones' der Harmonie zunächst 
nicht von Bedeutung. Trotzdem ist es der Verständlichkeit halber meistens 
besser, einfache Harmonien zu wählen.

\section{Der ausgezeichnete Parameter "`ABSTAND"'}
\index{ABSTAND}\label{sec:der-ausg-param}

In "`ABSTAND"' steht unmittelbar nach einer Harmonieformanalyse die Anzahl 
von Verschiebungsschritten, die notwendig waren, um das Muster der 
Projektionstonleiter auf die verglichene Harmonie abzubilden. 
Verdeutlichen wir uns dies an einem Beispiel:

Das aktuelle Tonsystem sei (wieder einmal) das Tonsystem \sourcecode{C\_Dur} mit der 
Verankerungstaste 60 (=c') und der Breite 12. Eine Harmonieformanalyse der 
Harmonie \{0,3,7\} führt bei jedem beliebigen Mollakkord zu der 
angegebenen Aktion. Eine Stimmungslogik der Form
\begin{verbatim}

HARMONIE 
      Moll = {0,3,7}
LOGIK
      Xantippe Taste X = C_Dur [
                                    FORM Moll -> Umst ( 7 )
                               ]

\end{verbatim}

führt beim Erkennen eines beliebigen Mollakkordes die Umstimmung \sourcecode{Umst}
durch und übergibt an deren ersten Platzhalter den Wert 7. Wir haben hier 
also einen stets konstanten Übergabewert vorliegen. Mit dem 
ausgezeichneten Parameter 'ABSTAND' haben wir einen variablen 
Übergabewert, welcher an den entsprechenden Platzhalter der Umstimmung 
genau die Anzahl von Verschiebungen übergibt, die nötig waren, um die 
Projektionstonleiter auf die abgefragte Harmonie 
abzubilden. Was bedeutet dies?

Nehmen wir an, die Stimmungslogik \sourcecode{Xantippe} sei aktiv, und wir
schlagen einen c-moll-Akkord an. Um die Projektionstonleiter (in ihr
steht die Harmonie \{0,3,7\}) auf die abgefragte Harmonie Moll
abzubilden ist keine Verschiebung nötig, in dem ausgezeichneten
Parameter 'ABSTAND' steht nun \index{Parameter}\index{Übergabewert}
also der Wert 0. Dieser Wert wird an die Umstimmung \sourcecode{Umst}
übergeben. Wir wollen an dieser Stelle noch nicht darauf eingehen, was
diese Umstimmung macht, sondern nur ermitteln, mit welchem
Übergabewert sie aufgerufen wird. Wenn nun ein d-moll-Akkord
angeschlagen wird (die Projektionstonleiter hat nun den Inhalt
\{2,6,9\}), so sind \emph{zwei} Verschiebungen nötig, um die PT auf die
abgefragte Harmonie ( = \{0,3,7\} ) abzubilden, also beide Harmonien
'zur Deckung zu bringen'. Bei einem angeschlagenen fis-moll-Akkord hat
ABSTAND den Wert 6, bei b-moll den Wert 10\footnote{Dies hei"st
  natürlich, dass die Verschiebungen zyklisch durchgeführt werden, da
  der b-moll-Akkord nicht die Harmonie \{10,13,17\} hat, sondern
  \{10,1,5\}. Gerechnet wird also immer modulo 12, bzw.  allgemein
  modulo 'Breite'.}.

Was passiert nun, wenn sich die Verankerungstaste des aktuellen Tonsystem 
verändert, also nicht mehr auf c' steht? In diesem Fall bezieht sich die 
Anzahl nötiger Verschiebungen nicht mehr auf die Taste 60, sondern eben 
auf die neue Verankerungstaste. Setzen wir in unserem Beispiel die 
Verankerungstaste auf 62 (=d'), z.\,B. durch eine Umstimmung, die durch 
einen Tastendruck ausgelöst wurde\footnote{Beispielprogramm xantippe.mut}:
\label{XANTIPPE}

\begin{verbatim}
HARMONIE 
  Moll = {0,3,7}
UMSTIMMUNG
  Anker62 = 62 [ ]
LOGIK
  Xantippe Taste X = C_Dur [ FORM Moll -> Umst(ABSTAND)
                             Taste A   -> Anker62
                           ]
\end{verbatim}
Der Anker des aktuellen Tonsystems sei also 62. Welchen Wert erhält 
ABSTAND bei einem angeschlagenen d-moll-Akkord? Natürlich den Wert 0, da 
die zu vergleichende Harmonie \{0,3,7\} mit der 0 immer die 
Verankerungstaste identifiziert. Ein fis-moll-Akkord setzt ABSTAND auf 4, 
ein c-moll-Akkord auf 10.

Bei einer Harmonieformanalyse ist der Wert, der in ABSTAND festgehalten 
wird, ist also immer gleich der Anzahl Verschiebungen bezogen auf die \index{Verankerungstaste}
momentane Verankerungstaste. Tonal interpretiert liefert uns ABSTAND die 
\emph{Stufe}, auf der der angeschlagene Akkord innerhalb der momentanen 
Tonart (durch den Anker angegeben) steht.

Die Benutzung von ABSTAND ist nur unmittelbar nach einer 
Harmonieformanalyse sinnvoll, da zu einem späteren Zeitpunkt kein Bezug 
mehr zur zuletzt erkannten Harmonieform existiert; dennoch kann ABSTAND 
theoretisch auch ohne Harmonieform-Auslöser benutzt werden, z.\,B.
\begin{verbatim}
LOGIK Meier Taste M = [ Taste X -> Umst(ABSTAND) ]
\end{verbatim}

Eine sinnvolle Anwendung findet der ABSTAND-Parameter bei der
Programmierung einer besonderen mutierenden Stimmung, dem Wandern im
\index{Netz!tonales} tonalen Netz. In der Beschreibung der
Demonstrationslogiken im Anhang finden Sie eine Kurzbeschreibung des
Prinzip des tonalen Netzes.

Dieser Sachverhalt lässt sich in einer sehr kompakten Form 
programmieren\footnote{Beispielprogramm mininetz.mut}:
\label{MININETZ}
\begin{verbatim}
LOGIK
       Tonales_Netz Taste T = C_Dur
           [ FORM Dur  -> Transponiere(ABSTAND)
             FORM Moll -> Transponiere(ABSTAND) ]
UMSTIMMUNG
       Transponiere(Distanz) = @ + Distanz [ ]
HARMONIE
       Dur  = {0,4,7}
       Moll = {0,3,7}
\end{verbatim}

In diesem Programmierbeispiel ist die Harmonieformanalyse noch nicht sehr 
flexibel, da sie nur auf vollständig angeschlagene Dur- bzw. moll-Akkorde 
reagiert. Doch gerade diese Tatsache lässt sich ausnutzen, um Ihnen einen 
Eindruck davon zu vermitteln, wie die Harmonieanalyse bei \mutabor{} 
vonstatten geht. Wenn Sie nämlich die Logik 'Tonales\_Netz' aktivieren und 
die Quinte c---g anschlagen, so wird diese im Frequenzverhältnis 3:2 (rein) 
intoniert. Aufgrund der Konstruktion des tonalen Netzes ist aber die Quinte 
d---a nicht rein. Diesen Effekt sollten Sie einmal ausprobieren. Lassen 
sie die Quinte d---a liegen und schlagen Sie nun ganz kurz den Ton Fis an. 
In diesem Moment erkennt \mutabor{} die Harmonie \{2,6,9\} als eine um 2 
Tasten verschobene Harmonieform von \{0,4,7\} und stimmt um, indem die 
Intervallstruktur des reinen C-Dur nun auf den Ton D übertragen wird. 
Sofort wird die Frequenz des Tones a korrigiert und die Quinte d---a ist 
rein gestimmt.

Für eine Anwendung in der Praxis ist es jedoch wünschenswert, dass auch 
Quinten und Terzen erkannt werden. Hierzu müssen wir die Harmonien neu 
deklarieren\footnote{Beispielprogramm netz.mut}:
\label{NETZ}

\begin{verbatim}
HARMONIE
	 Quinte = {0,7}
	 Dur    = {0,4,*7}
	 Moll   = {0,3,7}
LOGIK 
	 Netz Taste N = C_Dur
	     [ FORM Quinte -> Transponiere(ABSTAND)
	       FORM Dur    -> Transponiere(ABSTAND)
	       FORM Moll   -> Transponiere(ABSTAND) ]
\end{verbatim}

Nun werden Quinten und Terzen sofort erkannt und rein 
gestimmt.\footnote{Hierbei ergibt sich ein sehr netter Effekt: man schlage 
die Terz c---e an. Diese wird von der Harmonieanalyse erkannt und rein 
gestimmt. Nun nehme man das gis hinzu. Aufgrund der Struktur des tonalen 
Netzes wird dieses als as interpretiert und viel höher als die reine Terz 
über e intoniert. Sobald aber das c losgelassen wird, erkennt der Computer 
wieder das Tastenmuster einer Terz und stimmt diese rein: das as wird zum 
gis. Nun können wir dasselbe Spielchen wiederholen, indem wir e---gis---c 
anschlagen (eigentlich e-gis-his) und dann das e loslassen, und so weiter. 
auf diese Weise sackt die Stimmung immer weiter ab.} Sie spielen nun in 
einem mutierenden Tonsystem der reiner Stimmung. Dies ist ein Beispiel 
dafür, dass mutierende Stimmungen nicht zwangsweise etwas exotisches sind, 
sondern durchaus auch in Bereichen tonaler Harmonik angewandt werden 
können.


\chapter{Anweisungen, Auslöser und Aktionen}\label{cha:anwe-ausl-und}
\index{Anweisung}\index{Auslöser}\index{Aktion}
Die allgemeinste Form einer Stimmungslogik lautet

\sourcecode{LOGIK name Auslöser = Einstimmung [ Anweisungen ]}

Eine Anweisung hat immer die Form

\sourcecode{Auslöser -> Aktion}

Bisher haben Sie bereits einige verschiedene Auslösertypen kennen gelernt, 
nämlich Tastendruck, Harmonie und Harmonieform, 
sowie den Aktionstyp Aufruf eines 
Tonsystems/Umstimmung/Logik. In diesem Kapitel werden Sie noch weitere 
Auslöser und Aktionen, sowie Erweiterungen bereits bekannter 
Konstruktionen kennen lernen.

\section{Umstimmungsbünde mit Parametern}\label{sec:umst-mit-param}
\index{Umstimmungsbund!mit Parametern}
übertragen wir nun die Prinzipien der Umstimmung mit Parametern auf 
Umstimmungsbünde und verdeutlichen dies an einem einfachen Beispiel:

Eine Umstimmung soll die Verankerungstaste auf einen neuen, als Platzhalter 
übergebenen Wert setzen und anschließend -- sozusagen 
"`sicherheitshalber"' -- die Breite der Fundamentaltonleiter auf den Wert 9 
setzen. Wenn die Breitenkorrektur nicht notwendig wäre, könnten wir einfach
\begin{verbatim}
UMSTIMMUNG Mephisto(Faust) = Faust [ ]
\end{verbatim}
schreiben. Da aber außerdem noch der Parameter 'Breite' verändert werden 
soll, müssen wir einen Umstimmungsbund benutzen. Ein Umstimmungsbund kann 
-- genau wie eine normale Umstimmung -- Platzhalter benutzen. Diese werden 
hinter dem Namen des Umstimmungsbundes in runden Klammern angegeben. 
Ein Umstimmungsbund dient nur als Zwischenstation der tatsächlichen 
Umstimmung, da die eigentlichen Umstimmungen nicht im Bund, sondern in den 
einzelnen Umstimmungen passieren. Der Umstimmungsbund kann die ihm 
übergebenen Platzhalter also nicht selber benutzen, sondern nur an die 
Einzelumstimmungen weitergeben. Wir können nun unseren Umstimmungsbund 
formulieren:
\begin{verbatim}
UMSTIMMUNG
         Anker_neu_setzen(wert) = wert [ ]
         Breite_fest_setzen     = [ << 9 >> ]
         Mephisto(Faust)        = { Anker_neu_setzen(Faust),
                                    Breite_fest_setzen }
\end{verbatim}
Überlegen wir uns nun, was passiert, wenn wir von einer Logik aus 
diesen Umstimmungsbund mit dem Übergabewert 6 aufrufen, z.\,B. durch 
Drücken der Taste X bei folgender (als aktiv angenommener) Stimmungslogik:
\begin{verbatim}
LOGIK Goethe Taste G = [ Taste X -> Mephisto(6) ]
\end{verbatim}
Die Umstimmung \sourcecode{Mephisto} werde mit dem 
Übergabewert 6 aufgerufen. Im 
Platzhalter 'Faust' wird also der Wert 6 festgehalten. 
Die Angaben im Umstimmungsbund besagen nun, dass zunächst die Umstimmung 
\sourcecode{Anker\_neu\_setzen} mit dem in 'Faust' gespeicherten Wert -- in 
unserem Fall 6 -- durchgeführt werden soll. Was passiert nun 
in der Umstimmung \sourcecode{Anker\_neu\_setzen} ? Diese wird mit dem 
Übergabewert 6 aufgerufen, 
welcher dann im Platzhalter 'wert' vermerkt wird; 
und laut Vorschrift wird die Verankerungstaste auf eben diesen Wert gesetzt. 
Wenn dies geschehen ist, so ist nun gemäß unseres Umstimmungsbundes die 
Umstimmung \sourcecode{Breite\_fest\_setzen} an der Reihe. Diese setzt 
die Breite der FT absolut auf den Wert 9.

Dieses Prinzip lässt sich nun um noch eine Stufe erweitern. 
Ein Umstimmungsbund kann auch \emph{mehrere} Platzhalter verwenden. 
Dies ist notwendig, wenn z.\,B. nacheinander zwei Umstimmungen aufgerufen 
sollen, die ihrerseits Platzhalter benutzen. Soll z.\,B. zunächst die 
Verankerungstaste auf einen (als Platzhalter zu übergebenden) Wert gesetzt 
werden, und danach die Breite der Fundamentaltonleiter neu eingestellt 
werden, und zwar ebenfalls variabel mit Platzhalterübergabe, so muss der 
Umstimmungsbund, der beide Umstimmungen nacheinander durchführt, zwei 
Platzhalter benutzen, deren Werte dann an die Umstimmungen weitergegeben 
werden. Im Programmtext also:
\begin{verbatim}
UMSTIMMUNG
         Anker_neu ( wert )  = wert [ ]
         Breite_neu ( wert ) = [ << wert >> ]
         Aufruf(para1,para2) = { Anker_neu(para1),
                                 Breite_neu(para2) }
\end{verbatim}
Wenn dieser Umstimmungsbund in einer Logik aufgerufen werden soll, 
so müssen (zwingend!) auch \emph{zwei} Werte übergeben werden, z.\,B.:
\begin{verbatim}
LOGIK Meier Taste M = [ Taste X -> Aufruf(64,12) ]
\end{verbatim}
Das Drücken der Taste X bewirkt also nacheinander das Neusetzen des 
Ankers auf 64 und der Breite auf 12.

Ein Umstimmungsbund kann beliebig viele Platzhalter benutzen, sofern 
sichergestellt ist, dass
\begin{itemize}
\item jeder rechts vom Gleichheitszeichen benutzte Platzhalter auch links in 
den Klammern angegeben wurde,
\item der Aufruf mit genau der Zahl an Werten passiert, wie 
Platzhalter angegeben sind.
\end{itemize}

Ferner gelten folgende Freiheiten:
\begin{itemize} 
\item Die Reihenfolge, in der die Platzhalter in den Klammern angegeben 
wurden, muss nicht mit der Reihenfolge übereinstimmen, in der diese 
benutzt werden.
\item Ein Platzhalter darf mehrmals für Aufrufe benutzt werden.
\end{itemize}

Alle folgenden Umstimmungsbünde sind syntaktisch korrekt (wenn auch 
praktisch unbrauchbar):
\begin{verbatim}
UMSTIMMUNG
    Anker (neu) = neu [ ]
    Breite(neu) = [<< neu >>]
    Periode     = [ ] Quint

    Simpel(x,y)={ Breite(y) , Anker (x) }
    Naja(a,b,c)={ Breite(b) , Anker (c), Periode, Anker(b),
                  Breite(a) }
\end{verbatim}


\section{Auswählende Umstimmungsbünde}\label{sec:ausw-umst}
\index{Umstimmung!auswählend}
Ein Umstimmungsbund ist eine sequenzielle Auflistung von 
Umstimmungen\footnote{oder Tonsystemen}, die nacheinander 
durchgeführt werden sollen. Wir wollen nun noch eine weitere 
Konstruktion eines Umstimmungsbundes einführen: den \emph{auswählenden 
Umstimmungsbund}.

Ein auswählender Umstimmungsbund ist ein Umstimmungsbund, dem ein 
besonderer Parameter übergeben wird, der ein Auswahlkriterium dafür 
darstellt, \emph{welche} der im Bund aufgelisteten Umstimmungen durchgeführt 
werden sollen. Dieser Parameter ist (wie gewöhnlich) eine ganze Zahl, 
die durch die aufrufende Logik entweder explizit angegeben oder in der 
Form des variablen ABSTAND-Parameter übergeben wird.

Innerhalb des auswählenden Umstimmungsbundes stehen dann 
Wenn"=Dann"=Zuweisungen, die angeben, welche Umstimmungen bei 
welchem Wert des Auswahlparameters durchgeführt werden sollen. Doch 
nun erst einmal ein Beispiel:

\begin{verbatim}
UMSTIMMUNG
    Auswahl ( X ) = X { 1 -> Susi
                        2 -> Otto
                        3 -> Fritz
                        7 -> Irene }
\end{verbatim}  

Wenn dieser Umstimmungsbund mit dem Übergabewert 3 aufgerufen wird, also 
\sourcecode{Auswahl(3)} , so wird die Umstimmung \sourcecode{Fritz} durchgeführt. Bei 
einem Übergabewert von 7 wir entsprechend \sourcecode{Irene} aktiviert. Jeder 
Wert, der nicht 1,2,3 oder 7 ist bewirkt keine Umstimmung.

Die Syntax dieser Konstruktion erinnert an eine Mischung aus 
Stimmungslogik und Umstimmungsbund. Und genau das stellt der auswählende 
Umstimmungsbund auch dar. Wieder haben wir eine Art "`Auslöser"' und 
eine "`Aktion"', nämlich den Vergleichswert und die angegebene Umstimmung. 
Vor der geschweiften Klammer wird der Name des Parameters angegeben, 
der mit den in den Klammern angegebenen Werten verglichen werden soll.

Diese Konstruktion können wir nun erweitern, denn es kann durchaus 
erwünscht sein, dass auch der auswählende Umstimmungsbund mehr als 
nur einen Parameter bekommt, und alle bis auf den auswählenden 
Parameter an die aufgelisteten Umstimmungen weiter gibt. Hier wird 
ganz analog den mehrparametrigen Umstimmungsbünden verfahren, und dabei 
bekommt die explizite Angabe des Namens des Auswahlparameters vor 
den geschweiften Klammern ihren Sinn:

\begin{verbatim}
UMSTIMMUNG
   Transpo ( X , Wohin ) = X { 0 -> Transponiere_hoch(Wohin)
                               1 -> Transponiere_runter(Wohin) }
\end{verbatim}

Wird diese Umstimmung mit den Übergabewerten \sourcecode{Transpo(1,6)} 
aufgerufen, so wird die Umstimmung \sourcecode{Trans\-po\-nie\-re\-\_run\-ter} mit dem 
weitergereichten Übergabewert 6 durchgeführt.

Wie in einem Umstimmungsbund können Sie auch mehrere Umstimmungen 
auflisten, die beim Erkennen eines Vergleichswertes nacheinander ausgeführt \index{Umstimmung!Bund}
werden sollen. Dies erspart die Programmierung eines zusätzlichen 
Umstimmungsbundes, falls mehrere Parameter verändert werden sollen:

\begin{verbatim}
UMSTIMMUNG
    Auswahl(X) = X { 0 -> Halb1,Expandiere
                     1 -> Reset
                     2 -> Reset
                     3 -> Reset }
\end{verbatim}

Der auswählende Parameter muss nicht unbedingt als erster in der 
Parameterliste angegeben werden. Deshalb wird sein Name vor die 
geschweifte Klammer geschrieben. Außerdem können Sie in einem 
auswählenden Umstimmungsbund auch die ANSONSTEN--Option benutzen. 
Dies bedeutet, dass die der ANSONSTEN--Option zugeordneten Umstimmungen 
durchgeführt werden, wenn kein anderer Vergleich passt:

\begin{verbatim}
UMSTIMMUNG
    Verschiebe(X) = X { 0         -> Halb1,Expandiere
                        ANSONSTEN -> Transponiere }
\end{verbatim}

Hier würden also im Falle von X=0 nacheinander die Umstimmungen \sourcecode{Halb1} 
und \sourcecode{Expandiere} durchgeführt, und \emph{ansonsten} immer 
die Umstimmung \sourcecode{Transponiere}.

Insbesondere im Zusammenhang mit dem ABSTAND--Parameter und der 
ANSONSTEN"=Option bildet der aus\-wähl\-en\-de Umstimmungsbund 
vielfältige Möglichkeiten. Man denke z.\,B. an eine mutierende Stimmung, 
die Septime des Dominantseptakkordes zu groß nehmen soll. 
Mit Hilfe des auswählenden Umstimmungsbundes ist 
dies kein Problem\footnote{Beispielprogramm auswahl.mut}:

\label{AUSWAHL}
\begin{verbatim}
LOGIK
     Septbetonung Taste S = Halb12 [ FORM Sept -> Betone(ABSTAND) ]

HARMONIE
     Sept = { 0,4,*7,10 }

UMSTIMMUNG
     Betone(dist) = dist { 7         -> GrosseSept
                           ANSONSTEN -> Halb12 }

     GrosseSept = [ @,@,@,@,@,@+Faktor ]

INTERVALL
     Faktor = 24 Wurzel 2

UMSTIMMUNG
     Halb12     = { Halb,Expandiere }
     Expandiere = [ << 12 >> ]

TON
     a = 440

TONSYSTEM
     Halb = 69 [ a ] 2 Faktor
\end{verbatim}



\section{Kommunikation -- MIDIOUT und MIDIIN}\label{sec:komm-midi-und}
\label{STATUSBYTE}
\index{Kommunikation!MIDI}\index{MIDIIN}\index{MIDIOUT}
Normalerweise verarbeitet \mutabor{} nur die MIDI-Informationen 'NOTE ON' 
und 'NOTE OFF', also nur Informationen darüber, welche Tasten auf der 
Klaviatur angeschlagen bzw. losgelassen wurden. Alle anderen 
MIDI-Informationen, z.\,B. Program Change -- Meldungen, etc\dots{} werden 
ignoriert.

Sie können jedoch solche Meldungen gezielt als Auslöser verwenden. 
Auf diese Weise können Sie \mutabor{} von Ihrem Masterkeyboard aus 
"`fernsteuern"'. Oder Sie können mit Hilfe eines Sequenzer-Programms auf einem 
zweiten Computer alle Umstimmvorgänge gezielt beeinflussen.

Der Auslöser MIDIIN kann an allen Stellen eingesetzt werden, 
wo ein Auslöser erforderlich ist, also als Auslöser einer 
Stimmungslogik oder als Bestandteil einer Anweisung innerhalb einer 
Stimmungslogik. Ein Beispiel:
\begin{verbatim}
LOGIK Otto MIDIIN ( #C0, #06 ) = C_Dur [ ]
\end{verbatim}
Das Doppelkreuz \# bedeutet, dass die darauf folgende Zahl zur Basis 16
interpretiert werden soll (Sedezimal, Hexadezimal). \#C0 entspricht
also (dezimal) 182.\footnote{In der MIDI-Dokumentation sind die
  Angaben meistens im hexadezimalen Zahlenformat angegeben, und um
  Ihnen das Umrechnen zu ersparen, können Sie MIDI-Bytes direkt im
  Hex-Format eingeben.} In diesem Beispiel ist der Auslöser für die
Logik Otto \index{MIDI!event} der MIDI-Event \#C0 \#06. Das bedeutet,
dass die Logik Otto aktiviert wird, wenn über MIDI die Zahlenfolge 182,
6 an \mutabor{} geschickt wurde. Auf diese Weise können Sie z.\,B. Program
Change-Meldungen benutzen, um Logiken umzuschalten oder Umstimmungen
durchzuführen. Die meisten Masterkeyboards können gezielte Program
Change Meldungen senden, so dass diese richtiggehend ideal zum
Umschalten von Logiken und Umstimmungen benutzt werden können.

Das Gegenstück zum MIDIIN-Auslöser ist die MIDIOUT-Aktion. Hier
werden\index{Aktion!MIDIOUT} die angegebenen Daten direkt an die
MIDI-Schnittstelle ausgegeben. Damit können dem angeschlossenen
Synthesizer/Sampler Spezialkommandos z.\,B. zur Änderung der
Klangfarbe übermittelt werden. Die Syntax ist analog zur
MIDIIN-Anweisung:
\begin{verbatim}
LOGIK Meier Taste M = [ Taste Q -> MIDIOUT(160,#54,23) ]
\end{verbatim}
Hier wird, wenn die Logik Meier (durch Drücken der Taste M) aktiv ist, 
beim jedem Drücken der Taste Q an die MIDI-Schnittstelle die 
Zahlenfolge 160, 84\footnote{Entspricht hexadezimal \#54 !} 
und 23 ausgegeben.

Bei dem Auslöser \sourcecode{MIDIIN} unterliegt die Folge von MIDI-Bytes
folgenden Einschränkungen 
im Sinne des MIDI-Stan\-dards:
\begin{itemize}
\item Die erste Zahl muss ein Statusbyte sein, also im Bereich zwischen
  128 und 255 liegen. Bei MIDIIN ist es außerdem nur sinnvoll, Zahlen
  aus den Bereich zwischen 160 und 239 zu wählen, da die MIDI-Bytes
  128 bis 159 für NOTE-ON bzw. NOTE-OFF-Meldungen reserviert ist und
  nicht in die Analyse miteinbezogen wird. Außerdem sind
  Systemmeldungen \index{Systemmeldungen} (größer 239) ebenfalls von
  der Analyse ausgeschlossen.
\item Das untere Nibble (die unteren 4 Bits) des Statusbytes muss 0
  sein; falls es das nicht ist, wird der Wert vom Compiler entsprechen
  korrigiert und eine Warnung ausgegeben. Praktisch gelten also nur
  die Zahlen \#A0, \#B0, \#C0, \#D0 und \#E0\footnote{Diese
    anscheinende Beschränkung auf nur fünf verschiedene MIDI-Auslöser
    trügt. Da diesen Statusbytes nämlich noch ein bis zwei Datenbytes
    folgen, beträgt die Anzahl verschiedener analysierbarer
    MIDI-Kommandos weit über 60000, was wohl völlig ausreichend ist.}.
  Dies ist für die MIDI-Kanal-Zuordnung wichtig (siehe weiter unten,
  Abschnitt "`Simulation verschiedener Instrumente"').
\item Alle nachfolgenden Zahlen müssen Datenbytes, also kleiner als
  128 sein. Negative Zahlen sind als MIDI-Byte natürlich nicht
  zulässig.
\end{itemize}

Die Byte-Folgen bei der Aktion \sourcecode{MIDIOUT} unterliegen nur der
Einschränkung, dass sie im Bereich von 0 -- 255 liegen müssen.

Denkbar ist eine \emph{Zusammenschlie"sung (Vernetzung) mehrerer Computer}, \index{Master!Vernetzung}
von denen einer als "`Master"' alle Vorgänge mit einem Sequenzer-Programm 
kontrolliert. Die übrigen Rechner sind mit je einem Synthesizer/Sampler 
verbunden. Mittels der MIDI-Kommunikation durch den MIDIIN-Auslöser 
und die MIDIOUT-Aktion können sich die auf diese Weise vernetzten 
\mutabor{}"=Systeme aufeinander einstimmen und Informationen austauschen. 
So können Sie pro Rechner eine 16-fache mikrotonale Polyphonie erzeugen 
-- bei nur drei Computern (und Samplern) stehen Ihnen damit schon fast 
unbegrenzte Möglichkeiten der Klangfülle zur Verfügung, nämlich 48 Stimmen!

Um gezielte Nachrichten über gerade geschehene Umstimmungen an weitere
\mutabor{}"=Geräte in der Kette senden zu können, ist es erlaubt,
MIDIOUT"=Nachrichten auch in Stimmungsbünde einzubauen:
\begin{verbatim}
UMSTIMMUNG
  Hoch_hinaus = [ @+Terz, @+Terz, @+Terz, @+Terz, @+Terz ]
  Aber_schmal = [ ] @ - 10 cent
  
  Umstimmungsbund_mit_Nachricht = { Hoch_hinaus ,
                                    MIDIOUT ( #B0, #05, #60 ) ,
                                    Aber_schmal }
\end{verbatim}
Wenn nun die Umstimmung \sourcecode{Umstimmungsbund\_mit\_Nachricht}
aufgerufen wird, so werden zunächst die ersten 5 Töne der
Fundamentaltonleiter um das Intervall Terz (was auch immer das sein
mag) erhöht, dann die MIDI-Nachricht \#B0,\#05,\#60 (=Controller 5 auf
\#60 setzen) geschickt und anschließend das Periodenintervall um 10
Cent verkleinert.

\chapter{Noch mehr über Aufrufe}\index{Aufruf!mehrere}\label{cha:noch-mehr-uber}
\section{Direkte Umstimmungsbünde}\label{sec:direkte-umst}
Manchmal erscheint es zu umständlich, einen Umstimmungsbund zu
deklarieren, falls ein Auslöser mehrere Aktionen auslösen soll. Wir
haben deshalb in die Logiksprache ab der Version 2.1 die Möglichkeit
eingebaut, Umstimmungsbünde direkt hinter den Auslöserpfeil in einer
Logik zu schreiben. Die beiden folgenden Logiken \sourcecode{Entweder} und
\sourcecode{Oder} arbeiten völlig identisch und unterscheiden sich nur in der
syntaktischen Formulierung:
\begin{verbatim}
INTERVALL
  Viertelton = 24 wurzel 2

TON
  a' = 440

TONSYSTEM
  Viertel = 69 [ a' ] Viertelton

UMSTIMMUNG
  Allerhand = { MIDIOUT(254),
                Viertel,
                MIDIOUT(#C0,#2B),
                Expandiere }
HARMONIE
  Grunz = { 1,*2,3 }

LOGIK
  Entweder Taste A = [ FORM Grunz -> Allerhand ]
  Oder     Taste B = [ FORM Grunz ->  { MIDIOUT(254),
                                        Viertel,
                                        MIDIOUT(#C0,#2B),
                                        Expandiere } ]
\end{verbatim}
Die Konstruktion eines "`direkten Umstimmungsbundes"' arbeitet die
innerhalb der geschweiften Klammern aufgeführten Aktionen -- genau wie
ein normaler Umstimmungsbund -- sequentiell ab.

Da in komplexen Stimmungslogiken meistens mehrere Klammerebenen
ineinander geschachtelt werden, sollten Sie sehr sorgfältig prüfen, dass
jede geöffnete Klammer auch an der richtigen Stelle die
korrespondierende geschlossene Klammer hat.

\section{Wer darf wen aufrufen ?}\label{sec:wer-darf-wen}
Dies ist eine wichtige Frage, auf die es eine einfache Antwort gibt:
sofern es keine Selbstbezüglichkeiten gibt, darf quasi jeder jeden
aufrufen, also:
\begin{itemize}
\item (einzige Ausnahme:) Die Einstimmung einer Logik darf nur ein
  Tonsystem oder eine Umstimmung (inkl. Um\-stim\-mungs\-bund)
  aufrufen.
\item Ein Umstimmungsbund darf eine Umstimmung, ein Tonsystem, eine
  MIDIOUT-Nachricht und sogar eine Logik aufrufen.
\item Ein direkter Umstimmungsbund darf natürlich alles, was ein
  Umstimmungsbund darf.
\item Eine Aktion darf ebenfalls eine Umstimmung, ein Tonsystem, eine
  MIDIOUT-Nachricht und eine Logik aufrufen.
\end{itemize}\index{Pathologische Fälle}
Dabei können theoretisch gewisse pathologische Fälle auftreten, z.\,B.
die Logik, die in ihrer Einstimmung einen Umstimmungsbund aufruft,
welcher eine MIDI"=Nachricht sendet und anschließend -- denn dem
Umstimmungsbund ist dies ja gestattet -- eine Logik aktiviert, die
ihrerseits in der Einstimmung\dots{} \hspace{1cm} also wie gesagt:
pathologisch. Da solche Fälle jedoch keinerlei musikalische Relevanz
haben, wird hier nicht weiter darauf eingegangen, dem Benutzer wird
außerdem geraten, solche Fälle nicht auszuprobieren\dots

\chapter{Beschreibung interner Vorgänge}\label{cha:beschr-intern-vorg}
\index{Interne Vorgänge}
Um komplexe Stimmungslogiken gezielt einsetzen zu können, kann es sehr
wichtig werden, genau zu wissen, wie das Laufzeitmodul intern
arbeitet. Was passiert z.\,B., wenn eine Änderung der Breite auf 0
verlangt wird, oder wie werden die Frequenzen liegender Töne
korrigiert? Mit diesen Fragen befasst sich das nun folgende Kapitel.

\section{Grundsätzliches}\label{sec:grundsatzliches}

Wenn Sie eine Stimmungslogik programmiert haben und diese im freien
Spiel anwenden, so können Sie sich jederzeit
den momentanen Status des Instruments
anzeigen lassen. Diese Funktionen sind im Beiheft "`Die Oberfläche"' beschrieben.


\section{Handhabung von Grenzfällen}\label{sec:handh-von-grenzf}
\index{Grenzfälle}
Es kann zur Laufzeit passieren, dass gewisse Parameter unzulässige
Werte einnehmen sollen. "`Was passiert, wenn\dots"'  lautet die Frage; es
ist sicher, dass sich das aktuelle Tonsystem nach dem hundertsten
Aufruf der Umstimmung
\begin{verbatim}
UMSTIMMUNG
               Aufi_gehts = [ @ + Halbton ]
\end{verbatim}
oder ähnlichen, meist harmloseren Konstruktionen nicht mehr wie
vorgeschrieben verhalten kann.

\paragraph{Die Breite der Fundamentaltonleiter} kann nicht kleiner als
eins und nicht größer als 60 werden. Wenn eine Umstimmung diese Grenze
unter- bzw. überschreiten möchte, so wird der Umstimmungsauftrag
schlichtweg ignoriert. Ein Tonsystem der Breite null ist wohl auch
theoretisch undenkbar, und die Beschränkung auf eine maximale Breite
von 60 Tasten -- das sind immerhin fünf Oktaven -- dürfte in der
Praxis keine Schwierigkeiten bereiten und ist eine rein technische
Grenze.

\paragraph{Tonhöhen.} Der tiefste Ton, den \mutabor{} ausgeben
kann,\index{Tonhöhe!Grenzen der} entspricht der MIDI-Taste 0 (das sind
etwa 19 Hz), der höchste Ton entspricht der MIDI-Taste 127 (etwa 12000
Hz). Größere oder kleinere Frequenzwerte werden ausgegeben, aber aus
technischen Gründen in den hörbaren Bereich projiziert, so dass eine
'MIDI-Frequenz' von 137 mit (137 modulo 128) , also 9 erklingt. Ebenso
mit Noten, die tiefer als die 'MIDI-Frequenz' 0 kommen. Mutierende
Stimmungen, die über den höchsten Ton hinaus wollen, werden also
gelassen und nicht -- wie bei der Änderung der Breite -- einfach
beschnitten. Dieses geschlossene Verhalten führt manchmal zu lustigen
Effekten, insbesondere wenn das Kreisen in diesem Tonraum mittels
wenigen Umstimmungen passiert. Wir möchten Ihnen deshalb folgende
Stimmungslogik nicht vorenthalten, die ein geradezu chaotisches --
aber sehr interessantes -- Verhalten an den Tag
legt\footnote{Beispielprogramm chaos.mut}:
\label{CHAOS}
\begin{verbatim}
INTERVALL
 i1=1:1   i2=2:1   i3=3:1   i4=4:1   i5=5:1   i6=6:1
 i7=7:1   i8=8:1   i9=9:1   i10=10:1 i11=11:1 i12=12:1
 i13=13:1 i14=14:1 i15=15:1 i16=16:1 i17=17:1 i18=18:1
 i19=19:1 i20=20:1 i21=21:1 i22=22:1 i23=23:1 i24=24:1
 i25=25:1 i27=27:1 i28=28:1 i30=30:1

TON
 a=110
 o2=a+i2   o3 =a+i3  o4 =a+i4  o5 =a+i5  o6 =a+i6  o7 =a+i7
 o8=a+i8   o9 =a+i9  o10=a+i10 o11=a+i11 o12=a+i12 o13=a+i13
 o14=a+i14 o15=a+i15 o16=a+i16 o17=a+i17 o18=a+i18 o19=a+i19
 o20=a+i20 o21=a+i21 o22=a+i22 o23=a+i23 o24=a+i24 o25=a+i25
 o27=a+i27 o28=a+i28 o30=a+i30
 b=a+5i2
 u2=b-i2   u3=b-i3   u4=b-i4   u5=b-i5   u6=b-i6   u7=b-i7
 u8=b-i8   u9=b-i9   u10=b-i10 u11=b-i11 u12=b-i12 u13=b-i13
 u14=b-i14 u15=b-i15 u16=b-i16 u17=b-i17 u18=b-i18 u19=b-i19
 u20=b-i20 u21=b-i21 u22=b-i22 u23=b-i23 u24=b-i24

TONSYSTEM
 Mischung = 60 [ a,u15,o2,u14,o3,o4,u13,o5,u12,o6,u11,o7,
                o8,u10,o9,u9,o10,o11,u8,o12,u7,o13,u6,o14,
                o15,u5,o16,u4,o17,o18,u3,o19,u2,o20,b,o21] i1

HARMONIE
 Alles={0,*1,*2,*3,*4,*5,*6,*7,*8,*9,*10,*11,*12,*13,*14,
    *15,*16,*17,*18,*19,*20,*21,*22,*23,*24,*25,*26,
    *27,*28,*29,*30,*31,*32,*33,*34,*35,*36,*37,*38,
    *39,*40,*41,*42,*43,*44,*45,*46,*47,*48}

UMSTIMMUNG
 Verschiebe(Dist) = @+Dist [ ] 

LOGIK
    Chaos Taste C = Mischung 
         [ FORM 0~Alles -> Verschiebe(ABSTAND) ]
\end{verbatim}

Diese Stimmungslogik berechnet eine Oberton- und eine Untertonreihe ab
der tiefsten gedrückten Taste. Sobald sich also die tiefste gedrückte
Taste ändert, werden nahezu alle liegenden Tasten umgestimmt; die
Stimmung steigt sofort stark in die Höhe, sehr bald wird wieder in
einen sehr tiefen Tonbereich gesprungen, \dots{} Die Stimmung ist nur so
lange stabil, wie Sie die tiefste Taste konstant lassen.  Zum
Programmtext dieser Stimmungslogik sollte noch gesagt werden, dass der
tabellarisch anmutende Anblick nur durch den Umstand nötig wurde, dass
hier tatsächlich \emph{48 verschiedene Töne} deklariert wurden, die auf
24 verschiedenen Intervallen basieren. Ein solches Logikprogramm wirkt
zwar wenig elegant, lässt sich aber leider nicht kürzer fassen.
Glücklicherweise befindet es sich auf der Beispieldiskette, so dass Sie
es nicht abtippen müssen, um in der Genuss dieser exotischen Stimmung
zu kommen.

\paragraph{Wenn die Verankerungstaste} im Zuge einer relativen
Umstimmung größer als 96 werden soll, so wird vom geforderten Wert so
oft die Breite der Fundamentaltonleiter subtrahiert, bis die
Verankerungstaste kleiner als 96 geworden ist. Falls der Wert kleiner
als 36 werden soll, wird entsprechend addiert. Diese Korrektur hat
keinerlei hörbaren Effekt und ist nur aus Gründen der internen
Rundungsoptimierung notwendig. Ein Tonsystem z.\,B. der Breite 21
reagiert auf die beiden folgenden Umstimmungen:
\begin{verbatim}
UMSTIMMUNG
	      Auf = @ + 17 [ ]
	      Ab  = @ - 4  [ ]
\end{verbatim}
ohne hörbaren Unterschied, da in beiden Fällen die gleiche 'Tonigkeit'
angesprungen wurde.

\paragraph{Die Anzahl gleichzeitig liegender Töne} wird durch das
Laufzeitmodul nicht begrenzt. Alle liegenden Tasten, und es spielt
dabei keine Rolle, wie viele es derer sind, werden in die
Klaviaturanalyse einbezogen. So gesehen würde ein
Elefant\index{Elefant}, der seinen Fuß auf die Klaviatur stellt, eine \index{Töne!Anzahl}
gültige Harmonie spielen, die auch theoretisch auswertbar ist.
Tatsächlich ist aber die maximale Zahl \emph{hörbarer} Töne je nach
Ausgabegerät und MIDI-Kanal-Zuordnung\footnote{Siehe Abschnitt
"`Simulation verschiedener Instrumente"'} begrenzt, gewöhnlicherweise
auf sechzehn, manchmal auch auf acht oder vier Töne.

\paragraph{Die Anzahl verschiedener Stimmungslogiken, Tonsysteme und
Umstimmungen} ist nur durch die Speicherkapazität Ihres Computers
beziehungsweise durch die maximale Anzahl verschiedener Auslöser
begrenzt. Wenn Sie also keine MIDI- oder Harmonieauslöser benutzen,
sondern alle Logiken per Tastendruck aktivieren, so ist deren Anzahl
auf die sechsundzwanzig Buchstaben (A bis Z) begrenzt.

\paragraph{Die Anzahl der Ton-Umstimmungen} kann möglicherweise
nicht mit der Breite des Tonsystems übereinstimmen. Überzählige
Ton-Umstimmungen werden dann ignoriert, ebenso bleiben
überzählige Töne der Fundamentaltonleiter einfach unverändert.
\index{Ton!Umstimmung}

\paragraph{Harmonieanalyse.} Es kann vorkommen, dass eine
Harmonieanalyse auf eine Harmonie testen soll, dessen eine
Tonigkeit\index{Harmonieanalyse!Tonigkeit} jenseits der Breite der
momentanen Fundamentaltonleiter liegt. Was passiert, wenn z.\,B. die
Harmonie \sourcecode{Dur = \{ 0,4,7 \} } in einem Tonsystem der Breite 5
getestet werden soll? Es gibt hier prinzipiell zwei Möglichkeiten:
entweder kann eine solche Harmonie im genannten Tonsystem niemals
erkannt werden, da es dort keine 7. Tonigkeit gibt, oder alle über die
Breite hinausgehenden Tonigkeiten der Harmonie werden ignoriert, in
unserem Beispiel also würde praktisch auf die Harmonie \sourcecode{\{ 0,4 \}
} getestet werden. Wir haben uns für die zweite Lösung entschieden,
aber erst der Einsatz in der Praxis kann darüber entscheiden, ob nicht
vielleicht doch die erste Möglichkeit sinnvoller ist.

\section{Über Rundungsfehler}\label{sec:uber-rundungsfehler}
\index{Rundungsfehler}
Das Laufzeitmodul von \mutabor{} arbeitet bei der internen Speicherung
von Tonfrequenzen mit einer Genauigkeit von $10^{-8}$ Cent. Es ist
zwar möglich, dass bei ständig durchgeführten relativen Umstimmungen
unter bestimmten Umständen irgendwann einmal hörbare Fehl-Stimmungen
auftreten, die durch Rundungsfehler verursacht wurden, jedoch wird
dieser Fall in der Praxis wohl kaum eintreten.

Falls Sie das Gefühl haben sollten, dass das Laufzeitmodul trotz
korrekter Stimmungslogik falsch intoniert, so ist es ratsam, erst
einmal den internen Zustand mittels der Protokollfunktionen abzufragen
und erst bei Nichtübereinstimmung mit der
Erwartung und nach eingehender Prüfung des Logikprogramms auf einen
Programmfehler in der \mutabor{}-Software zu schließen\ldots

Noch eine Bemerkung über absichtliche Rundungen: wenn ein Tonsystem
der Breite 15 der Umstimmung
\begin{verbatim}
UMSTIMMUNG  Halbieren = [ << @/2 >> ]
\end{verbatim}
ausgesetzt ist, so wird der bei der Rechnung entstehende Nachkommawert
einfach abgeschnitten. Das resultierende Tonsystem wäre also hier von
der Breite 7.


\section{Arbeitsweise der Synthesizer-Treiber}\label{sec:arbe-der-synth}
\index{Synthesizer!Arbeitsweise}
\label{EINSTELLUNG}
Leider haben es die Hersteller handelsüblicher Synthesizer versäumt,
in das MIDI-Protokoll Befehle zur direkten Ansteuerung von Mikrotönen
einzubauen.\footnote{\label{FBNULLEINS}Einzige Ausnahme bildet hier der
Synthesizer FB-01 von Yamaha. Hier gibt es eine System-Exklusiv-
Meldung der Gestalt "`Schalte die Note c mit der Feinstimmung +42
Feinstimmeinheiten an"', wobei eine Feinstimmeinheit einem
Hundertachtundzwanzigstel Halbton entspricht. Leider lässt die
Klangqualität dieses Gerätes nach heutigen Maßstäben zu wünschen übrig
und außerdem wird es nicht mehr produziert. Wahrlich schade, dass
Yamaha diese für unsere Zwecke so sinnvolle Option aus ihrem Programm
gestrichen hat. Im Gegensatz zu früheren Versionen von \textbf{\mutabor{}} 
wird dieser Synthesizer nicht mehr speziell unterstützt.} Wir müssen
daher einen kleinen Trick anwenden, um dem Synthesizer/Sampler
trotzdem Mikrotöne entlocken zu können.

\subsection{Die Entfesselung der Mikrotöne}\label{sec:die-entfesselung-der}
\index{Mikrotöne!Erzeugung}\index{Pitch-Bender}
Die meisten Synthesizer haben einen "`Pitch-Bender"'. Das ist ein
Drehrad, mit dessen Hilfe man die Frequenz der Töne während des Spiels
anheben oder absenken kann. Und zwar je nach Einstellung am
Synthesizer zwischen einem und zwölf Halbtönen. Diese Anhebung erfolgt
in Schritten von einem Vierundsechzigstel des Maximalintervalls. Wenn
man also die Weite des Pitch-Bendings auf einen Halbton einstellt, so
kann man damit eine Auflösung von 1/64 Halbton erreichen, was
theoretisch 1,6 Cent entspricht.

Für ein solches Pitch-Bending existiert glücklicherweise auch ein
MIDI-Code, mit dem man diese Funktion über MIDI ansteuern kann. Leider
verändert eine Pitch-Bend-Nachricht nicht nur einen gezielten Ton,
sondern alle momentan liegenden Töne. Auf diese Weise ist es also noch
nicht möglich, das c' um 4 Cent zu erhöhen und gleichzeitig das f'' um
35 Cent zu erniedrigen.

Doch zu guter Letzt ist auch dieses Problem in den Griff zu bekommen.
Wenn man sechzehn Synthesizer
hintereinander schaltet, so dass jeder Synthesizer nur \emph{einen} Ton
spielt, so könnte man jedem Gerät eine \emph{eigene} 
Pitch-Bend-Information geben --
und das Problem wäre gelöst. Nun hat zwar nicht
jeder sechzehn gleiche Synthesizer, aber in diesem Punkt kommt uns die
moderne Technik zur Hilfe: die meisten Synthesizer lassen sich in
einer "`Multi-Mode"'-Betriebsart ansteuern. D.\,h. der Synthesizer tut
so, als ob er aus sechzehn eigenständigen Synthesizern gleichen Typs
bestehen würde. Man kann nun jeden der sechzehn 'Synthesizer im
Synthesizer' gesondert ansprechen, und somit erreichen, dass bis zu
sechzehn gleichzeitig liegende Töne jeweils eine \emph{eigene}
Feinstimminformation bekommen können.

\subsection{Unbedingt beachten\dots}\label{sec:unbed-beacht}

Dabei sind jedoch folgende Dinge von großer Wichtigkeit:
\label{PITCH}
\begin{itemize}\index{Kanal!MIDI}
\item Auf jedem MIDI-Kanal muss dasselbe Instrument (Klangfarbe)
  eingestellt werden, damit mehrere gleichzeitig liegende Töne auch
  mit derselben Klangfarbe erklingen.
\item Auf jedem Kanal muss die Reichweite des Pitch-Benders auf 1
  Halbton eingestellt werden (bzw. den Wert, den Sie im Setup-Dialog
  von \mutabor{} eingestellt haben), damit die Feinstimmungen
  korrekt durchgeführt werden. Diese "`Pitch Bend Range"' ist bei
  den meisten Synthesizern ein Parameter der Klangfarbe und nicht des
  Multis.
\end{itemize}

Diese Einstellungen erfordern gewisse Kenntnisse über die Bedienung
des benutzten Synthesizers, über die dieses Handbuch aufgrund der
Mannigfaltigkeit des Angebotes von Synthesizern und Samplern leider
keine genaue Beschreibung liefern kann. Glücklicherweise sind nur die
beiden oben genannten Einstellungen nötig, und wenn Sie erst einmal
wissen, wie Sie Ihren Synthesizer richtig einzustellen haben, dürfte
dies kein Problem mehr darstellen. Falls Sie dennoch mit der
Einstellung Ihres Synthesizers Schwierigkeiten haben, so stehen wir
Ihnen gerne mit Rat und Tat beiseite, denn an solchen technischen
Kleinigkeiten sollte der Einsatz von \mutabor{} nicht scheitern.

\subsection{Korrektur von Frequenzen}\label{sec:korr-von-freq}
Eine wichtige Frage ist noch ungeklärt. Wie soll sich \mutabor{}
verhalten, wenn sich aufgrund einer Umstimmung die Frequenz einer
liegenden Note ändern muss. Es steht natürlich außer Frage, dass der
Frequenzwert korrigiert werden muss. Und hierzu gibt es zwei
verschiedene Fälle:

\begin{itemize}
\item[a)] Die korrigierte Frequenz liegt im Einzugsbereich derselben
  MIDI-Taste (z.\,B. soll von Taste $54$ mit Feinstimmung $\frac{4}{64}$
  auf Taste $54$ mit Feinstimmung $\frac{9}{64}$ umgestimmt werden).
  Es muss lediglich die Pitch-Bend-Information mit dem neuen Wert an
  den Synthesizer geschickt werden. Der (liegen bleibende) Ton wird
  einfach in seiner Tonhöhe korrigiert.
\item[b)] Wenn die korrigierte Frequenz dagegen nicht im
  Einzugsbereich derselben MIDI-Taste liegt (z.\,B. von Taste $54$ mit
  $\frac{4}{64}$ auf Taste $53$ mit $\frac{62}{64}$), so muss der Ton
  ausgeschaltet werden und mit der korrigierten Frequenz neu
  angeschlagen werden.
\end{itemize}

So klingen die meisten kleinen Frequenzänderungen sehr natürlich,
während größere Änderungen durch ein Neuanschlagen des Tones je nach
eingestellter Klangfarbe mehr oder minder unangenehm auffallen. Dies
ist ein technisches Problem bei der Ansteuerung von Synthesizern durch
MIDI, aber kein prinzipielles Problem von \mutabor{}.

\chapter{Simulation verschiedener Instrumente}\label{cha:simul-versch-instr}
\index{Instrumente!Simulation verschiedener}
Die meisten modernen Synthesizer oder Sampler können in einem
"`Multi-Mode"' betrieben werden. 
Hierbei kann jedem MIDI-Kanal eine
eigene Klangfarbe zugeordnet werden. Auf diese Weise kann ein einziges
Gerät bis zu sechzehn verschiedene Instrumente (Klangfarben) simulieren. 
Wie Sie im vorigen Kapitel gelesen haben, nutzt \mutabor{} diese Option aus, 
um auf diese Weise mikrotonale Polyphonie zu ermöglichen.

Leider beansprucht \mutabor{} zur Ansteuerung der Mikrotöne für jeden 
gleichzeitig liegenden Ton einen eigenen 
MIDI-Kanal\footnote{Ausnahme: der FB-01-Synthesizer von Yamaha, 
siehe Fußnote auf Seite \pageref{FBNULLEINS}}. 
(Ein vierstimmiger Akkord belegt also immer vier MIDI-Kanäle.) 
Deshalb muss der Synthesizer zum Musizieren mit \mutabor{} so 
eingestellt werden, dass er auf jedem MIDI-Kanal dieselbe Klangfarbe 
spielt\footnote{Wenn Sie den vorigen Abschnitt noch nicht gelesen haben, 
so sollten Sie dies jetzt nachholen. Die Kenntnis der internen Vorgänge 
ist zum Verständnis dieses Abschnittes notwendig.}. 
Um diese Reduktion auf nur eine Klangfarbe im Betrieb mit \mutabor{} zu 
verhindern, können Sie die Kanalzuordnung selbst definieren. 
Stellen Sie die Klangfarbenzuordnung an Ihrem Synthesizer/Sampler z.\,B. 
so ein, da"s die MIDI-Kanäle eins bis acht ein Klavier und die Kanäle 
neun bis sechzehn ein Streichorchester spielen. Wenn Sie im normalen Betrieb 
nun einen 12-stimmigen Akkord anschlagen, so werden die ersten acht Töne 
vom Klavier und die letzten vier vom Streichorchester gespielt. 
Dies macht offensichtlich keinen Sinn. Nun können Sie aber dem 
\mutabor{}"=Compiler mitteilen, dass die Kanäle 1--8 vom ersten 
Instrument belegt werden und die Kanäle 9--16 vom zweiten. 
Dies reduziert die maximale Polyphonie auf acht Stimmen pro Instrument, 
ermöglicht aber dafür die Simulation mehrerer Instrumente.


Sie können so z.\,B. ein Streichquartett oder eine beliebige andere 
Kombination verschiedener Instrumente simulieren.

Sie treffen in der MidiKanal-Deklaration eine Zuordnung, welcher 
MIDI-Eingabekanal welchen MIDI-Ausgabekanälen zugeordnet werden soll. 
Wenn Sie keine MidiKanal"=Deklaration angeben, so wird diese Zuordnung mit\index{MIDI!Kanal}
\begin{verbatim}
  MIDIKANAL
	     1 -> 1-16
\end{verbatim}
initialisiert. Das bedeutet, dass alle NOTE-ON bzw. NOTE-OFF-Meldungen, 
die über die MIDI-Schnittstelle auf MIDI-Kanal 1 vom Masterkeyboard 
empfangen werden, an Kanäle 1 bis 16 weitergeleitet werden. 

In unserem Beispiel eines Musikstücks für "`Klavier"' und
"`Streichorchester"' würde die MidiKanal-Zuordnung lauten:
\begin{verbatim}
MIDIKANAL
	     1 -> 1-8
	     2 -> 9-16
\end{verbatim}
Dies bedeutet, dass alle NOTE-ON- bzw. NOTE-OFF-Meldungen, die auf
MIDI-Kanal 1 empfangen werden, an das "`Klavier"' weitergeleitet werden
und alle auf Kanal 2 empfangenen an das "`Streichorchester"'.
Grundvoraussetzung hierfür ist, dass Sie an Ihrem Synthesizer/Sampler
den Multi-Mode richtig eingestellt haben und die Pitchbend-Range
bei jeden Instrument einen Halbton beträgt.



Das Besondere bei der Benutzung mehrerer MidiKanäle ist, da"s 
\mutabor{} bei dieser Aufspaltung jeden MidiKanal als 
\emph{völlig eigenständig} behandelt. Verschiedene MidiKanäle 
können zur gleichen Zeit unterschiedliche Stimmungslogiken oder 
Tonsysteme spielen.

Auf dem Bildschirm wird aber nur eines der Instrumente/Kanäle
 angezeigt.
Welches das ist, kann mit den Zifferntasten 1--9 gewählt werden.
Auf diesen angezeigten MidiKanal (Instrument) bezieht sich auch der Auslöser
\sourcecode{Taste}. 
\begin{verbatim}
LOGIK
    Hans  Taste h = drittel_ton [ ]
    Meier Taste m = halb_ton [ ]
MIDIKANAL
    1 -> 1-8    "Klavier"
    2 -> 9-16   "Streicher"
\end{verbatim}
Wenn das Klavier in der Logik Hans spielen soll, so drückt man
zunächst die Tasten `1' und `h'. Wenn dann die Streicher in der 
Logik Meier spielen sollen, so drückt man die Tasten
`2' und `m'. Dabei wechselt man zunächst zum MidiKanal 2
und aktiviert dann die Logik Meier. Der Wechsel zum MidiKanal
2 bleibt bestehen, so dass Sie nicht vor jeder einzelnen Umstimmung
den MidiKanal \emph{vorwählen} müssen. Da bisher keine 
MidiKanal-Deklaration benutzt wurde, gab es nur den MidiKanal 1
und man brauchte nie eine Ziffer \emph{vorzuwählen}.
Natürlich kann durch eine erneute Zifferntaste zum nächsten
MidiKanal\index{Instrument!vorwählen}
gewechselt werden. Es sind aber nur so viele MidiKanäle
vorhanden, wie in dem Logik-Programm deklariert wurden.
Auf MidiKanäle, die nicht deklariert wurden, reagiert \mutabor{} 
einfach nicht.

Die MidiKanäle müssen nicht fortlaufend durchnummeriert werden.
Man kann z.\,B. auch deklarieren:

\sourcecode{MIDIKANAL 4 -> 1-16}

und schon reagiert \mutabor{} \emph{nur} auf den MIDI-Kanal 4. Dies ist für die
Zusammenschaltung mehrerer \mutabor{} bzw. mehrerer
Synthesizer in einer MIDI-Kette wichtig.

\chapter{Kommentare}\label{cha:kommentare}

\index{Kommentare}
Wie in dem vorherigen Programmierbeispiel zu sehen war, kann man
so genannte \emph{Kommentare} in ein Programm hineinschreiben.
Kommentare dienen dazu, Bemerkungen und Gedanken zum Programm
festzuhalten. Sie sind nur für den menschlichen Leser
von Bedeutung und beeinflussen in keinster Weise das
Logikprogramm. Kommentare sind dadurch gekennzeichnet,
dass sie in doppelten Anführungszeichen (\dq) stehen. Ein Kommentar
darf seinerseits keine Anführungszeichen enthalten, aber durchaus
über mehrere Zeilen gehen. Beispiel:

\begin{verbatim}
INTERVALL
      Drittelton = 18 Wurzel 2  "Teilt die Oktave in 
                                 achtzehn gleiche Teile"
TON
      a = 440 "Der Kammerton wird auf 440Hz gesetzt"
     
TONSYSTEM
      Drittel = 69 [ a ] Drittelton
         "Das Tonsystem Drittel beginnt beim
          eingestrichenen a' (=69), die FT besteht aus
          einem Ton (dem Ton a=440Hz) und diese Struktur
          wird um einen Drittelton verschoben wiederholt."

LOGIK
      Drittel Taste D = Drittel [ ]
         "Die nicht-mutierende, statische Logik Drittel wird
          durch Druck der Taste D aktiviert."
\end{verbatim}
Dies ist ein Musterbeispiel für gut dokumentiertes Programmieren. Mit 
Hilfe von Kommentaren im Logikprogrammtext ist es auch einem 
anderen Anwender möglich, zu verstehen, wie die 
Stimmungslogik funktioniert -- selbst wenn dieser 
noch nie Stimmungslogiken unter \mutabor{} programmiert hat.

Eine weitere Verwendungsmöglichkeit für Kommentare ist das 
"`Auskommentieren"' von Programmteilen, die nicht \index{Auskommentieren}
oder noch nicht übersetzt werden sollen. Wenn die 
konzeptuelle Planung eines Logikprogramms bereits so weit fortgeschritten 
ist, dass die Logiken fertig programmiert sind, aber noch 
nicht die von ihnen benutzten Tonsysteme und Umstimmungen, z.\,B. das 
Logikprogramm
\begin{verbatim}
LOGIK
     Garten Taste G = Gras [
                             Blume       -> Strauch(4)
                             Kanne       -> Gras
                             FORM Akkord -> Umstimmaktion(ABSTAND)
                           ]
TONSYSTEM 
     Gras = 60 [ c ] Murksintervall
TON
     c = 266
INTERVALL
     Murksintervall = 1.088463542657 : 1
\end{verbatim}
würde vom Compiler nicht übersetzt werden können, da in der Logik 
\sourcecode{Garten} die Auslöser \sourcecode{Blume}, \sourcecode{Kanne} und \sourcecode{Akkord}, 
sowie die Aktionen \sourcecode{Strauch} und \sourcecode{Umstimmaktion} noch 
nicht deklariert worden sind.

Wenn Sie aber den bereits fertig programmierten Bestandteil 
des Logikprogramms testen möchten, ohne das in der Logik 
formulierte Konzept zu löschen, so müssen Sie bloß die noch nicht 
deklarierten Bezeichner "`auskommentieren"':

\begin{verbatim}
LOGIK
     Garten Taste G = Gras [
  "
                       Blume       -> Strauch(4)
                       Kanne       -> Gras
                       FORM Akkord -> Umstimmaktion(ABSTAND)
  "
                           ]
TONSYSTEM 
     Gras = 60 [ c ] Murksintervall
TON
     c = 266
INTERVALL
     Murksintervall = 1.088463542657 : 1
\end{verbatim}

Der gesamte Text, welcher \emph{zwischen} den Anführungszeichen steht
wird vom Compiler als Kommentar behandelt und nicht weiter beachtet.
Dadurch sind die Bezeichner \sourcecode{Blume}, \sourcecode{Kanne}, \sourcecode{Akkord},
\sourcecode{Strauch} und \sourcecode{Umstimmaktion} nicht benutzt und
werden vom Compiler auch nicht als "`nicht deklariert"' bemängelt. 
Das Programm mit den auskommentierten Stellen kann jetzt fehlerfrei 
übersetzt werden.

Dies ist sehr wichtig, wenn Sie Ihre Programme auf der höchsten 
Ebene zu schreiben beginnen (also mit den Logiken) und Objekte benutzen, 
die Sie noch nicht deklariert haben. Wenn Sie die entsprechenden 
Stellen auskommentieren, so können Sie das Logikprogramm-Fragment 
trotzdem austesten. 

\newcounter{savedchapter}
\setcounter{savedchapter}{\value{chapter}}
\let\savedchaptercommand\chapter

\appendix

\chapter[Programmiersprache -- Zusammenfassung]{Zusammenfassung aller Elemente der Programmiersprache}\label{cha:zusamm-aller-elem}
\label{anhang:programmieren}
Dieses Kapitel gibt Ihnen noch einmal eine vollständige
Zusammenfassung aller programmtechnischer Konstruktionen, mit denen
Sie \mutabor{} Mikrotöne entlocken können. Sie werden hier vielem
Bekannten begegnen, aber auch Neues entdecken. Da Sie bereits mit dem
Konzept und den meisten Elementen der Programmiersprache von \mutabor{}
vertraut sind, werden die grundlegenden Konstruktionen nicht noch
einmal im Detail erläutert. Sie sollten also wissen, was unter einer
Tondeklaration, einer Umstimmung oder einer MIDI-Nachricht, \dots{}  zu
verstehen ist. Wir wollen Sie in diesem Kapitel auf mögliche
Konstruktionen aufmerksam machen, die die Flexibilität und
Universalität von \mutabor{} hervorstellen und dazu dienen sollen, {\em
  kreative Anregungen} für das Experimentieren mit ausgefallenen
Stimmungslogiken zu geben.

Die vollständige Definition der Syntax der Programmiersprache finden
Sie im Referenzhandbuch, dort jedoch in der abstrakten Form der
Syntaxdiagramme. Um Ihnen also noch einmal eine "`letzte Chance"' zu
geben, die Programmiersprache durch das Beschauen von Beispiel zu
lernen, wurde dieses Kapitel konzipiert.

\section{Intervalle und Töne}\label{sec:intervalle-und-tone}
Ein Intervall ist bei \mutabor{} ein Frequenzverhältnis und kann
als Zahlenverhältnis angegeben werden. \index{Intervalle}\index{Töne}
\begin{verbatim}
    INTERVALL  meier = 5 : 4
\end{verbatim}
Wenn das Intervall sozusagen "`in die andere Richtung"' weisen
soll, so braucht man nur das 
Zahlenverhältnis umzukehren:
\begin{verbatim}
    INTERVALL umgekehrt = 4 : 5
\end{verbatim}
Wenn ein Intervall der soundsovielte Teil eines anderen
Intervalles sein soll, so muss man wegen dem logarithmischen
Gehör die entsprechende Wurzel ziehen. Wenn also die Oktave in
genau zwölf gleichgroße Teile geteilt werden soll, so muss
man die zwölfte Wurzel ziehen:
\begin{verbatim}
    INTERVALL halbton = 12 WURZEL 2
\end{verbatim}
Natürlich ist es auch möglich, sich ein beliebiges 
Phantasie-Intervall zu definieren. Wenn dies z.\,B. die Kommazahl
3.14159 sein soll, so schreibt man einfach
\begin{verbatim}
    INTERVALL phantasie = 3.14159 : 1
\end{verbatim}
Mit dem bisherigen kann man Intervalle definieren, die einfach
einen festen Wert haben. Des weiteren kann man Intervalle auch aus
anderen Intervallen zusammensetzen.\index{Intervalle!zusammengesetzte}
\begin{verbatim}
    INTERVALL seltsam = phantasie - 3 halbton + 2 cent
\end{verbatim}
Dieses Intervall"=Zusammensetzen muss in sich konsistent sein.
So kann man nicht zwei Intervalle gegenseitig voneinander abhängen
lassen.
\begin{verbatim}
    INTERVALL eins = phantasie + zwei + 5.1 cent
              zwei = eins - 3 halbton
\end{verbatim}
Die Intervalle \sourcecode{eins} und \sourcecode{zwei} sind in dieser
Konstruktion nicht eindeutig definiert und deshalb werden
solche Kreis-Bezüglichkeiten überprüft.

Kommen wir nun zu den Tönen: Ein Ton ist bei \mutabor{} ein Name oder
Bezeichner, der letztendlich eine Frequenz repräsentiert. Mit welchem
Instrument oder in welcher Klangfarbe die Töne gespielt werden, ist
Sache des angeschlossenen Synthesizers, hat also mit \mutabor{} nichts zu
tun. Hier deklarieren wir, dass ein Ton eine bestimmte Frequenz hat:
\begin{verbatim}
  TON c   = 130.81
      d   = 146.84
      e   = 164.8
      f   = 174.6
      g   = 196
      a   = 220
      h   = 246.9
\end{verbatim}
Dasselbe hätten wir mit der folgenden Deklaration erreicht:
\begin{verbatim}
  INTERVALL halbton = 12 WURZEL 2

  TON c   = a - 9 halbton
      d   = a - 7 halbton
      e   = a - 5 halbton
      f   = a - 4 halbton
      g   = a - 2 halbton
      a   = 220
      h   = a + 2 halbton
\end{verbatim}
Man sieht also, dass Töne sowohl eine feste Frequenz haben
können, als auch relativ zu anderen Tönen definiert werden können.
Die relativ definierten Töne werden von \mutabor{} berechnet,
so dass auch diese Töne letztendlich eine feste Frequenz haben.
Die relativ angegebenen Töne können beliebig komplex sein:
\begin{verbatim}
  TON otto = c + 2 seltsam - 4 phantasie + 3 halbton + 0.2 cent
\end{verbatim}
Es muss nur gewährleistet sein, dass die Töne nicht gegenseitig
voneinander abhängen und dass die benutzten Bezugstöne und
Intervalle korrekt deklariert werden. Es gibt also eine Menge
Möglichkeiten, um sich Töne zusammenzustellen.

\section{Tonsysteme}\label{sec:tonsysteme}
\index{Tonsystem}
Tonsysteme geben an, welche Taste der Klaviatur welchen Ton haben
soll. Dazu werden einfach die Töne der Reihenfolge nach
aufgelistet. Die folgende Deklaration \footnote{
Beispielprogramm \filename{komentar.mut}}\label{KOMENTAR}
\begin{verbatim}
   TONSYSTEM gleichstufig = 48 [c, ,d, ,e,f, ,g, ,a, ,h] oktave
\end{verbatim}
wird im folgenden nochmal geschrieben, aber so, dass die einzelnen
Elemente mit Kommentaren versehen sind. Diese Schreibweise wird
von \mutabor{} genauso verstanden, wie obiges Beispiel:

\begin{tabular}{ll}
	     &         \\
 \sourcecode{TONSYSTEM}    & \sourcecode{    ''dieses Wort bestimmt, dass jetzt'' } \\
		& \sourcecode{    ''Tonsystem-Deklarationen folgen''  } \\
		&                                       \\
 \sourcecode{gleichstufig} & \sourcecode{    ''das ist der Name des Tonsystems'' } \\
		&                                       \\
\sourcecode{=}         & \sourcecode{    ''ein Gleichheitszeichen, damit sichtbar'' } \\
		& \sourcecode{    ''ist, dass hier etwas definiert wird'' } \\
		&                                           \\
\sourcecode{48}      & \sourcecode{  ''das ist die MIDI-Nummer der ersten Taste,'' } \\
		& \sourcecode{   ''die hier einen Ton zugeordnet bekommt'' } \\
		&                                            \\
\sourcecode{[}         & \sourcecode{    ''Die Liste der Töne beginnt''         } \\
		&                                           \\
\sourcecode{c}         & \sourcecode{    ''Taste 48 bekommt Ton c''              } \\
\sourcecode{,}         & \sourcecode{    ''nächster Ton folgt'' } \\
		& \sourcecode{    ''Taste 49 ist stumm''  } \\
\sourcecode{,}         & \sourcecode{   ''nächster Ton folgt'' } \\
\sourcecode{d}         & \sourcecode{   ''Taste 50 bekommt Ton d'' } \\
\sourcecode{,}         & \sourcecode{   ''nächster Ton folgt'' } \\
		& \sourcecode{    ''Taste 51 ist stumm'' } \\
\sourcecode{,}         & \sourcecode{   ''nächster Ton folgt'' } \\
\sourcecode{e}         & \sourcecode{   ''Taste 52 bekommt Ton e'' } \\
\sourcecode{,}         & \sourcecode{   ''nächster Ton folgt'' } \\
\sourcecode{f}         & \sourcecode{   ''Taste 53 bekommt Ton f'' } \\
\sourcecode{,}         & \sourcecode{   ''nächster Ton folgt'' } \\
	     & \sourcecode{       ''Taste 54 ist stumm'' } \\
\sourcecode{,}         & \sourcecode{   ''nächster Ton folgt'' } \\
\sourcecode{g}         & \sourcecode{   ''Taste 55 bekommt Ton g'' } \\
\sourcecode{,}         & \sourcecode{   ''nächster Ton folgt'' } \\
	      & \sourcecode{      ''Taste 56 ist stumm'' } \\
\sourcecode{,}         & \sourcecode{   ''nächster Ton folgt'' } \\
\sourcecode{a}         & \sourcecode{   ''Taste 57 bekommt Ton a'' } \\
\sourcecode{,}         & \sourcecode{   ''nächster Ton folgt'' } \\
	      & \sourcecode{      ''Taste 58 ist stumm'' } \\
\sourcecode{,}         & \sourcecode{   ''nächster Ton folgt'' } \\
\sourcecode{h}         & \sourcecode{   ''Taste 59 bekommt Ton h'' } \\
		&                                       \\
\sourcecode{]}         & \sourcecode{   ''die Liste der Töne ist zu Ende'' } \\
		&                                       \\
\sourcecode{oktave}   & \sourcecode{   ''die Tasten 48 bis 59 wiederholen sich'' } \\
		& \sourcecode{   ''nach beiden Seiten bis zum Anschlag'' } \\
		& \sourcecode{   ''und verschieben die Tonhöhe dabei'' } \\
		& \sourcecode{   ''um jeweils eine Oktave''} \\
		&   \\
\end{tabular}

Man sieht hier, wie die einzelnen Töne den Tasten des Keyboards
zugeordnet werden. Nach rechts geht das Schema folgendermaßen weiter:

\begin{tabular}{l@{ $\rightarrow$ }l}
  Taste 60 & c + oktave \\
  Taste 61 & Stumm \\
  Taste 62 & d + oktave \\
  Taste 63 & Stumm \\
  Taste 64 & e + oktave \\+
  \dots{}    &  \dots{} \\
  Taste 72 & c + 2 oktave \\
  Taste 73 & Stumm \\
  Taste 74 & d + 2 oktave \\
  \dots{}   & \dots{} \\
  Taste 84 & c + 3 oktave \\
\end{tabular}

und nach links geht das Schema auch entsprechend weiter.
Wenn man die Einschränkung fallen lassen will, dass sich das Muster
der Töne periodisch wiederholt, so braucht man nur das Tonsystem
so breit zu wählen, dass es die ganze Klaviatur abdeckt.
\begin{verbatim}
 TONSYSTEM sehr_breit = 36 [ ton36,ton37,ton38, ... ,ton96 ] dummy
\end{verbatim}
In diesem Fall kann als Periodenintervall (\sourcecode{dummy}) ein
beliebiges Intervall eingesetzt werden, da es im Bereich der
verfügbaren Klaviaturtasten sowieso keine Auswirkungen hat.
Man muss jetzt 61 Töne\footnote{Vorausgesetzt, dass ein Keyboard
 mit 61 Tasten benutzt wird.} (\sourcecode{ton36}--\sourcecode{ton96}) einzeln
deklarieren, aber dafür hat man die Tonhöhe jeder einzelnen Taste
in der Hand. Man kann auch das Tonsystem kürzer wählen und z.\,B.
ein Muster aus 3 Tasten wiederholen lassen.
\begin{verbatim}
    TONSYSTEM drei = 50 [ t1, t2, t3 ] halbton
    TON t1 = t2 - 10 cent
        t2 = 440
        t3 = t2 + 10 cent
    INTERVALL halbton = 12 wurzel 2
              cent = 1200 wurzel 2
\end{verbatim}
Jetzt liegt der Kammerton a' auf der Taste 51 (=es) und die weiteren 
Halbtöne sind nicht direkt daneben, sondern jeweils drei Tasten
weiter. Dafür ist jeder Ton sozusagen "'eingerahmt"' von 
zwei weiteren Tönen, die um $\pm$10 Cent verstimmt sind.

\section{Umstimmungen und Verwandte}\label{sec:umst-und-verw}
\index{Umstimmung}
Eine Umstimmung verändert das aktuelle Tonsystem, indem einzelne
Parameter der Fundamentaltonleiter geändert werden. So kann man die
Verankerungstaste verschieben, die Breite oder das
 Periodenintervall ändern und einzelne Töne ändern.
Dies sind die vier Grundmöglichkeiten, um ein Tonsystem zu ändern.
\begin{verbatim}
  UMSTIMMUNG  
     u1 = 51 [ ]               "Setzt Verankerungstaste neu"
     u2 = [ << 5 >> ]          "Setzt die Breite auf 5"
     u3 = [ << @ + 1 >> ]      "Verbreitert um einen Ton"
     u4 = [ @, xy, @ ]         "Setzt den zweiten Ton neu"
     u5 = [ @, @, @ - 3 cent ] "Erniedrigt den dritten Ton"
     u6 = [ ] ganzton - 5 cent "Setzt das Periodenintervall neu"
     u7 = [ ] @ + halbton      "Vergroessert das Periodenintervall"
\end{verbatim}
Da eine einzelne Umstimmung im allgemeinen nicht ausreichend ist, 
kann man auch mehrere Umstimmungen auf einmal durchführen.
\begin{verbatim}
  UMSTIMMUNG mehrfach = { u1, u3, u7 }
\end{verbatim}
Wenn jetzt die Umstimmung \sourcecode{mehrfach} durchgeführt wird,
so werden der Reihe nach die Umstimmungen \sourcecode{u1, u3} und
\sourcecode{u7} durchgeführt. Sollte zusätzlich zu den drei Umstimmungen
noch eine Nachricht an den Synthesizer geschickt werden, so 
schreibt man:
\begin{verbatim}
  UMSTIMMUNG mehrfach = { u1, u3, u7, MidiOut (#C0, 17, 22) }
\end{verbatim}
und schon bekommt der Synthesizer eine drei Bytes lange
MIDI-Nachricht geschickt, sobald die Umstimmung
\sourcecode{mehrfach} durchgeführt wird.
Bei der Angabe von Verankerungstaste und Breite darf keine
Kommazahl sondern nur eine ganze Zahl angegeben werden.
Diese Zahl muss aber nicht konstant sein, sondern sie kann
der Umstimmung auch als Parameter mitgegeben werden.
\begin{verbatim}
  UMSTIMMUNG  setze_anker (x) = x [ ] 
\end{verbatim}
Diese Umstimmung kann nun von verschiedenen Stellen aus mit
unterschiedlichen Parametern aufgerufen werden.
\begin{verbatim}
    setze_anker (48)
    setze_anker (36)
    setze_anker (59)
\end{verbatim}\index{Parameter}
Eine Umstimmung kann auch mehrere Parameter haben, und
diese Parameter an andere Umstimmungen weitergeben.
\begin{verbatim}
   UMSTIMMUNG meier (x,y,z) = {   setze_anker (z),
                                  setze_breite (x),
                                  verschiebe (y),
                                  midiout (#D0, 17, 4)
                              }

       setze_breite (x) = [ << x >> ]

       verschiebe (j) = @ + j [ ]
\end{verbatim}
Es ist auch möglich, in Abhängigkeit von einem Parameter
eine Fallunterscheidung zu treffen. Hier soll die 
Verankerungstaste nur dann verschoben werden, wenn der 
Parameter \emph{ungerade} ist.
\begin{verbatim}
    UMSTIMMUNG ungerade (x) = x {
                                  -5 -> verschiebe (x)
                                  -3 -> verschiebe (x)
                                  -1 -> verschiebe (x)
                                   1 -> verschiebe (x)
                                   3 -> verschiebe (x)
                                   5 -> verschiebe (x)
                                   7 -> verschiebe (x)
                                   9 -> verschiebe (x)
                                 }
\end{verbatim}
Man kann auch je nach Parameter verschiedene andere Umstimmungen
oder MIDI"=Nachrichten auslösen:
\begin{verbatim}
     UMSTIMMUNG super (x) = x {
                               0 -> u1 , midiout (#c0, 0, 0)
                               1 -> u2, u3, verschiebe (4)
                               2 -> midiout (#b0, 2, 5), u4
                       ansonsten -> midiout (#b0, 0, 127)
                    }
\end{verbatim}
Es ist allerdings verboten, dass Umstimmungen sich gegenseitig
aufrufen. Das würde zu einem unendlichen PingPong von
Umstimmungen führen und ist nicht sinnvoll. Sinnvoll ist jedoch,
dass eine Umstimmung ein komplettes Tonsystem aufruft.
In diesem Fall wird das angegebene Tonsystem eingestellt. Es
ist sogar möglich in einer Umstimmung eine andere
Logik aufzurufen. In diesem Fall wird in diese Logik gewechselt
und, falls vorhanden, deren Einstimmung durchgeführt.

\section{Harmonien und Logiken}\label{sec:harm-und-logik}
\index{Harmonien}\index{Logiken}
Eine Logik besteht aus einer Einstimmung und einer Liste von
Stimmungsregeln. Die Regeln wiederum geben an, was bei
welchem Ereignis zu geschehen hat.
\begin{verbatim}
  LOGIK einfach  TASTE e = drei [ ]
\end{verbatim}
Die Logik \sourcecode{einfach} wird aktiviert, wenn man auf der Computertastatur die Taste E
drückt und hat als Einstimmung das Tonsystem \sourcecode{drei}.
Weiteres passiert in dieser Logik nicht. Wenn man jedoch einige
Anweisungen in die eckigen Klammern schreibt, so arbeitet die
Logik.
\begin{verbatim}
  LOGIK einfach  TASTE e = drei [ 
                                  taste x -> u1
                                  taste y -> u3 , u5
                                  taste z -> midiout (#C0, 1)
				]
\end{verbatim}
Jetzt kann man mit der Taste E die Logik \sourcecode{einfach} aufrufen
und dann mit den Tasten \sourcecode{x, y} und \sourcecode{z} die entsprechenden
Umstimmungen oder das MidiOut aufrufen.
Aber Tasten sind nicht die einzigen Ereignisse, auf die eine
Logik reagieren kann.
\begin{verbatim}
  LOGIK program_change  TASTE p = drei [ 
                           midiin (#C0, 05) -> u1
                           midiin (#C0, 09) -> u3 , u5
                           midiin (#C0, 22) -> midiout (#C0, 1)
                      ]
\end{verbatim}
Die Logik \sourcecode{program\_change} wird mit der Taste P aktiviert und
reagiert dann auf die drei angegebenen MIDI-Nachrichten.
Es können auch gemischte Ereignisse von einer Logik bearbeitet
werden:
\begin{verbatim}
  LOGIK mehrfach  TASTE m =  [ 
                       taste x          -> u1
                       taste y          -> u3 , u5
                       taste z          -> midiout (#C0, 1)
                       midiin (#C0, 05) -> u4
                       midiin (#C0, 09) -> u3 , u2
                       midiin (#C0, 22) -> midiout (#B0, 127, 4)
                   ]
\end{verbatim}
Die Logik \sourcecode{mehrfach} wird mit der Taste M aktiviert und
reagiert dann auf drei Tasten und drei MIDI-Nachrichten.
Ferner enthält sie keine Einstimmung. 
Es bleibt bei ihrem Aufruf also
die bisherige Stimmung bestehen. 
Die dritte Möglichkeit, in einer Logik auf etwas zu reagieren,
ist die Harmonie-Analyse. 

\subsection{Komplizierte Fälle}\label{sec:komplizierte-falle}

Kompliziertere Auslöser als Computer-Tasten oder MIDI-Nachrichten sind
Harmonien und Harmonieformen. (Eigentlich handelt es sich hierbei um
"'Tastenmuster"`, denn eine Harmonieanalyse im Sinne der
Musikwissenschaft ist mit \mutabor{} noch nicht möglich.) Hier wird
eine Harmonie als Menge von gedrückten Tasten in der
Projektionstonleiter verstanden. Beim Erkennen der angegebenen
Harmonie gilt der Auslöser als erfüllt. Beispiel:
\begin{verbatim}
    HARMONIE 
          c_dur   = {0, 4, 7}
          cis_dur = {1, 5, 8}
          d_dur   = {2, 6, 9}
          dis_dur = {3, 7, 10}
          e_dur   = {4, 8, 11}

    LOGIK mutiere_dur Taste M = [
                 c_dur   -> u1
                 cis_dur -> u2
                 d_dur   -> u3
                 dis_dur -> u4
                 e_dur   -> u5
               ansonsten -> u99
             ]
\end{verbatim}
Sobald die Logik \sourcecode{mutiere\_dur} aktiviert ist, wird
bei jeder der fünf angegebenen Harmonien eine Umstimmung
durchgeführt. Falls in einem Moment mal keine der Harmonien
zutrifft (sondern ein anderes Tastenmuster gedrückt ist)
 wird die Umstimmung u99 durchgeführt.

Man kann auch eine Harmonie-Form Analyse durchführen, bei der es
nicht auf die Verschiebung der Harmonie bezüglich des Grundtons
ankommt:
\begin{verbatim}
    HARMONIE 
          dur   = {0, 4, 7}

    LOGIK mutiere_dur_2 Taste M = [
                 FORM dur   -> umst (ABSTAND)
                 ansonsten  -> u99
             ]
\end{verbatim}
Hier wird jede Dur-Harmonieform erkannt, und zwar nicht nur die fünf
Harmonien aus dem vorigen Beispiel, sondern alle zwölf Harmonieformen
-- vorausgesetzt, die Breite des Tonsystems ist zwölf. In dem
Parameter \sourcecode{ABSTAND} wird als Nummer mitgegeben, um wieviele
Tastenschritte die erkannte Harmonie vom Prototyp der Harmonieform
entfernt ist, ABSTAND liefert also die Stufe, auf der die erkannte
Harmonie bezüglich dem aktuellen Zentrum (=Verankerungstaste) steht.
Man könnte dies für eine \emph{Auswählende Umstimmung} verwenden und
die Umstimmung \sourcecode{umst} definieren als:
\begin{verbatim}
    UMSTIMMUNG
        umst (x) = x {
                       0 -> u1  "Fall c_dur"
                       1 -> u2  "Fall cis_dur"
                       2 -> u3  "Fall d_dur"
                       3 -> u4  "Fall dis_dur"
                       4 -> u5  "Fall e_dur"
                 }
\end{verbatim}
Möchte man zusätzlich erreichen, dass nicht jeder 
\sourcecode{Dur}-Dreiklang erkannt wird, so kann man noch folgende
Einschränkungen machen:
\begin{verbatim}
    HARMONIE 
          dur   = {0, 4, 7}

    LOGIK mutiere_dur_3 Taste M = [
                 FORM 4 ~ dur  -> umst (ABSTAND)
             ]
\end{verbatim}
Jetzt wird die Harmonie-Form Analyse genauso durchgeführt, aber
der tiefste Ton muss vom Tonigkeitstyp 4 sein. Ebenso kann einschränkend gesagt werden, dass der
höchste Ton von einer bestimmten Art sein soll:
\begin{verbatim}
    HARMONIE 
          dur   = {0, 4, 7}

    LOGIK mutiere_dur_3 Taste M = [
                 FORM dur ~ 7  -> umst (ABSTAND)
             ]
\end{verbatim}
Jetzt reagiert der Auslöser \sourcecode{dur \sourcecode{~} 7} nur dann, wenn der
höchste Ton ein g ist. (Bei zwölfstufiger Fundamentaltonleiter).  Es
sind auch beide Einschränkungen gleichzeitig zugelassen. Durch diese
einschränkende Analyse können verschiedene Lagen und Umkehrungen
derselben Harmonieform unterschiedliche Umstimmungen auslösen.

\chapter[Änderungen der Sprachsyntax (2.0 $\rightarrow$ 2.1/3.0)]
{Änderungen der Sprachsyntax von Version 2.0 zu Version 2.1 und Version 3.0}\label{cha:ander-der-sprachsynt}

Leider ließen sich folgende Konstruktionen nicht "`aufwärtskompatibel"' 
herstellen:
\begin{itemize} 
\item In Intervalldeklarationen sind reine Ganz- oder Kommazahlen
  nicht mehr erlaubt.  Die Intervalldeklarationen \sourcecode{INTERVALL
    Oktave = 2 } oder \sourcecode{Halbton = 1.05946} ist ab Version 2.1
  syntaktisch falsch.  Nichtsdestotrotz können Sie solche Intervalle
  erzeugen, indem Sie sie als Verhältnis zu eins darstellen: \sourcecode{
    INTERVALL Oktave = 2:1} und \sourcecode{Halbton = 1.05946:1}.
\item Die Deklaration von INSTRUMENT ist durch die Deklaration von
  MIDIKANAL ersetzt worden. Die Bedeutung hat sich jedoch nicht
  geändert.
\item An verschiedenen Stellen wo bisher nur ein einzelnes Intervall
  zulässig war, kann jetzt auch eine Kombination von mehreren
  Intervallen angegeben werden.  Diese Fälle sind:
  \begin{description}
  \item[Intervalldeklaration]
\begin{verbatim} 
 Intervall kombi = 2 Terz - Oktave + 4 cent
\end{verbatim}
  \item[Tonsystemdeklaration]
\begin{verbatim}
 Tonsystem Meier = 69 [ a ] Halbton + 0.5 cent
\end{verbatim}
  \item[Umstimmungsdeklaration]
\begin{verbatim}
 Umstimmung Otto_absolut = [ ] Halbton + 0.7 cent
\end{verbatim}
\begin{verbatim}
 Umstimmung Otto_relativ = [ ] @ + 3 cent
\end{verbatim}
  \end{description}
\end{itemize}

Falls also Logikprogramme, die Sie unter der Version 2.0 erfolgreich 
geschrieben haben, bei neueren Versionen plötzlich Syntaxfehler 
erzeugen, so liegt die Ursache mit Sicherheit an einer dieser 
Nicht-Aufwärtskompatibilitäten. Es sei noch einmal betont, dass sich durch 
diese Änderung der Sprachsyntax keinerlei Einschränkungen für die 
Funktionalität oder die Leistungsfähigkeit von \mutabor{} ergeben haben.

Eine weitere Änderung ab der Version 2.1 ist, dass zu hohe oder zu
tiefe Frequenzen nicht mehr unterdrückt werden, sondern in den
"`hörbaren Bereich"' hineinprojiziert werden. Näheres siehe Abschnitt
"`Handhabung von Grenzfällen"'.

\chapter{Fehlerursachen}\label{cha:fehlerursachen}

\section{Fehlermeldungen des Compilers}\label{sec:fehl-des-comp}
\index{Fehler!Compiler}

In diesem Anhang werden sämtliche Fehlermeldungen des Compilers
aufgeführt und erläutert, wie sie entstehen und wie sie
zu beheben sind. Anstelle der drei Fragezeichen (???)
erscheint der fehlerhafte Wert oder ein Hinweis auf
die Stelle wo sich der Fehler befindet.

\subsection{Allgemeine Fehler}\label{sec:allgemeine-fehler}

\paragraph{Datei- und Systemfehler} sollten eigentlich nie
auftreten. Sollte \mutabor{} trotzdem unter \textbf{reproduzierbaren}
Umständen abstürzen, so informieren Sie uns, damit
wir den Fehler beseitigen können.

\paragraph{Undefinierter Fehler in ???} ist ein Fehler, der
auf Programmierfehlern innerhalb von \mutabor{} beruht. Falls
dieser Fehler auftreten sollte, so prüfen Sie bitte, unter welchen
\emph{reproduzierbaren} Bedingungen der Fehler erscheint und melden
uns diesen Fehler. Wir sind bemüht, solche Fehler umgehend zu
beheben.

\paragraph{Syntaktischer Fehler,\ nicht näher spezifiziert.\ (Zeile ???)}
tritt auf, wenn das Lo\-gik-\-Pro\-gramm sich in gravierender Weise
nicht an die vorgegebene Syntax hält. Zur Behebung des Fehlers lesen
Sie bitte im Referenzhandbuch die Syntaxgraphen. Es kann sich bei
diesem Fehler um vergessene Kommata, um falsche Namensschreibungen, um
falsche Wahl der Klammern, um Vergessen von Zeichen, \dots{} -- kurz: um
so ziemlich alles handeln, was der Compiler nicht versteht.

\paragraph{Ungültiges Zeichen\ im Quelltext: asc=???\ (Zeile ???)}
tritt auf, wenn in der Datei, die das Logik-Programm enthält,
Zeichen enthalten sind, die nicht zum eigentlichen Text gehören.
Manche Editoren schreiben zusätzliche Steuerzeichen in den Text,
um z.\,B. Absätze, Fettschrift \ldots\ zu markieren. Solche
Steuerzeichen stören das Logik-Programm und müssen entfernt
 werden. Der Text des Logik-Programms muss also im ASCII-Format
gespeichert sein. Als Fehlercode wird der
Dezimalwert des fehlerhaften Zeichens und die Zeilen-Nummer
angezeigt, in der das fehlerhafte Zeichen auftrat.

\paragraph{Kann Datei nicht öffnen:\ ???} tritt auf, wenn die
angegebene Datei nicht existiert, oder fehlerhaft ist.
Es kann auch
sein, dass die Diskette voll ist oder dass die Datei
schreibgeschützt ist.

\paragraph{Speichermangel !} tritt auf, wenn der Speicher Ihres
Computers nicht mehr ausreicht. Es ist prinzipiell möglich,
dass einfach das Logik-Programm zu komplex ist.
Wahrscheinlicher ist jedoch, dass der Speicher von anderen
Programmen mitbenutzt wird.

\subsection{Doppeldeklarationen}\label{sec:doppeldeklarationen}
treten als Fehler auf, wenn mehrere Objekte gleichen Typs und gleichen
Namens deklariert werden. Wenn man diese Namen benutzen würde, ist
nicht mehr eindeutig, welche Deklaration gemeint ist.
\paragraph{Der Intervallname\ ???\ wurde doppelt benutzt}
\paragraph{Der Tonname ???\ wurde doppelt benutzt}
\paragraph{Der Tonsystemname\ ???\ wurde doppelt benutzt}
\paragraph{Der Umstimmungsname\ ???\ wurde doppelt benutzt}
\paragraph{Der Harmoniename ???\ wurde doppelt benutzt}
\paragraph{Der Logikname ???\ wurde doppelt benutzt}
\paragraph{Der Parametername\ ???\ wurde doppelt benutzt}
\paragraph{Das MIDI-Instrument ???\ wurde doppelt benutzt}
\paragraph{Überlappungsfehler bei\ der MIDI-Zuordnung:\ Kanal ???\ wurde doppelt benutzt}
\paragraph{Mehr als ein ANSONSTEN-Auslöser\ in Logik ???}
\paragraph{Mehr als ein ANSONSTEN in\ Umstimmung ???}
\paragraph{Die Alternative ???\ ist doppelt in ???}

\subsection{Undefinierte Symbole}\label{sec:undefinierte-symbole}
treten als Fehler auf, wenn man etwas benutzt, ohne deklariert 
zu haben, um was es sich handeln soll. Zur Fehlerbehebung
kann man die Deklaration des Symbols nachholen oder die
Benutzung des Symbols ausschließen, z.\,B. indem man
den Programmteil, der das undefinierte Symbol benutzt,
in Kommentar (\dq) einschließt.
\paragraph{Unbekanntes Symbol:\ ???}
\paragraph{Unbekanntes Intervall:\ ???}
\paragraph{Unbekannter Ton:\ ???}
\paragraph{Unbekanntes Tonsystem:\ ???}
\paragraph{Unbekannte Umstimmung:\ ???}
\paragraph{Unbekannte Harmonie:\ ???\ (in Logik ???)}
\paragraph{Unbekannter Parametername:\ ??? in ???}
\paragraph{Unbekannte Periode:\ ???\ (in Tonsystem ???)}
\paragraph{Unbekannter Ton:\ ???\ (in Tonsystem ???)}
\paragraph{Unbekannter Ton:\ ???\ (in Umstimmung ???)}
\paragraph{Unbekanntes Intervall:\ ???\ (in Umstimmung ???)}
\paragraph{Unbekannte Umstimmung:\ ???\ (in Umstimmung ???)}
\paragraph{Unbekannte Einstimmung:\ ???\ (in Logik ???)}
\paragraph{Unbekannte Aktion:\ ???\ (in Logik ???)}
	 
\subsection{Bereichsüber- bzw. -Unterschreitungen}\label{sec:bere-bzw.-unterschr}
treten auf, wenn der angegebene Wert nicht in den
erlaubten Grenzen liegt.
\paragraph{MIDI-Kanal ungültig\ (zulässig: 1 bis 16)}
\paragraph{Unzulässiger Wert\ von Intervall ???}
\paragraph{Taste ??? liegt außerhalb\ des Wertebereichs 36..96\ (in Tonsystem ???)}
\paragraph{In Umstimmung ???\ ist der Ton ???\ nicht komplex}
\paragraph{Unzulässiger Wert\ in ???}
\paragraph{Unzulässiger MIDI-Code\ in Logik ???\ (erwarte ???)}
\paragraph{Logik ??? darf nicht mit\ ANSONSTEN aufgerufen werden}
\paragraph{Als Taste eines Auslösers\ wurde kein einzelner\ Buchstabe angegeben\ TASTE ???}
     
\subsection{Parameterfehler}\label{sec:parameterfehler}
treten auf, wenn der Mechanismus zur Übergabe von Parametern innerhalb
von Aufrufen oder Umstimmungen nicht passt, oder wenn Parameter
benutzt werden, die nicht ordnungsgemäß deklariert wurden. Prüfen Sie
in diesem Fall die gemeldete Fehlerstelle und die Deklaration der
Umstimmung.
\paragraph{In Umstimmungsbund/case\ ???\ stimmen in\ Umstimmung ???\ die Parameter nicht}
\paragraph{In Logik ???\ stimmt die Parameter-\ anzahl von ??? nicht}
\paragraph{In Logik ???\ ist die Einstimmung\ ??? nicht\ parameterlos}
\paragraph{in Umstimmungsbund ???\ ist das Tonsystem ???\ nicht parameterlos}
\paragraph{in Umstimmungs\_case ???\ ist das Tonsystem ???\ nicht parameterlos}
	 
\subsection{Gegenseitige Abhängigkeiten}\label{sec:gegens-abhang}
sind innerhalb von Tönen und Umstimmungen möglich, aber verboten.
Stellen Sie die Struktur der Abhängigkeiten so um, dass gegenseitige
Abhängigkeiten nicht mehr vorkommen. Die Struktur der Abhängigkeiten
muss einen gerichteten, azyklischen Graphen ergeben, der nicht
notwendigerweise zusammenhängend ist.\footnote{Zu Deutsch: zeichnet
  man sich die Abhängigkeiten im Programm folgendermaßen auf: Die
  Töne, Intervalle usw. werden als Punkte dargestellt. Immer dann,
  wenn ein Punkt (A) in der Deklaration eines anderen Punktes (B)
  auftaucht, zeichnen wir einen Pfeil von B nach A. Diese Zeichnung
  ergibt am Ende einen gerichteten Graphen. Finden wir einen Punkt C,
  von dem ein Pfeil weggeht und man entlang der Pfeile auch wieder
  zurückkommt, so ist das ein Kreis. Der Graph ist dann nicht
  azyklisch. Es darf also keinen solchen Kreis geben.}

\paragraph{Die Töne ??? und\ ??? hängen gegenseitig\ voneinander ab}
\paragraph{Die Umstimmungen/Logiken ???\ und ??? hängen\ gegenseitig\ voneinander ab}
\paragraph{Die Intervalle ???\ und ??? hängen\ gegenseitig\ voneinander ab}
	 
\subsection{Syntaxfehler}\label{sec:syntaxfehler}
treten auf, wenn das Logik-Programm nicht den Syntaxgraphen des
Referenzhandbuchs entspricht.  Im Gegensatz zu \emph{Syntaktischer
  Fehler,\ nicht näher spezifiziert.\ (Zeile ???)} wird
hier Bezug auf das gerade gelesene Programmkonstrukt genommen.
\paragraph{Falsches Zeichen!\ Erwarte ??? (Zeile ???)}
\paragraph{Fehlerhafte Intervall-\ deklaration in\ Intervall ???}
\paragraph{Fehlerhafte Tondeklaration\ von Ton ???}
\paragraph{Fehlerhafte Tonsystem-\ deklaration. (Zeile ???)}
\paragraph{Ungültige Pameterliste\ in Zeile ???}
\paragraph{Ungültige Umstimmung\ in Zeile ???}
\paragraph{Ungültige Harmoniedekl.\ bei ???}
\paragraph{Fehlerhafte MIDI-Liste\ in Zeile ???}
\paragraph{Nach einem \protect\# folgt keine\ Hex-Ziffer (Zeile ???)}


\section{Warnungen}\label{sec:warnungen}
sind Meldungen an den Benutzer, dass das Programm zwar korrekt
funktioniert, aber zweifelhafte Konstruktionen enthält.  Es ist
nützlich, diese Warnungen zu beachten, es sei denn, man beabsichtigt
gerade die dadurch hervorgerufenen Effekte.
\paragraph{MIDI-Auslöser in Logik ???\ beginnt nicht mit Kanal 0-\ Statusbyte.\ Wert wurde korrigiert!}
Siehe Abschnitt \ref{STATUSBYTE}
\paragraph{Unmöglicher Harmonieauslöser\ in Logik ???}
Siehe Abschnitt \ref{UNMOEGLICHER}
\paragraph{Mehrdeutiger Auslöser ??? in Logik ???}

\paragraph{Auslöser ANSONSTEN ist nicht der letzte in Logik ???}
Diese Warnung ist belanglos, es dient aber der besseren
Lesbarkeit, wenn diese Alternative am Ende und nicht
irgendwo zwischen den Alternativen steht.
\paragraph{Undefinierte Compilerwarnung}
ist eine Warnung, die auf Programmierfehlern innerhalb von \mutabor{}
beruht. Falls diese Warnung auftreten sollte, so prüfen Sie bitte,
unter welchen \textbf{reproduzierbaren} Bedingungen die Warnung erscheint
und melden uns diesen Fehler. Wir sind bemüht, solche Fehler umgehend
zu beheben.

\paragraph{Verschwendung bei Kanalzuordnung} bedeutet, dass Sie bei der
MIDI"=Kanal"=Deklaration nicht alle 16 möglichen MIDI-Kanäle benutzen.

\paragraph{Konfigurationsdatei fehlerhaft} In der Datei \filename{MUTABOR.CFG} sind fehlerhafte Einträge. Sie können diese Datei
löschen und danach \mutabor{} erneut starten und wieder verlassen.
Dabei wird die Datei mit den Grundeinstellungen neu erzeugt.

\section{Sonstige Fehler}\label{sec:sonstige-fehler}

\index{Fehler!Laufzeit}
\subsection{Es ist nichts zu hören}\label{sec:es-ist-nichts}

Wenn Sie nichts hören, so kann das vielfältige Ursachen haben.  Solche
Pannen sind natürlich ärgerlich, und wir sind deswegen bemüht, hier
eine möglichst vollständige Liste an Ursachen anzugeben, die schuld
daran sein könnten, dass Sie nichts hören.

\begin{itemize}
\small

\item Sind alle Geräte angeschaltet und mit Strom versorgt ? Ist der
  Verstärker richtig eingestellt und die Lautstärke aufgedreht ? \dots{}
  Auch wenn es lächerlich erscheint --- dieser Fehler ist
  erfahrungsgemäß der häufigste.

\item Das Masterkeyboard sendet auf einem MIDI-Kanal, welcher nicht
  als MidiKanal bei \mutabor{} eingetragen ist. Gewöhnlicherweise
  sollte das Masterkeyboard auf MIDI-Kanal 1 senden.

\item Die Klangerzeugungseinheit (Synthesizer/Sampler) ist nicht
  korrekt für den Betrieb mit \mutabor{} eingestellt. Es muss auf jedem
  MIDI-Kanal die gleiche Klangfarbe mit Pitch-Bend-Range 1 eingestellt
  sein (Multi-Mode). 
\iffalse
 Ausnahme: der Yamaha FB-01. Hier könnte es sein,
  dass der Synthesizer angeschaltet wurde, \emph{nachdem} das
  Laufzeitmodul aktiviert wurde.  Der FB-01 hat den
  Initialisierungscode verpasst. Abhilfe: verlassen Sie das
  Laufzeitmodul und starten Sie es erneut.
\fi

\item Es ist eine Stimmungslogik aktiv, die nur aus stumm geschalteten
  Tasten besteht, oder aus Tönen, die außerhalb des spielbaren
  Bereichs liegen.

\item Sie befinden sich noch gar nicht im Laufzeitmodul. Nur das
  Laufzeitmodul erzeugt hörbare Töne, wenn Sie in der normalen
  Benutzerumgebung sind und auf der Klaviatur spielen, so wird keine
  Note an den Synthesizer geschickt!

\item Sie haben, bevor Sie das Laufzeitmodul aufgerufen haben, sehr
  viele MIDI-Daten vom Masterkeyboard an den Computer geschickt (durch
  wildes Anschlagen von Tasten), so dass der MIDI-Puffer des Computers
  überlastet und das System abgestürzt ist. Dieser Effekt taucht zwar
  nur selten auf, trotzdem ist es sinnvoll, nur dann etwas auf der
  Klaviatur zu spielen, wenn das Laufzeitmodul auch aktiviert wurde
  und somit in der Lage ist, Ihre Noten auch auszuwerten.

\item Eine Hardwarekomponente ist defekt. Z.\,B. ein MIDI-Kabel oder gar
  der Synthesizer oder das Masterkeyboard.

\end{itemize}

Wenn gar nichts hilft, so besteht die letzte Rettung darin, den
Computer und alle anderen Komponenten auszuschalten und erneut zu
starten, und zwar in der Reihenfolge Computer -- Synthesizer. Sollten
Sie auch dann noch kein positives Ergebnis erzielen, so sollten Sie
mit uns Kontakt aufnehmen, um den Fehler zu beheben.

\subsection{Die Intonation ist verkehrt}\label{sec:die-intonation-ist}
\index{Intonation!fehlerhafte}
Dies liegt meistens daran, dass die Pitch-Bend-Range entweder bei allen 
oder bei einem MIDI-Kanal nicht korrekt eingestellt ist. Jede Klangfarbe 
muss eine Pitch-Bend-Range von 1 Halbton haben. Da die Pitch-Bend-Range 
meistens ein Parameter der eingestellten Klangfarbe und nicht ein globaler 
Parameter des Synthesizers ist, müssen Sie die Klangfarbe entsprechend 
modifizieren. Diese Einstellung ist nur in den seltensten Fällen 
kompliziert und ist dem Bedienungshandbuch des benutzten Synthesizers zu 
entnehmen. Informationen über Sinn und Zweck der Pitch-Bend-Range 
entnehmen Sie dem Abschnitt "`Arbeitsweise der Synthesizertreiber"'.

\subsection{Es sind immer zwei Töne zu hören}\label{sec:es-sind-immer}

Dieser Effekt tritt immer dann auf, wenn Klaviatur und Klangerzeugung in 
demselben Gerät stecken. Der eine Ton kommt vom Synthesizer selbst und 
intoniert die gewöhnliche gleichstufige Stimmung, der andere Ton ist 
der von \mutabor{} produzierte Mikroton. Um nur die Töne von \mutabor{} zu 
hören, müssen Sie Ihren Synthesizer in die MIDI-Betriebsart "`LOCAL~OFF"' 
versetzen, damit nur Töne erzeugt werden, die über MIDI geschickt wurden. 
Wie Sie diese Einstellung vornehmen, die bei einigen Synthesizern nach 
jedem neuen Anschalten des Gerätes vorgenommen werden muss, kann hier 
leider nicht allgemein gültig angegeben werden. Schlagen Sie hierzu im 
Bedienungshandbuch des Synthesizers nach.

\subsection{Das Instrument ist nur monophon spielbar}\label{sec:das-instrument-ist}

Sie haben vergessen, den Synthesizer in den Multi-Mode zu versetzen und 
auf allen sechzehn MIDI-Kanälen dieselbe Klangfarbe einzustellen. Wie Sie 
Ihren Synthesizer richtig einstellen, lesen Sie auf Seite 
\pageref{EINSTELLUNG} und im Bedienungshandbuch des Synthesizers.

\subsection{Es bleiben ungewollt Töne liegen}\label{sec:es-bleiben-ungewollt}

Dies ist ein Effekt, der eigentlich nicht auftreten sollte. Er kommt
normalerweise nur bei einer Überlastung des Systems mit angeschlagenen
Tasten (sehr viele Anschläge in extrem kurzen Zeiträumen) vor. Drücken
Sie entweder die Panic-Funktion oder schalten Sie den Synthesizer aus-
und wieder an oder wechseln Sie kurz die Klangfarbe und stellen Sie
sie wieder her, da bei diesem Vorgang die meisten Synthesizer alle
liegenden Töne ausschalten. Nur im größten Notfall ist es nötig, das
Laufzeitmodul zu verlassen und erneut aufzurufen.

\chapter{Beschreibung der Demonstrationslogiken "`demo.mut"'}\label{cha:beschr-der-demonstr}

Es ist von größter Wichtigkeit, dass Sie vor dem Aufruf der
Demonstration alle Geräte angeschaltet, richtig miteinander verbunden
und korrekt eingestellt haben. (\emph{MULTI-MODE mit gleicher
  KLANGFARBE auf allen 16 Kanälen, PITCH-BEND-RANGE überall auf 1,
  MASTERKEYBOARD sendet auf KANAL 1} \/ !!! Falls Sie Probleme mit der
Einstellung haben, so lesen Sie sicherheitshalber die Abschnitte
"`Fehlerursachen"' und "`Arbeitsweise der Synthesizer-Treiber"'. Wenn
Sie das Demonstrationsprogramm starten und alles funktioniert, so wird
die Stimmung des Instruments auf eine gleichstufige
Halbton-Temperierung eingestellt -- also die ganz gewöhnliche Stimmung
heutiger Instrumente.

\section{G -- Gleichstufige Temperaturen}\label{sec:g-gleichst-temp}
\index{Temperaturen!gleichschwebende}
Im Mittelpunkt der Stimmungslogik "`GLEICH"' stehen
gleichstufige Teilungen der Oktave\footnote{Frequenzfaktor $\sqrt[n]{2}$}. 
Diese Stimmungslogik können Sie aktivieren, indem Sie die 
Taste 'G' drücken. 

Innerhalb dieser Stimmungslogik stehen Ihnen fünf verschieden abgestufte 
gleichstufige Tonsysteme zur Ver\-fü\-gung:
Halbtöne (12-Teilung der Oktave), Dritteltöne (18-Teilung 
der Oktave), Vierteltöne (24-Teilung der Oktave), Achteltöne 
(48-Teilung der Oktave) und Sechzehnteltöne (96-Teilung der Oktave).
Außerdem können Sie auf den schwarzen Tasten eine Pentatonik (5-Teilung der 
Oktave) spielen. 

\section{H -- Historische Stimmungen}\label{sec:h-hist-stimm}
\index{Stimmung!historische}
Hier haben wir für Sie verschiedene historische Stimmungen von 
Musikinstrumenten (meist Klavier oder Orgel) 
vorbereitet\footnote{Den Stimmungen liegen die Schriften 
"`Musikalische Temperierungen"' von Müller Schulze und 
"`Das Stimmen"' von Heribert Schmitt zugrunde.}. Sie 
können diese sehr leicht mit der heute üblichen 
gleichstufigen Temperierung vergleichen, indem Sie einfach -- 
auch bei liegenden Noten -- zwischen der 
Stimmungslogik GLEICH und einem Tonsystem der Stimmungslogik HISTORISCH 
hin- und her schalten. Der Ton c hat in allen Tonsystemen die gleiche 
Frequenz. Bitte beachten Sie, dass das Aktivieren der Logik 
HISTORISCH noch \emph{kein} neues Tonsystem einstellt. Wenn Sie also 
gerade eine gleichstufige Stimmung eingestellt haben und dann die 
Logik HISTORISCH durch Drücken der Taste 'H' aktivieren, so werden die 
Töne noch nicht verändert. Erst wenn Sie die Taste für eine 
konkrete Stimmung (innerhalb der Logik HISTORISCH) drücken, so 
wird diese eingestimmt.

Besonders bei diesen Stimmungen ist die Benutzung der
Protokollfunktionen zum Anzeigen der aktuellen Stimmung sehr nützlich,
um die Intervallstruktur der Tonsysteme zu erforschen (siehe
Begleithandbuch "`Die Benutzeroberfläche"'.

Folgende historische Stimmungen sind programmiert: Pythagore\"{\i}sche
Tonleiter, mitteltönige Stimmung, Silbermann, Werckmeister, Schlick,
Kirnberger 3, "`Bach"'-Stimmung.  \index{Tonleiter!Pythagoreische}
\index{Stimmung!mitteltönige} \index{Mitteltönig}
\index{Stimmung!Silbermann} \index{Silbermann}
\index{Stimmung!Werckmeister} \index{Werckmeister}
\index{Stimmung!Schlick} \index{Schlick} \index{Stimmung!Kirnberger 3}
\index{Kirnberger 3} \index{Stimmung!Bach} \index{Bach}


\section{N -- Tonales Netz}\label{sec:n-tonales-netz}
\index{Netz!tonales}
Diese mutierende Stimmung ermöglicht es, in immer reinen Dur-- bzw. 
Mollakkorden zu spielen. Der Computer erkennt die Akkorde und 
intoniert Terzen und Quinten rein\footnote{Dies entspricht einem auf 
"`Modulation"' geschalteten einfachen Automaten zur Intonation in reiner 
Stimmung nach M.\,Vogel.}. Das dieser mutierenden Stimmung 
zugrunde liegende Prinzip ist das "`Wandern im tonalen Netz 
des (2,3,5)"=frei erzeugten Tonsystems"'. Man hört jeweils einen 
rhombusartigen Ausschnitt aus dem zweidimensionale Tonnetz (nach rechts 
sind reine Quinten, nach oben reine Terzen aufgetragen), wobei der 
aktuelle harmonische Grundton im Zentrum des Rhombus steht:

\input{netz.pic}

Das System wird auf C-Dur initialisiert. Es werden immer genau die 
Töne gespielt, die innerhalb des Rhombus liegen. Wird nun eine andere 
Tonalität gespielt, z.\,B. ein D-Dur-Dreiklang, so verschiebt sich der 
Rhombus, so dass sein Mittelpunkt über dem Ton D liegt. Dies bedeutet, dass 
gewisse Töne umgestimmt werden müssen, nämlich alle, die nicht in der 
Schnittmenge beider Rhomben liegen. Das wiederholte Spielen gewisser 
Kadenzen, z.\,B. C--d--F--G--C bewirkt somit eine immer weitere 
Entfernung vom Ausgangspunkt, was an einem deutlich hörbaren 
"`Absinken"' der Stimmung zu bemerken ist (immerhin etwa 
einen Viertelton pro Kadenz).

\section{O - Obertöne}\label{sec:o-obertone}
Eine sehr interessante Hörerfahrung bietet das Spielen mit der
Obertonreihe. Die weißen Tasten dieses statischen Tonsystems sind aus
einer Obertonreihe auf der Frequenz 110Hz aufgebaut, welche auf dem
eingestrichenen c' beginnt und sich über drei Oktavlagen erstreckt.
Auf den schwarzen Tasten liegt eine Untertonreihe auf demselben
Grundton, allerdings um einige Oktaven nach oben verschoben. Probieren
Sie einfach mal diese "`Naturklänge"' aus - z.\,B. indem Sie Klänge aus
geradzahligen mit ungeradzahligen Obertönen vergleichen, oder einen
Cluster auf hohen schwarzen Tasten spielen, gefolgt von einem Cluster
weißer Tasten ab dem c'. Das einfache Prinzip der Obertonreihe birgt
zahllose neue Tonhöhenerlebnisse in sich \ldots

\section{Modifizieren der Demo-Logiken}\label{sec:modif-der-demo}

Die Demonstrationslogiken können Sie sehr einfach verändern und somit 
Ihre eigene Demonstration programmieren, da wir den Quelltext der 
Demonstration im \mutabor{}"=Verzeichnis unter dem Namen \filename{demo.mut}
gespeichert haben. 

Wenn Sie die Demonstrationslogiken verändern ist dieser Abschnitt des
Handbuchs natürlich nicht mehr gültig. Er bezieht sich ausdrücklich
auf die von uns ausgelieferte Version von \filename{demo.mut}. Es
könnte sehr lehrreich sein, wenn Sie sich einmal das Logikprogramm
\filename{demo.mut} im Editor ansehen würden, auch ohne es zu
verändern, da in diesem Programm viele Elemente der Programmiersprache
von \mutabor{} benutzt werden, die abzuschauen sich lohnen könnte.

\chapter{Beispiele auf Diskette}\label{cha:beisp-auf-disk}
Im Lieferumfang von \mutabor{} sind verschiedene
Beispielprogramme enthalten. Viele Programmierbeispiele aus dem
Handbuch finden Sie hier bereits fertig getippt. Wenn dies der Fall
ist, so ist in einer Fußnote der Programmname der Beispiellogik auf
Diskette angegeben.

Um die Stelle im Handbuch zu finden, an der die Beispiele auf Diskette
näher erläutert sind, finden Sie hier ein Verzeichnis der Beispiele
auf der Beispieldiskette, sofern die Logikprogramme im Handbuch
erwähnt werden.

\vspace{5mm}

\begin{center}
\begin{tabular}{l|c}
Das Logikprogramm \dots{} & finden Sie auf Seite \dots{} \\ \hline
anker.mut & \pageref{ANKER} \\
aufab.mut & \pageref{AUFAB} \\
auswahl.mut & \pageref{AUSWAHL} \\
c\_b.mut & \pageref{CB} \\
c\_dur.mut & \pageref{CDUR} \\
chaos.mut & \pageref{CHAOS} \\
drittel.mut  &  \pageref{DRITTEL} \\
gleichsh.mut & \pageref{GLEICHSCH} \\
komentar.mut & \pageref{KOMENTAR} \\
mininetz.mut & \pageref{MININETZ} \\
naturspt.mut & \pageref{NATURSPT} \\
netz.mut & \pageref{NETZ} \\
penta.mut & \pageref{PENTA} \\
periode.mut & \pageref{PERIODE} \\
pythago.mut & \pageref{PYTHAGO} \\
xantippe.mut & \pageref{XANTIPPE} \\
\end{tabular}
\end{center}

\chapter{Glossar}\label{cha:glossar}

\begin{description}
\item[{Grundparameter}] Die vier Grundparameter, die ein Tonsystem
  vollständig beschreiben sind Verankerungstaste, Breite, Töne der
  Fundamentaltonleiter und das Periodenintervall.

\item[{Fundamentaltonleiter}] Dieser im Handbuch vielleicht am
  häufigsten erwähnte Begriff bezeichnet einen bestimmten Blockbereich
  der Klaviatur, beginnend bei der Verankerungstaste, welcher
  genauso viele Tasten umfasst, wie die FT breit ist. In der
  Fundamentaltonleiter sind alle Informationen enthalten, die nötig
  sind, um den gesamten Klaviaturtasten Frequenzen zuweisen zu können.

\item[{Laufzeitmodul}] Der Teil von \mutabor{}, welcher für das
  Live-Musizieren zuständig ist. Der Compiler generiert aus einem
  Logikprogramm ein eigenständiges Maschinenprogramm, das
  "`Laufzeitmodul"'.

\item[{MIDI-Kanal}] Vergleichbar mit einem Radio kann der empfangende
  Synthesizer aus allen ankommenden MIDI-Meldungen die für ihn
  bestimmten herausfiltern. Wird z.\,B. über die MIDI-Leitung eine
  NOTE-ON-Meldung auf Kanal 5 gesendet, so reagieren nur diejenigen
  Geräte in der MIDI-Kette darauf, die als Empfangskanal ebenfalls den
  Kanal 5 eingestellt haben.

\item[{MIDI-Kette}] Werden mehrere MIDI-Geräte so miteinander
  verbunden, dass die Meldungen von Gerät A nach Gerät B nach Gerät C
  \dots{} geschickt werden, so spricht man von einer MIDI-Kette.

\item[{MIDI-Protokoll}] Genaue Spezifikation der Datenübertragung über
  die MIDI-Schnittstelle, z.\,B. Festlegung, dass Statusbytes das achte
  Bit gesetzt haben, dass \#9x eine Note-On-Meldung ist, \dots{}

\item[{MIDI-Schnittstelle}] Genormte Verbindung zwischen Computern und
  Synthesizern bzw.\ Samp\-lern. über diese Schnittstelle
  kommunizieren die Geräte und teilen sich z.\,B. mit, welche Tasten auf
  der Klaviatur gerade gedrückt wurden, welche Klangfarbe auf dem
  Synthesizer eingestellt werden soll, und vieles mehr.

\item[{Periodenintervall}] Das Periodenintervall bildet die Töne der
  Fundamentaltonleiter auf die restliche Klaviatur ab. Zwei Tasten,
  die im gleichen Abstand zueinander stehen, wie die
  Fundamentaltonleiter breit ist (in Halbtonschritten gerechnet),
  stehen zueinander im Frequenzverhältnis des Periodenintervalls.
  (siehe Seite \pageref{MATHE})

\item[{Pitch-Bender}] Die meisten Masterkeyboards haben einen
  Dreh-Regler, mit dem man die Tonhöhe aller liegenden Töne quasi
  stufenlos erhöhen bzw. erniedrigen kann. Spaltet man den Synthesizer
  intern in bis zu 16 eigenständige Instrumente mit derselben
  Klangfarbe auf, so kann jeder Kanal einen eigenen Pitch-Bend-Wert,
  also eine eigene Verstimmung bekommen. Auf diese Weise entlockt
  \mutabor{} nahezu jedem handelsüblichen Synthesizer/Sampler die
  Mikrotöne.

\item[{Sequenzer}] Ein Computerprogramm, das wie ein
  Mehrspur-Aufnahmegerät auf dem MIDI-Key\-board gespielte Noten
  aufzeichnen, bearbeiten, speichern und wiedergeben kann.

\item[{Töne der Fundamentaltonleiter}] Die Töne der
  Fundamentaltonleiter enthalten die Information, welche Frequenzen
  den Tasten der FT zugeordnet sind, bzw. welche Tasten gesperrt sind.

\item[{Verankerungstaste}] Die unterste Taste der Fundamentaltonleiter
  wird Verankerungstaste genannt. Sie muss nicht notwendigerweise einen
  Ton tragen, kann also durchaus gesperrt sein.
\end{description}

\chapter{Schlussbemerkung}\label{cha:schlu3bemerkung}
Wir haben versucht, die Programmiersprache zum Erstellen von
Tonsystemen und Stimmungslogiken klar und verständlich zu
gestalten, so dass die zugrunde liegenden Konzepte prägnant zum
Ausdruck kommen.

Der Leser bzw. die Leserin  sei an dieser Stelle ermutigt, sich
durch praktische Anwendung mit \mutabor{} vertraut zu
machen. Die Programmiersprache \mutabor{} ist eine
Art Universalinstrument, von dem wir selbst nicht vorhersehen
können, was damit alles möglich ist. Demzufolge hat
\mutabor{} auch den Charakter eines
Experimentalinstruments. Es ist nun die Sache des Musikers und
Komponisten, die verschiedenen Möglichkeiten dieses Instruments zu
erforschen.

Wir selbst betrachten \mutabor{} nicht als endgültig
feststehende Programmiersprache, sondern als Experimentiersprache,
die ihrerseits weiter entwickelbar ist. Wohin diese Entwicklung
gehen soll und wird, hängt maßgeblich von der Resonanz aus der
Musikwelt ab. Für Lob und Kritik, insbesondere aber für
Verbesserungsvorschläge haben wir stets ein offenes Ohr.

Wir wünschen Ihnen auf jeden Fall viele neue musikalische
Erfahrungen mit \mutabor{}.

%\end{document}

\ifhtml\else\cleardoublepage\fi


%\part{Referenzhandbuch}
%\include{referenz}
\iffalse
\pagestyle{empty}

\mbox{}\vfill
\begin{center}
\Huge \textbf{\mutabor{} II} \\[1cm]
 \large \sl Ein computergesteuertes Musikinstrument \\
 \large \sl zum Experimentieren mit\\
 \large \sl Stimmungslogiken und Mikrot"onen \\[1cm]
	{\huge Referenzhandbuch} \\[1cm]
\scriptsize \copyright 1991, 1992 Volker Abel \& Peter Reiss\\
\end{center}
\vfill\mbox{}
\ifhtml\else\clearpage\fi


% Nummerierung der Kapitel
% damit es linear weitergeht (Nach dem Anhang...)

\setcounter{chapter}{\value{savedchapter}}%
                           % Muss im Bedienungshandbuch 
                           % gesetzt werden (vor \appendix)
\setcounter{savedchapter}{\value{page}} % muss nach \mainmatter
                                % wiederhergestellt werden.
\setcounter{section}{0}%
\def\thechapter{\arabic{chapter}}% 
\global\let\chapter\savedchaptercommand
\mainmatter
\setcounter{page}{\value{savedchapter}}
\makeatletter
\xdef\@chapapp{\chaptername}%
\ifx\theHchapter\undefined
\else
\renewcommand\theHchapter    {\arabic{chapter}}
\renewcommand\theHfigure     {\theHchapter.\arabic{figure}}
\renewcommand\theHtable      {\theHchapter.\arabic{table}}
\renewcommand\theHsection    {\theHchapter.\arabic{section}}
\def\Hy@chapapp{\Hy@chapterstring}
\fi
\makeatother


\hyphenation{Har-mo-nie-na-me
	     Har-mo-nie-na-men
	     kon-sis-tent kon-sis-ten-te kon-sis-ten-ter}

\newcommand{\bi}{\begin{itemize}}
\newcommand{\ei}{\end{itemize}}


\pagestyle{headings}

\chapter{Sprachreferenz}\label{cha:sprachreferenz}
\label{eins}

Das Referenzhandbuch soll einerseits ein kurzes
Nachschlagewerk sein, andererseits stellt es die Basisbeschreibung
der Logiksprache \mutabor{} dar. Es ist nicht dafür geeignet, die
Logiksprache zu erlernen, dafür sei an dieser Stelle auf das
Bedienungshandbuch verwiesen, sondern gibt bei zweifelhaften
Fragen exakte Auskünfte. Das Referenzhandbuch ist dafür gedacht,
die Funktionalität der Logiksprache festzulegen. Wenn ein Benutzer
von \mutabor{} feststellt, dass eine von ihm programmierte Logik
nicht so funktioniert, wie er es wünscht, so ist dieses
Referenzhandbuch die Entscheidungsinstanz, anhand derer man
feststellen kann, ob das Logikprogramm falsch ist, oder ob
\mutabor{} nicht korrekt funktioniert.
Dieser Abschnitt beschreibt die Logiksprache \mutabor{} in der Version 3.0.


\chapter{Lexikalische Konventionen}\label{cha:lexik-konv}
\label{zwei}

Ein Programm besteht aus einem Text, der in einer Datei gespeichert ist.
Es wird in einer einzigen Phase übersetzt.

\section{Eingabesymbole}\label{sec:eingabesymbole}
\index{Eingabesymbole}
\label{zweieins}

Es  gibt fünf Klassen\index{Wortklassen} von Eingabesymbolen:
Namen\index{Namen},  reservierte  Worte\index{Worte!reservierte},
Zahlen,
 Operatoren  und  andere
Trenner.  Leerzeichen,\index{Leerzeichen} Tabulatorzeichen,
Vertikal-Tabulatoren,
Zeilentrenner,\index{Zeilentrenner} Seitenvorschub\index{Seitenvorschub}
sowie Kommentare\index{Kommentare},
allgemein mit Zwischenraum  bezeichnet,  werden
ignoriert,  abgesehen davon,  da\ss{} sie zur Trennung direkt
benachbarter Namen, Zahlen oder reservierter Worte n\"otig sind.

Wenn  die Eingabe bis zu einem bestimmten Zeichen schon in Eingabesymbole
zerlegt  wurde,  wird als n\"achstes
Eingabesymbol die l\"angstm\"ogliche  Kette
von Zeichen aufgefasst, die ein Eingabesymbol darstellen kann.


\section{Kommentare}\label{sec:kommentare}
\index{Kommentare}
\label{zweizwei}

Kommentare\index{Kommentar} werden in doppelte Anführungszeichen (\sourcecode{\dq})
\index{\dq (Anführungszeichen)}eingeschlossen.
Kommentare  k\"onnen  nicht
verschachtelt\index{Verschachtelung} werden.
Kommentare dürfen sich über mehrere Zeilen erstrecken.

\section{Namen}\label{sec:namen}
\label{zweidrei}

Ein Name\index{Namen} besteht aus einer Folge von
Buchstaben und Ziffern;  das
erste Zeichen mu\ss{} ein Buchstabe sein.  Dabei z\"ahlen
der Unterstrich  (\sourcecode{_})\index{\sourcecode{_} (Unterstrich)} und der Apostroph
(\textbf{'})\index{' (Apostroph)}
zu den Buchstaben.\footnote{Der Unterstrich zählt zu den Buchstaben, damit
man aussagekräftige Namen gut lesbar schreiben kann, z.\,B.
\sourcecode{LOGIK Drittel\_Ton\_Mutierend} ist besser als
\sourcecode{LOGIK dritteltonmutierend}
und der Apostroph zählt als Buchstabe,
damit man Töne mit einfach oder mehrfach gestrichenen
Namen versehen kann, z.\,B. \sourcecode{TON e${}'''$ }}
Gro\ss{}e und kleine Buchstaben werden gleich behandelt.
Namen können beliebig lang sein.
Zur  Unterscheidung von Namen wird der gesamte Name herangezogen,
und nicht nur die ersten paar Buchstaben, wie dies bei anderen
Programmiersprachen üblich ist\index{Signifikanz}.

In den Büchern von \mutabor{} werden die Begriffe \emph{Name}\index{Namen}
und \emph{Bezeichner}\index{Bezeichner}
parallel und mit gleicher Bedeutung verwendet. Ferner werden bei einigen
Syntaxgraphen auch Begriffe der Art \emph{Intervallname}, \emph{Tonname},\ldots
verwendet. Diese Begriffe sind syntaktisch gesehen einfache Namen.
Es soll damit zum Ausdruck kommen, dass Namen einer bestimmten Art von
Objekten gemeint sind. Dies gehört prinzipiell nicht zur Syntax der
Logiksprache, sondern wird erst später semantisch überprüft. Es erhöht
jedoch das Verständnis, wenn direkt erkennbar ist, welche Art von Namen
in Frage kommt.

\section{Reservierte Worte}\label{sec:reservierte-worte}
\label{zweivier}

Die  folgenden  Worte sind reserviert\index{Worte!reservierte}
 und k\"onnen  nur  mit  ihrer
vordefinierten Bedeutung verwendet werden:

%\vspace{1cm}
\begin{center}
\begin{tabular}{|llll|}
	      \hline
   ANSONSTEN & FORM   & HARMONIE & INTERVALL \\
   LOGIK  & MIDIIN   & MIDIKANAL & MIDIOUT    \\
   TASTE     & TON    & TONSYSTEM &   UMSTIMMUNG \\
   WURZEL    &        &           &              \\
 \hline
\end{tabular}
\end{center}
%\vspace{1cm}

Ebenso dürfen die folgenden englischen Schlüsselworte
benutzt werden. Sie sind ebenfalls reserviert und
dürfen nicht als Namen benutzt werden.

\begin{center}
\begin{tabular}{|llll|}
	      \hline
   ELSE & SHIFTED   & PATTERN & INTERVAL \\
   LOGIC  & MIDIIN   & MIDICHANNEL & MIDIOUT    \\
   KEY     & TONE    & TONESYSTEM &  RETUNING \\
   ROOT    &        &           &              \\
 \hline
\end{tabular}
\end{center}

Alle anderen Wörter dürfen als Namen oder Bezeichner verwendet werden.

\section{Zahlen}\label{sec:zahlen}
\index{Zahlen}
\label{zweifuenf}

Zahlen können ganzzahlig oder mit Dezimalpunkt geschrieben werden.
Eine Schreibweise mit Zehnerexponent ist nicht
zugelassen.\index{Zahlen} Ferner kann eine Zahl auch mit dem Zeichen
\textbf{\#}\index{\#} beginnen und weitere Hex-Ziffern\index{Hex-Ziffern}
enthalten. Diese Schreibweise bedeutet, dass die Zahl zur Basis 16
ausgewertet wird. In dieser Hex-Schreibweise ist kein Dezimalpunkt
erlaubt, folglich sind nur ganzzahlige Werte darstellbar.

Ähnlich wie bei Namen werden Zahlen im Syntaxdiagramm auch als \emph{Faktor}
oder \emph{Verankerungs-Taste} bezeichnet,
um ihre jeweilige semantische Bedeutung
hervorzuheben. Es handelt sich aber syntaktisch gesehen um ganz gewöhnliche
Zahlen.

An einigen Stellen sind nur ganze Zahlen zugelassen (z.\,B. als
Verankerungs-Taste, Tonsystem-Breite,\ldots) und an anderen Stellen
sind auch Komma-Zahlen zugelassen, (Tonfrequenz,
In\-ter\-vall\-gr"o"se,\ldots).



\chapter{Syntax Schreibweise}\label{cha:syntax-schreibweise}
\label{drei}

In der Syntax-Darstellung in diesem Buch werden die Grammatikbegriffe
\emph{in eckigen Kästchen} und reservierte Worte und Zeichen \emph{in
ovalen Kästchen} geschrieben. Die Syntax erhält man, indem man auf den
Pfeilen entlang durch das Syntaxdiagramm läuft.
Alternative Formulierungen erscheinen dabei als Verzweigungen
und Wiederholungen ergeben sich, indem das Syntaxdiagramm Zyklen enthält.
Ein Beispiel:

\input{syn1.pic}

Das Wort \emph{TON} muss genau so geschrieben werden, während das Konstrukt
\emph{EinzelTon} noch genauer aufgelöst werden muss. Ferner ergibt sich durch
die Schleife, dass mehrere Konstrukte von der Art \emph{EinzelTon}
aufeinander folgen können.


\chapter{Das Logikprogramm}\label{cha:das-logikprogramm}
Das Logikprogramm ist der oberste Grammatikbegriff und entspricht
dem gesamten Programm. Das Logikprogramm selbst ist eine beliebige
Folge von Deklarationen\index{Deklarationen}.
Insbesondere ist hervorzuheben, dass auch
garnichts, also ein leeres File, ein zulässiges Programm ist.
In diesem Fall ist keine Logik vorhanden und die normale
gleichstufig temperierte Stimmung eingestellt.

Es können Intervalle, Töne, Tonsysteme, Umstimmungen, Harmonien,
Logiken und Midi-Kanäle deklariert werden. Die einzelnen Elemente können
in beliebiger Reihenfolge deklariert werden. Insbesondere kann auf
Elemente Bezug genommen werden, die erst später deklariert werden.

\input{syn2.pic}

\chapter{Intervall}\index{Intervall}\label{cha:intervall}

Eine Intervalldeklaration definiert einzelne Intervalle und die
dazugehörenden Werte. Intervallnamen dürfen nicht doppelt definiert
werden, sie dürfen jedoch gleiche Namen tragen,
wie Objekte einer anderen Art.
Der Wert eines Intervalls ist ein Frequenzverhältnis.
Der Wert eines Intervalls kann auf eine von drei Arten angegeben werden:
\bi
  \item als Zahlenverhältnis,
  \item als Wurzelausdruck\index{Wurzel} oder
  \item als Kombination aus anderen Intervallen.
\ei
Es bedeutet dabei
\sourcecode{Zahl1 : Zahl2} das Verhältnis von Zahl1 zu Zahl2 und
\sourcecode{A WURZEL B} ist die \texttt{A}te Wurzel aus \texttt{B}.
Wenn Intervalle aus anderen Intervallen zusammengesetzt werden,
so dürfen keine gegenseitigen Abhängigkeiten entstehen, sondern
die Werte der Intervalle müssen in konsistenter Form berechenbar sein.
Alle Intervalle die zum Zusammensetzen anderer Intervalle benutzt werden,
müssen innerhalb des Logik-Programms deklariert werden.
Optional kann einem Intervall auch ein Faktor\index{Faktor}
 vorangestellt werden, dann gilt der Wert dieses Intervalls
entsprechend mehrfach. Da es sich bei Intervallen eigentlich um
Frequenzverhältnisse handelt, ergibt sich der Wert eines zusammengesetzten
Intervalles multiplikativ. Zum Beispiel wird das Intervall
\begin{verbatim} 
INTERVALL I = 3 oktave - 5 terz + 17 cent 
\end{verbatim}
 nach folgendem Schema
berechnet: \[I = \mathtt{oktave}^{3} * \mathtt{terz}^{-5} * \mathtt{cent}^{17}\]

\input{syn3.pic}

\input{syn5.pic}

\chapter{Ton}\label{cha:ton}
\index{Ton}

Eine Tondeklaration definiert einzelne Töne und die
dazugehörenden Werte. Tonnamen dürfen nicht doppelt definiert
werden, sie dürfen jedoch gleiche Namen tragen,
wie Objekte einer anderen Art.
Der Wert eines Tons ist eine Zahl, die der Frequenz des Tons
entspricht. Der Wert eines Tons kann eine einfache Zahl sein oder
sich aus einem Bezugston\index{Bezugston}
 ergeben, der möglicherweise
um bestimmte Intervalle verschoben wird.
Im ersteren Fall wird die Frequenz\index{Frequenz}
 des Tons direkt angegeben.
Im anderen Fall folgt nach dem Gleichheitszeichen der Name eines
Bezugstons, sowie eine Folge von Intervallen, mit denen die Frequenz des
Bezugstons multipliziert wird, bis sie den gewünschten Ton ergibt.
Grundsätzlich müssen alle Töne, die als Bezugston herangezogen werden,
im Logikprogramm definiert sein. Ferner darf die Definition aller Töne
keine gegenseitigen Abhängigkeiten enthalten.

Sämtliche Töne ergeben nach deren Berechnung immer eine feste Frequenz.
Ein Tonname steht also als Synonym für eine
absolute Frequenz\index{Frequenz!absolute}, deren
Wert sich multiplikativ ergibt. Beispiel:
\begin{verbatim} 
TON  b = c - 2 quint + oktave 
\end{verbatim}
wird berechnet als \[Frequenz(\mathtt{b}) = Frequenz(\mathtt{c}) * \mathtt{quint}^{-2} * \mathtt{oktave}\].

\input{syn4.pic}



\chapter{Tonsystem}\label{cha:tonsystem}
\index{Tonsystem}
Eine Tonsystemdeklaration definiert einzelne Tonsysteme und die
dazugehörenden Werte. Tonsystemnamen dürfen nicht doppelt definiert
werden, sie dürfen jedoch gleiche Namen tragen,
wie Objekte einer anderen Art.
Ein Tonsystem besteht aus einer Verankerungstaste\index{Verankerungstaste},
einer Menge von Tönen
und einem Periodenintervall\index{Periodenintervall}.
Das Periodenintervall kann auch aus mehreren Intervallen
kombiniert werden.
Die Fundamentaltonleiter wiederum
besteht aus einer Folge von Tönen, die in eckigen Klammern
``\textbf{[ ]}''\index{[ ] (eckige Klammern)}
 stehen und durch Kommata getrennt sind.
Sie entspricht einem Bereich auf der Klaviatur. Die Zuordnung der Töne
zu den Tasten ergibt sich durch die Verankerungstaste.
Sie entspricht der MIDI-Nummer\index{MIDI!Nummer} des ersten Tones der
Fundamentaltonleiter.\footnote{Das eingestrichene $c'$ entspricht
der MIDI-Tastennummer 60.}
Von dieser Position aus werden nach rechts folgend die weiteren Töne
den Klaviaturtasten zugeordnet. Die nächste Taste rechts daneben entspricht
folglich dem zweiten Ton der Fundamentaltonleiter, etc\ldots\ .
Wenn nach diesem Schema
alle Töne der Fundamentaltonleiter zugeordnet sind, wird die gesamte
Fundamentaltonleiter sowohl nach unten, wie nach oben wiederholt und
dabei um das Periodenintervall\label{Periodenintervall}
verschoben, bis die Grenzen der Klaviatur erreicht sind.

Bei der Deklaration der Fundamentaltonleiter\index{Fundamentaltonleiter}
dürfen einzelne Töne fehlen, nicht jedoch die trennenden Kommata.
Diese fehlenden Töne bewirken,
dass die entsprechende Taste keinen Ton erzeugt.
Fehlende Töne werden bei der Zuordnung
$Klaviaturtaste \rightarrow Ton$ normal
mitgezählt, als wären sie vorhanden.



\input{syn6.pic}


\chapter{Umstimmung}\label{cha:umstimmung}
\index{Umstimmung}
Eine Umstimmungsdeklaration definiert einzelne Umstimmungen und die
dazugehörenden Werte. Umstimmungsnamen dürfen nicht doppelt definiert
werden, sie dürfen jedoch gleiche Namen tragen,
wie Objekte einer anderen Art.
Eine Umstimmung enthält Regeln, nach denen die aktuelle Stimmung
\index{Stimmung!aktuelle} zu
verändern ist. Der Wert der Änderung kann sowohl absolut als auch
relativ zum alten Zustand angegeben werden. Bei einer absoluten Änderung
wird eine Zahl oder ein symbolischer Parameter angegeben, der den
neuen Wert darstellt, ohne Rücksicht auf den vorherigen Zustand.
Bei einer relativen Änderung wird der bisherige Wert als Bezugswert
genommen und um einen gewissen Betrag verändert. Semantisch bedeutet das
Symbol \textbf{@}\index{\sourcecode{@}} `\emph{der alte Wert}',
auf den Bezug genommen werden soll.

Falls die Umstimmung von einem oder mehreren Parametern\index{Parameter}
 abhängig sein soll,
so ist für jeden Parameter\index{Parameter!symbolischer Name}
 ein symbolischer Parameter-Name anzugeben.
Ein Parameter-Name kann später im Umstimmungs-Ausdruck benutzt werden.

\input{syn8.pic}


Es gibt sieben Arten von Umstimmungs-Ausdrücken, die jeweils einzelne
Merkmale der aktuellen Stimmung ver\-än\-dern:

\input{syn7.pic}

\begin{description}
  \item[Umstimmung-Verankerung]\index{Umstimmung!Verankerung}
      entspricht einer Verschiebung
      der Fundamentaltonleiter. Dabei wird angegeben, wo sich die
      neue Verankerungs-Taste befinden soll. Diese neue Verankerungstaste
      behält ihre Frequenz bei und alle anderen Töne ergeben sich,
      indem relativ zur neuen Verankerungstaste die gesamte
      bisherige Intervallstruktur\index{Intervallstruktur}
      übernommen wird.

      Im allgemeinen folgt aus einer Veränderung der Verankerungstaste auch
      eine Veränderung der Töne.

      \input{syn9.pic}

      \input{syn26.pic}

  \item[Umstimmung-Breite]\index{Umstimmung!Breite}
      entspricht einer Veränderung der Breite
      der Fundamentaltonleiter.\index{Fundamentaltonleiter!Breite der}
      Der erste Ton der Fundamentaltonleiter bleibt erhalten und alle
      Töne rechts davon werden im aktuellen Zustand festgehalten.
      Dann wird abgezählt, wieviele der festgehaltenen Töne benötigt
      werden, um die neue Breite der Fundamentaltonleiter zu erhalten.
      Diese Töne bilden dann die neue Fundamentaltonleiter.
      Das Intervall von der Verankerungstaste zur ersten Taste nach der
      Fundamentaltonleiter ergibt das neue Periodenintervall.

      Im allgemeinen folgt aus der Veränderung der Breite auch eine Veränderung
      des Periodenintervalls. Eine Verkürzung der Breite, die zu einer
      Breite kleiner 1 führt, wird ignoriert. Eine Verbreiterung, die
      zu einer Breite von mehr als 60 Tasten führt, wird ignoriert.

      \input{syn10.pic}


  \item[Umstimmung-Töne]\index{Umstimmung!Töne}
      entspricht einer Änderung einzelner Töne
      innerhalb der Fundamentaltonleiter.
      Hier wird für jeden Ton der Fundamentaltonleiter ein Rechenausdruck
      angegeben, um den sich der Ton zu verändern hat. Diese Rechenausdrücke
      sind durch Kommata getrennt und beziehen sich in ihrer Reihenfolge
      auf die entsprechenden Töne, wie sie bei einem Tonsystem
      angegeben werden.
      Soll ein Ton unverändert
      erhalten bleiben, so ist dafür das Symbol \sourcecode{@}\index{\sourcecode{@}}
      zu schreiben, weil
      `\emph{nichts}' bedeutet, dass der Ton stummgeschaltet wird.
      Falls die Anzahl der Rechenausdrücke kleiner ist, als die
      Breite der Fundamentaltonleiter, so bleiben die überzähligen Töne
      unverändert. Falls die Breite der Fundamentaltonleiter kleiner ist,
      als die Anzahl der Rechenausdrücke, so bleiben die
      überzähligen Rechenausdrücke ohne Effekt.

      \input{syn11.pic}

      \input{syn12.pic}

  \item[Umstimmung-Periode]\index{Umstimmung!Periode}
      entspricht einer Änderung des
      Periodenintervalls\index{Periodenintervall}.
      Hier wird der Wert des
      Periodenintervalls verändert, so dass die Töne der Fundamentaltonleiter
      erhalten bleiben, aber alle anderen Töne neu berechnet werden,
      da sie sich aus der Fundamentaltonleiter und der Periodenverschiebung
      ergeben.

      \input{syn13.pic}



  \item[Umstimmungs-Bund]\index{Umstimmung!Bund}
      Ein Umstimmungsbund umfasst eine Folge von
      Aktionen.\index{Aktionen} Die einzelnen
      Aktionen können andere Umstimmungen,
      Tonsysteme oder Logiken aufrufen, oder auch MIDI-Nachrichten
      senden. Dabei dürfen keine gegenseitigen
      Abhängigkeiten entstehen. Die einzelnen Aktionen werden in der
      aufgeführten Reihenfolge ausgeführt.
      Die Parameter\index{Parameter} eines
      Umstimmungsbundes können an die einzelnen Aufrufe weitergegeben
      werden, sofern der aufgerufene Name eine Umstimmung ist und
      diese Umstimmung auch entsprechend viele Parameter akzeptiert.
      Die mitgegebenen Parameter werden in der gleichen Reihenfolge
      den Parameter-Namen der Umstimmung zugeordnet.


      \input{syn14.pic}

      \input{syn28.pic}

      \input{syn23.pic}

      \input{syn17.pic}


    \item[Umstimmungs-Alternative]\index{Umstimmung!Alternative} Eine
      Umstimmungs-Alternative ermöglicht die Auswahl einer von
      mehreren Alternativen in Abhängigkeit eines
      Parameters\index{Parameter} der Umstimmung (der
      Selektor\index{Selektor}).  Die Alternativen bestehen jeweils
      aus der Konstanten, die einen der möglichen Fälle bestimmt, und
      einer Folge von Aktionen, welche durchgeführt werden, falls die
      Auswahl für diese Alternative zutrifft. Die einzelnen Aktionen
      einer Alternative sind mit Komma (\texttt{,}) getrennt, während
      komplette Alternativen nicht durch ein besonderes Trennzeichen
      getrennt sind. Die Konstanten der einzelnen Alternativen müssen
      ganzzahlig und eindeutig sein. Am Ende der Alternativen darf
      einmalig die Spezialkonstante \sourcecode{ansonsten}\index{Ansonsten}
      stehen. Falls der Selektor mit einer der Konstanten
      übereinstimmt, wird diese Alternative durchgeführt. Falls der
      Selektor mit keiner Konstanten übereinstimmt und die
      Spezialkonstante \sourcecode{ansonsten} vorhanden ist, wird die
      Alternative des \sourcecode{ansonsten}-Zweigs durchgeführt. Falls der
      Selektor mit keiner Konstanten übereinstimmt und die
      Spezialkonstante \sourcecode{ansonsten} nicht vorhanden ist, wird keine
      der Alternativen durchgeführt, der Aufruf dieser Umstimmung
      bleibt in diesem Fall folgenlos.

      \input{syn16.pic}

      \item[Aktion-Midi-Out]\index{Aktion!MIDI-Out}
	Midi-Ausgaben werden mit dem Wort \emph{MIDIOUT}\index{MIDIOUT}
	 eingeleitet und
sind eine in runde Klammern eingeklammerte
und durch Kommata getrennte Folge von Zahlen. Bei Zahlen ist die sedezimale
Schreibweise zulässig (\#\emph{ziffern}).\index{\#} Die Zahlen
müssen im Bereich von 0 bis 255 liegen, also Bytes sein. Es erfolgt keine
Kontrolle, ob es sich um eine sinnvolle Midi-Nachricht handelt.

      \input{syn22.pic}

\end{description}

\chapter{Harmonie}\label{cha:harmonie}
\index{Harmonie}

Eine Harmoniedeklaration definiert einzelne Harmonien und die
dazugehörenden Werte. Harmonienamen dürfen nicht doppelt definiert
werden, sie dürfen jedoch gleiche Namen tragen,
wie Objekte einer anderen Art.
Der Wert einer Harmonie ist eine Folge von Tasten-Nummern,\index{Tasten!Nummer}
welche einzelne Tasten innerhalb der Fundamentaltonleiter identifizieren.
Die Tasten der Klaviatur, welche außerhalb der Fundamentaltonleiter
stehen, werden in die Fundamentaltonleiter hineinprojiziert und
bilden die Projektionstonleiter,\index{Projektionstonleiter}
welche die gleiche Breite
hat, wie die Fundamentaltonleiter, aber stattdessen die Information
enthält, welche (projizierten) Tasten (Tonigkeiten\index{Tonigkeit})
gerade gedrückt sind.
Eine Harmonie wiederum wird erkannt, wenn in der Projektionstonleiter genau
die angegebenen Tastennummern gedrückt sind. Dabei können einzelne
Tasten unberücksichtigt bleiben. Die entsprechenden Tastennummern sind
durch einen Stern (*)\index{*} gekennzeichnet.
Optional kann eine Bezugstaste\index{Bezugstaste}\index{Tasten!Bezugs-}
mit angegeben werden.
Ihr Wert ändert nicht das Harmonie-Erkennungsmuster, sondern gibt
bei der Harmonieform-Analyse eine relative Verschiebung
der Bezugstonigkeit\index{Bezugstonigkeit}
 in dem Wert \texttt{ABSTAND}\index{ABSTAND} an.
Falls keine Bezugstaste angegeben ist, so wird ihr Wert als $0$ angenommen,
also ohne Verschiebung der Bezugstonigkeit.

\input{syn15.pic}



\chapter{Logik}\label{cha:logik}
\index{Logik}
Eine Logikdeklaration definiert einzelne Logiken und die
dazugehörenden Werte. Logiknamen dürfen nicht doppelt definiert
werden, sie dürfen jedoch gleiche Namen tragen,
wie Objekte einer anderen Art.
Logiken werden durch einen Auslöser\index{Auslöser} aktiviert und
bestehen aus einer Einstimmung\index{Einstimmung}
 und einer Menge von Anweisungen\index{Anweisung!Menge von}.
Die Einstimmung darf ein Tonsystem oder eine Umstimmung sein.
Die Einstimmung darf fehlen, in diesem Fall wird die vorherige Stimmung
beibehalten. Die Menge der Anweisungen darf ebenfalls leer sein, in diesem
Fall bleibt die Logik in der aktuellen Stimmung stehen.
Eine Anweisung wiederum besteht ebenfalls aus einem Auslöser, sowie einer
Liste von Aktionen.
Die Aktionen dürfen der Aufruf einer Umstimmung, eines Tonsystems
oder einer Logik sein, oder das Senden einer Midi-Nachricht.

Grundsätzlich existieren zwei Mengen von Auslösern: Die Auslöser, welche
eine der Logiken aktivieren, und die Auslöser, die innerhalb der
gerade aktiven Logik die Anweisungen anwählen. Erstere nennen wir
\emph{globale Auslöser}\index{Auslöser!globale}
 und letztere \emph{lokale Auslöser}\index{Auslöser!lokale}. Beim Eintreffen
eines Ereignisses wird zuerst die Menge der globalen Auslöser
und dann die Menge der lokalen Auslöser überprüft, ob das Ereignis
den Auslöser auslöst. Innerhalb der globalen und lokalen Auslöser wird
in der Reihenfolge ihrer Deklaration gesucht und beim ersten zutreffenden
Auslöser die Suche beendet.
Wenn ein globaler Auslöser zutrifft, so wird in die entsprechende Logik
gewechselt. Wenn ein lokaler Auslöser zutrifft, so werden die entsprechenden
Aktionen innerhalb der aktuellen Logik aufgerufen.
Innerhalb der Menge der globalen Auslöser ist der Spezialauslöser
\sourcecode{ansonsten}\index{ansonsten} nicht erlaubt.

\input{syn24.pic}

\input{syn27.pic}

\section{Auslöser}\label{sec:ausloser}
\index{Auslöser}

Auslöser können wahlweise eine der 26 Buchstabentasten, eine Harmonie,
eine Harmonieform,
ein MIDI-Ereignis oder der Spezialauslöser \sourcecode{ansonsten} sein.

\input{syn18.pic}

\subsection{Tasten}\index{Tasten}\label{sec:tasten}

Dieser Auslöser\index{Auslöser!Taste}
 besteht aus dem Wort \emph{Taste}\index{Tasten}, gefolgt von einem der
26 Buchstaben. Wenn innerhalb von \mutabor{} die angegebene Taste auf der
Tastatur gedrückt wird, so gilt dieser Auslöser als erfüllt.
Der Auslöser bezieht sich jeweils auf das Instrument bzw.\ MIDI-Kanal,
das vorher
mit den Zifferntasten angewählt wurde. Man kann also nur die
Instrumente/MIDI-Kanäle 1--9 mit einem Tasten-Auslöser steuern.

      \input{syn19.pic}

\subsection{Harmonien}\label{sec:harmonien}
\index{Harmonien}
Harmonie-Auslöser\index{Auslöser!Harmonie}
 können wahlweise eine Harmonie oder eine Harmonieform
\index{Auslöser!Harmonie-Form}
erkennen. Die Harmonie wird beim Namen genannt und die enthaltenen
Tasten-Nummern beziehen sich auf die entsprechenden Ton-Nummern der
Fundamentaltonleiter\index{Fundamentaltonleiter}.
Alle gedrückten Tasten werden in die
Fundamentaltonleiter hineinprojiziert und das entstehende Tastenmuster
wird mit der Harmonie verglichen. Bei Übereinstimmung gilt die
Harmonie als erkannt. Tasten, die in der Harmonie mit einem
Stern (*)\index{*} gekennzeichnet sind, werden bei diesem
Vergleich nicht berücksichtigt.
Einschränkenderweise kann man mit der Vortaste\index{Vortaste} angeben,
welche Tonigkeit\index{Tonigkeit}
der tiefste Ton haben muss, damit diese Harmonie gilt. Entsprechend muss
bei Angabe der Nachtaste\index{Nachtaste} der höchste Ton
die Tonigkeit der Nachtaste
haben. Bei Angabe von Vortaste und Nachtaste müssen beide Bedingungen
erfüllt sein. Die Tonigkeit eines Tons bezieht sich hier auf die Breite
der aktuellen Fundamentaltonleiter.\footnote{Diese Breite ist nicht
auf 12 festgelegt, sondern in weiten Grenzen frei wählbar.}
Wenn Vortaste oder Nachtaste eine Tonigkeit haben, die nicht in der
Harmonie vorkommt, so kann diese Harmonie nicht gespielt werden, da sie
sich selbst widerspricht. Solche Harmonien sind '\emph{unmögliche}'
Auslöser\index{Auslöser!unmögliche}
 und werden mit einer Warnung gekennzeichnet. Praktisch
bleiben sie ohne Effekt, da sie nie zutreffen können.

Bei einer Harmonieform\index{Harmonieform} wird überprüft,
ob die Harmonie relativ zu jedem
Ton der Fundamentaltonleiter zutrifft. Wird eine solchermaßen
verschobene Harmonie erkannt, gilt der Auslöser als erfüllt
und der Wert \texttt{ABSTAND}\index{ABSTAND} wird entsprechend der Harmonieanalyse gesetzt.

Wenn bei der Harmonie eine Bezugstaste\index{Bezugstaste}
 angegeben wurde, dann wird
die Harmonie-Analyse relativ um die Bezugstaste verschoben durchgeführt.

    \input{syn20.pic}

\subsection{MIDI-Ereignisse}\index{Auslöser!MIDI-Ereignis}\label{sec:midi-ereignisse}
Midi-Ereignisse werden mit dem Wort \emph{MIDIIN}\index{MIDIIN}
eingeleitet und
sind eine eingeklammerte Folge von Zahlen. Bei Zahlen ist die sedezimale
Schreibweise zulässig (\#\emph{ziffern}).\index{\#}
Die erste Zahl muss im Bereich von
128 bis 255 liegen, also ein Status-Byte sein. Alle weiteren Zahlen
müssen im Bereich von 0 bis 127 liegen, also Daten-Bytes sein.
Beim Status-Byte müssen die unteren vier Bits (Kanal-Nummer) auf Null stehen.
Die tatsächliche Kanalnummer wird dem jeweiligen
Instrument/MIDI-Kanal zugeordnet,
so dass ein Midi-Ereignis sich jeweils auf das richtige
Instrument/MIDI-Kanal bezieht.
Die Status-Bytes\index{Status-Byte}
 \emph{Note-On\index{Note-On} (\#90)}, \emph{Note-Off\index{Note-Off}
  (\#80)} und
\emph{System-Byte\index{System-Byte}
 (\#F0 - \#FF)} werden nicht in die Midi-Analyse
einbezogen.

    \input{syn21.pic}

\section{Aktionen}\index{Aktionen}\label{sec:aktionen}

Aktionen werden innerhalb einer Logik ausgelöst und können wahlweise
ein Tonsystem aufrufen, eine Umstimmung aufrufen,
zu einer Logik wechseln oder eine Midi-Nachricht
senden. Beim Aufruf einer Umstimmung müssen genau so viele Parameter\index{Parameter}
mitgegeben werden, wie die aufgerufene Umstimmung benötigt. Die mitgegebenen
Parameter werden in der gleichen Reihenfolge den Parameter-Namen
der Umstimmung zugeordnet. Als Parameter können Zahlen angegeben werden,
sowie der Wert \emph{ABSTAND}\index{ABSTAND}.
Der Wert ABSTAND entspricht den Gegebenheiten der
zuletzt durchgeführten Harmonie-FORM-Analyse, wobei der Wert
ABSTAND bis zur nächsten Harmonie-FORM-Analyse beibehalten wird.

Eine Logik kann eine einzelne Aktion ausführen oder eine in geschweiften
Klammern stehende Aktions-Liste aus\-füh\-ren. Im Falle der Aktions-Liste
werden die einzelnen Aktionen der Liste in genau der angegebenen
Reihenfolge ausgeführt.


\subsection{MIDI-Ausgaben}\index{MIDI!Ausgabe}\label{sec:midi-ausgaben}
\index{Aktion!MIDI Ausgabe}

Midi-Ausgaben werden mit dem Wort \emph{MIDIOUT}\index{MIDIOUT}
 eingeleitet und
sind eine eingeklammerte Folge von Zahlen. Bei Zahlen ist die sedezimale
Schreibweise zulässig (\#\emph{ziffern}). Die Zahlen
müssen im Bereich von 0 bis 255 liegen, also Bytes sein. Es erfolgt keine
Kontrolle, ob es sich um eine sinnvolle Midi-Nachricht handelt.

    \input{syn22.pic}

\chapter{MIDI-Kanäle}\label{cha:midi-kanale}
\index{MIDI!Kanal}
\label{MIDIKANAL}

Eine MIDI-Kanal-Deklaration definiert die Zuordnung, auf welchen
MIDI-In-Kanal \mutabor{} reagieren soll, und wieviele MIDI-Out-Kanäle
den einzelnen Eingabekanälen zugeordnet werden. Dies ist für die
Tonerzeugung nach dem Multi-16-Verfahren\index{Multi-16-Verfahren}
 nötig. Wird als
Synthesizer-Treiber das Modell FB-01\index{FB-01} eingestellt, so gilt nur
die Angabe des MIDI-In-Kanals, der Ausgabekanal wird ignoriert.
Die MIDI-In-Kanäle werden auch als \emph{Instrumente} bezeichnet, da
für jeden MIDI-In-Kanal separat das gesamte Logikprogramm gilt.
Es können bis zu 16 Instrumente gleichzeitig und in
verschiedenen Logiken spielen.
Werden alle 16 Instrumente/Midi-Kanäle
und das Multi-16-Verfahren benutzt, so kann jedes Instrument nur
noch einen einzigen
Ton gleichzeitig spielen, da nicht mehr als eben diese 16 Midikanäle
auch als Ausgabe-Kanäle verfügbar sind. Beim Modell FB-01 von Yamaha wird
eine spezielle System-exklusive Ansteuerungsmethode verwendet,
so dass hier bis zu 16 Instrumente zu je 16 gleichzeitigen Tönen
spielbar sind.

Beim Multi-16-Verfahren wird
jedem Eingabekanal ein Bereich von Midi-Out-Kanälen zugeordnet. Die Größe
dieses Bereichs entspricht der Anzahl Töne,
die mit diesem Eingabekanal gleichzeitig gespielen werden können.
Die Kanal-Information des empfangenen Statusbytes bestimmt, für
welchen Eingabekanal die Meldung gilt.
Jeder Eingabekanal hat für sich das gesamte Logik-Programm
zur Verfügung und arbeitet unabhängig von den anderen Eingabekanälen.
Bei der Analyse einer Harmonie und bei der Analyse einer Midi-in-Nachricht
wird berücksichtigt, auf welchen Eingabekanal es sich bezieht.
Auf dem Bildschirm kann mit den Zifferntasten 1 bis 9 zwischen den
Eingabekanälen 1 bis 9 umgeschaltet werden. Ein Tasten-Auslöser
bezieht sich immer auf den am Bildschirm angezeigten Eingabekanal.
Falls keine MIDI-Kanäle deklariert werden, und auch keine Kanalzuordnung
in der Konfigurationsdatei angegeben ist,
gilt die Deklaration
\begin{verbatim}
	MIDIKANAL  1  ->  1 - 16
\end{verbatim}
als automatisch gegeben.

\input{syn25.pic}

\section{Worterklärungen}\label{sec:worterklarungen}
\begin{description}
\item[Taste] Die Klaviatur-Tasten werden als Abstand in
  Tastatur-Halbtönen zur Verankerungstaste angegeben. Die
  Verankerungstaste hat somit die Nummer 0, die nächste Taste weiter
  rechts die Nummer 1, die nächste die 2 usw.

\item[Selektor] Der Selektor einer Umstimmungs-Alternative ist der
  Parameter, nach dem sich die Alternativen ausrichteten. In der Regel
  ist dies ein Parameter (d.\,h. der Parametername) der Umstimmung und
  repräsentiert so eine ganze Zahl.

\item[Buchstabe] Hier können sie einen der 26 Standard-Buchstaben des
  lateinischen Alphabets angeben (also keine Umlaute). Wenn Sie dann
  auf der Computer"=Tastatur diese Taste drücken wird das zugeordnete
  Ereignis ausgelöst. Es wird dabei nicht zwischen Klein- und
  Großbuchstaben unterschieden.

\item[Verankerungs-Taste] Die Verankerungs-Taste ist die
  Klaviatur-Taste, bei der die Fundamentaltonleiter beginnt und wird
  als MIDI-Nummer der Taste angegeben (also als Zahl). Das
  eingestrichene c (c') hat die Nummer 60, die anderen Tasten sind
  aufsteigend bzw. absteigend durchnummeriert, d.\,h.  c\#' hat die
  Nummer 61, d' die 62 usw.

\item[Konstanten] Die Konstanten ganze Zahlen, die festlegen, bei
  welchem Wert des Selektors in die
  darauffolgende Umstimmung verzweigt wird.

\item[Namen] Viele Aufrufe und Zuordnungen laufen in \mutabor{}
  über die Namen ab. Beim deklarieren eines jeden Objektes müssen Sie
  diesem einen Namen geben, den Sie später als Referenz in anderen
  Deklarationen angeben können.

\item[Zahlen] Zahlen können in Mutabor als Ganze Zahlen oder
  Dezimalzahlen (mit Punkt als Komma) angegeben werden.

\item[MIDI-Werte] MIDI-Werte sind ganze Zahlen zwischen 0 und 255, die
  von MIDI-Geräten als Befehle und Daten ausgewertet werden.
  Zusätzlich ist hier die hexadezimale Schreibweise erlaubt, der ein
  Gatterzeichen (\#) voranzustellen ist, also z.\,B. \#9C, \#A0, \#78,
  \#E0 \dots
\end{description}

%\chapter{Konfiguration}
%\index{Konfiguration}
%
%Die Datei \filename{MUTABOR.CFG} enthält die globalen Voreinstellungen bezüglich
%der an den Instrumenten angeschlossenen Synthesizer-Treiber und
%der Voreinstellungen der Kanalzuordnung bei den MIDI-Kanälen.
%Als Treibernummern sind bisher vorgesehen:
%\begin{description}
% \item[1] System-Exclusiv, Typ Yamaha FB01
% \item[2] Standard Midi nach Multi-16-Methode
%\end{description}
%Falls hier MIDI-Kanäle deklariert sind, so gilt deren Kanal-Zuordnung,
%es sei denn, das Logikprogramm gibt eine eigene Kanal-Zuordnung vor.
%Falls keinerlei Kanalzuordnung getroffen wird, so gilt die Zuordnung
%\sourcecode{1 -> 1 - 16} als gegeben.

\fi


\printindex
%\input{handbuch.ind}

\end{document}
