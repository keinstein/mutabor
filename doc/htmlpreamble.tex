\setcounter{tocdepth}{1}
% Grundeinstellungen liegen in der Datei „tshtml.cfg“
\usepackage[%tshtml,
mutabor,
% Es folgt das Hauptformat: hier xhtml
            html,
% CSS2 verwenden
%            css2,
% in der Log-Datei Informationen zur Konfiguration von tex4ht ausgeben
          info,
% Fußnoten (und Literaturzitate) erscheinen beim überfahren der Marke
%            mouseover,
% Auf der ersten Seite wird auch ein „Weiter“-Pfeil angezeigt
            next,
            sections+,
% HTML-Dateien möglichst kurz -- maximale Aufsplittungstiefe
            4,
 %            nominitoc,
% Index 2-Spaltig
            index=2,
            Gin-dim+,
% HTML-Zeichensatz: UTF-8
            charset=utf-8,
            hyperref,
            NoFonts,
%            fonts+
%            fonts,
]{tex4ht}
\makeatletter
\edef\ts@savecatcode{\noexpand\catcode`\noexpand\:\the\catcode`\:\relax} 
%\catcode`\:11\relax
\AtEndDocument{\makehhk}
\newcommand\tshhkentry{}
\newcommand\tsatoclink{}

%\:::HRefTag=macro:
%#1#2->\if \relax #2\relax \else \:TagHTag {#2}\fi \HCode {<\tag:A \:newlnch \if
% \relax #1\relax \NOHREF: {#2}\else \HREF: "\get:hfile {#1}\:sharp #1"\fi \if \
%relax #2\relax \else \space \NAME: "#2"\fi \:attr \empty:lnk >}.
%<insert>    \show\:::HRefTag

\expandafter\def\expandafter\mut@sharp\expandafter{\csname :sharp\endcsname}

\newif\ifwx@inrange
\wx@inrangefalse
\def\wx@item#1{%
  \def\wx@item@{#1}%
  \let\wx@indextext\wx@item@
}
\def\wx@subitem#1{%
  \def\wx@subitem@{\wx@item@{}: #1}
  \let\wx@indextext\wx@subitem@
}
\def\wx@subsubitem#1{%
  \def\wx@subsubitem@{\wx@subitem@{} -- #1}%
  \let\wx@indextext\wx@subsubitem@
}
\def\wx@LNK#1#2#3#4{%
  \ifwx@inrange
  \wx@inrangefalse
  \else
  \HCode{<li>^^J}%
  \HCode{<object type="text/sitemap" >^^J%
\space<param name="Name" value="}\wx@indextext\HCode{" >^^J%
\space<param name="Local" value="}#1\mut@sharp #2\HCode{" >^^J%
</object>^^J}% 
%  \HCode{</li>^^J}%
  \fi
}
\def\loadwxindex{%
  {%
    \let\item\wx@item
    \let\subitem\wx@subitem
    \let\subsubitem\wx@subsubitem
    \let\LNK\wx@LNK
    \let\rangeto\wx@inrangetrue
    \InputIfFileExists{\jobname.wxi}{}{}%
  }%
}



\def\tsatoclink#1#2#3#4{%
  \typeout{Setting: \csname get:hfile\endcsname{#2} at \csname :sharp\endcsname #2}%
  \HCode{%
\tslinestart<object type="text/sitemap" >^^J
\tslinestart\space<param name="Name" value="}#4\HCode{" >^^J
\tslinestart\space<param name="Local" value="}\csname get:hfile\endcsname{#2}\csname :sharp\endcsname #2\HCode{" >^^J
\tslinestart</object>^^J}% "<
%   \expandafter\ifx \csname #3-def\endcsname\relax
%      \global \expandafter\let \csname #3-def\endcsname\def
%      \Link {#2}{#3}
%   \else
%      \Link {#2}{}
%   \fi 
%   {\Configure {ref}{}{}{}%
%     \let \EndLink =\empty
%     \let \H:Tag:attr \:gobbleII
%     \let \:::HRef \empty
%     \def \::hRef [##1]##2{}%
%     \def \::hRefTag [##1]##2##3{}%
%     \def \:::HRefTag ##1##2{}%
%     \Configure {cite}{}{}{}{}%
%     #4}%
%   \EndLink
}
\def\ts@hhkentry#1#2#3#4{%
  \def\tslinestart{#1\space}%
  \HCode{#1<li>^^J}%
  #3%
%  \HCode{#1</li>^^J}%
}
\def\tshhkentry{\ts@hhkentry{}}
\def\tstocendsubparagraph{}
\def\tstocendparagraph{\endsubparagraph}

\def\deftocmacro#1#2#3{
  \expandafter\def\csname tstocstart#1\endcsname{%
    \HCode{#3\space<ul><!-- u#1 -->^^J}%
    \expandafter\let\csname tocstart#1\endcsname\relax
    \expandafter\def\csname tocstop#1\expandafter\endcsname{\csname tstocstop#1\endcsname}
  }

  \expandafter\def\csname tstocskip#1\endcsname{\csname tocstop#2\endcsname}

  \expandafter\def\csname tstocstop#1\endcsname{%
    \csname tocstop#2\endcsname
    \HCode{#3\space</ul><!-- i#1 -->^^J}%
    \expandafter\def\csname tocstop#1\endcsname{\csname tstocskip#1\endcsname}
  }

  \expandafter\def\csname tstoc#2\endcsname{%
    \csname tocstop#2\endcsname
    \csname tocstart#1\endcsname
    \expandafter\def\csname tocstart#2\endcsname{\csname tstocstart#2\endcsname}
    \HCode{#3\space\space<!-- a#2 -->^^J}%
    \ts@hhkentry{#3\space\space}%
  }%
}

\def\tstocpart{
  \tocstoppart
  \ts@hhkentry{}%
}

\deftocmacro{part}{chapter}{}
\deftocmacro{chapter}{section}{\space}
\deftocmacro{section}{subsection}{\space\space}
\deftocmacro{subsection}{subsubsection}{\space\space\space}
\deftocmacro{subsubsection}{paragraph}{\space\space\space\space}
\deftocmacro{paragraph}{subparagraph}{\space\space\space\space\space}
\def\tocstopsubparagraph{}

\def\gobblenl{\@ifnextchar[\@gobblenl{}}
\def\@gobblenl[#1]{}
{\catcode`\^^J=\active
  \gdef\mknlsp{%
    \def^^J{ }}%
}
\def\@nl@end{nl@end}
\def\@removenl#1#2\@nl@end{%
  \ifx#1^^J
  \ 
  \else
  #1
  \fi
  \ifx#2\relax
  \else
  \expandafter\@removenl
  \fi
}

\newcommand\removenewline[1]{%
  \ifx#1\relax
  \else
    \@removenl#1\@nl@end
  \fi
}

\newcommand\mutabortitle{}
\ifhelp
  \let\mutaborsavesubject\subject
  \def\subject#1{%
    \def\mutabortitle{#1}%
    \mutaborsavesubject{#1}%
  }
\else
  \let\mutaborsavetitle\title
  \def\title#1{%
    \def\mutabortitle{#1}%
    \mutaborsavetitle{#1}%
  }
\fi
\def\mutabordefaulttopic{\jobname.html}
\newcommand\tsarg{}
\def\tsarg#1{#1}
\let\tsdotocentry\tsarg
\expandafter\let\expandafter\mut@gobbleIV\csname :gobbleIV\endcsname
\expandafter\let\expandafter\mut@gobbleIII\csname :gobbleIII\endcsname
\expandafter\let\expandafter\mut@gobble\csname :gobble\endcsname
\newcommand\makehhk{%
  {\ignorespaces
    \let\par\relax
    \def\showname##1{\expandafter\show\csname ##1\endcsname}%
    \let\@gnewline\space
    \def\@newline{\space\mut@gobbleIII}%
    \let\texorpdfstring\@secondoftwo%
    \let\fontencoding\mut@gobble
    \let\fontfamily\mut@gobble
    \let\fontseries\mut@gobble
    \let\fontshape\mut@gobble
    \let\fontsize\mut@gobble
    \let\@setfontsize\mut@gobbleIII
    \let\usefont\mut@gobbleIV 
    \let\selectfont\relax
    \let\LARGE\relax
    \let\normalsize\relax
    \Configure{PROLOG}{}
    \Configure{HTML}{}{}
    \Configure{HEAD}{}{}
    \Configure{HEAD}{\Configure{@HEAD}{}}{}
    \Configure{VERSION}{}
    \Configure{DOCTYPE}{}
    \Configure{@DOCTYPE}{}
    \Configure{@HEAD}{}
    \Configure{HTML}{}{}
    \Configure{BODY}{}{}
    \Configure{TITLE}{}{}
    % .hhp-Datei erstellen.
    \IgnorePar
%    \special{t4ht>\jobname.hhp}%
    \FileStream+{\jobname.hhp}%
    \HCode{%
Contents file=\jobname.hhc^^J%
Index file=\jobname.hhk^^J%
Title=}{%
      \removenewline\expandafter{\mutabortitle}}%
    \HCode{^^J%
Default Topic=\mutabordefaulttopic^^J
Charset=UTF-8^^J}%
    \EndFileStream{\jobname.hhp}%
%\special{t4ht<\jobname.hhp}%
\typeout{done}%
%\expandafter\show\csname a:TocLink\endcsname
    \let\doTocEntry\tsdotocentry
    \expandafter\let\csname a:TocLink\endcsname\tsatoclink
    \let\toclikesection\tshhkentry
    \let\toclikechapter\tshhkentry
    \let\tocaddchap\tshhkentry
    \let\tocpart\tshhkentry
    \let\tocsection\tshhkentry
    \let\tocchapter\tshhkentry
    \let\tocsubsection\tshhkentry
    \let\tocparagraph\tshhkentry
    \let\tocappendix\tshhkentry
    \let\tocminisec\tshhkentry
    \let\textsc\tsarg%
    \IgnorePar
    \FileStream+{\jobname.hhk}%
    \HCode{<ul>^^J}%
    \loadwxindex 
%    \catcode`\#11
    \InputIfFileExists{\jobname.4ct}{}{}%
    \HCode{</ul>^^J}%
    \EndFileStream{\jobname.hhk}%
%
    \let\toclikesection\tstocsection
    \let\toclikechapter\tstocchapter
    \let\tocaddchap\tstocchapter
    \let\tocpart\tstocpart
    \let\tocsection\tstocsection
    \let\tocchapter\tstocchapter
    \let\tocsubsection\tstocsubsection
    \let\tocparagraph\tstocparagraph
    \let\tocappendix\tstocchapter
    \def\tocminisec{\tocparagraph}%
%
    \let\tocstartpart\relax
    \let\tocstoppart\tstocskippart
    \let\tocstartchapter\relax
    \let\tocstopchapter\tstocskipchapter
    \let\tocstartsection\relax
    \let\tocstopsection\tstocskipsection
    \let\tocstartsubsection\relax
    \let\tocstopsubsection\tstocskipsubsection
    \let\tocstartsubsubsection\relax
    \let\tocstopsubsubsection\tstocskipsubsubsection
    \let\tocstartparagraph\relax
    \let\tocstopparagraph\tstocskipparagraph
    \let\tocstartsubparagraph\relax
    \let\tocstopsubparagraph\tstocskipsubparagraph
%
    \IgnorePar
    \FileStream+{\jobname.hhc}%
    \HCode{<ul>^^J}%
%    \catcode`\#11
    \InputIfFileExists{\jobname.4ct}{}{}%
    \HCode{</ul>^^J}%
    \EndFileStream{\jobname.hhc}%
  }%
  \relax
}
\endinput

